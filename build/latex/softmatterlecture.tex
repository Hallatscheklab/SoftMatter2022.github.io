%% Generated by Sphinx.
\def\sphinxdocclass{report}
\documentclass[letterpaper,10pt,english]{sphinxmanual}
\ifdefined\pdfpxdimen
   \let\sphinxpxdimen\pdfpxdimen\else\newdimen\sphinxpxdimen
\fi \sphinxpxdimen=.75bp\relax
\ifdefined\pdfimageresolution
    \pdfimageresolution= \numexpr \dimexpr1in\relax/\sphinxpxdimen\relax
\fi
%% let collapsible pdf bookmarks panel have high depth per default
\PassOptionsToPackage{bookmarksdepth=5}{hyperref}

\PassOptionsToPackage{warn}{textcomp}
\usepackage[utf8]{inputenc}
\ifdefined\DeclareUnicodeCharacter
% support both utf8 and utf8x syntaxes
  \ifdefined\DeclareUnicodeCharacterAsOptional
    \def\sphinxDUC#1{\DeclareUnicodeCharacter{"#1}}
  \else
    \let\sphinxDUC\DeclareUnicodeCharacter
  \fi
  \sphinxDUC{00A0}{\nobreakspace}
  \sphinxDUC{2500}{\sphinxunichar{2500}}
  \sphinxDUC{2502}{\sphinxunichar{2502}}
  \sphinxDUC{2514}{\sphinxunichar{2514}}
  \sphinxDUC{251C}{\sphinxunichar{251C}}
  \sphinxDUC{2572}{\textbackslash}
\fi
\usepackage{cmap}
\usepackage[T1]{fontenc}
\usepackage{amsmath,amssymb,amstext}
\usepackage{babel}



\usepackage{tgtermes}
\usepackage{tgheros}
\renewcommand{\ttdefault}{txtt}



\usepackage[Bjarne]{fncychap}
\usepackage{sphinx}

\fvset{fontsize=auto}
\usepackage{geometry}


% Include hyperref last.
\usepackage{hyperref}
% Fix anchor placement for figures with captions.
\usepackage{hypcap}% it must be loaded after hyperref.
% Set up styles of URL: it should be placed after hyperref.
\urlstyle{same}

\addto\captionsenglish{\renewcommand{\contentsname}{Course Information:}}

\usepackage{sphinxmessages}
\setcounter{tocdepth}{0}


% Jupyter Notebook code cell colors
\definecolor{nbsphinxin}{HTML}{307FC1}
\definecolor{nbsphinxout}{HTML}{BF5B3D}
\definecolor{nbsphinx-code-bg}{HTML}{F5F5F5}
\definecolor{nbsphinx-code-border}{HTML}{E0E0E0}
\definecolor{nbsphinx-stderr}{HTML}{FFDDDD}
% ANSI colors for output streams and traceback highlighting
\definecolor{ansi-black}{HTML}{3E424D}
\definecolor{ansi-black-intense}{HTML}{282C36}
\definecolor{ansi-red}{HTML}{E75C58}
\definecolor{ansi-red-intense}{HTML}{B22B31}
\definecolor{ansi-green}{HTML}{00A250}
\definecolor{ansi-green-intense}{HTML}{007427}
\definecolor{ansi-yellow}{HTML}{DDB62B}
\definecolor{ansi-yellow-intense}{HTML}{B27D12}
\definecolor{ansi-blue}{HTML}{208FFB}
\definecolor{ansi-blue-intense}{HTML}{0065CA}
\definecolor{ansi-magenta}{HTML}{D160C4}
\definecolor{ansi-magenta-intense}{HTML}{A03196}
\definecolor{ansi-cyan}{HTML}{60C6C8}
\definecolor{ansi-cyan-intense}{HTML}{258F8F}
\definecolor{ansi-white}{HTML}{C5C1B4}
\definecolor{ansi-white-intense}{HTML}{A1A6B2}
\definecolor{ansi-default-inverse-fg}{HTML}{FFFFFF}
\definecolor{ansi-default-inverse-bg}{HTML}{000000}

% Define an environment for non-plain-text code cell outputs (e.g. images)
\makeatletter
\newenvironment{nbsphinxfancyoutput}{%
    % Avoid fatal error with framed.sty if graphics too long to fit on one page
    \let\sphinxincludegraphics\nbsphinxincludegraphics
    \nbsphinx@image@maxheight\textheight
    \advance\nbsphinx@image@maxheight -2\fboxsep   % default \fboxsep 3pt
    \advance\nbsphinx@image@maxheight -2\fboxrule  % default \fboxrule 0.4pt
    \advance\nbsphinx@image@maxheight -\baselineskip
\def\nbsphinxfcolorbox{\spx@fcolorbox{nbsphinx-code-border}{white}}%
\def\FrameCommand{\nbsphinxfcolorbox\nbsphinxfancyaddprompt\@empty}%
\def\FirstFrameCommand{\nbsphinxfcolorbox\nbsphinxfancyaddprompt\sphinxVerbatim@Continues}%
\def\MidFrameCommand{\nbsphinxfcolorbox\sphinxVerbatim@Continued\sphinxVerbatim@Continues}%
\def\LastFrameCommand{\nbsphinxfcolorbox\sphinxVerbatim@Continued\@empty}%
\MakeFramed{\advance\hsize-\width\@totalleftmargin\z@\linewidth\hsize\@setminipage}%
\lineskip=1ex\lineskiplimit=1ex\raggedright%
}{\par\unskip\@minipagefalse\endMakeFramed}
\makeatother
\newbox\nbsphinxpromptbox
\def\nbsphinxfancyaddprompt{\ifvoid\nbsphinxpromptbox\else
    \kern\fboxrule\kern\fboxsep
    \copy\nbsphinxpromptbox
    \kern-\ht\nbsphinxpromptbox\kern-\dp\nbsphinxpromptbox
    \kern-\fboxsep\kern-\fboxrule\nointerlineskip
    \fi}
\newlength\nbsphinxcodecellspacing
\setlength{\nbsphinxcodecellspacing}{0pt}

% Define support macros for attaching opening and closing lines to notebooks
\newsavebox\nbsphinxbox
\makeatletter
\newcommand{\nbsphinxstartnotebook}[1]{%
    \par
    % measure needed space
    \setbox\nbsphinxbox\vtop{{#1\par}}
    % reserve some space at bottom of page, else start new page
    \needspace{\dimexpr2.5\baselineskip+\ht\nbsphinxbox+\dp\nbsphinxbox}
    % mimic vertical spacing from \section command
      \addpenalty\@secpenalty
      \@tempskipa 3.5ex \@plus 1ex \@minus .2ex\relax
      \addvspace\@tempskipa
      {\Large\@tempskipa\baselineskip
             \advance\@tempskipa-\prevdepth
             \advance\@tempskipa-\ht\nbsphinxbox
             \ifdim\@tempskipa>\z@
               \vskip \@tempskipa
             \fi}
    \unvbox\nbsphinxbox
    % if notebook starts with a \section, prevent it from adding extra space
    \@nobreaktrue\everypar{\@nobreakfalse\everypar{}}%
    % compensate the parskip which will get inserted by next paragraph
    \nobreak\vskip-\parskip
    % do not break here
    \nobreak
}% end of \nbsphinxstartnotebook

\newcommand{\nbsphinxstopnotebook}[1]{%
    \par
    % measure needed space
    \setbox\nbsphinxbox\vbox{{#1\par}}
    \nobreak % it updates page totals
    \dimen@\pagegoal
    \advance\dimen@-\pagetotal \advance\dimen@-\pagedepth
    \advance\dimen@-\ht\nbsphinxbox \advance\dimen@-\dp\nbsphinxbox
    \ifdim\dimen@<\z@
      % little space left
      \unvbox\nbsphinxbox
      \kern-.8\baselineskip
      \nobreak\vskip\z@\@plus1fil
      \penalty100
      \vskip\z@\@plus-1fil
      \kern.8\baselineskip
    \else
      \unvbox\nbsphinxbox
    \fi
}% end of \nbsphinxstopnotebook

% Ensure height of an included graphics fits in nbsphinxfancyoutput frame
\newdimen\nbsphinx@image@maxheight % set in nbsphinxfancyoutput environment
\newcommand*{\nbsphinxincludegraphics}[2][]{%
    \gdef\spx@includegraphics@options{#1}%
    \setbox\spx@image@box\hbox{\includegraphics[#1,draft]{#2}}%
    \in@false
    \ifdim \wd\spx@image@box>\linewidth
      \g@addto@macro\spx@includegraphics@options{,width=\linewidth}%
      \in@true
    \fi
    % no rotation, no need to worry about depth
    \ifdim \ht\spx@image@box>\nbsphinx@image@maxheight
      \g@addto@macro\spx@includegraphics@options{,height=\nbsphinx@image@maxheight}%
      \in@true
    \fi
    \ifin@
      \g@addto@macro\spx@includegraphics@options{,keepaspectratio}%
    \fi
    \setbox\spx@image@box\box\voidb@x % clear memory
    \expandafter\includegraphics\expandafter[\spx@includegraphics@options]{#2}%
}% end of "\MakeFrame"-safe variant of \sphinxincludegraphics
\makeatother

\makeatletter
\renewcommand*\sphinx@verbatim@nolig@list{\do\'\do\`}
\begingroup
\catcode`'=\active
\let\nbsphinx@noligs\@noligs
\g@addto@macro\nbsphinx@noligs{\let'\PYGZsq}
\endgroup
\makeatother
\renewcommand*\sphinxbreaksbeforeactivelist{\do\<\do\"\do\'}
\renewcommand*\sphinxbreaksafteractivelist{\do\.\do\,\do\:\do\;\do\?\do\!\do\/\do\>\do\-}
\makeatletter
\fvset{codes*=\sphinxbreaksattexescapedchars\do\^\^\let\@noligs\nbsphinx@noligs}
\makeatother



\title{Soft Matter Lecture}
\date{Oct 09, 2022}
\release{21}
\author{Frank Cichos, Ralf Seidel}
\newcommand{\sphinxlogo}{\vbox{}}
\renewcommand{\releasename}{Release}
\makeindex
\begin{document}

\pagestyle{empty}
\sphinxmaketitle
\pagestyle{plain}
\sphinxtableofcontents
\pagestyle{normal}
\phantomsection\label{\detokenize{index::doc}}
\begin{figure}[htbp]
\centering

\noindent\sphinxincludegraphics[width=960\sphinxpxdimen,height=269\sphinxpxdimen]{{FluffyParticles}.png}
\end{figure}




\chapter{Course Information}
\label{\detokenize{course-info/info:course-information}}\label{\detokenize{course-info/info::doc}}

\section{Lecture Schedule}
\label{\detokenize{course-info/info:lecture-schedule}}
\sphinxAtStartPar
Tuesday:        13:15 \sphinxhyphen{} 14:45 Uhr SR 221

\sphinxAtStartPar
Thursday:       09:15 \sphinxhyphen{} 10:45 Uhr SR 221


\section{Seminar Schedule}
\label{\detokenize{course-info/info:seminar-schedule}}
\sphinxAtStartPar
Wednesday:      13:15 \sphinxhyphen{} 14:45 Uhr SR 224

\sphinxAtStartPar
The lectures will be supported by a seminar where current topics of soft matter are discussed by the students in form of seminar talks.


\section{Exercise Schedule}
\label{\detokenize{course-info/info:exercise-schedule}}
\sphinxAtStartPar
Tuesday:        9:15 \sphinxhyphen{} 10:45 Uhr SR 218

\sphinxAtStartPar
As all student will most likely fit into one seminar room, we will use only one of the dates.
We will therefore offer one exercise class per week (starting October 25).


\chapter{Instructors}
\label{\detokenize{course-info/instructors:instructors}}\label{\detokenize{course-info/instructors::doc}}

\section{Lectures}
\label{\detokenize{course-info/instructors:lectures}}
\sphinxAtStartPar
Email: \sphinxstyleemphasis{firstname.lastname@physik.uni\sphinxhyphen{}leipzig.de}
\begin{itemize}
\item {} 
\sphinxAtStartPar
Prof. Dr. Oskar Hallatschek (substituting for Prof. Dr. Ralf Seidel)
\begin{itemize}
\item {} 
\sphinxAtStartPar
Linnéstr. 5, 04103 Leipzig

\item {} 
\sphinxAtStartPar
Office: 429

\item {} 
\sphinxAtStartPar
Phone: +0341 97 32571

\end{itemize}

\end{itemize}

\sphinxAtStartPar
The October 25/27 lectures are given by
* Dr. Henri Franquelim


\section{Seminar}
\label{\detokenize{course-info/instructors:seminar}}
\sphinxAtStartPar
Email: \sphinxstyleemphasis{firstname.lastname@physik.uni\sphinxhyphen{}leipzig.de}
\begin{itemize}
\item {} 
\sphinxAtStartPar
Dr. Giulio Isacchini
\begin{itemize}
\item {} 
\sphinxAtStartPar
Linnéstr. 5, 04103 Leipzig

\item {} 
\sphinxAtStartPar
Office: xxx

\item {} 
\sphinxAtStartPar
Phone: +0341 97

\end{itemize}

\item {} 
\sphinxAtStartPar
Dr. Valentin Slepukhin
\begin{itemize}
\item {} 
\sphinxAtStartPar
Linnéstr. 5, 04103 Leipzig

\item {} 
\sphinxAtStartPar
Office: xxx

\item {} 
\sphinxAtStartPar
Phone: +0341 97

\end{itemize}

\end{itemize}


\section{Exercise}
\label{\detokenize{course-info/instructors:exercise}}
\sphinxAtStartPar
Email: \sphinxstyleemphasis{firstname.middlename\_lastname@physik.uni\sphinxhyphen{}leipzig.de}
\begin{itemize}
\item {} 
\sphinxAtStartPar
Dr. Julene Madariaga Marcos
\begin{itemize}
\item {} 
\sphinxAtStartPar
Linnéstr. 5, 04103 Leipzig

\item {} 
\sphinxAtStartPar
Office: 424

\item {} 
\sphinxAtStartPar
Phone: +0341 97 32506

\end{itemize}

\end{itemize}

\sphinxAtStartPar
Email: \sphinxstyleemphasis{firstname.lastname@physik.uni\sphinxhyphen{}leipzig.de}
\begin{itemize}
\item {} 
\sphinxAtStartPar
Dr. Patrick Irmisch
\begin{itemize}
\item {} 
\sphinxAtStartPar
Linnéstr. 5, 04103 Leipzig

\item {} 
\sphinxAtStartPar
Office: 435

\item {} 
\sphinxAtStartPar
Phone: +0341 32558

\end{itemize}

\end{itemize}


\chapter{Resources}
\label{\detokenize{course-info/resources:resources}}\label{\detokenize{course-info/resources::doc}}
\sphinxAtStartPar
There are several books available on the topic of Soft Matter. Many set their own focus. Here is a list of books useful for the lecture.

\sphinxAtStartPar
Jacob N. Israelachvili: \sphinxstylestrong{Intermolecular and Surface Forces: With Applications to Colloidal and Biological Systems} (Academic Press)

\sphinxAtStartPar
Rob Pillips, Jane Kondev, Julie Theriot: \sphinxstylestrong{Physical Biology of the Cell} (Garland Science)

\sphinxAtStartPar
Richard A.L. Jones: \sphinxstylestrong{Soft Condensed Matter} (Oxford University Press)

\sphinxAtStartPar
Michael Rubinstein, Ralph H Colby: \sphinxstylestrong{Polymer Physics} (Oxford University Press)

\sphinxAtStartPar
Jonathan Howard: \sphinxstylestrong{Mechanics of Motor proteins and the Cytoskeleton} (Sinauer Associates)
\begin{enumerate}
\sphinxsetlistlabels{\Alph}{enumi}{enumii}{}{.}%
\setcounter{enumi}{12}
\item {} 
\sphinxAtStartPar
Doi und S.F. Edwards: \sphinxstylestrong{The Theory of Polymer Dynamics} (Oxford Academic Press)

\end{enumerate}

\sphinxAtStartPar
P.G. de Gennes and J. Prost: \sphinxstylestrong{The Physics of Liquid Crystals} (Oxford Academic Press)

\sphinxAtStartPar
In addition, we will make current literature from different journals available.


\chapter{Exercises and Exam}
\label{\detokenize{course-info/exam:exercises-and-exam}}\label{\detokenize{course-info/exam::doc}}

\section{Exercises}
\label{\detokenize{course-info/exam:exercises}}
\sphinxAtStartPar
The exercises will start in the week of October 25, 2021.
The seminar will start on October 25, 2021
The first exercise sheet will be be available on October 18, 2021 \sphinxstylestrong{online}
The solutions to the exercise sheets have to be handed in before the lecture on Tuesday in the following week
For every solved problem 1 point can be obtained.


\section{Exam}
\label{\detokenize{course-info/exam:exam}}
\sphinxAtStartPar
Prerequisite for the admission to the examination are 50\% of the total points.
Examination, 180 min written (knowledge + problem solving)
The final grade will be calculated from 66\% of the mark + 33\% seminar talk mark.

\sphinxAtStartPar
The following section was created from \sphinxcode{\sphinxupquote{notebooks/L1/1\_introduction.ipynb}}.


\chapter{Soft Matter Physics 2022/23}
\label{\detokenize{notebooks/L1/1_introduction:Soft-Matter-Physics-2022/23}}\label{\detokenize{notebooks/L1/1_introduction::doc}}

\section{Introduction}
\label{\detokenize{notebooks/L1/1_introduction:Introduction}}
\sphinxAtStartPar
\sphinxincludegraphics[width=800\sphinxpxdimen]{{intro}.png}

\sphinxAtStartPar
\sphinxhref{https://home.uni-leipzig.de/~physik/sites/mona/teaching/periodic-lectures/soft-matter-physics-ws-2021-22/}{Lecture Slides}

\sphinxAtStartPar
The following section was created from \sphinxcode{\sphinxupquote{notebooks/L1/2\_Thermodynamics\_Statistics.ipynb}}.


\chapter{Thermodynamics and Statistical Physics Revisited}
\label{\detokenize{notebooks/L1/2_Thermodynamics_Statistics:Thermodynamics-and-Statistical-Physics-Revisited}}\label{\detokenize{notebooks/L1/2_Thermodynamics_Statistics::doc}}

\section{Thermodynamics}
\label{\detokenize{notebooks/L1/2_Thermodynamics_Statistics:Thermodynamics}}

\subsection{Fundamental Quantities}
\label{\detokenize{notebooks/L1/2_Thermodynamics_Statistics:Fundamental-Quantities}}
\sphinxAtStartPar
We can define different thermodynamic systems depending on their ability to exchange energy and particles with the environment.

\sphinxAtStartPar
\sphinxincludegraphics[width=500\sphinxpxdimen]{{systems}.png}

\sphinxAtStartPar
A system is called \sphinxstyleemphasis{isolated} if it can neither exchange energy nor particles with its environment. This ensemble of particles is also called \(micro-canonical\). The energy in such a system can not fluctuate. It is a \sphinxstyleemphasis{closed system} if it can exchange energy with the surrounding, i.e., a heat bath keeping it at a constant temperature. The ensemble of particles in such a system corresponds to a \sphinxstyleemphasis{canonical} ensemble. Energy can fluctuate. In case the system can exchange also particles, it is
called an \sphinxstyleemphasis{open system} or \sphinxstyleemphasis{grand\sphinxhyphen{}canonical} ensemble.

\sphinxAtStartPar
A thermodynamic system can be defined by \sphinxstylestrong{state variables} \(p\), \(V\) and/or \(T\). \sphinxstylestrong{State functions} are useful to formulate the laws of thermodynamics. A state function is independent of how one arrived at that particular state. They are most often used to characterize an equilibrium state.

\begin{sphinxadmonition}{warning}{}\unskip
\sphinxAtStartPar
\sphinxstylestrong{Note:} State Function and State Variables
\begin{itemize}
\item {} 
\sphinxAtStartPar
State function is a function of state variables that depends only on the state of the system.

\item {} 
\sphinxAtStartPar
State variables characterize the state of a system and are independent of how the system got there.

\end{itemize}
\end{sphinxadmonition}

\sphinxAtStartPar
We will refer to a number of state functions during the course. Here is a list of the most useful ones:
\begin{itemize}
\item {} 
\sphinxAtStartPar
\sphinxstylestrong{Internal energy} \(U\) (state function):

\end{itemize}

\sphinxAtStartPar
State function. The total energy contained in a thermodynamic system. It is the energy necessary to create the system with the major components kinetic energy and potential energy.

\sphinxAtStartPar
\begin{equation}
U=Q+W
\end{equation}

\sphinxAtStartPar
The internal energy can be changed by supplying heat or doing work on the system such that

\sphinxAtStartPar
\begin{equation}
dU=dQ-pdV
\end{equation}
\begin{itemize}
\item {} 
\sphinxAtStartPar
\sphinxstylestrong{Enthalpy} \(H\) (state function):

\end{itemize}

\sphinxAtStartPar
The enthalpy \(H\) of a system is measuring the internal energy plus the product of volume and pressure.

\sphinxAtStartPar
\begin{equation}
H=U+pV
\end{equation}

\sphinxAtStartPar
The enthalpy is especially useful for a system at constant pressure \(p={\rm const.}\), i.e., for processes where the volume can not be controlled.

\sphinxAtStartPar
\begin{equation}
\mathrm{d}H=\mathrm{d}U+p\mathrm{d}V+V\mathrm{d}p=\delta Q + V\mathrm{d}p
\end{equation}

\sphinxAtStartPar
In this case the change in enthalpy of a system corresponds to the exchanged heat.
\begin{itemize}
\item {} 
\sphinxAtStartPar
\sphinxstylestrong{Entropy} \(S\) (state function):

\end{itemize}

\sphinxAtStartPar
The change of entropy of a system is given by \begin{equation}
\mathrm{d}S=\frac{\delta Q}{T},
\end{equation}

\sphinxAtStartPar
where \(T\) denotes the temperature and \(\delta Q\) the heat exchanged in a reversible process.
\begin{itemize}
\item {} 
\sphinxAtStartPar
\sphinxstylestrong{Free Energy} \(G\) or \(F\) (state function):

\end{itemize}

\sphinxAtStartPar
Free energies are useful for systems which can exchange energy with the environment but not particles. In such non\sphinxhyphen{}isolated systems we define the

\sphinxAtStartPar
\sphinxstylestrong{Gibbs Free Energy}

\sphinxAtStartPar
\begin{equation}
G=H-TS
\end{equation}

\sphinxAtStartPar
for a system with constant pressure and

\sphinxAtStartPar
\sphinxstylestrong{Helmholtz Free Energy}

\sphinxAtStartPar
\begin{equation}
F=U-TS
\end{equation}

\sphinxAtStartPar
for a system with constant volume. Both quantities are used to indicate in which direction physical or chemical processes proceed. After we have written down the two first laws of thermodynamics, we will have a look at a specific example.


\subsection{Laws of Thermodynamics}
\label{\detokenize{notebooks/L1/2_Thermodynamics_Statistics:Laws-of-Thermodynamics}}

\subsubsection{First Law of Thermodynamics}
\label{\detokenize{notebooks/L1/2_Thermodynamics_Statistics:First-Law-of-Thermodynamics}}
\sphinxAtStartPar
The internal energy \(U\) is connected to the first law of thermodynamics:

\sphinxAtStartPar
\begin{equation}
\mathrm dU=\delta Q+\delta W=\delta Q-p\mathrm dV
\end{equation}

\sphinxAtStartPar
which states, that the energy contained in a system can be changed by providing heat or performing mechanical work on the system. Here \(\delta Q\) and \(\delta W\) are no state functions. They characterize a process of delivering heat or performing mechanical work.

\begin{sphinxadmonition}{warning}{}\unskip
\sphinxAtStartPar
\sphinxstylestrong{Note:} First Law of Thermodynamics
\begin{itemize}
\item {} 
\sphinxAtStartPar
The internal energy of a system can only be changed by exchanging heat with the environment or by doing mechanical work on it (energy cannot appear or disappear).

\end{itemize}
\end{sphinxadmonition}


\subsubsection{Second Law of Thermodynamics}
\label{\detokenize{notebooks/L1/2_Thermodynamics_Statistics:Second-Law-of-Thermodynamics}}
\sphinxAtStartPar
While the internal energy \(U\) may be fixed (i.e., \(\mathrm{d}U=0\)), the system may exist in different configurations (see, for example, and ideal gas with different arrangements of molecules). Such an isolated system will always evolve such that the change in entropy is positive or zero (\(\mathrm{d}S\ge 0\)). This is equivalent to the formulation that the entropy becomes maximized.



\sphinxAtStartPar
\sphinxstylestrong{Note:} Second Law of Thermodynamics
\begin{itemize}
\item {} 
\sphinxAtStartPar
When an isolated system is left alone long enough it evolves to thermal equilibrium whose entropy is at least as great as before (\(\mathrm{d}S \gt 0\)).

\end{itemize}

\begin{sphinxadmonition}{note}{}\unskip
\sphinxAtStartPar
\sphinxstylestrong{Example: Closed System in Contact with Reservoir}

\sphinxAtStartPar
Look at the above figure of the closed systems (inner part) inside a temperature bath (reservoir). If we apply the 2nd Law of thermodynamics to reservoir and system we write down

\sphinxAtStartPar
\begin{equation}
\mathrm{d}S_{\rm tot}=\mathrm{d}S_{\rm R}+ \mathrm{d}S_{\rm S}\ge 0.
\end{equation}

\sphinxAtStartPar
The system may exchange energy with the reservoir and, thus, according to the 1st Law of thermodynamics we have

\sphinxAtStartPar
\begin{equation}
\mathrm{d}U_{\rm R}=T\mathrm{d}S_{\rm R}-p \mathrm{d}V_{\rm R}
\end{equation}

\sphinxAtStartPar
with \(T\) being the constant temperature of the reservoir. We may transform this equation and insert it into the 2nd law to obtain:

\sphinxAtStartPar
\begin{equation}
\mathrm{d}S_{\rm S}+\frac{\mathrm{d}U_{\rm R}}{T}+\frac{p}{T}\mathrm{d}V_{\rm R}\ge 0.
\end{equation}

\sphinxAtStartPar
Using \(\mathrm{d}U_{\rm R}=-\mathrm{d}U_{\rm S}\) and \(\mathrm{d}V_{\rm R}=-\mathrm{d}V_{\rm S}\) we can transform the equation for constant pressure and temperature to

\sphinxAtStartPar
\begin{equation}
\mathrm{d}G=\mathrm{d}(U_{\rm S}+pV_{\rm S}-TS_{\rm S})\le 0.
\end{equation}

\sphinxAtStartPar
This indicates the direction in which a closed system, that can exchange energy with a bath, develops according to the 2nd Law of thermodynamics.

\sphinxAtStartPar
\sphinxstylestrong{If we bring a small system into thermal contact with a reservoir, which is in equilibrium at temperature T, then the reservoir will stay in equilibrium at the same temperature, but the small system will come to a new equilibrium, which minimizes the free energy.}

\sphinxAtStartPar
Thus, in equilibrium the free energy reaches a minimum value. Free energy minimization does two things at the same time: It tries to minimize the internal energy/enthalpy as well as to maximize the entropy (due to the negative sign in front of the entropy term) of a non\sphinxhyphen{}isolated object. The free energy minimum is, thus, the optimal balance between the two extrema.
\end{sphinxadmonition}



\sphinxAtStartPar
The following section was created from \sphinxcode{\sphinxupquote{notebooks/L2/3\_Statistical\_Physics\_Definitions.ipynb}}.


\chapter{Statistical Physics Definitions}
\label{\detokenize{notebooks/L2/3_Statistical_Physics_Definitions:Statistical-Physics-Definitions}}\label{\detokenize{notebooks/L2/3_Statistical_Physics_Definitions::doc}}
\sphinxAtStartPar
The field of statitical physics uses approaches of statistics and probability theory to address physical problems. It considers large populations and derives expressions for the ensemble (or the macrostate) of a system from the microscopic states in the system.


\section{Entropy}
\label{\detokenize{notebooks/L2/3_Statistical_Physics_Definitions:Entropy}}

\subsection{Entropy Definition by Boltzmann}
\label{\detokenize{notebooks/L2/3_Statistical_Physics_Definitions:Entropy-Definition-by-Boltzmann}}
\sphinxAtStartPar
The term of entropy becomes very important in that context. It measures the number of different ways a system can be rearranged to yield the same macrostate. It is, thus, an indicator for the microscopic degeneracy of a macrostate. In this context the definition of entropy by \sphinxstylestrong{Boltzmann} is well known, i.e.,

\sphinxAtStartPar
\begin{equation}
S=k_\mathrm{B} \ln(W)
\end{equation}

\sphinxAtStartPar
where \(W\) is the number of microstates corresponding to a system’s macrostate with and energy \(E\). Here \(k_\mathrm{B}=1.38064852 × 10^{-23} {\rm m^2\, kg\, s^{-2}\, K^{-1}}\) is the Boltzmann constant. Below are two examples of how to use the formula for the calculation of the entropy.

\begin{sphinxadmonition}{note}{}\unskip
\sphinxAtStartPar
\sphinxstylestrong{Example: Entropy of an N letter word}

\noindent\sphinxincludegraphics[width=254\sphinxpxdimen,height=100\sphinxpxdimen]{{letters}.png}

\sphinxAtStartPar
Consider the number of states \(W\) of a word with \(N\) letters of \(M\) different characters. The letter can be arranged in

\sphinxAtStartPar
\begin{equation}
W=M^N
\end{equation}

\sphinxAtStartPar
different ways such that the entropy is given by

\sphinxAtStartPar
\begin{equation}
S=N\, k_\mathrm{B} \ln(M) = k_{\rm B}\sum_{1}^{N} \ln(M)
\end{equation}

\sphinxAtStartPar
which just tells that entropy is an additive quantity.
\end{sphinxadmonition}

\begin{sphinxadmonition}{note}{}\unskip
\sphinxAtStartPar
\sphinxstylestrong{Example: Arrangement of molecules along a chain}

\noindent\sphinxincludegraphics[width=400\sphinxpxdimen,height=73\sphinxpxdimen]{{binding}.png}

\sphinxAtStartPar
We consider a linear molecules (perhaps a DNA) that has \(N\) binding sites for, e.g., proteins. \(N_\mathrm{p}\) sites are occupied with a protein where the binding energy is equal for each site. The number of different ways in which the \(N_\mathrm{p}\) proteins can be arranged on the \(N\) sites is given by the binomial coefficient

\sphinxAtStartPar
\begin{equation}
W(N_\mathrm{p};N)=\frac{N!}{N_\mathrm{p}!(N-N_\mathrm{p})!}.
\end{equation}

\sphinxAtStartPar
Therefore, the entropy is given by

\sphinxAtStartPar
\begin{equation}
S=k_\mathrm{B} \ln\left ( \frac{N!}{N_\mathrm{p}!(N-N_\mathrm{p})!}\right )
\end{equation}

\sphinxAtStartPar
which can be further simplified using the identity

\sphinxAtStartPar
\begin{equation}
\ln(N!)=\sum_{n=1}^{N}\ln(n)
\end{equation}

\sphinxAtStartPar
and the \sphinxstyleemphasis{Stirling approximation}

\sphinxAtStartPar
\begin{equation}
\sum_{n=1}^{N}\ln(n)\approx \int_1^{N}\ln(x)\,\mathrm{d}x\approx N\ln(N)-N.
\end{equation}

\sphinxAtStartPar
This, finally, leads to

\sphinxAtStartPar
\begin{equation}
S=-k_\mathrm{B} N [c \ln(c)+(1-c)\ln(1-c)]
\end{equation}

\sphinxAtStartPar
with \(c=N_\mathrm{p}/N\) being the mean occupation of each site or the probability to find a state occupied.
\end{sphinxadmonition}

\sphinxAtStartPar
Below you just find some Python code calculating the entropy as a function of “concentration” using the Stirling approximation and the original formula. You also recognize there, that the Stirling formula is not yet very good, since \(N=100\).

\noindent\sphinxincludegraphics{{stirling}.pdf}


\subsection{Shannon Entropy}
\label{\detokenize{notebooks/L2/3_Statistical_Physics_Definitions:Shannon-Entropy}}
\sphinxAtStartPar
A different access to entropy comes from the field of information theory and has been devised by Claude Shannon. Information theory is trying to mathematically assess the information content of a measurement facing uncertainty. It will turn out further below that this alternative description results in the Boltzmann distribution and effectively amounts to making a best guess about the probability distribution given some limited knowledge about the system such as the average energy.

\sphinxAtStartPar
The Shannon entropy is defined by

\sphinxAtStartPar
\begin{equation}
S\left(p_{1}, p_{2}, \ldots, p_{N}\right)=S\left(\left\{p_{i}\right\}\right)=-\sum_{i=1}^{N} p_{i} \ln p_{i}
\end{equation}

\sphinxAtStartPar
and relates to its thermodynamic version, the Gibbs entropy

\sphinxAtStartPar
\begin{equation}
S\left(\left\{p_{i}\right\}\right)=-k_\mathrm{B}\sum_{i=1}^{N} p_{i} \ln p_{i},
\end{equation}

\sphinxAtStartPar
where \(p_i\) is the probability of the \(i\)th microstate (or outcome). The example below will show, that if only the normalization of the probability is known, maximization of the Shannon entropy will directly lead to an equal probability of events (uniform distribution). Later, we see that similar calculations can be done to yield the Boltzmann distribution.

\begin{sphinxadmonition}{note}{}\unskip
\sphinxAtStartPar
\sphinxstylestrong{Example: Uniform distribution}

\sphinxAtStartPar
To figure out that the Shannon entropy is indeed delivering some useful measure, we will have a look at a measurement which has \(N\) outcomes (e.g., rolling a dice). We, of course, know that in this case all numbers of the dice have equal probability, but we can test this by maximizing the Shannon entropy as required by our thermodynamic considerations earlier.

\sphinxAtStartPar
To do so, we use the technique of Lagrange multipliers, which allows us to set a constraint while maximizing the entropy. This contraint is for this example, that

\sphinxAtStartPar
\begin{equation}
\sum_{i}^{N} p_{i}=1
\end{equation}

\sphinxAtStartPar
i.e., that the probability is normalized to 1. With this constraint we maximize the entropy by adding an additional term with the contrain multilied by the Lagrange multiplier \(\lambda\):

\sphinxAtStartPar
\begin{equation}
S^{\prime}=-\sum_{i} p_{i} \ln p_{i}-\lambda\left(\sum_{i} p_{i}-1\right).
\end{equation}

\sphinxAtStartPar
We see, that if the probability is normalized to 1, we do not change the entropy. Our procedure is to find that set of probabilities \(p_i\) which maximizes this augmented entropy function.

\sphinxAtStartPar
The derivative of the augmented entropy with respect to \(\lambda\) yields the normalization condition, i.e.,

\sphinxAtStartPar
\begin{equation}
\frac{\partial S^{\prime}}{\partial \lambda}=-\left(\sum_{i} p_{i}-1\right)\overset{!}{=} 0.
\end{equation}

\sphinxAtStartPar
Differentiation with respect to the probabilities yields

\sphinxAtStartPar
\begin{equation}
\frac{\partial S^{\prime}}{\partial p_{i}}=-\ln p_{i}-1-\lambda\overset{!}{=} 0
\end{equation}

\sphinxAtStartPar
which directly gives

\sphinxAtStartPar
\begin{equation}
p_{i}=\mathrm{e}^{-1-\lambda}.
\end{equation}

\sphinxAtStartPar
Together with the normalization condition we therefore obtain

\sphinxAtStartPar
\begin{equation}
\sum_{i=1}^{N} \mathrm{e}^{-1-\lambda}=1
\end{equation}

\sphinxAtStartPar
and since the exponent does not depend on \(i\) we find

\sphinxAtStartPar
\begin{equation}
\mathrm{e}^{-1-\lambda}=\frac{1}{N}
\end{equation}

\sphinxAtStartPar
or

\sphinxAtStartPar
\begin{equation}
p_{i}=\frac{1}{N}
\end{equation}

\sphinxAtStartPar
which is the expected equal probability of finding one of the \(N\) outcomes.
\end{sphinxadmonition}


\section{Boltzmann Distribution}
\label{\detokenize{notebooks/L2/3_Statistical_Physics_Definitions:Boltzmann-Distribution}}
\sphinxAtStartPar
Our previous consideration of the state functions has shown, that thermal equibrium is associated with a minimum in free energy. As the free energy consists of internal energy \(U\) (or enthalpy \(H\)) and an entropic term (\(-TS\)), we may understand this minimization as a competition between the minimization of the internal energy and a maximization of the entropy (since it is \(-TS\)). The figure below illustrates this competition for a gas in the gravity field.

\noindent\sphinxincludegraphics[width=500\sphinxpxdimen,height=295\sphinxpxdimen]{{entropy_comp}.png}

\sphinxAtStartPar
The internal energy minimization yields just a condensed layer at the bottom of the container, while the entropy maximization will try to spread the particles evenly (middle picture). The compromise of both at finite temperature is given by the \sphinxstylestrong{barometric height formula}, i.e.,

\sphinxAtStartPar
\begin{equation}
p(z)=p_0\exp\left ( -\frac{m g z}{k_\mathrm{B} T}\right ),
\end{equation}

\sphinxAtStartPar
where \(p(z)\) is the probability to find a particle at height \(z\), \(m\) is the mass of a particle, \(g\) is the gravitational acceleration and \(p_0\) is a normalization constant. The result actually gives a hint at some very fundamental distribution, which always provides the free energy minimum in thermal equilibirum. This distribution is the Boltzmann distribution.

\sphinxAtStartPar
The Boltzmann distribution is an approach of statistical physics to describe a thermodynamic system in equilibrium. The idea is hereby to deliver probability distributions for the probability of all different microstates. Key distinguishing feature of different microstates is their energy \(E_i\) (that was so far neglected in the examples above), where \(i\) indicates the \(i\)th microstate.

\sphinxAtStartPar
The Boltzmann distribution tells us precisely the probability of finding a given microstate with energy \(E_i\): If a particle is in equilibrium with its environment then the probability of finding the particle in state \(i\) with energy \(E_i\) is

\sphinxAtStartPar
\begin{equation}
p(E_i)=\frac{1}{Z}\exp\left ( -\frac{E_i}{k_\mathrm{B} T}\right ).
\end{equation}

\sphinxAtStartPar
The normalization factor \(1/Z\) contains the so\sphinxhyphen{}called partition function \(Z\):

\sphinxAtStartPar
\begin{equation}
Z=\sum_{i}\exp\left ( -\frac{E_i}{k_\mathrm{B} T}\right )={\rm const}.
\end{equation}

\sphinxAtStartPar
It ensures that the total probablity to find a system in any of the states is

\sphinxAtStartPar
\begin{equation}
\sum_i p(E_i)=1.
\end{equation}


\subsection{Mean Energy}
\label{\detokenize{notebooks/L2/3_Statistical_Physics_Definitions:Mean-Energy}}
\sphinxAtStartPar
The Boltzmann distribution is useful to calculate also expectation values, for example, of the total energy of the system (the mean energy \(\langle E\rangle\)).

\sphinxAtStartPar
The mean is defined by: \begin{equation}
\langle E \rangle=\frac{1}{Z}\sum_{i=1}^{N}E_{i}\exp\left ( -\frac{E_i}{k_\mathrm{B} T}\right).
\end{equation}

\sphinxAtStartPar
Abbrevating \(\beta=(k_\mathrm{B} T)^{-1}\) we find

\sphinxAtStartPar
\begin{equation}
\langle E \rangle=\frac{1}{Z}\sum_{i=1}^{N}\bigg(- \frac{\partial }{\partial \beta}\exp\left ( -\beta E_i\right)\bigg),
\end{equation}

\sphinxAtStartPar
where the sum is nothing else than the derivative of the partition function

\sphinxAtStartPar
\begin{equation}
\langle E \rangle=-\frac{1}{Z} \frac{\partial }{\partial \beta}Z
\end{equation}

\sphinxAtStartPar
or just

\sphinxAtStartPar
\begin{equation}
\langle E \rangle=-\frac{\partial }{\partial \beta}\ln(Z).
\end{equation}


\subsection{Free Energy}
\label{\detokenize{notebooks/L2/3_Statistical_Physics_Definitions:Free-Energy}}
\sphinxAtStartPar
Employ the Gibbs entropy \(S=-k_\mathrm{B}\sum_{i=1}^{N} p_{i} \ln p_{i}\) to find a relation between the free energy \(F\) (or \(G\)) and the partition function. Inserting the probability \(p_i=Z^{-1}\exp(-\beta E_i)\) and doing some transformations yields

\sphinxAtStartPar
\begin{equation}
S=k_\mathrm{B} (\ln(Z)+\beta <E>).
\end{equation}

\sphinxAtStartPar
Using

\sphinxAtStartPar
\begin{equation}
F=U-TS
\end{equation}

\sphinxAtStartPar
for the free energy, we can insert the above result for the entropy and obtain

\sphinxAtStartPar
\begin{equation}
F=-k_\mathrm{B} T \ln(Z)
\end{equation}

\sphinxAtStartPar
(or \(G\) in the same way).

\sphinxAtStartPar
Note that this is the total energy and not the mean energy (internal, enthalpy or free energy) of the states in the system. The partition function thus allows us to calculate the free energy.


\subsection{Deriving the Boltzmann Distribution}
\label{\detokenize{notebooks/L2/3_Statistical_Physics_Definitions:Deriving-the-Boltzmann-Distribution}}
\sphinxAtStartPar
There are a number of ways to derive the Boltzmann distribution. We will have a quick look at a classical derivation of the Boltzmann distribution for a closed system, e.g., a system which is in contact with a reservoir as depicted below.

\noindent\sphinxincludegraphics[width=186\sphinxpxdimen,height=200\sphinxpxdimen]{{entropy_calc}.png}

\sphinxAtStartPar
System (index s) and reservoir (index r) have total energy \(E_{\rm tot}=E_{\rm r}+E_{\rm s}\). We assert now, that the probability to find the system in a specific microstate \(p(E_{\rm s}^{(i)})\) with the energy \(E_{\rm s}^{(i)}\) is directly proportional to the number of states available to the reservoir, when the system is in that state. The ratio of the probabilities of two states is then equal to the ratio of the number of states of the reservoir, i.e.,

\sphinxAtStartPar
\begin{equation}
\frac{p(E_{\rm s}^{(1)})}{p(E_{\rm s}^{(2)})}=\frac{W_{\rm r}(E_{\rm tot}-E_{\rm s}^{(1)})}{W_{\rm r}(E_{\rm tot}-E_{\rm s}^{(2)})}.
\end{equation}

\sphinxAtStartPar
Here, the function \(W_{\rm r}(E_{\rm tot}-E_{\rm s}^{(1)})\) is the number of states available to the reservoir, when the system is having the energy \(E_{\rm s}^{(1)}\).

\sphinxAtStartPar
We can now rewrite the above equation in terms of the entropy using \(W(S(E))=\exp(S(E)/k_\mathrm{B})\) such that

\sphinxAtStartPar
\begin{equation}
\frac{W_{\rm r}(E_{\rm tot}-E_{\rm s}^{(1)})}{W_{\rm r}(E_{\rm tot}-E_{\rm s}^{(2)})}=\frac{\exp(S_{\rm r}(E_{\rm tot}-E_{\rm s}^{(1)})/k_{\rm B})}{\exp(S_{\rm r}(E_{\rm tot}-E_{\rm s}^{(2)})/k_{\rm B})}.
\end{equation}

\sphinxAtStartPar
We may now expand the entropy to first order

\sphinxAtStartPar
\begin{equation}
S_{\rm r}(E_{\rm tot}-E_{\rm s})\approx S_{\rm r}(E_{\rm tot})-\frac{\partial S_{\rm r}}{\partial E}E_{\rm s}
\end{equation}

\sphinxAtStartPar
considering that \(E_{\rm s}\) is only very tiny as compared to the total energy of the reservoir. Using the thermodynamic identity that

\sphinxAtStartPar
\begin{equation}
\frac{\partial S_{\rm r}}{\partial E}|_{V,N}=\frac{1}{T}
\end{equation}

\sphinxAtStartPar
we finally find

\sphinxAtStartPar
\begin{equation}
\frac{p(E_{\rm s}^{(1)})}{p(E_{\rm s}^{(2)})}=\frac{\exp(-E_{\rm s}^{(1)}/k_{\rm B}T)}{\exp(-E_{\rm s}^{(2)}/k_{\rm B}T)}
\end{equation}

\sphinxAtStartPar
which corresponds to the ratio of two Boltzmann distributions

\sphinxAtStartPar
\begin{equation}
p(E_{\rm s}^{(i)})=\frac{1}{Z}\exp\left (- \frac{E_{\rm s}^{(i)}}{k_{\rm B} T}\right),
\end{equation}

\sphinxAtStartPar
where \(Z\) is the previously mentioned normalization factor, which is called \sphinxstyleemphasis{partition function}.

\begin{sphinxadmonition}{note}{}\unskip
\sphinxAtStartPar
\sphinxstylestrong{Example: Barometric height formula}

\sphinxAtStartPar
We have mentioned already the barometric height formula giving the probability of finding a particle at a height \(z\). To derive that, we consider a constant gravitational force \(F=-mg\) along the \(z\)\sphinxhyphen{}direction such that the potential energy is given by \(E=mgz\) assuming that \(E=0\) at \(z=0\).

\sphinxAtStartPar
The probability for finding the particle at position \(z\) is therefore

\sphinxAtStartPar
\begin{equation}
p(z)=\frac{1}{Z}\exp\left( -\frac{mgz}{k_{\rm B} T}\right)=\frac{1}{<z>}\exp\left ( -\frac{z}{<z>}\right ).
\end{equation}

\sphinxAtStartPar
Normalization provides the value of the partition function

\sphinxAtStartPar
\begin{equation}
Z=\int_0^\infty\exp\left( -\frac{mgz}{k_{\rm B} T}\right)dz=\frac{k_{\rm B}T}{mg}.
\end{equation}

\sphinxAtStartPar
We may further calculate the mean height, which is the sedimentation length in sedimentation problems:

\sphinxAtStartPar
\begin{equation}
<z>=\frac{1}{Z}\int_0^\infty h\exp\left( -\frac{mgz}{k_{\rm B} T}\right)\mathrm{d}z=\frac{k_{\rm B}T}{mg}
\end{equation}

\sphinxAtStartPar
and the mean energy

\sphinxAtStartPar
\begin{equation}
<E>=-\frac{\partial }{\partial \beta} \ln (Z)=+\frac{\partial}{\partial \beta}\ln(\beta m g)=\frac{1}{\beta}=k_{\rm B} T=mg <z>.
\end{equation}
\end{sphinxadmonition}

\begin{sphinxadmonition}{note}{}\unskip
\sphinxAtStartPar
\sphinxstylestrong{Example: Boltzmann distribution is the maximum entropy distribution in which the average energy is prescribed as a constraint}

\sphinxAtStartPar
We can also obtain the Boltzmann distribution from the Shannon entropy by constraining the Shannon entropy. With only the normalization as a constraint in entropy maximization, we obtained equally likely microsates. If we now constrain the mean energy \(\langle E\rangle\) of the system, we obtain a distribution which maximizes the entropy under this condition. The mean energy is given by

\sphinxAtStartPar
\begin{equation}
\langle E\rangle=\sum_{i} E_{i} p_{i},
\end{equation}

\sphinxAtStartPar
such that we can add another constraint to our augmented entropy. We now have a Lagrange multiplier \(\lambda\) for the normalization of the probability and a second one \(\beta\) which is multiplied by the energy constraint.

\sphinxAtStartPar
\begin{equation}
S^{\prime}=-\sum_{i} p_{i} \ln p_{i}-\lambda\left(\sum_{i} p_{i}-1\right)-\beta\left(\sum_{i} p_{i} E_{i}-\langle E\rangle\right).
\end{equation}

\sphinxAtStartPar
Taking the derivative

\sphinxAtStartPar
\begin{equation}
\frac{\partial S^{\prime}}{\partial p_{i}}=-\ln p_{i}-1-\lambda-\beta E_i\overset{!}{=} 0
\end{equation}

\sphinxAtStartPar
results in

\sphinxAtStartPar
\begin{equation}
p_{i}=\mathrm{e}^{-1-\lambda-\beta E_{i}}
\end{equation}

\sphinxAtStartPar
and together with the normalization condition \(\sum p_i=1\) finally

\sphinxAtStartPar
\begin{equation}
\mathrm{e}^{-1-\lambda}=\frac{1}{\sum_{i} \mathrm{e}^{-\beta E_{i}}},
\end{equation}

\sphinxAtStartPar
where we already recognized that we can replace the prefactor \(\mathrm{e}^{-1-\lambda}\) by \(1/Z\) with \(Z\) being the partition function

\sphinxAtStartPar
\begin{equation}
Z=\sum_{i} \mathrm{e}^{-\beta E_{i}}.
\end{equation}

\sphinxAtStartPar
Overall this, therefore, leads to the Boltzmann distribution

\sphinxAtStartPar
\begin{equation}
p_{i}=\frac{\mathrm{e}^{-\beta E_{i}}}{\sum_{i} \mathrm{e}^{-\beta E_{i}}}
\end{equation}

\sphinxAtStartPar
which is quite interesting. We have just fixed the mean energy of the system and maximized the entropy. The Boltzmann distribution is therefore the probability distribution which maximizes the entropy under as little as possible additional information (just the mean energy).

\sphinxAtStartPar
The only thing that is missing in the above formula is an expression for the value of \(\beta\), the Lagrange multiplier. This can be obtained when knowning the mean energy. In thermal euilibrium, this mean energy can be obtained from the equipartition theorem.
\end{sphinxadmonition}



\sphinxAtStartPar
The following section was created from \sphinxcode{\sphinxupquote{notebooks/L3/1\_Equipartition.ipynb}}.


\chapter{Equipartition}
\label{\detokenize{notebooks/L3/1_Equipartition:Equipartition}}\label{\detokenize{notebooks/L3/1_Equipartition::doc}}
\sphinxAtStartPar
In the previous section, we showed how the probability distribution for a system with average energy \(\langle E \rangle\) could be guessed by using the principle of maximum entropy. However, to finish that calculation, we need to determine the meaning and significance of the Lagrange multiplier \(\beta\). We can derive a value for \(\beta\) with the help of the equipartition theorem.

\begin{sphinxadmonition}{warning}{}\unskip
\sphinxAtStartPar
The equipartition theorem, also known as the law of equipartition, equipartition of energy or simply equipartition, states that every degree of freedom that appears only quadratically in the total energy has an average energy of \(\frac{1}{2}k_B T\).
\end{sphinxadmonition}

\sphinxAtStartPar
To obtain the Lagrange parameter we just consider one degree of freedom for a monoatomic gas. This degree of freedom is the kinetic energy of one atom along the x\sphinxhyphen{}direction, which is given by

\sphinxAtStartPar
\begin{equation}
E=\frac{p_{X}^{2}}{2 m}
\end{equation}

\sphinxAtStartPar
According to the Boltzmann distribution, the probability to find an atom with a certain momentum is given by

\sphinxAtStartPar
\begin{equation}
P\left(p_{x}\right)=\frac{\mathrm{e}^{-\beta\left(p_{x}^{2} / 2 m\right)}}{\sum_{\text {states }} \mathrm{e}^{-\beta\left(p_{x}^{2} / 2 m\right)}}
\end{equation}

\sphinxAtStartPar
We can thus calculate the mean energy by summing up (or integrating when going to continuous states) over alls possible momenta

\sphinxAtStartPar
\begin{equation}
\sum_{\text {states }} \rightarrow \int_{-\infty}^{\infty} \mathrm{d} p_{x}
\end{equation}

\sphinxAtStartPar
This yields

\sphinxAtStartPar
\begin{equation}
\int_{-\infty}^{\infty} \mathrm{e}^{-\beta p_{x}^{2} / 2 m} \mathrm{~d} p_{x}=\sqrt{\frac{2 m \pi}{\beta}}
\end{equation}

\sphinxAtStartPar
To obtain the value of \(\beta\) we now constrain the mean energy to the value given by the equiparition principle

\sphinxAtStartPar
\begin{equation}
\langle E\rangle=\frac{1}{2} k_{\mathrm{B}} T
\end{equation}

\sphinxAtStartPar
This is the mean energy per degree of freedom. One can show that each degree of freedom, independent of the object (atom, colloid, parking car) is carrying this mean energy. It provides actually our measure of temperature.

\sphinxAtStartPar
Using the momentum to calculate the mean energy we write down

\sphinxAtStartPar
\begin{equation}
\langle E\rangle=\frac{\int_{-\infty}^{\infty} \frac{p_{x}^{2}}{2 m} \mathrm{e}^{-\beta\left(p_{x}^{2} / 2 m\right)} \mathrm{d} p_{x}}{\sqrt{2 m \pi / \beta}}
\end{equation}

\sphinxAtStartPar
which can be slightly simplified

\sphinxAtStartPar
\begin{equation}
\langle E\rangle=\frac{\alpha^{3 / 2}}{\beta \sqrt{\pi}}\left(-\frac{\partial}{\partial \alpha}\right) \int_{-\infty}^{\infty} \mathrm{e}^{-\alpha p_{x}^{2}} \mathrm{~d} p_{x}
\end{equation}

\sphinxAtStartPar
and calculated via some tricks valid for integrals over Gaussian functions. This finally leads us to the result that the value of the Lagrangian multiplier must be

\sphinxAtStartPar
\begin{equation}
\beta=\frac{1}{k_{\mathrm{B}} T}
\end{equation}

\begin{sphinxadmonition}{warning}{}\unskip
\sphinxAtStartPar
A good example for a non\sphinxhyphen{}quadratic degree of freedom is the barometric height formula with a potential energy that is linear in position.
\end{sphinxadmonition}

\sphinxAtStartPar
Equipartition is useful in many ways. The fundamental degrees of freedom, e.g. the vibrations of a single molecule are quadratic in the bond length. Similarly all rotational degrees of freedom are as well. Their occupation is determined by a Boltzmann distribution and readily visible in molecular spectra. On more macroscopic scales it is very useful in the field of force measurements using optical tweezers.

\begin{sphinxadmonition}{note}{}\unskip
\sphinxAtStartPar
\sphinxstylestrong{Example: Two\sphinxhyphen{}level\sphinxhyphen{}system}

\sphinxAtStartPar
As another frequent application of Boltzmann’s law are state populations of two state systems, as we find them frequently in physics, e.g., for spin systems. Such two level spin systems are, for example, very important for nuclear magnetic resonance (NMR), which is an important tool to study the structure and dynamics of soft matter.

\noindent\sphinxincludegraphics[width=438\sphinxpxdimen,height=358\sphinxpxdimen]{{two_level}.png}

\sphinxAtStartPar
Consider the image above, where a single energy level at zero magnetic field (\(B=0\)) splits into two energy levels due to the interaction of a proton spin (red arrow) with the external magnetic field.

\sphinxAtStartPar
The magnetic moment of the proton spin may take two expectation values in the magnetic field, which are characterized by the magnetic quantum number \(m_Z=\pm 1/2\). The magnetic moment projected along the magnetic field direction is then
\begin{equation*}
\begin{split}\mu_Z=\gamma m_Z \hbar\end{split}
\end{equation*}
\sphinxAtStartPar
and the energy of the states
\begin{equation*}
\begin{split}E(m_Z)=-\mu_Z B=-\gamma m_Z \hbar B\end{split}
\end{equation*}
\sphinxAtStartPar
with \(\gamma=2.675222005\times 10^{8}\; \mathrm{s}^{-1} \mathrm{T}^{-1}\) being the gyromagnetic ratio of the proton. The energy difference for a nonzero magnetic field is therefore given by
\begin{equation*}
\begin{split}\Delta E=\gamma \hbar B=\hbar \omega_\mathrm{L}\end{split}
\end{equation*}
\sphinxAtStartPar
which results for a magnetic field of \(B=1\; \mathrm{T}\) in \(\Delta E\approx 1.76\times 10^{-7}\;{\rm eV}\) or a Larmor frequency of \(\omega_\mathrm{L}=42\; {\rm MHz}\). This energy difference is almost negligible as compared to the thermal energy at room temperature \(k_\mathrm{B}T=2.6\times 10^{-2}\;{\rm eV}\). Yet, this small energy difference is used to give the contrast in NMR and related techniques such as MRI.

\sphinxAtStartPar
Using the Boltzmann distribution we can now calculate the ratio of the population of spins in the lower or excited state
\begin{equation*}
\begin{split}\frac{N_{-\frac{1}{2}}}{N_{+\frac{1}{2}}}=\exp\left (-\frac{\Delta E}{k_\mathrm{B} T} \right )\end{split}
\end{equation*}
\sphinxAtStartPar
which is very close to one:
\begin{equation*}
\begin{split}\frac{N_{-\frac{1}{2}}}{N_{+\frac{1}{2}}}=0.99999332.\end{split}
\end{equation*}
\sphinxAtStartPar
If you consider now a volume of \(V=1\;{\rm {\mu m}}^3\) water, then you would roughly have about \(N=6.7\times 10^{19}\) protons. This then means that the excess number of protons in the excited state is just \(N_{+\frac{1}{2}}-N_{-\frac{1}{2}}=4.5\times 10^{12}\), which is extremely low. Thus, to detect something in NMR or MRI, a certain number of protons in the volume is required.
\end{sphinxadmonition}

\begin{sphinxadmonition}{note}{}\unskip
\sphinxAtStartPar
\sphinxstylestrong{Example: Position of a Bead in an Optical Tweezer}

\sphinxAtStartPar
In an optical tweezer, a polarizable object (e.g. a polymer bead) is hold in the intensity gradient of a focused laser beam. The nearly Gaussian intensity distribution of a focused beam leads, in first approximation to a linear force a parabolic potential and can be employed to measure tiny forces.

\noindent\sphinxincludegraphics[width=800\sphinxpxdimen,height=353\sphinxpxdimen]{{tweezers}.png}

\sphinxAtStartPar
For one dimension of the 3D optical potential it can thus be written as

\sphinxAtStartPar
\begin{equation}
F=-k x
\end{equation}

\sphinxAtStartPar
and

\sphinxAtStartPar
\begin{equation}
E=\frac{1}{2}k x^2
\end{equation}

\sphinxAtStartPar
Using the Boltznmann distribution for the potential provides the probability distribution for finding the particle at a certain position \(x\)

\sphinxAtStartPar
\begin{equation}
p(x)=\frac{1}{Z}\exp\left(-\frac{kx^2}{2k_B T} \right)
\end{equation}

\sphinxAtStartPar
which resembles a Gaussian distribution with a variance of

\sphinxAtStartPar
\begin{equation}
\sigma^2=\langle x^2 \rangle =\frac{k_B T}{k}
\end{equation}

\sphinxAtStartPar
We also readily recognize that the partition function \(Z\) is the normalization factor of the Gaussian

\sphinxAtStartPar
\begin{equation}
Z=\sqrt{2\pi }\sigma=\sqrt{2\pi\frac{k_B T }{k}}
\end{equation}

\sphinxAtStartPar
The mean potential energy is then calculated by

\sphinxAtStartPar
\begin{equation}
\langle E \rangle =\int_{-\infty}^{\infty} E\, p(x)\,dx=\frac{1}{2}k\int_{-\infty}^{\infty} x^2 p(x)dx=\frac{1}{2}k \frac{k_B T}{k}=\frac{1}{2}k_B T
\end{equation}

\sphinxAtStartPar
With the help of the variance of the distribution mentioned above, we also recognize that the trap stiffness can be obtained by dividing th ethermal energy \(k_B T\) by the variance of the positional fluctuations.

\sphinxAtStartPar
\begin{equation}
k=\frac{k_B T}{\langle x^2\rangle }
\end{equation}

\begin{DUlineblock}{0em}
\item[] which is very helpful for the experiment calibrating the trap stiffness for using the tweezers for force measurements. According to that you just have to observe the position of the trapped object in the potential and measure the variance.
\item[] We will deal with this relation later again in the section about the fluctuation dissipation relation.
\end{DUlineblock}

\sphinxAtStartPar
The second degree of freedom of a particle in an optical tweezer is given by its velocity \(v\). This is as well a quadratic degree in \(v\), yet its measurement turns out to be tricky. As the particle is carrying out Brownian motion in the trap, its velocity is not given by the difference of positions divided by the difference in observation times. To measure the velocity accurately one has to got to very short times, when the particle is carrying out ballistic motion. This has been
achieved only about 10 years ago, while the distribution connected to the velocities, the \sphinxstylestrong{Maxwell\sphinxhyphen{}Boltzmann distribution} is already known for a very long time.

\sphinxAtStartPar
Take some time to have a look at the Maxwell\sphinxhyphen{}Boltzmann distribution again.
\end{sphinxadmonition}


\chapter{When a Macrostate is a Microstate}
\label{\detokenize{notebooks/L3/1_Equipartition:When-a-Macrostate-is-a-Microstate}}
\sphinxAtStartPar
In practice, we are often interested in the likelihood that the system is in a state that is described by some macroscopic parameter \(X\) that we can measure. For example, for a DNA molecule inside a cell, an interesting quantity, which can be measured using fluorescent markers, is the distance R between two sites on the DNA chain. Repeated measurements of \(R\) can to construct the probability distribution p(R).

\sphinxAtStartPar
In general, the probability of the macrostate \(X\) is given by the sum of probabilities of all the microstates of the system that adopt the specified value \(X\),

\sphinxAtStartPar
\begin{equation}
p(X)=\sum_{i_{X}} p_{i}=\sum_{i_{X}} \frac{1}{Z} \mathrm{e}^{-\beta E_{i}}
\end{equation}

\sphinxAtStartPar
For the DNA example, the sum in the above equation would run over only those microstates \(i_X\) that have the prescribed distance between the two labeled sites on the polymer, e.g. \(X=R\). Using the basic relation between the partition function and the free energy, \(G = −k_BT \ln(Z)\), we can express the probability of the macrostate X as

\sphinxAtStartPar
\begin{equation}
p(X)=\frac{1}{Z} \mathrm{e}^{-\beta G(X)}
\end{equation}

\sphinxAtStartPar
where

\sphinxAtStartPar
\begin{equation}
G(X)=-k_{\mathrm{B}} T \ln \left(\sum_{i_{X}} \mathrm{e}^{-\beta E_{i}}\right)
\end{equation}

\sphinxAtStartPar
is the free energy of the macrostate \(X\). Note that the formula for \(p(X)\) is identical to the Boltzmann formula for the probability of a microstate, with the energy of the microstate replaced by the free energy of the macrostate. Note that the sum on the right side of the last equation is not the partition function \(Z\) but that of the subensemble of microstates fulfilling the condition X, i.e. \(Z_X\). Similarly, when writing down the states and weights for the macrostates
\(X\), the energy is replaced by the free energy, as shown in in the figure below. In this sense, one person’s macrostate is truly another person’s microstate.

\noindent\sphinxincludegraphics[width=500\sphinxpxdimen,height=252\sphinxpxdimen]{{micro_macro}.png}



\sphinxAtStartPar
The following section was created from \sphinxcode{\sphinxupquote{notebooks/L3/2\_Chemical\_Potential.ipynb}}.


\chapter{Chemical Potential}
\label{\detokenize{notebooks/L3/2_Chemical_Potential:Chemical-Potential}}\label{\detokenize{notebooks/L3/2_Chemical_Potential::doc}}
\sphinxAtStartPar
So far, we have considered either isolated systems, meaning that no energy nor particle exchange is possible, and closed systems, which allow an exchange of energy with a reservoir (bath). We will now have a look at a system that also allows an exchange of particles with a reservoir. For these systems, we will define the \sphinxstylestrong{chemical potential} \(\mu\) which we will identify as the \sphinxstylestrong{free energy change when adding one molecule of a given species to a thermodynamic system} at constant
\(p\) and \(T\).

\sphinxAtStartPar
We consider a system of two components A and B in contact with a heat bath and a diffusive equilibrium between A and B.

\noindent\sphinxincludegraphics[width=300\sphinxpxdimen,height=153\sphinxpxdimen]{{chem_pot}.png}

\sphinxAtStartPar
The total free energy of our system is then defined as

\sphinxAtStartPar
\begin{equation}
G=G_{\mathrm A}+G_{\mathrm B}.
\end{equation}

\sphinxAtStartPar
The total number of particles in the system is also conserved and given by \(N_{\rm tot}=N_{\mathrm A}+N_{\mathrm B}\). In equilibrium we require the free energy of the system to be a minimum with respect to the particle number such that:

\sphinxAtStartPar
\begin{equation}
{\mathrm d}G=\frac{ \partial G_{\mathrm A}}{\partial N_{\mathrm A}}\bigg|_{T,p}{\mathrm d}N_{\mathrm A}+\frac{\partial G_{\mathrm B}}{\partial N_{\mathrm B}}\bigg|_{T,p}{\mathrm d}N_{\mathrm B}=\frac{ \partial G_{\mathrm A}}{\partial N_{\mathrm A}}\bigg|_{T,p}{\mathrm d}N_{\mathrm A}-\frac{\partial G_{\mathrm B}}{\partial N_{\mathrm B}}\bigg|_{T,p}{\mathrm d}N_{\mathrm A}=0.
\end{equation}

\sphinxAtStartPar
with \({\mathrm d}N_{\mathrm A}=-{\mathrm d}N_{\mathrm B}\), since the total number of particles is conserved (i.e., \(\mathrm{d}N_{\rm tot}=0\)):

\sphinxAtStartPar
\begin{equation}
\frac{\mathrm dG_{\mathrm A}}{\mathrm dN_{\mathrm A}}=\frac{\mathrm dG_{\mathrm B}}{\mathrm dN_{\mathrm B}}.
\end{equation}

\sphinxAtStartPar
Apparently, in equilibrium the change in the free energy of system \(A\) needs to be the same of system \(B\) when exchanging particles at constant \(T,p\).

\begin{sphinxadmonition}{warning}{}\unskip
\sphinxAtStartPar
\sphinxstylestrong{Chemical Potential}

\sphinxAtStartPar
We can define a quantity

\sphinxAtStartPar
\begin{equation}
\mu(T,p,N)=\frac{\mathrm dG}{\mathrm dN}\bigg|_{T,P}
\end{equation}

\sphinxAtStartPar
which is termed the \sphinxstylestrong{chemical potential}, which is the change in the free energy when adding a particle to the system. The chemical potential may be interpreted as the cost for adding one more particle (at constant \(T,p\)) to the system.

\sphinxAtStartPar
Note
\begin{itemize}
\item {} 
\sphinxAtStartPar
if \(\mu_l<\mu_v\) at constant \(T,p\), we can lower the free energy by moving a particle from \(v\) to \(l\),

\item {} 
\sphinxAtStartPar
\(\mu_l\neq \mu_v\) means non\sphinxhyphen{}equilibrium and partricles will diffuse until \(\mu_l=\mu_v\).

\end{itemize}
\end{sphinxadmonition}

\sphinxAtStartPar
We will have a closer look at the chemical potential with an example below. The previous condition states that the chemical potential of the two systems is the same in equilibrium. The chemical potential is useful in many situations. For example, to determine


\section{Phase equilibria}
\label{\detokenize{notebooks/L3/2_Chemical_Potential:Phase-equilibria}}
\noindent\sphinxincludegraphics[width=318\sphinxpxdimen,height=323\sphinxpxdimen]{{phase_equilibria}.png}

\sphinxAtStartPar
In this case, we have a boundary between two phases (e.g., liquid and vapor) and particles of the vapour phase may join the liquid phase and vice versa. In this case, the number of vapor particles \(N_{\rm vapor}\) and the number of liquid particles \(N_{\rm liquid}\) is not fixed, but the total number of particles \(N=N_{\rm vapor}+N_{\rm liquid}\) is.


\section{Chemical equilibria}
\label{\detokenize{notebooks/L3/2_Chemical_Potential:Chemical-equilibria}}
\noindent\sphinxincludegraphics[width=309\sphinxpxdimen,height=314\sphinxpxdimen]{{chemical_equilibria}.png}

\sphinxAtStartPar
In the case of chemical equilibria, two species \(A\) and \(B\) may react to form a new species \(AB\) by a chemical reaction:

\sphinxAtStartPar
\begin{equation}
A+B \rightleftharpoons A B.
\end{equation}

\sphinxAtStartPar
Here the individual numbers of particles \(N_A\),\(N_B\) and \(N_{AB}\) are not fixed, but \(N_A + N_{AB}\) and \(N_B+N_{AB}\) are.

\begin{sphinxadmonition}{note}{}\unskip
\sphinxAtStartPar
\sphinxstylestrong{Example: Free energy of a dilute solution}

\sphinxAtStartPar
We would like to calculate the free energy and the chemical potential for a dilute solution of some \sphinxstyleemphasis{solutes} in a solvent, which we just term H\(_2\)O. Actually, no additional information on the details of the solute and solvent are currently required, though if we want to have numbers, we would need to know which solute or solvent we are looking at.

\noindent\sphinxincludegraphics[width=300\sphinxpxdimen,height=299\sphinxpxdimen]{{solute_solvent}.png}

\sphinxAtStartPar
Suppose we have
\begin{itemize}
\item {} 
\sphinxAtStartPar
\(N_{\rm H_2 O}\) solvent molecules (e.g., water),

\item {} 
\sphinxAtStartPar
\(N_{\rm s}\) solute molecules (e.g., proteins).

\end{itemize}

\sphinxAtStartPar
The solute chemical potential is defined by

\sphinxAtStartPar
\begin{equation}
\mu_{\rm s}=\left( \frac{\partial G_{\rm tot}}{\partial N_{\rm s}}\right)\bigg|_{p,T}
\end{equation}

\sphinxAtStartPar
or intuitively as

\sphinxAtStartPar
\begin{equation}
\mu_{\rm s}=G_{\rm tot}(N_{\rm s}+1)-G_{\rm tot}(N_{\rm s}).
\end{equation}

\sphinxAtStartPar
The total free energy \(G_{\rm tot}\) consists of the enthalpy of forming the solvent molecules \(N_{\rm{H_{2}O}}\varepsilon_{\rm H_{2}O}\) and the enthalpy solvating the solute proteins \(N_{\rm s}\varepsilon_{\rm s}\).

\sphinxAtStartPar
\begin{equation}
G_{\rm tot}=N_{\rm{H_{2}O}}\varepsilon_{\rm H_{2}O}+N_{\rm s}\varepsilon_{\rm s}-TS_{\rm mix}.
\end{equation}

\begin{DUlineblock}{0em}
\item[] The last term \(TS_{\rm mix}\) denotes the entropy for mixing the solute and the solvent molecules.
\item[] In the following, we are interested in the last term, which is the entropic contribution to the free energy containing the entropy of mixing solute and solvent.
\end{DUlineblock}

\sphinxAtStartPar
The mixing entropy can be calculated by either the Gibbs definition

\sphinxAtStartPar
\begin{equation}
S_{\rm mix}=-k_{\rm B}\sum_{i}p_{i}\ln(p_{i})
\end{equation}

\sphinxAtStartPar
or the Boltzmann definition

\sphinxAtStartPar
\begin{equation}
S_{\rm mix}=k_{\rm B}\ln(W).
\end{equation}

\sphinxAtStartPar
We will use the latter definition including the number of possible configurations to obtain the mixing entropy. If the total number of molecules in the volume is \(N=N_{\rm H_2O}+N_{\rm s}\), we can write down the number of different ways to arrange the molecules as previously done. We obtain:

\sphinxAtStartPar
\begin{equation}
W(N_{\rm H_2O}, N_{\rm s} )=\frac{N!}{N_{\rm H_2O}! N_{\rm s}!}.
\end{equation}

\sphinxAtStartPar
As in our example with the DNA binding, we can apply the Stirling formula for large \(N\) and find

\sphinxAtStartPar
\begin{equation}
S_{\rm mix}=-k_\mathrm{B} \left ( N_{\rm H_2O} \ln\left (\frac{N_{\rm H_2O}}{N_{\rm H_2O}+N_{\rm s}}\right) +N_{\rm s}\ln\left (\frac{N_{\rm s}}{N_{\rm H_2O}+N_{\rm s}} \right )\right ).
\end{equation}

\sphinxAtStartPar
To really go in the dilute limit, the number of solute molecules should be much larger than the number of solvent molecules, i.e.,
\begin{equation*}
\begin{split}\frac{N_{\rm s}}{N_{\rm H_2O}}\ll 1\end{split}
\end{equation*}
\sphinxAtStartPar
which allows us to approximate the solution with

\sphinxAtStartPar
\begin{equation}
S_{\rm mix}\approx -k_\mathrm{B} \left ( N_{\rm H_2O} \ln \left ( 1-\frac{N_{\rm s}}{N_{\rm H_2O}}\right) +N_{\rm s}\ln\left ( \frac{N_{\rm s}}{N_{\rm H_2O}}\right)\right )
\end{equation} employing the Taylor series expansion of \(N_{\rm H_2O}/(N_{\rm H_2O}+N_{\rm s})=1/(1+N_{\rm s}/N_\mathrm{H_2O})\) at \(N_{\rm s}/N_\mathrm{H_2O}\approx 0\).

\sphinxAtStartPar
Using
\begin{equation*}
\begin{split}\ln(1+\epsilon)\approx \epsilon\end{split}
\end{equation*}
\sphinxAtStartPar
we may finally write

\sphinxAtStartPar
\begin{equation}
S_{\rm mix}\approx -k_\mathrm{B} \left ( N_{\rm s} \ln \left ( \frac{N_{\rm s}}{N_{\rm H_2O}}\right) -N_{\rm s}\right )
\end{equation}

\sphinxAtStartPar
and the total free energy is then given by

\sphinxAtStartPar
\begin{equation}
G_{\rm tot}=N_{\rm{H_{2}O}}\varepsilon_{\rm H_{2}O}+N_{\rm s}\varepsilon_{\rm s}+k_\mathrm{B} T \left ( N_{\rm s} \ln \left ( \frac{N_{\rm s}}{N_{\rm H_2O}}\right) -N_{\rm s}\right ).
\end{equation}

\sphinxAtStartPar
We can now come back and calculate the chemical potential of the solute as noted above

\sphinxAtStartPar
\begin{equation}
\mu_{\rm s}=\epsilon_{\rm s}+k_{\rm B} T \ln \left ( \frac{N_{\rm s}}{N_{\rm H_2O}}\right ).
\end{equation}

\sphinxAtStartPar
As we rather work with concentrations than number of molecules we can use \(c=N_{\rm s}/V\) and \(c_0=c_{\rm H_2O}=N_{\rm H_2O}/V\) to write

\sphinxAtStartPar
\begin{equation}
\mu_{\rm s}=\epsilon_{\rm s}+k_{\rm B} T \ln \left ( \frac{c}{c_0}\right ).
\end{equation}

\sphinxAtStartPar
The value of \(c_0\) thereby acts as a reference point, which is commonly chosen to be at \(c_0=1\,{\rm M}\). The value of \(\epsilon_{\rm s}\) is termed the standard chemical potential \(\mu_{\rm s}^{0}\). The standard chemical potential is measured at standard thermodynamic conditions, i.e. \(p_0=101.3\, {\rm kPa}\) and \(T=293.15\, {\rm K}\), and \(c_0=1\, {\rm M}\).
\end{sphinxadmonition}

\sphinxAtStartPar
In a mixed system, we define the chemical potential of a component \(i\) as the sum of two components

\sphinxAtStartPar
\begin{equation}
\mu_i=\mu_i^0+k_\mathrm{B} T \ln \left ( \frac{c_i}{c_{i,0}}\right ).
\end{equation}
\begin{itemize}
\item {} 
\sphinxAtStartPar
The standard chemical potential is a molecular property (see slides) that contains
\begin{itemize}
\item {} 
\sphinxAtStartPar
the internal energy/enthalpy to create the molecule,

\item {} 
\sphinxAtStartPar
the conformational entropy of the molecule, and

\item {} 
\sphinxAtStartPar
the enthalpy and entropy contributions when bringing the molecule into contact with the solvent (solvation free energy).

\end{itemize}

\item {} 
\sphinxAtStartPar
The logarithm term \(k_\mathrm{B} T \ln \left ( \frac{c_i}{c_{i0}}\right )\) is the pure mixing/dilution entropy reflecting the different abundance of solute and solvent due to the resulting numbers of microstates.

\end{itemize}

\sphinxAtStartPar
For a solvent we consequently write

\sphinxAtStartPar
\begin{equation}
\mu_{\rm H_2O}=\frac{\partial G_{\rm tot}}{\partial N_{\rm H_2O}}\bigg|_{T,p}=\mu_{\rm H_2O}^{0}-k_\mathrm{B} T\ln\left ( \frac{c_{\rm s}}{c_{\rm H_2O}}\right).
\end{equation}

\sphinxAtStartPar
Finally, a chemical potential may be also defined for gases, e.g., for the ideal gas:

\sphinxAtStartPar
\begin{equation}
\mu_{\rm iG}=k_\mathrm{B} T \ln\left (\frac{p}{p_0}\right)
\end{equation}

\sphinxAtStartPar
which only has entropic contributions.

\begin{sphinxadmonition}{note}{}\unskip
\sphinxAtStartPar
\sphinxstylestrong{Example: Barometric height formula}

\sphinxAtStartPar
We can have a look again at the barometric height formula using the framework of the chemical potential. Each height \([h,h+\mathrm{d}h]\) is exchanging particles with other heights until the chemical potential in all regions is the same. We can thus write

\sphinxAtStartPar
\begin{eqnarray}
\mu(h=0)&=&\mu_{0}(h=0)+k_\mathrm{B} T\ln \left ( \frac{n(h=0)}{n_0}\right )\\
&=&\mu_{0}(h)+k_\mathrm{B} T \ln \left ( \frac{n(h)}{n_0}\right )
\end{eqnarray}

\sphinxAtStartPar
The standard chemical potential is just given by the potential energy, i.e.:
\begin{equation*}
\begin{split}\mu_{0}(h)=mgh\end{split}
\end{equation*}
\sphinxAtStartPar
and
\begin{equation*}
\begin{split}\mu_{0}(h=0)=0\end{split}
\end{equation*}
\sphinxAtStartPar
which then leads us to

\sphinxAtStartPar
\begin{equation}
-mgh=k_\mathrm{B} T\left ( \ln \left( \frac{n(h)}{n_0} \right )    -\ln \left( \frac{n(h=0)}{n_0} \right ) \right).
\end{equation}

\sphinxAtStartPar
This finally leads us to

\sphinxAtStartPar
\begin{equation}
\frac{n(h)}{n(h=0)}=\exp\left (-\frac{mgh}{k_B T} \right )
\end{equation}

\sphinxAtStartPar
which is, of course, the barometric height formula.
\end{sphinxadmonition}



\sphinxAtStartPar
The following section was created from \sphinxcode{\sphinxupquote{notebooks/L4/1\_Osmotic\_Pressure.ipynb}}.


\chapter{Osmotic Pressure}
\label{\detokenize{notebooks/L4/1_Osmotic_Pressure:Osmotic-Pressure}}\label{\detokenize{notebooks/L4/1_Osmotic_Pressure::doc}}
\sphinxAtStartPar
The chemical potential as defined in the previous section also allows us to understand the osmotic pressure. Consider a volume that is seperated into two equally sized parts by a semipermeable membrane. One of the compartments is filled with pure water, the other compartment contains a number of additional solute molecules. The solute molecules thereby cannot pass the membrane while the water can easily pass it.

\sphinxAtStartPar
The chemical potential of the water
\begin{equation*}
\begin{split}\mu_{\rm H_2O}=\frac{\partial G_{\rm tot}}{\partial N_{\rm H_2O} }\end{split}
\end{equation*}
\sphinxAtStartPar
needs to be the same on both sides.

\sphinxAtStartPar
The chemical potential on the solute side is given by
\begin{equation*}
\begin{split}\mu_{\rm H_2O}=\mu_{\rm H_2O}^{o}(T,p_2)-\frac{N_{\rm s}}{N_{\rm H_2O}}k_{\rm B} T\end{split}
\end{equation*}
\sphinxAtStartPar
and on the solvent side
\begin{equation*}
\begin{split}\mu_{\rm H_2O}=\mu_{\rm H_2O}^{o}(T,p_1)\end{split}
\end{equation*}
\sphinxAtStartPar
from which follows that
\begin{equation*}
\begin{split}\mu_{\rm H_2O}^o(T,p_1)=\mu_{\rm H_2O}^o(T,p_2)-\frac{N_{\rm s}}{N_{\rm H_2O}} k_{\rm B} T.\end{split}
\end{equation*}
\sphinxAtStartPar
Here we have already assumed that on both sides of the membrane we have a different pressure (\(p_1\neq p_2\)). If both pressures are only slightly different we can write
\begin{equation*}
\begin{split}\mu_{\rm H_2O}^o(T,p_2)\approx \mu_{\rm H_2O}^o (T,p_1)+\left ( \frac{\partial \mu_{\rm H_2O}^o }{\partial p}\right )(p_2-p_1)\end{split}
\end{equation*}
\sphinxAtStartPar
which is just a Taylor expansion. It turns out, that the derivative
\begin{equation*}
\begin{split}\frac{\partial \mu }{\partial p}=v\end{split}
\end{equation*}
\sphinxAtStartPar
is nothing else than the volume occupied by one molecule. Inserting the Taylor expansion into the original equality of the chemical potentials, we find

\sphinxAtStartPar
\begin{equation}
p_2-p_1=\frac{N_{\rm s}}{v N_{\rm H_2O}} k_{\rm B} T.
\end{equation}

\sphinxAtStartPar
With \(V=v N_{\rm H_2O}\) as the toral volume of water we have

\sphinxAtStartPar
\begin{equation}
\Pi =p_2-p_1=\frac{N_{\rm s}}{V}k_{\rm B} T = n_{\rm s}k_{\rm B} T
\end{equation}

\sphinxAtStartPar
which is the van’t Hoff formula for the \sphinxstylestrong{osmotic pressure}.

\begin{sphinxadmonition}{warning}{}\unskip
\sphinxAtStartPar
\sphinxstylestrong{Osmotic Pressure}

\sphinxAtStartPar
The van’t Hoff formula for the osmotic pressure is

\sphinxAtStartPar
\begin{equation}
\Pi = n_{\rm s}k_{\rm B} T
\end{equation}

\sphinxAtStartPar
where \(n\) is the number density of suspended objects and \$k\_B T \$ is the thermal energy.
\end{sphinxadmonition}

\sphinxAtStartPar
Note that the equation looks very much like the equation of state of the ideal gas, however, the pressure is not generated by the collisions of the solute molecules with the membrane. The pressure is generated by the water molecules which try to dilute the solute molecules.

\noindent\sphinxincludegraphics[width=824\sphinxpxdimen,height=339\sphinxpxdimen]{{osmotic_pressure}.jpeg}

\sphinxAtStartPar
Image taken from \sphinxhref{https://chem.libretexts.org/Bookshelves/General\_Chemistry/Map\%3A\_General\_Chemistry\_(Petrucci\_et\_al.)/13\%3A\_Solutions\_and\_their\_Physical\_Properties/13.07\%3A\_Osmotic\_Pressure}{chemistry libre texts}

\sphinxAtStartPar
The concept of osmotic pressures play a very important role in soft matter physics, biology but also environmental science.


\section{Shape of red blood cells}
\label{\detokenize{notebooks/L4/1_Osmotic_Pressure:Shape-of-red-blood-cells}}
\sphinxAtStartPar
One particular and often mentioned example of the importance of osmotic pressure is the shape of red blood cells.

\noindent\sphinxincludegraphics[width=1010\sphinxpxdimen,height=239\sphinxpxdimen]{{cells}.jpeg}

\sphinxAtStartPar
Image taken from \sphinxhref{https://chem.libretexts.org/Bookshelves/General\_Chemistry/Map\%3A\_General\_Chemistry\_(Petrucci\_et\_al.)/13\%3A\_Solutions\_and\_their\_Physical\_Properties/13.07\%3A\_Osmotic\_Pressure}{chemistry libre texts}


\section{Reverse Osmosis}
\label{\detokenize{notebooks/L4/1_Osmotic_Pressure:Reverse-Osmosis}}
\sphinxAtStartPar
A second very important application for osmosis is actually reverse osmosis. Reverse osmosis is heavily employed for water desalination. Salt water is pumped through a membrane, which is selectively preventing ions from passing the membrane. Due to the osmotic pressure difference between the salt free and the salty regions, high pressures are required to pump the water through these membranes. Yet, the technical realizations of this method come close to the thermodynamic limit of this process.

\noindent\sphinxincludegraphics[width=801\sphinxpxdimen,height=339\sphinxpxdimen]{{reverse_osmosis}.jpeg}

\sphinxAtStartPar
Image taken from \sphinxhref{https://chem.libretexts.org/Bookshelves/General\_Chemistry/Map\%3A\_General\_Chemistry\_(Petrucci\_et\_al.)/13\%3A\_Solutions\_and\_their\_Physical\_Properties/13.07\%3A\_Osmotic\_Pressure}{chemistry libre texts}



\sphinxAtStartPar
The following section was created from \sphinxcode{\sphinxupquote{notebooks/L4/2\_Gibbs\_Distribution.ipynb}}.


\chapter{Gibbs Distribution}
\label{\detokenize{notebooks/L4/2_Gibbs_Distribution:Gibbs-Distribution}}\label{\detokenize{notebooks/L4/2_Gibbs_Distribution::doc}}
\sphinxAtStartPar
So far, we have considered closed systems and derived and studied the Boltzmann distribution. Now our systems are open and the particle number might change. The Boltzmann distribution is therefore not anymore sufficient to describe the probability to find a system with an energy \(E_\mathrm{s}\) and a particle number \(N_\mathrm{s}\) in contact with a reservoir. To derive a distribution for this new situation, we can use one of the ways, which also allowed the derivation of the Boltzmann
distribution. This route is, again, considering the ratio of the proababilities of two arbitrary states of the system (index s) which is equal to the ratio of the numbers of possible states of the reservoir (index \(r\)):

\sphinxAtStartPar
\begin{equation}
\frac{p(E_\mathrm{s}^{(1)},N_\mathrm{s}^{(1)})}{p(E_\mathrm{s}^{(2)},N_\mathrm{s}^{(2)})}=\frac{W_\mathrm{r}(E_\mathrm{tot}-E_\mathrm{s}^{(1)},N_\mathrm{tot}^{(1)})}{W_\mathrm{r}(E_\mathrm{tot}-E_\mathrm{s}^{(2)},N_\mathrm{tot}^{(2)})}
\end{equation}

\sphinxAtStartPar
Hereby, the \(E_\mathrm{s}^{(i)}, N_\mathrm{s}^{(i)}\) denote the energy and the number of particles in the system in the \(i\)th state. \(W_\mathrm{r}\) indicates the number of available states to the reservoir, when the system is in a specific energy state with a specific number of particles. Similar to the derivation of the Boltzmann distribution, we can now assert that the total number of available states of the system and the reservoir is given by

\sphinxAtStartPar
\begin{equation}
W_\mathrm{tot}(E_\mathrm{tot}-E_\mathrm{s}^{(1)},N_\mathrm{tot}-N_\mathrm{s}^{(1)})=1\times W_\mathrm{r}(E_\mathrm{tot}-E_\mathrm{s}^{(1)},N_\mathrm{tot}-N_\mathrm{s}^{(1)}).
\end{equation}

\sphinxAtStartPar
Note that the system is in exactly one state in this case. Utilizing again that \(S=k_\mathrm{B} \ln(W)\) or \(W=\exp(S/k_\mathrm{B})\), we find

\sphinxAtStartPar
\begin{equation}
\frac{W_\mathrm{r}^{(1)}}{W_\mathrm{r}^{(2)}}=\frac{\exp(S_\mathrm{r}^{(1)}/k_\mathrm{B})}{\exp(S_\mathrm{r}^{(2)}/k_\mathrm{B})}.
\end{equation}

\sphinxAtStartPar
Once more, we can use a first order expansion to get

\sphinxAtStartPar
\begin{equation}
S_\mathrm{r}(E_\mathrm{tot}-E_\mathrm{s},N_\mathrm{tot}-N_\mathrm{s})\approx S_\mathrm{r}(E_\mathrm{tot},N_\mathrm{tot})-\frac{\partial S_\mathrm{r}}{\partial E}E_\mathrm{s}-\frac{\partial S_\mathrm{r}}{\partial N}E_{N_\mathrm{s}}.
\end{equation}

\sphinxAtStartPar
The partial derivatives can be converted with the help of the known thermodynamic relations

\sphinxAtStartPar
\begin{equation}
\frac{\partial S_\mathrm{r}}{\partial E}\bigg|_{V,N} =\frac{1}{T}
\end{equation}

\sphinxAtStartPar
and

\sphinxAtStartPar
\begin{equation}
\frac{\partial S_\mathrm{r}}{\partial N}\bigg|_{V,E}=-\frac{\mu}{T}
\end{equation}

\sphinxAtStartPar
which finally yields the Gibbs distribution:

\begin{sphinxadmonition}{warning}{}\unskip
\sphinxAtStartPar
\sphinxstylestrong{Gibbs distribution}

\sphinxAtStartPar
\begin{equation}
p(E_\mathrm{s}^{(1)},N_\mathrm{s}^{(1)})=\frac{1}{Z}\exp\left ( -\frac{E_\mathrm{s}^{(1)}-\mu N_\mathrm{s}^{(1)} }{k_\mathrm{B} T} \right )
\end{equation}

\sphinxAtStartPar
with the \sphinxstylestrong{grand partition function}

\sphinxAtStartPar
\begin{equation}
Z=\sum_{i}\exp \left ( -\frac{E_\mathrm{s}^{(i)}-\mu N_\mathrm{s}^{(i)}}{k_\mathrm{B} T}\right ).
\end{equation}
\end{sphinxadmonition}

\begin{sphinxadmonition}{note}{}\unskip
\sphinxAtStartPar
\sphinxstylestrong{Ligand Binding}

\sphinxAtStartPar
As an example, we have a look at the binding of a ligand (green) to a receptor, which is indicated in the figure below.

\noindent\sphinxincludegraphics[width=400\sphinxpxdimen,height=144\sphinxpxdimen]{{two_state_binding}.png}

\sphinxAtStartPar
The binding is reported by the variable \(\sigma\), which can be either 0 or 1. The binding energy is therefore

\sphinxAtStartPar
\begin{equation}
E=\sigma \varepsilon_\mathrm{b}
\end{equation}

\sphinxAtStartPar
with \(\varepsilon_\mathrm{b}<0\), since the ligand wants to bind. The grand partition function

\sphinxAtStartPar
\begin{equation}
Z=\sum_{\rm states}\exp\left ( -\beta (E_{\rm state}- \mu N_{\rm state}) \right ),
\end{equation}

\sphinxAtStartPar
in which the chemical potential \(\mu\) reflects the contact with the particle reservoir and \(\beta=(k_\mathrm{B}T)^{-1}\), can therefore be written as:

\sphinxAtStartPar
\begin{equation}
Z=\sum_{\sigma=0}^{1}\exp\left ( -\beta (\varepsilon_\mathrm{b} \sigma- \mu \sigma) \right )=1+\exp(-\beta (\varepsilon_\mathrm{b} -\mu )).
\end{equation}

\sphinxAtStartPar
From that we can calculate the mean number of bounded ligands to be

\sphinxAtStartPar
\begin{equation}
\langle N \rangle =\frac{\exp(-\beta (\varepsilon_\mathrm{b} -\mu ))}{1+\exp(-\beta (\varepsilon_\mathrm{b} -\mu ))}.
\end{equation}

\sphinxAtStartPar
Using

\sphinxAtStartPar
\begin{equation}
\mu = \mu_0 + k_\mathrm{B} T \ln \left ( \frac{c}{c_0}\right )
\end{equation}

\sphinxAtStartPar
we can finally write

\sphinxAtStartPar
\begin{equation}
\langle N \rangle =\frac{c/c_0 \exp(-\beta \Delta \varepsilon_\mathrm{b})}{1+\frac{c}{c_0}\exp(-\beta \Delta \varepsilon_\mathrm{b})}
\end{equation}

\sphinxAtStartPar
with \(\Delta \varepsilon_\mathrm{b}=\varepsilon_\mathrm{b}-\mu_0\), which is the energy freed when taking the ligand from the solution and placing it at the acceptor.
\end{sphinxadmonition}



\sphinxAtStartPar
The following section was created from \sphinxcode{\sphinxupquote{notebooks/L5/1\_Phase\_Transitions.ipynb}}.


\chapter{Phase Transitions}
\label{\detokenize{notebooks/L5/1_Phase_Transitions:Phase-Transitions}}\label{\detokenize{notebooks/L5/1_Phase_Transitions::doc}}
\sphinxAtStartPar
The importance of looking at phase transitions resides in the multicomponent nature of soft matter and its interaction energy scales.

\begin{sphinxuseclass}{nbinput}
\begin{sphinxuseclass}{nblast}
{
\sphinxsetup{VerbatimColor={named}{nbsphinx-code-bg}}
\sphinxsetup{VerbatimBorderColor={named}{nbsphinx-code-border}}
\begin{sphinxVerbatim}[commandchars=\\\{\}]
\llap{\color{nbsphinxin}[5]:\,\hspace{\fboxrule}\hspace{\fboxsep}}\PYG{k+kn}{import} \PYG{n+nn}{numpy} \PYG{k}{as} \PYG{n+nn}{np}
\PYG{k+kn}{from} \PYG{n+nn}{ipywidgets} \PYG{k+kn}{import} \PYG{n}{interact}\PYG{p}{,} \PYG{n}{interactive}\PYG{p}{,} \PYG{n}{fixed}\PYG{p}{,} \PYG{n}{interact\PYGZus{}manual}
\PYG{k+kn}{import} \PYG{n+nn}{ipywidgets} \PYG{k}{as} \PYG{n+nn}{widgets}
\PYG{k+kn}{import} \PYG{n+nn}{matplotlib}\PYG{n+nn}{.}\PYG{n+nn}{pyplot} \PYG{k}{as} \PYG{n+nn}{plt}

\PYG{o}{\PYGZpc{}}\PYG{k}{matplotlib} inline
\PYG{o}{\PYGZpc{}}\PYG{k}{config} InlineBackend.figure\PYGZus{}format = \PYGZsq{}retina\PYGZsq{}

\PYG{n}{plt}\PYG{o}{.}\PYG{n}{rcParams}\PYG{o}{.}\PYG{n}{update}\PYG{p}{(}\PYG{p}{\PYGZob{}}\PYG{l+s+s1}{\PYGZsq{}}\PYG{l+s+s1}{font.size}\PYG{l+s+s1}{\PYGZsq{}}\PYG{p}{:} \PYG{l+m+mi}{12}\PYG{p}{,}
                     \PYG{l+s+s1}{\PYGZsq{}}\PYG{l+s+s1}{axes.titlesize}\PYG{l+s+s1}{\PYGZsq{}}\PYG{p}{:} \PYG{l+m+mi}{18}\PYG{p}{,}
                     \PYG{l+s+s1}{\PYGZsq{}}\PYG{l+s+s1}{axes.labelsize}\PYG{l+s+s1}{\PYGZsq{}}\PYG{p}{:} \PYG{l+m+mi}{16}\PYG{p}{,}
                     \PYG{l+s+s1}{\PYGZsq{}}\PYG{l+s+s1}{axes.labelpad}\PYG{l+s+s1}{\PYGZsq{}}\PYG{p}{:} \PYG{l+m+mi}{14}\PYG{p}{,}
                     \PYG{l+s+s1}{\PYGZsq{}}\PYG{l+s+s1}{lines.linewidth}\PYG{l+s+s1}{\PYGZsq{}}\PYG{p}{:} \PYG{l+m+mi}{1}\PYG{p}{,}
                     \PYG{l+s+s1}{\PYGZsq{}}\PYG{l+s+s1}{lines.markersize}\PYG{l+s+s1}{\PYGZsq{}}\PYG{p}{:} \PYG{l+m+mi}{10}\PYG{p}{,}
                     \PYG{l+s+s1}{\PYGZsq{}}\PYG{l+s+s1}{xtick.labelsize}\PYG{l+s+s1}{\PYGZsq{}} \PYG{p}{:} \PYG{l+m+mi}{16}\PYG{p}{,}
                     \PYG{l+s+s1}{\PYGZsq{}}\PYG{l+s+s1}{ytick.labelsize}\PYG{l+s+s1}{\PYGZsq{}} \PYG{p}{:} \PYG{l+m+mi}{16}\PYG{p}{,}
                     \PYG{l+s+s1}{\PYGZsq{}}\PYG{l+s+s1}{xtick.top}\PYG{l+s+s1}{\PYGZsq{}} \PYG{p}{:} \PYG{k+kc}{True}\PYG{p}{,}
                     \PYG{l+s+s1}{\PYGZsq{}}\PYG{l+s+s1}{xtick.direction}\PYG{l+s+s1}{\PYGZsq{}} \PYG{p}{:} \PYG{l+s+s1}{\PYGZsq{}}\PYG{l+s+s1}{in}\PYG{l+s+s1}{\PYGZsq{}}\PYG{p}{,}
                     \PYG{l+s+s1}{\PYGZsq{}}\PYG{l+s+s1}{ytick.right}\PYG{l+s+s1}{\PYGZsq{}} \PYG{p}{:} \PYG{k+kc}{True}\PYG{p}{,}
                     \PYG{l+s+s1}{\PYGZsq{}}\PYG{l+s+s1}{ytick.direction}\PYG{l+s+s1}{\PYGZsq{}} \PYG{p}{:} \PYG{l+s+s1}{\PYGZsq{}}\PYG{l+s+s1}{in}\PYG{l+s+s1}{\PYGZsq{}}\PYG{p}{,}\PYG{p}{\PYGZcb{}}\PYG{p}{)}
\end{sphinxVerbatim}
}

\end{sphinxuseclass}
\end{sphinxuseclass}

\section{Liquid\sphinxhyphen{}Liquid Unmixing}
\label{\detokenize{notebooks/L5/1_Phase_Transitions:Liquid-Liquid-Unmixing}}
\sphinxAtStartPar
Consider two liquids consisting of molecules A and B, which we will bring together and mix. The mixing should happen at constant volume and temperature. We can therefore use the Helmholtz free energy as the quantity that is minimized.

\sphinxAtStartPar
Initially, the separated components have the free energies \(F_{\rm A}\) and \(F_{\rm B}.\) After the mixing, the free energy that derives from the mixing is \(F_{\rm A+B}-(F_{\rm A}+F_{\rm B}).\) If the volume fraction of component A is \(\phi_{\rm A}\) and the one of component B is \(\phi_{\rm B}\) such that \(\phi_{\rm A}+\phi_{\rm B}=1\) (incompressibility of the mixture), then we can calculate the mixing free energy .

\sphinxAtStartPar
First we calculate the mixing entropy \(S_{\rm mix}\) from

\sphinxAtStartPar
\begin{equation}
S=-k_{\rm B}\sum_{i}p_{i}\ln(p_{i})
\end{equation}

\sphinxAtStartPar
which results in

\sphinxAtStartPar
\begin{equation}
S_{\rm mix}=-k_{\rm B}(\phi_{\rm A}\ln(\phi_{\rm A})+\phi_{\rm B}\ln(\phi_{\rm B})).
\end{equation}

\sphinxAtStartPar
In addition to the mixing entropy, we have to calculate the change of the internal energy as well. This internal energy consists of an energy term that comprises the interaction of two molecules of the same type A (\(\epsilon_{\rm AA}\)), one of two molecules of the same type B (\(\epsilon_{\rm BB}\)), and one between the molecules A and B (\(\epsilon_{\rm AB}\)). If we assume that each site in the liquid has a number of neighbors, then the interaction energy of this site is

\sphinxAtStartPar
\begin{equation}
z\phi_{\rm A}\epsilon_{\rm AA}.
\end{equation}

\sphinxAtStartPar
A fraction of \(\phi_{\rm A}\) sites is occupied with molecules A and, thus, the interaction energy reads

\sphinxAtStartPar
\begin{equation}
\frac{z}{2}\phi_{\rm A}^2\epsilon_{\rm AA}.
\end{equation}

\sphinxAtStartPar
Similarly, expressions for the interaction of B molecules and the interaction of A with B can be obtained. To calculate the change in internal energy when mixing the two species, we still have to subtract the internal energy of the two separated components, \(z\phi_{\rm A}\epsilon_{\rm AA}/2\) and \(z\phi_{\rm B}\epsilon_{\rm BB}/2\), such that we obtain

\sphinxAtStartPar
\begin{equation}
U_{\rm mix}=\frac{z}{2}\left [(\phi_{\rm A}^2-\phi_{\rm A})\epsilon_{\rm AA}+(\phi_{\rm B}^2-\phi_{\rm B})\epsilon_{\rm BB}+2\phi_{\rm A}\phi_{\rm B}\epsilon_{\rm AB}\right].
\end{equation}

\sphinxAtStartPar
Using this expression as well as \(\phi_{\rm A}+\phi_{\rm B}=1\), we can define an interaction parameter

\sphinxAtStartPar
\begin{equation}
\chi=\frac{z}{2k_{\rm B}T}(2\epsilon_{\rm AB}-\epsilon_{\rm AA}-\epsilon_{\rm BB})
\end{equation}

\sphinxAtStartPar
such that

\sphinxAtStartPar
\begin{equation}
\frac{U_{\rm mix}}{k_{\rm B}T}=\chi\phi_{\rm A}\phi_{\rm B}
\end{equation}

\sphinxAtStartPar
by utilizing, for example, \((\phi_{\rm A}^2-\phi_{\rm A})\epsilon_{\rm AA}=\phi_{\rm A}(\phi_{\rm A}-1)\epsilon_{\rm AA}=-\phi_{\rm A}\phi_{\rm B}\epsilon_{\rm AA}\).

\sphinxAtStartPar
This finally yields the free energy of mixing

\sphinxAtStartPar
\begin{equation}
\frac{F_{\rm mix}}{k_{\rm B}T}=\phi_{\rm A}\ln(\phi_{\rm A})+\phi_{\rm B}\ln(\phi_{\rm B})+\chi\phi_{\rm A}\phi_{\rm B}.
\end{equation}

\sphinxAtStartPar
We can now plot this free energy of mixing as a function of the volume fraction of component A.

\begin{sphinxuseclass}{nbinput}
\begin{sphinxuseclass}{nblast}
{
\sphinxsetup{VerbatimColor={named}{nbsphinx-code-bg}}
\sphinxsetup{VerbatimBorderColor={named}{nbsphinx-code-border}}
\begin{sphinxVerbatim}[commandchars=\\\{\}]
\llap{\color{nbsphinxin}[6]:\,\hspace{\fboxrule}\hspace{\fboxsep}}\PYG{c+c1}{\PYGZsh{} define volume fraction variable}
\PYG{n}{phi}\PYG{o}{=}\PYG{n}{np}\PYG{o}{.}\PYG{n}{linspace}\PYG{p}{(}\PYG{l+m+mf}{0.01}\PYG{p}{,}\PYG{l+m+mf}{0.99}\PYG{p}{,}\PYG{l+m+mi}{100}\PYG{p}{)}
\end{sphinxVerbatim}
}

\end{sphinxuseclass}
\end{sphinxuseclass}
\begin{sphinxuseclass}{nbinput}
\begin{sphinxuseclass}{nblast}
{
\sphinxsetup{VerbatimColor={named}{nbsphinx-code-bg}}
\sphinxsetup{VerbatimBorderColor={named}{nbsphinx-code-border}}
\begin{sphinxVerbatim}[commandchars=\\\{\}]
\llap{\color{nbsphinxin}[7]:\,\hspace{\fboxrule}\hspace{\fboxsep}}\PYG{c+c1}{\PYGZsh{} calculate mixing free energy}
\PYG{k}{def} \PYG{n+nf}{mix}\PYG{p}{(}\PYG{n}{x}\PYG{p}{)}\PYG{p}{:}
    \PYG{n}{m}\PYG{o}{=}\PYG{n}{phi}\PYG{o}{*}\PYG{n}{np}\PYG{o}{.}\PYG{n}{log}\PYG{p}{(}\PYG{n}{phi}\PYG{p}{)}\PYG{o}{+}\PYG{p}{(}\PYG{l+m+mi}{1}\PYG{o}{\PYGZhy{}}\PYG{n}{phi}\PYG{p}{)}\PYG{o}{*}\PYG{n}{np}\PYG{o}{.}\PYG{n}{log}\PYG{p}{(}\PYG{l+m+mi}{1}\PYG{o}{\PYGZhy{}}\PYG{n}{phi}\PYG{p}{)}\PYG{o}{+}\PYG{n}{x}\PYG{o}{*}\PYG{n}{phi}\PYG{o}{*}\PYG{p}{(}\PYG{l+m+mi}{1}\PYG{o}{\PYGZhy{}}\PYG{n}{phi}\PYG{p}{)}
    \PYG{n}{plt}\PYG{o}{.}\PYG{n}{plot}\PYG{p}{(}\PYG{n}{phi}\PYG{p}{,}\PYG{n}{m}\PYG{p}{)}
    \PYG{n}{plt}\PYG{o}{.}\PYG{n}{ylabel}\PYG{p}{(}\PYG{l+s+sa}{r}\PYG{l+s+s2}{\PYGZdq{}}\PYG{l+s+s2}{\PYGZdl{}}\PYG{l+s+s2}{\PYGZbs{}}\PYG{l+s+s2}{frac}\PYG{l+s+s2}{\PYGZob{}}\PYG{l+s+s2}{F\PYGZus{}}\PYG{l+s+s2}{\PYGZob{}}\PYG{l+s+s2}{\PYGZbs{}}\PYG{l+s+s2}{rm mix\PYGZcb{}\PYGZcb{}}\PYG{l+s+s2}{\PYGZob{}}\PYG{l+s+s2}{k\PYGZus{}B T\PYGZcb{}\PYGZdl{}}\PYG{l+s+s2}{\PYGZdq{}}\PYG{p}{)}
    \PYG{n}{plt}\PYG{o}{.}\PYG{n}{ylabel}\PYG{p}{(}\PYG{l+s+sa}{r}\PYG{l+s+s2}{\PYGZdq{}}\PYG{l+s+s2}{volume fraction \PYGZdl{}}\PYG{l+s+s2}{\PYGZbs{}}\PYG{l+s+s2}{phi\PYGZdl{}}\PYG{l+s+s2}{\PYGZdq{}}\PYG{p}{)}
    \PYG{n}{plt}\PYG{o}{.}\PYG{n}{show}\PYG{p}{(}\PYG{p}{)}

\end{sphinxVerbatim}
}

\end{sphinxuseclass}
\end{sphinxuseclass}
\sphinxAtStartPar
\sphinxstyleemphasis{Free energy of mixing as a function of the composition of the binary liquid mixture for different interaction parameters.}

\begin{sphinxuseclass}{nbinput}
{
\sphinxsetup{VerbatimColor={named}{nbsphinx-code-bg}}
\sphinxsetup{VerbatimBorderColor={named}{nbsphinx-code-border}}
\begin{sphinxVerbatim}[commandchars=\\\{\}]
\llap{\color{nbsphinxin}[8]:\,\hspace{\fboxrule}\hspace{\fboxsep}}\PYG{c+c1}{\PYGZsh{} display interaction slider}
\PYG{n}{interact}\PYG{p}{(}\PYG{n}{mix}\PYG{p}{,} \PYG{n}{x}\PYG{o}{=}\PYG{n}{widgets}\PYG{o}{.}\PYG{n}{FloatSlider}\PYG{p}{(}\PYG{n+nb}{min}\PYG{o}{=}\PYG{o}{\PYGZhy{}}\PYG{l+m+mi}{1}\PYG{p}{,} \PYG{n+nb}{max}\PYG{o}{=}\PYG{l+m+mi}{5}\PYG{p}{,} \PYG{n}{step}\PYG{o}{=}\PYG{l+m+mf}{0.1}\PYG{p}{)}\PYG{p}{)}\PYG{p}{;}
\end{sphinxVerbatim}
}

\end{sphinxuseclass}
\begin{sphinxuseclass}{nboutput}
\begin{sphinxuseclass}{nblast}
{

\kern-\sphinxverbatimsmallskipamount\kern-\baselineskip
\kern+\FrameHeightAdjust\kern-\fboxrule
\vspace{\nbsphinxcodecellspacing}

\sphinxsetup{VerbatimColor={named}{white}}
\sphinxsetup{VerbatimBorderColor={named}{nbsphinx-code-border}}
\begin{sphinxuseclass}{output_area}
\begin{sphinxuseclass}{}


\begin{sphinxVerbatim}[commandchars=\\\{\}]
interactive(children=(FloatSlider(value=0.0, description='x', max=5.0, min=-1.0), Output()), \_dom\_classes=('wi…
\end{sphinxVerbatim}



\end{sphinxuseclass}
\end{sphinxuseclass}
}

\end{sphinxuseclass}
\end{sphinxuseclass}
\sphinxAtStartPar
The free energy function has a minimum at a composition of \(\phi=0.5\) for all \(\chi\) smaller than 2. In this region of interaction parameters, the two liquids are always miscible at any composition. For \(\chi\) larger than two, the free energy function reveals two minima, i.e., there are two compositions under which the mixture is stable (one left in phase B and one right in A). The location of these two minima defines what is called the \sphinxstylestrong{binodal} or \sphinxstylestrong{coexistence curve},
which separates the unstable from the stable region in a phase diagram (Figure 4). As the interaction parameter is proportional to \(1/T\), these stable compositions are a function of the temperature.

\sphinxAtStartPar
To understand the stability of a mixture, let us shorten the notation first. In all of the following considerations, we will denote \(\phi=\phi_{\rm A}.\) We will start with two mixtures we have prepared, which have a composition \(\phi_1\) and \(\phi_2\). If we take a volume \(V_1\) of the first mixture and a volume \(V_2\) of the second one and combine them to a volume \(V_0\), then the two mixtures form again a new composition with

\sphinxAtStartPar
\begin{equation}
\phi_{0}V_{0}=\phi_{1}V_{1}+\phi_{2}V_{2}.
\end{equation}

\sphinxAtStartPar
We can divide by \(V_0\) and introduce two new volume fractions \(\alpha_1\), \(\alpha_2\) to yield

\sphinxAtStartPar
\begin{equation}
\phi_{0}=\frac{V_{1}}{V_{0}}\phi_{1}+\frac{V_{2}}{V_{0}}\phi_{1}=\alpha_{1}\phi_{1}+\alpha_{2}\phi_{2}.
\end{equation}

\sphinxAtStartPar
The total free energy of the system of the two separated mixtures at these volume fractions can then be written as

\sphinxAtStartPar
\begin{equation}
F_{\rm sep}=\alpha_{1}F_{\rm mix}(\phi_{1})+\alpha_{2}F_{\rm mix}(\phi_{2}).
\end{equation}

\sphinxAtStartPar
We can add this free energy as a function of the volume fraction \(\phi_0\) to our free energy curve, for example, for an interaction parameter \(\chi<2\). The two initial concentrations lie on the coexistence curve, but the starting composition \(\phi_0\) is not. Its initial free energy is \(F_{\rm sep}(\phi_{0})\) and it is on a line connecting the points with the initial compositions, \sphinxhref{PhaseTr\_2.png}{Figure 2}. Since the actual free energy curve of the mixture is always
lower than any point of the line denoting the separate free energies, the mixture is \sphinxstylestrong{stable}. This is true as long as the free energy curve is convex.

\sphinxAtStartPar
\sphinxincludegraphics[width=400\sphinxpxdimen]{{PhaseTr_2}.png}

\sphinxAtStartPar
Lets change the free energy curve to a situation with a maximum at \(\phi=0.5\). If we look at two situations, where one is in the region of the maximum and the second is next to the minimum (\sphinxstylestrong{???}). Close to the maximum, the line with the free energy of the separate components always falls below the free energy curve of the mixture. The composition is therefore \sphinxstylestrong{unstable}. Close to the minimum, however, the composition is stable, yet, the mixture is not at the minimum of the free energy
curve. Eventually, the system will evolve towards the free energy minimum and thus the mixture is called \sphinxstylestrong{metastable}.

\noindent\sphinxincludegraphics[width=420\sphinxpxdimen,height=315\sphinxpxdimen]{{29573a33c89b3ee48bdb022463032c9cafcc1d57}.png}

\sphinxAtStartPar
Apparently the above analysis has done nothing else than measuring the sign of the curvature of the free energy curve. We can summarize the conditions as

\sphinxAtStartPar
\(\begin{eqnarray} \frac{\mathrm d^2F_{\rm mix}}{\mathrm d\phi^2}&>&0\ldots {\rm stable}\\ \frac{\mathrm d^2F_{\rm mix}}{\mathrm d\phi^2}&<&0\ldots {\rm unstable}\\ \frac{\mathrm d^2F_{\rm mix}}{\mathrm d\phi^2}&=&0\ldots {\rm spinodal\, line}\\ \frac{\mathrm d^3F_{\rm mix}}{\mathrm d\phi^3}&=&0\ldots {\rm critical\, point} \end{eqnarray}\)

\sphinxAtStartPar
The spinodal line therefore separates the metastable from the unstable region. It corresponds to the deflection points of the free energy curve. The critical point, beyond which no phase separation can be achieved at any concentration \(\phi\) is defined by setting the third derivative to zero.

\noindent\sphinxincludegraphics[width=420\sphinxpxdimen,height=315\sphinxpxdimen]{{4283678890bdc0da5341752e00a7bba222dbe4b7}.png}

\sphinxAtStartPar
\sphinxstyleemphasis{Phase diagram of an ideal mixture. The interaction parameter is plotted as a function of the composition. The individual regions as defined by the above equation are indicated.}

\sphinxAtStartPar
We can now plot this free energy of mixing as a function of the volume fraction of component A.



\sphinxAtStartPar
The following section was created from \sphinxcode{\sphinxupquote{notebooks/L5/2\_Kinetics\_LL\_Unmixing.ipynb}}.


\chapter{Kinetics of Liquid\textendash{}Liquid Unmixing}
\label{\detokenize{notebooks/L5/2_Kinetics_LL_Unmixing:Kinetics-of-Liquid_Liquid-Unmixing}}\label{\detokenize{notebooks/L5/2_Kinetics_LL_Unmixing::doc}}
\sphinxAtStartPar
If two liquids are brought together at a composition that is either unstable or metastable, they will phase separate. The kinetics of this phase separation can proceed in different ways.
\begin{enumerate}
\sphinxsetlistlabels{\arabic}{enumi}{enumii}{}{.}%
\item {} 
\sphinxAtStartPar
\sphinxstylestrong{Spinodal Decomposition} \textendash{} In this case, thermal fluctuations will amplify density fluctuations. The material will diffuse “uphill” to the regions of higher concentration to yield phase separation. The boundaries of the two phases are blurred in this case. Spinodal decomposition is the process of phase separation in the unstable region.

\item {} 
\sphinxAtStartPar
\sphinxstylestrong{Nucleation} \textendash{} In the case of nucleation, a thermal fluctuation will create a nucleus of a critical size, which further grows in time. The nucleus has a sharp boundary with the other phase. Nucleation is the process of phase separation in the metastable region.

\end{enumerate}

\sphinxAtStartPar
We will treat both kinetic regimes separately.


\section{Spinodal decomposition}
\label{\detokenize{notebooks/L5/2_Kinetics_LL_Unmixing:Spinodal-decomposition}}
\sphinxAtStartPar
Based on Fick’s law, we know that the diffusion currents occur in the direction against a composition gradient, i.e., \(J=-D∇\phi\). The result of that diffusion process is an equal composition in the whole sample. For phase separation to happen, material needs to flow from regions of low concentration to regions of higher concentration, so in the direction of the composition gradient. Material transport is therefore not balancing composition gradients. In the section about the chemical
potential, we learned that in an open system the chemical potential of all species is balanced. The chemical potential is given by

\sphinxAtStartPar
\begin{equation}
\mu=\frac{\mathrm dF}{\mathrm d\phi}|_{V,T}.
\end{equation}

\sphinxAtStartPar
If now the second derivative of the free energy with respect to \(\phi\) is positive, the regions with larger \(\phi\) have larger free energy. Thus regions of lower concentration are favored by the free energy and the diffusion happens to be downhill as given also by Ficks law. If the second derivate of the free energy is, however, negative, then fluctuations in the composition will lower the free energy and, thus, the fluctuations will grow in strength. The fluctuations with different
length scales will, however, not grow at the same rate. The spectrum of growth rates can be described by a phenomenological theory known as the Cahn\textendash{}Hilliard equation, which is applied in many fields. To arrive at that equation, we need to describe the free energy in terms of composition gradients

\sphinxAtStartPar
\begin{equation}
F=A\int \left [f_{0}(\phi)+\kappa \left( \frac{\mathrm d\phi}{\mathrm dx}\right)^2\right]\mathrm dx.
\end{equation}

\sphinxAtStartPar
Here, \(f_0\) is the free energy density, \(A\) an area and \(x\) a linear coordinate. To balance the chemical potentials we need to modify Ficks laws

\sphinxAtStartPar
\begin{eqnarray}
J &=& -D\frac{\mathrm d\phi}{\mathrm dx},\\ \frac{\mathrm d\phi}{\mathrm dt} &=& -\frac{\mathrm dJ}{\mathrm dx},\\ \frac{\partial\phi}{\partial t} &=& D\frac{\partial^{2}\phi}{\partial x^2}
\end{eqnarray}

\sphinxAtStartPar
to an appropriate new law

\sphinxAtStartPar
\begin{equation}
J=-M \frac{\mathrm d\mu}{\mathrm dx},
\end{equation}

\sphinxAtStartPar
where \(\mu=\mu_{\rm A}-\mu_{\rm B}\) is the difference between the chemical potentials of \(\mathrm{A}\) and \(\mathrm{B}\). This is apparently a linear approximation. According to the definition of the chemical potential, we obtain

\sphinxAtStartPar
\begin{equation}
\mu=\frac{\mathrm d}{\mathrm d\phi}\left [f_{0}(\phi)+\kappa \left( \frac{\mathrm d\phi}{\mathrm dx}\right)^2\right]
\end{equation}

\sphinxAtStartPar
from which follows

\sphinxAtStartPar
\begin{equation}
\mu=\frac{\mathrm{d}f_{0}}{\mathrm d\phi}+2\kappa\frac{\mathrm d^2\phi}{\mathrm dx^2},
\end{equation}

\sphinxAtStartPar
where it was applied that

\sphinxAtStartPar
\begin{equation}
\frac{\mathrm{d}}{\mathrm{d}\phi}\bigg(\frac{\mathrm{d}\phi}{\mathrm{d}x}\bigg)^2=\frac{\mathrm{d}x}{\mathrm{d}\phi}\frac{\mathrm{d}}{\mathrm{d}x}\bigg(\frac{\mathrm{d}\phi}{\mathrm{d}x}\bigg)^2=\frac{\mathrm{d}x}{\mathrm{d}\phi}\cdot \bigg(2\frac{\mathrm{d}\phi}{\mathrm{d}x}\bigg)\cdot\frac{\mathrm{d}}{\mathrm{d}x}\frac{\mathrm{d}\phi}{\mathrm{d}x}=2\frac{\mathrm{d}^2\phi}{\mathrm{d}x^2}.
\end{equation}

\sphinxAtStartPar
Inserting this into our modified Ficks law yields

\sphinxAtStartPar
\begin{equation}
J=-M f_{0}^{''}\frac{\mathrm d\phi}{\mathrm dx}-2M\kappa \frac{\mathrm d\phi^{3}}{\mathrm dx^3}
\end{equation}

\sphinxAtStartPar
with \(f_0^″=(\mathrm{d}^2 f_0)/(\mathrm{d}ϕ^2)\). Finally, inserting the above equation into the continuity equation yields

\sphinxAtStartPar
\begin{equation}
\frac{\partial \phi}{\partial t}=M f_{0}^{''}\frac{\partial ^2\phi}{\partial x^2}+2M\kappa \frac{\partial^4 \phi}{\partial x^4}.
\end{equation}

\sphinxAtStartPar
This is the \sphinxstylestrong{Cahn\textendash{}Hilliard} equation, which extends the diffusion equation for interacting species. The term in front of the second derivative yields an effective diffusion coefficient, i.e., \(D_{\rm eff}=M f_{0}^{''}\). While the coefficient \(M\) (the Onsager coefficient) is always positive, the second derivative of the free energy density \(f_0\) can be positive or negative. Inside the spinodal line, for example, the curvature of the free energy is negative. Following that, the
effective diffusion coefficient is negative which allows for our uphill diffusion. The Cahn\textendash{}Hilliard equation can be solved by a Fourier decomposition which yields

\sphinxAtStartPar
\begin{equation}
\phi(x,t)=\phi_{0}+A\cos(qx)\exp\left [ -D_{\rm eff} q^{2}\left (1+\frac{2\kappa q^{2}}{f_{0}^{''}}\right ) t\right ].
\end{equation}

\sphinxAtStartPar
Here \(q\) is a wavenumber defining the wavelength of the fluctuations. The exponent can be abbreviated by

\sphinxAtStartPar
\begin{equation}
R(q)= -D_{\rm eff} q^{2}\left (1+\frac{2\kappa q^{2}}{f_{0}^{''}}\right )
\end{equation}

\sphinxAtStartPar
which is nothing else as an inverse time or a rate coefficient. It provides the relaxation rate for a specific wavenumber. This rate coefficient is positive up to a certain wavenumber \(q_0\). All fluctuations with wavelengths larger than \(λ_0=2π/q_0\) are amplified in time. The fastest\sphinxhyphen{}growing fluctuation is obtained at \(q_{\rm max}\). This wavenumber sets the length scale of the structures, which typically appear during the spinodal decomposition.


\section{Nucleation}
\label{\detokenize{notebooks/L5/2_Kinetics_LL_Unmixing:Nucleation}}
\sphinxAtStartPar
Nucleation is the phenomenon of spontaneously generating a phase\sphinxhyphen{}separated volume by thermal fluctuations. By doing that, we create an interface between the two phases, which costs energy. We, therefore, have a first simple look at the interfacial energy.


\subsection{Interfacial Tension}
\label{\detokenize{notebooks/L5/2_Kinetics_LL_Unmixing:Interfacial-Tension}}
\sphinxAtStartPar
If we have a system of two components A and B and those components do not mix but form an interface, then, due to the fact that both sorts of molecules repel each other, we have to spend energy to bring them together. The work done to create an interfacial area is

\sphinxAtStartPar
\begin{equation}
W=F \Delta x=\gamma L \Delta x,
\end{equation}

\sphinxAtStartPar
where \(γ\) denotes the interfacial energy. If work is done, this would require an energy exchange with a reservoir to keep the temperature constant. That, however, also means that the interfacial energy is a free energy per unit area rather than an internal energy. Interfacial energies may contain entropic contributions. In the case of a sharp interface, we can use the interaction parameter \(χ\) and the molecular volume \(v\) to show

\sphinxAtStartPar
\begin{equation}
\gamma=\frac{1}{2v^{2/3}}(2\epsilon_{\rm AB}-\epsilon_{\rm AA}-\epsilon_{\rm BB})=\frac{\chi k_{\rm B}T}{zv^{2/3}}.
\end{equation}


\section{Nucleation}
\label{\detokenize{notebooks/L5/2_Kinetics_LL_Unmixing:nucleation-1}}\label{\detokenize{notebooks/L5/2_Kinetics_LL_Unmixing:id1}}
\sphinxAtStartPar
If we have a liquid mixture now at a certain composition, which is in the metastable region, i.e., \((∂^2 F)/(∂ϕ^2 )>0\), all of the density fluctuations in our system are damped and the spinodal decomposition can not happen. Therefore, the system has to phase separate by nucleation, i.e., that spontaneously a droplet of a certain size is formed and grows further. As we need energy to create the interface but also gain free energy by forming a volume phase, we need to have a look at the
total free energy change, when the droplet is formed.
\begin{itemize}
\item {} 
\sphinxAtStartPar
the volume part is delivering a free energy gain per volume of \(ΔF_v<0\), which has to be multiplied with the volume of the droplet.

\item {} 
\sphinxAtStartPar
the creation of an interface requires interfacial energy \(γ\), which is always positive and has to be multiplied with the droplet surface area.

\end{itemize}

\sphinxAtStartPar
Thus, the total free energy change reads

\sphinxAtStartPar
\begin{equation}
\Delta F(r)=\frac{4}{3}\pi r^3\Delta F_{v}+4\pi r^2 \gamma.
\end{equation}

\sphinxAtStartPar
Note, that the first term is negative as \(ΔF_v\) is negative. A typical free energy change with the particle size is shown in the figure below. It has a maximum at the position \(r^*\),which creates a nucleation barrier \(ΔG^*\). Droplets which are spontaneously formed at \(r<r^*\) can decrease their free energy by dissolving.

\noindent\sphinxincludegraphics[width=420\sphinxpxdimen,height=315\sphinxpxdimen]{{58eac6b9f0ed721532be595d1a9556f74a979945}.png}

\begin{sphinxadmonition}{note}{}\unskip
\sphinxAtStartPar
\sphinxstylestrong{Liquid\sphinxhyphen{}Liquid Unmixing \sphinxhyphen{} Ouzo Effect}

\sphinxAtStartPar
One example of liquid\sphinxhyphen{}liquid unmixing by nucleation is the Ouzo or Louche effect. The name denotes the spontaneous formation of a turbid microemulsion when a solution of anethol in ethanol is diluted with water. The figure below shows the phase digram of this ternary mixture.

\noindent\sphinxincludegraphics[width=591\sphinxpxdimen,height=414\sphinxpxdimen]{{ouzo}.png}

\sphinxAtStartPar
When the emulsification happens, the solution rapidly moves to the metastable region between the binodal and spinodal lines. It occurs because the alcohol, as it diffuses from the anethol into the water, carries with it some of the anethol molecules that still have a finite solubility in alcohol and alcohol/water mixtures. As the alcohol diffuses further into the water, the associating anethol(oil) molecules become expelled from the water\sphinxhyphen{}rich solution and are stranded in the form of fine
emulsion droplets. Thus, adding water to a solution of homogeneous oil/ethanol mixtures leads to an abrupt decrease of the solubility of oil in the water\sphinxhyphen{}rich continuous phase. This effect causes the strong local concentration fluctuations of solute molecules and homogeneous nucleation can start.

\sphinxAtStartPar
The image below shows the turbidity of the emulsion to appear as the water content is changed.

\noindent\sphinxincludegraphics[width=514\sphinxpxdimen,height=161\sphinxpxdimen]{{ouzo_1}.png}

\sphinxAtStartPar
The data has been taken from: Sitnikova, N. L., Sprik, R., Wegdam, G. \& Eiser, E. Spontaneously Formed trans\sphinxhyphen{}Anethol/Water/Alcohol Emulsions: Mechanism of Formation and Stability. Langmuir 21, 7083\textendash{}7089 (2005).
\end{sphinxadmonition}



\sphinxAtStartPar
The following section was created from \sphinxcode{\sphinxupquote{notebooks/L6/1\_Solid\_Liquid\_Phase\_Transitions.ipynb}}.


\chapter{Solid\textendash{}Liquid Phase Transitions}
\label{\detokenize{notebooks/L6/1_Solid_Liquid_Phase_Transitions:Solid_Liquid-Phase-Transitions}}\label{\detokenize{notebooks/L6/1_Solid_Liquid_Phase_Transitions::doc}}
\begin{sphinxuseclass}{nbinput}
\begin{sphinxuseclass}{nblast}
{
\sphinxsetup{VerbatimColor={named}{nbsphinx-code-bg}}
\sphinxsetup{VerbatimBorderColor={named}{nbsphinx-code-border}}
\begin{sphinxVerbatim}[commandchars=\\\{\}]
\llap{\color{nbsphinxin}[16]:\,\hspace{\fboxrule}\hspace{\fboxsep}}\PYG{k+kn}{import} \PYG{n+nn}{numpy} \PYG{k}{as} \PYG{n+nn}{np}
\PYG{k+kn}{import} \PYG{n+nn}{matplotlib}\PYG{n+nn}{.}\PYG{n+nn}{pyplot} \PYG{k}{as} \PYG{n+nn}{plt}

\PYG{o}{\PYGZpc{}}\PYG{k}{config} InlineBackend.figure\PYGZus{}format = \PYGZsq{}retina\PYGZsq{}
\PYG{c+c1}{\PYGZsh{} the lines below set a number of parameters for plotting, such as label font size,}
\PYG{c+c1}{\PYGZsh{} title font size, which you may find useful}
\PYG{n}{plt}\PYG{o}{.}\PYG{n}{rcParams}\PYG{o}{.}\PYG{n}{update}\PYG{p}{(}\PYG{p}{\PYGZob{}}\PYG{l+s+s1}{\PYGZsq{}}\PYG{l+s+s1}{font.size}\PYG{l+s+s1}{\PYGZsq{}}\PYG{p}{:} \PYG{l+m+mi}{14}\PYG{p}{,}
                     \PYG{l+s+s1}{\PYGZsq{}}\PYG{l+s+s1}{font.family}\PYG{l+s+s1}{\PYGZsq{}}\PYG{p}{:}\PYG{l+s+s1}{\PYGZsq{}}\PYG{l+s+s1}{sans\PYGZhy{}serif}\PYG{l+s+s1}{\PYGZsq{}}\PYG{p}{,}
                     \PYG{l+s+s1}{\PYGZsq{}}\PYG{l+s+s1}{axes.titlesize}\PYG{l+s+s1}{\PYGZsq{}}\PYG{p}{:} \PYG{l+m+mi}{16}\PYG{p}{,}
                     \PYG{l+s+s1}{\PYGZsq{}}\PYG{l+s+s1}{axes.labelsize}\PYG{l+s+s1}{\PYGZsq{}}\PYG{p}{:} \PYG{l+m+mi}{18}\PYG{p}{,}
                     \PYG{l+s+s1}{\PYGZsq{}}\PYG{l+s+s1}{axes.labelpad}\PYG{l+s+s1}{\PYGZsq{}}\PYG{p}{:} \PYG{l+m+mi}{14}\PYG{p}{,}
                     \PYG{l+s+s1}{\PYGZsq{}}\PYG{l+s+s1}{lines.linewidth}\PYG{l+s+s1}{\PYGZsq{}}\PYG{p}{:} \PYG{l+m+mi}{1}\PYG{p}{,}
                     \PYG{l+s+s1}{\PYGZsq{}}\PYG{l+s+s1}{lines.markersize}\PYG{l+s+s1}{\PYGZsq{}}\PYG{p}{:} \PYG{l+m+mi}{10}\PYG{p}{,}
                     \PYG{l+s+s1}{\PYGZsq{}}\PYG{l+s+s1}{xtick.labelsize}\PYG{l+s+s1}{\PYGZsq{}} \PYG{p}{:} \PYG{l+m+mi}{18}\PYG{p}{,}
                     \PYG{l+s+s1}{\PYGZsq{}}\PYG{l+s+s1}{ytick.labelsize}\PYG{l+s+s1}{\PYGZsq{}} \PYG{p}{:} \PYG{l+m+mi}{18}\PYG{p}{,}
                     \PYG{l+s+s1}{\PYGZsq{}}\PYG{l+s+s1}{xtick.top}\PYG{l+s+s1}{\PYGZsq{}} \PYG{p}{:} \PYG{k+kc}{True}\PYG{p}{,}
                     \PYG{l+s+s1}{\PYGZsq{}}\PYG{l+s+s1}{xtick.direction}\PYG{l+s+s1}{\PYGZsq{}} \PYG{p}{:} \PYG{l+s+s1}{\PYGZsq{}}\PYG{l+s+s1}{in}\PYG{l+s+s1}{\PYGZsq{}}\PYG{p}{,}
                     \PYG{l+s+s1}{\PYGZsq{}}\PYG{l+s+s1}{ytick.right}\PYG{l+s+s1}{\PYGZsq{}} \PYG{p}{:} \PYG{k+kc}{True}\PYG{p}{,}
                     \PYG{l+s+s1}{\PYGZsq{}}\PYG{l+s+s1}{ytick.direction}\PYG{l+s+s1}{\PYGZsq{}} \PYG{p}{:} \PYG{l+s+s1}{\PYGZsq{}}\PYG{l+s+s1}{in}\PYG{l+s+s1}{\PYGZsq{}}\PYG{p}{,}\PYG{p}{\PYGZcb{}}\PYG{p}{)}
\end{sphinxVerbatim}
}

\end{sphinxuseclass}
\end{sphinxuseclass}
\sphinxAtStartPar
When a liquid is cooled down and reaches the (freezing) melting temperature, the liquid may enter the solid phase. Directly at the melting temperature, the free energy of the solid and the liquid are the same (also the chemical potentials). When forming a crystal, the system needs to spend energy on creating the interface between liquid and solid, which requires a surface free energy to come from somewhere. The transition from the liquid to the solid, however, is not freeing any energy. As a
result, the liquid cannot freeze directly at the freezing temperature. To freeze the liquid, undercooling is required. An undercooling by a temperature \(ΔT\) would create a free energy difference \(ΔG\) between liquid and solid.

\noindent\sphinxincludegraphics[width=420\sphinxpxdimen,height=315\sphinxpxdimen]{{freezing}.png}

\sphinxAtStartPar
\sphinxstyleemphasis{Free energy of solid and liquid phase of a material as a function of temperature}


\section{Kinetics of the Liquid\textendash{}Solid Phase Transition}
\label{\detokenize{notebooks/L6/1_Solid_Liquid_Phase_Transitions:Kinetics-of-the-Liquid_Solid-Phase-Transition}}
\sphinxAtStartPar
To start the phase transition from a liquid to a solid in the undercooled liquid, a nucleus of a certain size has to appear from thermal fluctuations. This goes along the lines we discussed for the liquid\sphinxhyphen{}liquid phase transitions. To create a nucleus of a radius \(r\) we need a free energy change

\sphinxAtStartPar
\begin{equation}
\Delta G= \frac{4}{3}\pi r^3\Delta G_\mathrm{b}+4\pi r^2\gamma_{\rm sl},
\end{equation}

\sphinxAtStartPar
where \(ΔG_\mathrm{b}\) is the free energy change per unit volume, when creating the volume phase of the solid material. \(γ_{\rm sl}\) denotes the interfacial free energy for the solid/liquid contact. We also know that at a first\sphinxhyphen{}order phase transition all energy inserted into the system (the latent heat \(ΔH_\mathrm{m}\) at the melting temperature \(T_\mathrm{m}\)) is going into a change in the entropy of the system, i.e.,

\sphinxAtStartPar
\begin{equation}
\Delta S_\mathrm{m}=\left (\frac{\partial G_{\rm s}}{\partial T}\right )_{p}-\left (\frac{\partial G_{\rm l}}{\partial T}\right )_{p}=\frac{\Delta H_\mathrm{m}}{T_\mathrm{m}}.
\end{equation}

\sphinxAtStartPar
We can, therefore, extrapolate the free energy change at the undercooled temperature by

\sphinxAtStartPar
\begin{equation}
\Delta G_{\rm b}=-\frac{\Delta H_{\rm m}}{T_{\rm m}}\Delta T
\end{equation}

\sphinxAtStartPar
which yields

\sphinxAtStartPar
\begin{equation}
\Delta G= -\frac{4}{3}\pi r^3\frac{\Delta H_{\rm m}}{T_{\rm m}}\Delta T +4\pi r^2\gamma_{\rm sl}.
\end{equation}

\sphinxAtStartPar
According to our nucleation theory, this yields a critical radius of the nucleus of

\sphinxAtStartPar
\begin{equation}
r^{*}=\frac{2\gamma_{\rm sl}T_{\rm m}}{\Delta H_{\rm m}\Delta T}.
\end{equation}

\sphinxAtStartPar
A nucleus of size \(r<r^*\) would thus have the tendency to dissolve again, while a nucleus of size \(r>r^*\) is ready to grow continuously. The critical radius of the nucleus corresponds to a free energy barrier

\sphinxAtStartPar
\begin{equation}
\Delta G^{*}=\frac{16\pi}{3}\gamma_{\rm sl}^3\left (\frac{T_{\rm m}}{\Delta H_{\rm m}}\right )^2\frac{1}{\Delta T^2}.
\end{equation}

\sphinxAtStartPar
The probability of critical nucleus formation can thus be determined from the Boltzmann factor

\sphinxAtStartPar
\begin{equation}
e^{-\frac{\Delta G^{*}}{k_{\rm B}T}}
\end{equation}

\sphinxAtStartPar
which is a very strong function of temperature. Inserting typical values, e.g., for water/ice we find the following free energies as a function of the nucleus radius (\(\Delta T= 10\, {\rm K}\)).

\begin{sphinxuseclass}{nbinput}
\begin{sphinxuseclass}{nblast}
{
\sphinxsetup{VerbatimColor={named}{nbsphinx-code-bg}}
\sphinxsetup{VerbatimBorderColor={named}{nbsphinx-code-border}}
\begin{sphinxVerbatim}[commandchars=\\\{\}]
\llap{\color{nbsphinxin}[3]:\,\hspace{\fboxrule}\hspace{\fboxsep}}\PYG{k}{def} \PYG{n+nf}{dG}\PYG{p}{(}\PYG{n}{r}\PYG{p}{,}\PYG{n}{dT}\PYG{p}{)}\PYG{p}{:}
    \PYG{n}{gamma\PYGZus{}sl}\PYG{o}{=}\PYG{l+m+mf}{38e\PYGZhy{}3} \PYG{c+c1}{\PYGZsh{}(38mJ/m\PYGZca{}2)}
    \PYG{n}{Tm}\PYG{o}{=}\PYG{l+m+mf}{273.15}
    \PYG{n}{dHm}\PYG{o}{=}\PYG{l+m+mf}{3.34e8} \PYG{c+c1}{\PYGZsh{} (J/m\PYGZca{}3)}
    \PYG{k}{return}\PYG{p}{(}\PYG{o}{\PYGZhy{}}\PYG{l+m+mi}{4}\PYG{o}{*}\PYG{n}{np}\PYG{o}{.}\PYG{n}{pi}\PYG{o}{*}\PYG{n}{r}\PYG{o}{*}\PYG{o}{*}\PYG{l+m+mi}{3}\PYG{o}{*}\PYG{n}{dHm}\PYG{o}{*}\PYG{n}{dT}\PYG{o}{/}\PYG{n}{Tm}\PYG{o}{/}\PYG{l+m+mi}{3}\PYG{o}{+}\PYG{l+m+mi}{4}\PYG{o}{*}\PYG{n}{np}\PYG{o}{.}\PYG{n}{pi}\PYG{o}{*}\PYG{n}{r}\PYG{o}{*}\PYG{o}{*}\PYG{l+m+mi}{2}\PYG{o}{*}\PYG{n}{gamma\PYGZus{}sl}\PYG{p}{)}
\end{sphinxVerbatim}
}

\end{sphinxuseclass}
\end{sphinxuseclass}
\begin{sphinxuseclass}{nbinput}
{
\sphinxsetup{VerbatimColor={named}{nbsphinx-code-bg}}
\sphinxsetup{VerbatimBorderColor={named}{nbsphinx-code-border}}
\begin{sphinxVerbatim}[commandchars=\\\{\}]
\llap{\color{nbsphinxin}[7]:\,\hspace{\fboxrule}\hspace{\fboxsep}}\PYG{n}{plt}\PYG{o}{.}\PYG{n}{figure}\PYG{p}{(}\PYG{n}{figsize}\PYG{o}{=}\PYG{p}{(}\PYG{l+m+mi}{8}\PYG{p}{,}\PYG{l+m+mi}{6}\PYG{p}{)}\PYG{p}{)}
\PYG{n}{r}\PYG{o}{=}\PYG{n}{np}\PYG{o}{.}\PYG{n}{linspace}\PYG{p}{(}\PYG{l+m+mi}{0}\PYG{p}{,}\PYG{l+m+mf}{1e\PYGZhy{}8}\PYG{p}{,}\PYG{l+m+mi}{100}\PYG{p}{)}
\PYG{n}{kbT}\PYG{o}{=}\PYG{l+m+mf}{3.769e\PYGZhy{}21}
\PYG{n}{plt}\PYG{o}{.}\PYG{n}{plot}\PYG{p}{(}\PYG{n}{r}\PYG{o}{*}\PYG{l+m+mf}{1e9}\PYG{p}{,}\PYG{n}{dG}\PYG{p}{(}\PYG{n}{r}\PYG{p}{,}\PYG{l+m+mi}{10}\PYG{p}{)}\PYG{o}{/}\PYG{n}{kbT}\PYG{p}{,}\PYG{l+s+s1}{\PYGZsq{}}\PYG{l+s+s1}{k\PYGZhy{}}\PYG{l+s+s1}{\PYGZsq{}}\PYG{p}{)}
\PYG{n}{plt}\PYG{o}{.}\PYG{n}{xlabel}\PYG{p}{(}\PYG{l+s+s1}{\PYGZsq{}}\PYG{l+s+s1}{ radius \PYGZdl{}r\PYGZdl{} [nm] }\PYG{l+s+s1}{\PYGZsq{}}\PYG{p}{)}
\PYG{n}{plt}\PYG{o}{.}\PYG{n}{ylabel}\PYG{p}{(}\PYG{l+s+s1}{\PYGZsq{}}\PYG{l+s+s1}{free energy change \PYGZdl{}}\PYG{l+s+s1}{\PYGZbs{}}\PYG{l+s+s1}{Delta G/k\PYGZus{}}\PYG{l+s+si}{\PYGZob{}B\PYGZcb{}}\PYG{l+s+s1}{T\PYGZdl{}}\PYG{l+s+s1}{\PYGZsq{}}\PYG{p}{)}
\PYG{n}{plt}\PYG{o}{.}\PYG{n}{axhline}\PYG{p}{(}\PYG{n}{y}\PYG{o}{=}\PYG{l+m+mi}{0}\PYG{p}{,}\PYG{n}{ls}\PYG{o}{=}\PYG{l+s+s1}{\PYGZsq{}}\PYG{l+s+s1}{\PYGZhy{}\PYGZhy{}}\PYG{l+s+s1}{\PYGZsq{}}\PYG{p}{)}
\PYG{n}{plt}\PYG{o}{.}\PYG{n}{tight\PYGZus{}layout}\PYG{p}{(}\PYG{p}{)}
\PYG{n}{plt}\PYG{o}{.}\PYG{n}{show}\PYG{p}{(}\PYG{p}{)}
\end{sphinxVerbatim}
}

\end{sphinxuseclass}
\begin{sphinxuseclass}{nboutput}
\begin{sphinxuseclass}{nblast}
\hrule height -\fboxrule\relax
\vspace{\nbsphinxcodecellspacing}

\makeatletter\setbox\nbsphinxpromptbox\box\voidb@x\makeatother

\begin{nbsphinxfancyoutput}

\begin{sphinxuseclass}{output_area}
\begin{sphinxuseclass}{}
\noindent\sphinxincludegraphics[width=557\sphinxpxdimen,height=413\sphinxpxdimen]{{notebooks_L6_1_Solid_Liquid_Phase_Transitions_5_0}.png}

\end{sphinxuseclass}
\end{sphinxuseclass}
\end{nbsphinxfancyoutput}

\end{sphinxuseclass}
\end{sphinxuseclass}
\sphinxAtStartPar
The free energy barrier is therefore about 1500 times bigger than the thermal energy. Homogeneous nucleation from the bulk liquid phase thus essentially never happens. Even at a strong undercooling of about 50 K as shown in the plot below, the free energy barrier is about 65 times bigger than the thermal energy.

\begin{sphinxuseclass}{nbinput}
\begin{sphinxuseclass}{nblast}
{
\sphinxsetup{VerbatimColor={named}{nbsphinx-code-bg}}
\sphinxsetup{VerbatimBorderColor={named}{nbsphinx-code-border}}
\begin{sphinxVerbatim}[commandchars=\\\{\}]
\llap{\color{nbsphinxin}[8]:\,\hspace{\fboxrule}\hspace{\fboxsep}}\PYG{k}{def} \PYG{n+nf}{dGstar}\PYG{p}{(}\PYG{n}{dT}\PYG{p}{)}\PYG{p}{:}
    \PYG{n}{gamma\PYGZus{}sl}\PYG{o}{=}\PYG{l+m+mf}{38e\PYGZhy{}3} \PYG{c+c1}{\PYGZsh{}(38mJ/m\PYGZca{}2)}
    \PYG{n}{Tm}\PYG{o}{=}\PYG{l+m+mf}{273.15}
    \PYG{n}{dHm}\PYG{o}{=}\PYG{l+m+mf}{3.34e8} \PYG{c+c1}{\PYGZsh{} (J/m\PYGZca{}3)}
    \PYG{k}{return}\PYG{p}{(}\PYG{l+m+mi}{16}\PYG{o}{*}\PYG{n}{np}\PYG{o}{.}\PYG{n}{pi}\PYG{o}{*}\PYG{n}{gamma\PYGZus{}sl}\PYG{o}{*}\PYG{o}{*}\PYG{l+m+mi}{3}\PYG{o}{*}\PYG{p}{(}\PYG{n}{Tm}\PYG{o}{/}\PYG{n}{dHm}\PYG{p}{)}\PYG{o}{*}\PYG{o}{*}\PYG{l+m+mi}{2}\PYG{o}{/}\PYG{n}{dT}\PYG{o}{*}\PYG{o}{*}\PYG{l+m+mi}{2}\PYG{o}{/}\PYG{l+m+mi}{3}\PYG{p}{)}
\end{sphinxVerbatim}
}

\end{sphinxuseclass}
\end{sphinxuseclass}
\begin{sphinxuseclass}{nbinput}
{
\sphinxsetup{VerbatimColor={named}{nbsphinx-code-bg}}
\sphinxsetup{VerbatimBorderColor={named}{nbsphinx-code-border}}
\begin{sphinxVerbatim}[commandchars=\\\{\}]
\llap{\color{nbsphinxin}[9]:\,\hspace{\fboxrule}\hspace{\fboxsep}}\PYG{n}{plt}\PYG{o}{.}\PYG{n}{figure}\PYG{p}{(}\PYG{n}{figsize}\PYG{o}{=}\PYG{p}{(}\PYG{l+m+mi}{8}\PYG{p}{,}\PYG{l+m+mi}{6}\PYG{p}{)}\PYG{p}{)}
\PYG{n}{dt}\PYG{o}{=}\PYG{n}{np}\PYG{o}{.}\PYG{n}{linspace}\PYG{p}{(}\PYG{l+m+mi}{10}\PYG{p}{,}\PYG{l+m+mi}{50}\PYG{p}{)}
\PYG{n}{kbT}\PYG{o}{=}\PYG{l+m+mf}{3.769e\PYGZhy{}21}
\PYG{n}{plt}\PYG{o}{.}\PYG{n}{plot}\PYG{p}{(}\PYG{n}{dt}\PYG{p}{,}\PYG{n}{dGstar}\PYG{p}{(}\PYG{n}{dt}\PYG{p}{)}\PYG{o}{/}\PYG{n}{kbT}\PYG{p}{)}
\PYG{n}{plt}\PYG{o}{.}\PYG{n}{xlabel}\PYG{p}{(}\PYG{l+s+sa}{r}\PYG{l+s+s1}{\PYGZsq{}}\PYG{l+s+s1}{\PYGZdl{}}\PYG{l+s+s1}{\PYGZbs{}}\PYG{l+s+s1}{Delta T\PYGZdl{} [K]}\PYG{l+s+s1}{\PYGZsq{}}\PYG{p}{)}
\PYG{n}{plt}\PYG{o}{.}\PYG{n}{ylabel}\PYG{p}{(}\PYG{l+s+sa}{r}\PYG{l+s+s1}{\PYGZsq{}}\PYG{l+s+s1}{\PYGZdl{}}\PYG{l+s+s1}{\PYGZbs{}}\PYG{l+s+s1}{Delta G\PYGZca{}}\PYG{l+s+s1}{\PYGZob{}}\PYG{l+s+s1}{*\PYGZcb{}/k\PYGZus{}}\PYG{l+s+si}{\PYGZob{}B\PYGZcb{}}\PYG{l+s+s1}{T\PYGZdl{}}\PYG{l+s+s1}{\PYGZsq{}}\PYG{p}{)}
\PYG{n}{plt}\PYG{o}{.}\PYG{n}{tight\PYGZus{}layout}\PYG{p}{(}\PYG{p}{)}
\PYG{n}{plt}\PYG{o}{.}\PYG{n}{show}\PYG{p}{(}\PYG{p}{)}
\end{sphinxVerbatim}
}

\end{sphinxuseclass}
\begin{sphinxuseclass}{nboutput}
\begin{sphinxuseclass}{nblast}
\hrule height -\fboxrule\relax
\vspace{\nbsphinxcodecellspacing}

\makeatletter\setbox\nbsphinxpromptbox\box\voidb@x\makeatother

\begin{nbsphinxfancyoutput}

\begin{sphinxuseclass}{output_area}
\begin{sphinxuseclass}{}
\noindent\sphinxincludegraphics[width=558\sphinxpxdimen,height=413\sphinxpxdimen]{{notebooks_L6_1_Solid_Liquid_Phase_Transitions_8_0}.png}

\end{sphinxuseclass}
\end{sphinxuseclass}
\end{nbsphinxfancyoutput}

\end{sphinxuseclass}
\end{sphinxuseclass}
\sphinxAtStartPar
According to that, homogeneous nucleation, i.e., the freezing of a liquid is very unlikely and actually does not happen. More likely are heterogeneous nucleation events, which occur at the boundaries of the container or at impurities.


\section{Heterogeneous Nucleation}
\label{\detokenize{notebooks/L6/1_Solid_Liquid_Phase_Transitions:Heterogeneous-Nucleation}}
\sphinxAtStartPar
To study the process of heterogeneous nucleation, we have a look at a container wall, which acts as a nucleation catalyst. At this container wall, a solid droplet in form of a spherical cap exists.

\noindent\sphinxincludegraphics[width=380\sphinxpxdimen,height=233\sphinxpxdimen]{{heterogeneous_nucleation}.png}

\sphinxAtStartPar
The solid material shall have a contact angle of \(θ\) with the catalyst surface. In this case, the contact angle obeys Young’s law \(γ_{\rm sl}\cos(θ)=γ_{\rm cl}-γ_{\rm cs}\). Further, for the calculation of the volume and interfacial contributions, we need the volume and the surfaces of the spherical cap as well as the surface at the catalyst interface, which all together read

\sphinxAtStartPar
\begin{eqnarray}
V&=& \frac{1}{3}\pi r^3 (1-\cos(\theta))^2(2+\cos(\theta)),\\ S_{\rm sl}&=&2\pi r^2 (1-\cos(\theta)),\\ S_{\rm cs}&=&\pi r^2\sin^2(\theta).
\end{eqnarray}

\sphinxAtStartPar
Following the earlier arguments, we can write down the free energy change upon nucleation of a spherical cap of a radius at the catalyst surface as

\sphinxAtStartPar
\begin{eqnarray}
\Delta G(r)&=& \frac{1}{3}\pi r^3 (1-\cos(\theta))^2(2+\cos(\theta))\\ &&+ \gamma_{\rm sl}2\pi r^2 (1-\cos(\theta))\\ &&+ \gamma_{\rm cs}\pi r^2 \sin^2(\theta)\\ &&-\gamma_{\rm cl}\pi r^2 \sin^2(\theta).\\
\end{eqnarray}

\sphinxAtStartPar
With the help of Young’s equation, we can simplify this to

\sphinxAtStartPar
\begin{equation}
\Delta G^{*}=\frac{16\pi}{3}\gamma_{\rm sl}^3\left (\frac{T_{\rm m}}{\Delta H_{\rm m}}\right)^2 \frac{1}{\Delta T^2}\left ( \frac{(1-\cos(\theta))^2(2+\cos(\theta))}{4}\right ).
\end{equation}

\sphinxAtStartPar
The latter fraction is actually only an additional geometrical factor that depends only on \(θ\),while all factors before just resemble the homogeneous nucleation result. The graph below displays this geometrical factor \(f(θ)\) as a function of the contact angle \(θ\) and indicates that the creation of a nucleus with a certain contact angle at a solid wall considerably lowers the nucleation barrier. The nucleation barrier can thus be easily as low as thermal energy as shown in the
graph below for the water/ice system. The nucleus size in that model stays the same, while a contact angle of about 14° yields a nucleation barrier of \(1\, k_{\rm B}T\).

\begin{sphinxuseclass}{nbinput}
\begin{sphinxuseclass}{nblast}
{
\sphinxsetup{VerbatimColor={named}{nbsphinx-code-bg}}
\sphinxsetup{VerbatimBorderColor={named}{nbsphinx-code-border}}
\begin{sphinxVerbatim}[commandchars=\\\{\}]
\llap{\color{nbsphinxin}[10]:\,\hspace{\fboxrule}\hspace{\fboxsep}}\PYG{k}{def} \PYG{n+nf}{f}\PYG{p}{(}\PYG{n}{theta}\PYG{p}{)}\PYG{p}{:}
    \PYG{k}{return}\PYG{p}{(}\PYG{p}{(}\PYG{l+m+mi}{1}\PYG{o}{\PYGZhy{}}\PYG{n}{np}\PYG{o}{.}\PYG{n}{cos}\PYG{p}{(}\PYG{n}{theta}\PYG{p}{)}\PYG{p}{)}\PYG{o}{*}\PYG{o}{*}\PYG{l+m+mi}{2}\PYG{o}{*}\PYG{p}{(}\PYG{l+m+mi}{2}\PYG{o}{+}\PYG{n}{np}\PYG{o}{.}\PYG{n}{cos}\PYG{p}{(}\PYG{n}{theta}\PYG{p}{)}\PYG{p}{)}\PYG{o}{/}\PYG{l+m+mi}{4}\PYG{p}{)}
\end{sphinxVerbatim}
}

\end{sphinxuseclass}
\end{sphinxuseclass}
\begin{sphinxuseclass}{nbinput}
\begin{sphinxuseclass}{nblast}
{
\sphinxsetup{VerbatimColor={named}{nbsphinx-code-bg}}
\sphinxsetup{VerbatimBorderColor={named}{nbsphinx-code-border}}
\begin{sphinxVerbatim}[commandchars=\\\{\}]
\llap{\color{nbsphinxin}[11]:\,\hspace{\fboxrule}\hspace{\fboxsep}}\PYG{n}{theta}\PYG{o}{=}\PYG{n}{np}\PYG{o}{.}\PYG{n}{linspace}\PYG{p}{(}\PYG{l+m+mi}{0}\PYG{p}{,}\PYG{l+m+mi}{180}\PYG{p}{,}\PYG{l+m+mi}{100}\PYG{p}{)}
\end{sphinxVerbatim}
}

\end{sphinxuseclass}
\end{sphinxuseclass}
\begin{sphinxuseclass}{nbinput}
{
\sphinxsetup{VerbatimColor={named}{nbsphinx-code-bg}}
\sphinxsetup{VerbatimBorderColor={named}{nbsphinx-code-border}}
\begin{sphinxVerbatim}[commandchars=\\\{\}]
\llap{\color{nbsphinxin}[12]:\,\hspace{\fboxrule}\hspace{\fboxsep}}\PYG{n}{plt}\PYG{o}{.}\PYG{n}{figure}\PYG{p}{(}\PYG{n}{figsize}\PYG{o}{=}\PYG{p}{(}\PYG{l+m+mi}{8}\PYG{p}{,}\PYG{l+m+mi}{6}\PYG{p}{)}\PYG{p}{)}
\PYG{n}{plt}\PYG{o}{.}\PYG{n}{semilogy}\PYG{p}{(}\PYG{n}{theta}\PYG{p}{,}\PYG{n}{f}\PYG{p}{(}\PYG{n}{theta}\PYG{o}{*}\PYG{n}{np}\PYG{o}{.}\PYG{n}{pi}\PYG{o}{/}\PYG{l+m+mi}{180}\PYG{p}{)}\PYG{p}{)}
\PYG{n}{plt}\PYG{o}{.}\PYG{n}{xlabel}\PYG{p}{(}\PYG{l+s+sa}{r}\PYG{l+s+s1}{\PYGZsq{}}\PYG{l+s+s1}{contact angle \PYGZdl{}}\PYG{l+s+s1}{\PYGZbs{}}\PYG{l+s+s1}{theta\PYGZdl{} [°]}\PYG{l+s+s1}{\PYGZsq{}}\PYG{p}{)}
\PYG{n}{plt}\PYG{o}{.}\PYG{n}{ylabel}\PYG{p}{(}\PYG{l+s+sa}{r}\PYG{l+s+s1}{\PYGZsq{}}\PYG{l+s+s1}{f(\PYGZdl{}}\PYG{l+s+s1}{\PYGZbs{}}\PYG{l+s+s1}{theta\PYGZdl{})}\PYG{l+s+s1}{\PYGZsq{}}\PYG{p}{)}
\PYG{n}{plt}\PYG{o}{.}\PYG{n}{tight\PYGZus{}layout}\PYG{p}{(}\PYG{p}{)}
\PYG{n}{plt}\PYG{o}{.}\PYG{n}{show}\PYG{p}{(}\PYG{p}{)}
\end{sphinxVerbatim}
}

\end{sphinxuseclass}
\begin{sphinxuseclass}{nboutput}
\begin{sphinxuseclass}{nblast}
\hrule height -\fboxrule\relax
\vspace{\nbsphinxcodecellspacing}

\makeatletter\setbox\nbsphinxpromptbox\box\voidb@x\makeatother

\begin{nbsphinxfancyoutput}

\begin{sphinxuseclass}{output_area}
\begin{sphinxuseclass}{}
\noindent\sphinxincludegraphics[width=557\sphinxpxdimen,height=412\sphinxpxdimen]{{notebooks_L6_1_Solid_Liquid_Phase_Transitions_12_0}.png}

\end{sphinxuseclass}
\end{sphinxuseclass}
\end{nbsphinxfancyoutput}

\end{sphinxuseclass}
\end{sphinxuseclass}
\sphinxAtStartPar
\sphinxstyleemphasis{Geometry factor for the heterogeneous nucleation of a droplet with a contact angle at a catalyst wall}

\begin{sphinxuseclass}{nbinput}
{
\sphinxsetup{VerbatimColor={named}{nbsphinx-code-bg}}
\sphinxsetup{VerbatimBorderColor={named}{nbsphinx-code-border}}
\begin{sphinxVerbatim}[commandchars=\\\{\}]
\llap{\color{nbsphinxin}[15]:\,\hspace{\fboxrule}\hspace{\fboxsep}}\PYG{n}{plt}\PYG{o}{.}\PYG{n}{figure}\PYG{p}{(}\PYG{n}{figsize}\PYG{o}{=}\PYG{p}{(}\PYG{l+m+mi}{8}\PYG{p}{,}\PYG{l+m+mi}{6}\PYG{p}{)}\PYG{p}{)}
\PYG{n}{r}\PYG{o}{=}\PYG{n}{np}\PYG{o}{.}\PYG{n}{linspace}\PYG{p}{(}\PYG{l+m+mi}{0}\PYG{p}{,}\PYG{l+m+mf}{1e\PYGZhy{}8}\PYG{p}{,}\PYG{l+m+mi}{100}\PYG{p}{)}
\PYG{n}{kbT}\PYG{o}{=}\PYG{l+m+mf}{3.769e\PYGZhy{}21}

\PYG{p}{[}\PYG{n}{plt}\PYG{o}{.}\PYG{n}{plot}\PYG{p}{(}\PYG{n}{r}\PYG{o}{*}\PYG{l+m+mf}{1e9}\PYG{p}{,}\PYG{n}{dG}\PYG{p}{(}\PYG{n}{r}\PYG{p}{,}\PYG{l+m+mi}{10}\PYG{p}{)}\PYG{o}{*}\PYG{n}{f}\PYG{p}{(}\PYG{n}{theta}\PYG{o}{*}\PYG{n}{np}\PYG{o}{.}\PYG{n}{pi}\PYG{o}{/}\PYG{l+m+mi}{180}\PYG{p}{)}\PYG{o}{/}\PYG{n}{kbT}\PYG{p}{,}\PYG{l+s+s1}{\PYGZsq{}}\PYG{l+s+s1}{\PYGZhy{}}\PYG{l+s+s1}{\PYGZsq{}}\PYG{p}{,}\PYG{n}{label}\PYG{o}{=}\PYG{n+nb}{str}\PYG{p}{(}\PYG{n}{theta}\PYG{p}{)}\PYG{o}{+}\PYG{l+s+s2}{\PYGZdq{}}\PYG{l+s+s2}{°}\PYG{l+s+s2}{\PYGZdq{}}\PYG{p}{)} \PYG{k}{for} \PYG{n}{theta} \PYG{o+ow}{in} \PYG{n+nb}{range}\PYG{p}{(}\PYG{l+m+mi}{10}\PYG{p}{,}\PYG{l+m+mi}{16}\PYG{p}{,}\PYG{l+m+mi}{2}\PYG{p}{)}\PYG{p}{]}
\PYG{n}{plt}\PYG{o}{.}\PYG{n}{xlabel}\PYG{p}{(}\PYG{l+s+s1}{\PYGZsq{}}\PYG{l+s+s1}{ radius \PYGZdl{}r\PYGZdl{} [nm] }\PYG{l+s+s1}{\PYGZsq{}}\PYG{p}{)}
\PYG{n}{plt}\PYG{o}{.}\PYG{n}{ylabel}\PYG{p}{(}\PYG{l+s+s1}{\PYGZsq{}}\PYG{l+s+s1}{free energy change \PYGZdl{}}\PYG{l+s+s1}{\PYGZbs{}}\PYG{l+s+s1}{Delta G/k\PYGZus{}}\PYG{l+s+si}{\PYGZob{}B\PYGZcb{}}\PYG{l+s+s1}{T\PYGZdl{}}\PYG{l+s+s1}{\PYGZsq{}}\PYG{p}{)}
\PYG{n}{plt}\PYG{o}{.}\PYG{n}{axhline}\PYG{p}{(}\PYG{n}{y}\PYG{o}{=}\PYG{l+m+mi}{0}\PYG{p}{,}\PYG{n}{ls}\PYG{o}{=}\PYG{l+s+s1}{\PYGZsq{}}\PYG{l+s+s1}{\PYGZhy{}\PYGZhy{}}\PYG{l+s+s1}{\PYGZsq{}}\PYG{p}{)}
\PYG{n}{plt}\PYG{o}{.}\PYG{n}{legend}\PYG{p}{(}\PYG{p}{)}
\PYG{n}{plt}\PYG{o}{.}\PYG{n}{tight\PYGZus{}layout}\PYG{p}{(}\PYG{p}{)}
\PYG{n}{plt}\PYG{o}{.}\PYG{n}{show}\PYG{p}{(}\PYG{p}{)}
\end{sphinxVerbatim}
}

\end{sphinxuseclass}
\begin{sphinxuseclass}{nboutput}
\begin{sphinxuseclass}{nblast}
\hrule height -\fboxrule\relax
\vspace{\nbsphinxcodecellspacing}

\makeatletter\setbox\nbsphinxpromptbox\box\voidb@x\makeatother

\begin{nbsphinxfancyoutput}

\begin{sphinxuseclass}{output_area}
\begin{sphinxuseclass}{}
\noindent\sphinxincludegraphics[width=557\sphinxpxdimen,height=413\sphinxpxdimen]{{notebooks_L6_1_Solid_Liquid_Phase_Transitions_14_0}.png}

\end{sphinxuseclass}
\end{sphinxuseclass}
\end{nbsphinxfancyoutput}

\end{sphinxuseclass}
\end{sphinxuseclass}
\sphinxAtStartPar
\sphinxstyleemphasis{Nucleation barrier as a function of nucleus radius for heterogeneous nucleation at a wall. The curves display the results for the water/ice interface.}


\chapter{Classification of phase transitions}
\label{\detokenize{notebooks/L6/1_Solid_Liquid_Phase_Transitions:Classification-of-phase-transitions}}
\sphinxAtStartPar
A classification of different phase transitions can be carried out using the scheme of \sphinxstylestrong{Ehrenfest}. According to that, a phase transition

\begin{sphinxuseclass}{nbinput}
\begin{sphinxuseclass}{nblast}
{
\sphinxsetup{VerbatimColor={named}{nbsphinx-code-bg}}
\sphinxsetup{VerbatimBorderColor={named}{nbsphinx-code-border}}
\begin{sphinxVerbatim}[commandchars=\\\{\}]
\llap{\color{nbsphinxin}[ ]:\,\hspace{\fboxrule}\hspace{\fboxsep}}
\end{sphinxVerbatim}
}

\end{sphinxuseclass}
\end{sphinxuseclass}


\sphinxAtStartPar
The following section was created from \sphinxcode{\sphinxupquote{notebooks/L7/1\_Forces and Interactions.ipynb}}.


\chapter{Forces and Interactions in Soft Matter}
\label{\detokenize{notebooks/L7/1_Forces and Interactions:Forces-and-Interactions-in-Soft-Matter}}\label{\detokenize{notebooks/L7/1_Forces and Interactions::doc}}
\sphinxAtStartPar
While we have discussed in the previous sections the thermodynamics of systems and the kinetics of phase transitions, we have made as few as possible assumptions on the interactions between the liquid components of a mixture to highlight the importance of entropic and other effects. Now, we would like to have a close look at the possible types of interactions and their order of magnitude in soft matter systems. We may classify the interactions in the following way: \sphinxhyphen{} covalent interaction
(chemical binding), \sphinxhyphen{} electrostatic (Coulomb), \sphinxhyphen{} dipolar (vdW), \sphinxhyphen{} hydrogen bonding, \sphinxhyphen{} dispersion (vdW), \sphinxhyphen{} fluctuation, depletion (entropic).

\sphinxAtStartPar
These interactions deliver the forces that hold soft matter together, even though the phases are characterized by \sphinxstyleemphasis{density}, \sphinxstyleemphasis{free energy} and \sphinxstyleemphasis{entropy}, but not by the forces.

\begin{sphinxuseclass}{nbinput}
\begin{sphinxuseclass}{nblast}
{
\sphinxsetup{VerbatimColor={named}{nbsphinx-code-bg}}
\sphinxsetup{VerbatimBorderColor={named}{nbsphinx-code-border}}
\begin{sphinxVerbatim}[commandchars=\\\{\}]
\llap{\color{nbsphinxin}[1]:\,\hspace{\fboxrule}\hspace{\fboxsep}}\PYG{k+kn}{import} \PYG{n+nn}{numpy} \PYG{k}{as} \PYG{n+nn}{np}
\PYG{k+kn}{import} \PYG{n+nn}{matplotlib}\PYG{n+nn}{.}\PYG{n+nn}{pyplot} \PYG{k}{as} \PYG{n+nn}{plt}
\PYG{k+kn}{from} \PYG{n+nn}{numpy}\PYG{n+nn}{.}\PYG{n+nn}{linalg} \PYG{k+kn}{import} \PYG{n}{norm}
\PYG{k+kn}{from} \PYG{n+nn}{scipy}\PYG{n+nn}{.}\PYG{n+nn}{constants} \PYG{k+kn}{import} \PYG{n}{c}\PYG{p}{,}\PYG{n}{epsilon\PYGZus{}0}\PYG{p}{,}\PYG{n}{e}\PYG{p}{,}\PYG{n}{physical\PYGZus{}constants}

\PYG{o}{\PYGZpc{}}\PYG{k}{config} InlineBackend.figure\PYGZus{}format = \PYGZsq{}retina\PYGZsq{}
\PYG{c+c1}{\PYGZsh{} the lines below set a number of parameters for plotting, such as label font size,}
\PYG{c+c1}{\PYGZsh{} title font size, which you may find useful}
\PYG{n}{plt}\PYG{o}{.}\PYG{n}{rcParams}\PYG{o}{.}\PYG{n}{update}\PYG{p}{(}\PYG{p}{\PYGZob{}}\PYG{l+s+s1}{\PYGZsq{}}\PYG{l+s+s1}{font.size}\PYG{l+s+s1}{\PYGZsq{}}\PYG{p}{:} \PYG{l+m+mi}{14}\PYG{p}{,}
                     \PYG{l+s+s1}{\PYGZsq{}}\PYG{l+s+s1}{font.family}\PYG{l+s+s1}{\PYGZsq{}}\PYG{p}{:}\PYG{l+s+s1}{\PYGZsq{}}\PYG{l+s+s1}{sans\PYGZhy{}serif}\PYG{l+s+s1}{\PYGZsq{}}\PYG{p}{,}
                     \PYG{l+s+s1}{\PYGZsq{}}\PYG{l+s+s1}{axes.titlesize}\PYG{l+s+s1}{\PYGZsq{}}\PYG{p}{:} \PYG{l+m+mi}{16}\PYG{p}{,}
                     \PYG{l+s+s1}{\PYGZsq{}}\PYG{l+s+s1}{axes.labelsize}\PYG{l+s+s1}{\PYGZsq{}}\PYG{p}{:} \PYG{l+m+mi}{18}\PYG{p}{,}
                     \PYG{l+s+s1}{\PYGZsq{}}\PYG{l+s+s1}{axes.labelpad}\PYG{l+s+s1}{\PYGZsq{}}\PYG{p}{:} \PYG{l+m+mi}{14}\PYG{p}{,}
                     \PYG{l+s+s1}{\PYGZsq{}}\PYG{l+s+s1}{lines.linewidth}\PYG{l+s+s1}{\PYGZsq{}}\PYG{p}{:} \PYG{l+m+mi}{1}\PYG{p}{,}
                     \PYG{l+s+s1}{\PYGZsq{}}\PYG{l+s+s1}{lines.markersize}\PYG{l+s+s1}{\PYGZsq{}}\PYG{p}{:} \PYG{l+m+mi}{10}\PYG{p}{,}
                     \PYG{l+s+s1}{\PYGZsq{}}\PYG{l+s+s1}{xtick.labelsize}\PYG{l+s+s1}{\PYGZsq{}} \PYG{p}{:} \PYG{l+m+mi}{18}\PYG{p}{,}
                     \PYG{l+s+s1}{\PYGZsq{}}\PYG{l+s+s1}{ytick.labelsize}\PYG{l+s+s1}{\PYGZsq{}} \PYG{p}{:} \PYG{l+m+mi}{18}\PYG{p}{,}
                     \PYG{l+s+s1}{\PYGZsq{}}\PYG{l+s+s1}{xtick.top}\PYG{l+s+s1}{\PYGZsq{}} \PYG{p}{:} \PYG{k+kc}{True}\PYG{p}{,}
                     \PYG{l+s+s1}{\PYGZsq{}}\PYG{l+s+s1}{xtick.direction}\PYG{l+s+s1}{\PYGZsq{}} \PYG{p}{:} \PYG{l+s+s1}{\PYGZsq{}}\PYG{l+s+s1}{in}\PYG{l+s+s1}{\PYGZsq{}}\PYG{p}{,}
                     \PYG{l+s+s1}{\PYGZsq{}}\PYG{l+s+s1}{ytick.right}\PYG{l+s+s1}{\PYGZsq{}} \PYG{p}{:} \PYG{k+kc}{True}\PYG{p}{,}
                     \PYG{l+s+s1}{\PYGZsq{}}\PYG{l+s+s1}{ytick.direction}\PYG{l+s+s1}{\PYGZsq{}} \PYG{p}{:} \PYG{l+s+s1}{\PYGZsq{}}\PYG{l+s+s1}{in}\PYG{l+s+s1}{\PYGZsq{}}\PYG{p}{,}\PYG{p}{\PYGZcb{}}\PYG{p}{)}
\end{sphinxVerbatim}
}

\end{sphinxuseclass}
\end{sphinxuseclass}

\section{Pairwise interaction energy}
\label{\detokenize{notebooks/L7/1_Forces and Interactions:Pairwise-interaction-energy}}
\sphinxAtStartPar
Before we go into further details, we may have a look at some general behavior again. Let us assume that the interaction energy between two atoms/molecules is given by

\sphinxAtStartPar
\begin{equation}
w(r)=-\frac{C}{r^n}
\end{equation}

\sphinxAtStartPar
with \(C\) being an interaction\sphinxhyphen{}specific constant, then the force between the two species at a distance \(r\) is given by

\sphinxAtStartPar
\begin{equation}
F(r)=-\frac{\mathrm dw(r)}{\mathrm dr}=-\frac{nC}{r^{n+1}}.
\end{equation}

\sphinxAtStartPar
For a material, which has a number density \(\rho\) and, thus, the total number \(\rho 4πr^2 \mathrm{d}r\) molecules in a shell between \(r,r+\mathrm{d}r\) around a molecule, we obtain the following total interaction energy per molecule (the standard chemical potential):

\sphinxAtStartPar
\begin{equation}
\mu^{0}=\int_{\sigma}^{L}w(r)\rho 4\pi r^2 \mathrm dr=\frac{-4\pi C\rho}{(n-3)\sigma^{n-3}}\left [ 1-\left (\frac{\sigma}{L}\right )^{n-3}\right ].
\end{equation}

\sphinxAtStartPar
The total interaction energy, and thus also the property of the system will, consequently, depend on the size \(L\) of the system, except we assume \(n>3\) and \(L>>σ\), where \(σ\) is the size of the molecule. This states nothing else, that long range interactions may yield system\sphinxhyphen{}dependent properties or bulk properties that do not depend on the volume size only if objects become small. Obviously, Coulomb interactions or dipolar interactions may not satisfy the above
assumptions.


\section{Cohesive energy of a liquid}
\label{\detokenize{notebooks/L7/1_Forces and Interactions:Cohesive-energy-of-a-liquid}}
\sphinxAtStartPar
We can find out some general rule about the cohesive energy of a molecule with its neighbors in a liquid, when comparing the molar gas and molar liquid volumes. A typical gas molar volume is \(22.400~ \rm cm^3/mol\), while this is only \(20~ \rm cm^3/mol\) for a liquid. If liquid and gas coexist at a certain temperature \(T\), then the chemical potential of gas and liquid have to be the same, i.e.:

\sphinxAtStartPar
\begin{equation}
\mu^{0}_{\rm gas}+k_{\rm B}T\ln(X_{\rm gas})=\mu^{0}_{\rm liq}+k_{\rm B}T\ln(X_{\rm liq})
\end{equation}

\sphinxAtStartPar
(with \(X\) the dimensionless concentration in the respective phases, e.g., \(X_\mathrm{gas}=1/22.400\)) or

\sphinxAtStartPar
\begin{equation}
\mu^{0}_{\rm gas}-\mu^0_{\rm liq}\approx -\mu^{0}_{\rm liq}=k_{\rm B}T \ln\left (\frac{X_{\rm liq}}{X_{\rm gas}}\right)\approx 7 k_{\rm B}T
\end{equation}

\sphinxAtStartPar
assuming that there is essentially no cohesive energy in the gas phase. At the vaporization temperature \(T_\mathrm{B}\), the energy required to release one mole of molecules from its cohesion with its neighboring molecules to the gas phase is thus

\sphinxAtStartPar
\begin{equation}
U_{\rm vap}=-N_{\rm A}\mu^0_{\rm liq}=7N_{\rm A}k_{\rm B}T_\mathrm{B}=7RT_\mathrm{B}.
\end{equation}

\sphinxAtStartPar
This allows us to estimate the latent heat of vaporization

\sphinxAtStartPar
\begin{equation}
\Delta H_{\rm vap}=U_{\rm vap}+pV\approx 7 R T_\mathrm{B}+RT_\mathrm{B}.
\end{equation}

\sphinxAtStartPar
According to that, the ratio of latent heat of vaporization and boiling temperature is \(\frac{\Delta H_{\rm vap}}{T_\mathrm{B}}\approx 8 R\approx 80 \frac{\rm J}{\rm K\, mol}\) per mole, or \(9 k_{\rm B} T\) per molecule. If we assume that each molecule has on average 6 neighbors in a liquid, then we obtain a value of \$:nbsphinx\sphinxhyphen{}math:\sphinxtitleref{frac\{3\}\{2\}\textasciigrave{}k\_\{:nbsphinx\sphinxhyphen{}math:}rm B\textasciigrave{}\}T \$ as energy molecular pair. This approximate rule is called \sphinxstylestrong{Trouton’s rule} and gives only a very rough
estimate of the cohesive energy, as it completely neglects the details of the interactions. However, it demonstrates, why the thermal energy is important in soft matter.

\begin{sphinxadmonition}{warning}{}\unskip
\sphinxAtStartPar
\sphinxstylestrong{Trouton’s rule}

\sphinxAtStartPar
The molar latent heat of vaporization, which is a measure for the cohesive energy of a liquid, is approximately

\sphinxAtStartPar
\begin{equation}
\frac{\Delta H_{\rm vap}}{k_\mathrm{B}T}\approx 8R
\end{equation}

\sphinxAtStartPar
with \(R\) being the gas constant.
\end{sphinxadmonition}


\section{Coulomb forces, charge\textendash{}charge interactions}
\label{\detokenize{notebooks/L7/1_Forces and Interactions:Coulomb-forces,-charge_charge-interactions}}
\sphinxAtStartPar
The simplest but at the same time also one of the most important types of interaction is the electrostatic interaction, e.g., of simple charges. This type of interaction is important not only due to its relevance in biological systems, but the electrostatic interaction is in principle the only one delivering a long range repulsive force.


\subsection{Charge\textendash{}charge interactions}
\label{\detokenize{notebooks/L7/1_Forces and Interactions:Charge_charge-interactions}}
\sphinxAtStartPar
Charge\textendash{}charge interactions are mediated by the electric fields. Assume that we have a charge \(Q_1\) that creates an electric field

\sphinxAtStartPar
\begin{equation}
E_{1}=\frac{Q_{1}}{4\pi \epsilon_{0}\epsilon r^2}.
\end{equation}

\sphinxAtStartPar
We neglect the vectorial character of the electric field to avoid further complications. The electric field is creating a force on a second charge \(Q_2\)

\sphinxAtStartPar
\begin{equation}
F(r)=Q_{2}E_{1}=\frac{Q_{1}Q_{2}}{4\pi \epsilon_{0}\epsilon r^2}.
\end{equation}

\sphinxAtStartPar
For such a charge assembly at the distance \(r\) is an energy stored, which is the potential energy of assembling these two charges from infinity. The free energy of the two charges thus reads

\sphinxAtStartPar
\begin{equation}
w(r)=\int_{\infty}^{r}-F(r)\mathrm dr=-\int_{\infty}^{r} \frac{Q_{1}Q_{2}}{4\pi \epsilon_{0}\epsilon r^{2}}\mathrm dr=\frac{Q_{1}Q_{2}}{4\pi \epsilon_{0}\epsilon r}.
\end{equation}

\sphinxAtStartPar
If we evaluate this energy, for example, for a sodium and a chlorine ion at a distance of \(r=0.276\, {\rm nm}\), we find a free energy of interaction of \(w=-8.4\cdot 10^{-19}\, {\rm J}\), which corresponds to about \(200\, k_{\rm B}T\) at \(300~\rm K\) temperature. This is on the same order of magnitude than covalent interactions. It requires about \(3~\rm nN\) to break this bond. The long range character of electrostatics becomes clear when evaluating the distance at which
this interaction becomes comparable to \(k_{\rm B}T\): \(r=56\, {\rm nm}\). This is only considering a pair of ions. In a NaCl crystal, multiple neighbors contribute to the interaction energy of one sodium ion with its surrounding. One sodium ion has 6 \(\mathrm{Cl^–}\) neighbors at a distance of \(r=0.276\, {\rm nm}\), 12 \(\mathrm{Na^+}\) neightbors at \(\sqrt{2}r\),and further 8 \(\mathrm{Cl^–}\) neighbors at \(\sqrt{3}r\) and so on. We have to sum up all the
interaction energies for the total cohesive energy of the sodium ion in the crystal

\sphinxAtStartPar
\begin{equation}
\mu^{0}=-\frac{e^2}{4\pi \epsilon_{0}r}\left [ 6-\frac{12}{\sqrt{2}}+\frac{8}{\sqrt{3}}-\frac{6}{2}+\ldots\right]=-1.748\frac{e^2}{4\pi \epsilon_{0}r}.
\end{equation}

\sphinxAtStartPar
The factor in front of the Coulomb term (the one in the square brackets) is termed Madelung constant and is known from solid state physics. It is characteristic for specific lattice types such as a simple cubic lattice in this case. Note that the cohesive energy of one sodium ion therefore is about \(350\, k_{\rm B}T\) and is thus much larger than the thermal energy keeping the NaCl crystal stable. Yet it can be dissolved in water very easily.


\subsection{Born energy of solvation}
\label{\detokenize{notebooks/L7/1_Forces and Interactions:Born-energy-of-solvation}}
\sphinxAtStartPar
The Born energy of solvation calculates the free energy of assembling a charge inside a dielectric medium of dielectric constant \(\epsilon\). Let us shortly reconsider the free energy:

\sphinxAtStartPar
\begin{eqnarray}
\mathrm dU&=&\mathrm dQ+\mathrm dW\\ \Rightarrow\qquad\mathrm dU&=&T\mathrm dS+\mathrm dW\\ \Leftrightarrow\qquad\mathrm dW&=&\mathrm dU-T\mathrm dS
\end{eqnarray}

\sphinxAtStartPar
which leads to

\sphinxAtStartPar
\begin{equation}
\mathrm dF=\mathrm dU-T\mathrm dS=\mathrm dW.
\end{equation}

\sphinxAtStartPar
Therefore, the free energy change is related to the energy to assemble a charge

\sphinxAtStartPar
\begin{equation}
\Delta F=\frac{\epsilon \epsilon_{0}}{2}\int_{V}E^{2}\mathrm dV=\int \mathrm dw=\int_0^Q\frac{q\mathrm dq}{4\pi\epsilon_0\epsilon a}=\frac{1}{2}\frac{Q^2}{4\pi\epsilon_0\epsilon a}.
\end{equation}

\sphinxAtStartPar
To find the free energy, we, hence, integrated the square of the electric field over the volume (this is actually the same as adding tiny charge elements against previously assembled parts of the charge). According to that, the free energy of a charge \(Q=z e\) per charge in a medium with dielectric constant \(\epsilon\) is

\sphinxAtStartPar
\begin{equation}
\mu^{0}=\frac{z^2e^2}{8\pi \epsilon \epsilon_{0}a}
\end{equation}

\sphinxAtStartPar
if the charge has a radius \(a\). If we look now at the difference of assembling the charge in vacuum (with \(\epsilon\) =1) and in a medium with \(\epsilon\), we find the following difference in the chemical potential (free energy):

\sphinxAtStartPar
\begin{equation}
\Delta \mu^{0}=-\frac{z^2e^2}{8\pi \epsilon_{0}a}\left (\frac{1}{\epsilon}-1\right )=-\frac{28z^2}{a}\left (\frac{1}{\epsilon}-1\right )k_\mathrm{B}T\mathrm{\, per\,ion\, at\,}T=300~\mathrm{K}.
\end{equation}

\sphinxAtStartPar
This is the Born free energy of solvation of a single ion. The molar free energy is then obtained by multiplication with the Avogadro number \(N_{\rm A}\)

\sphinxAtStartPar
\begin{equation}
\Delta G=N_{\rm A}\Delta \mu^{0}=-\frac{69z^2}{a}\left (\frac{1}{\epsilon}-1\right )\mathrm{kJ\,mol^{-1}}
\end{equation}

\sphinxAtStartPar
\begin{equation}
\Delta \mu^{0}\approx \frac{e^2}{4\pi\epsilon_{0}\epsilon (a_{+}+a_{-})}
\end{equation}

\sphinxAtStartPar
The mole fraction that is dissolved in water is then found by the Boltzmann factor

\sphinxAtStartPar
\begin{equation}
X_{\rm s}=e^{-\frac{\Delta \mu^{0}}{k_{\rm B}T}}
\end{equation}

\sphinxAtStartPar
which is a measure for the solubility of the ions. Inserting the formula for the chemical potential delivers the proportionality \(X_{\rm s}\propto \exp(-{\rm const}/\epsilon)\). This dependency on the dielectric constant of the solvent is indeed observed in the experiment, even though this is only a trend and there are solvents with marked deviations.

\noindent\sphinxincludegraphics[width=641\sphinxpxdimen,height=495\sphinxpxdimen]{{solubility}.png}


\begin{savenotes}\sphinxattablestart
\centering
\begin{tabulary}{\linewidth}[t]{|T|T|}
\hline
\sphinxstyletheadfamily 
\sphinxAtStartPar
Compound
&\sphinxstyletheadfamily 
\sphinxAtStartPar
dielectric constant \(\epsilon\)
\\
\hline
\sphinxAtStartPar
water
&
\sphinxAtStartPar
78.5
\\
\hline
\sphinxAtStartPar
ethanol
&
\sphinxAtStartPar
24.3
\\
\hline
\sphinxAtStartPar
acetone
&
\sphinxAtStartPar
20.7
\\
\hline
\sphinxAtStartPar
hexane
&
\sphinxAtStartPar
1.9
\\
\hline
\sphinxAtStartPar
polystyrene
&
\sphinxAtStartPar
2.4
\\
\hline
\sphinxAtStartPar
sodium chloride
&
\sphinxAtStartPar
6.0
\\
\hline
\end{tabulary}
\par
\sphinxattableend\end{savenotes}

\sphinxAtStartPar
The reasons for that are essentially
\begin{enumerate}
\sphinxsetlistlabels{\arabic}{enumi}{enumii}{}{)}%
\item {} 
\sphinxAtStartPar
that the theory just assumes continuous changes in the interactions, despite the fact that there is a near order, and

\item {} 
\sphinxAtStartPar
that additional interactions like hydrogen bonds exist.

\end{enumerate}


\subsection{Interactions involving polar molecules}
\label{\detokenize{notebooks/L7/1_Forces and Interactions:Interactions-involving-polar-molecules}}
\sphinxAtStartPar
Many molecules exhibit a dipole or even higher moments due to the fact that the charges are not evenly distributed over the molecular structure. Some of the atoms exhibit a stronger tendency to accept charges than others. This is typically measured by electronegativity and provides an idea of whether atoms rather donate or accept a charge when binding to other atoms. While homo\sphinxhyphen{}atomic bonds therefore do not have dipole moments, hetero\sphinxhyphen{}atomic bonds do (see table).


\begin{savenotes}\sphinxattablestart
\centering
\begin{tabulary}{\linewidth}[t]{|T|T|}
\hline
\sphinxstyletheadfamily 
\sphinxAtStartPar
Bond
&\sphinxstyletheadfamily 
\sphinxAtStartPar
Dipole moment {[}D{]}
\\
\hline
\sphinxAtStartPar
C\sphinxhyphen{}C
&
\sphinxAtStartPar
0
\\
\hline
\sphinxAtStartPar
C\sphinxhyphen{}N
&
\sphinxAtStartPar
0.22
\\
\hline
\sphinxAtStartPar
O\sphinxhyphen{}H
&
\sphinxAtStartPar
1.51
\\
\hline
\sphinxAtStartPar
F\sphinxhyphen{}H
&
\sphinxAtStartPar
1.94
\\
\hline
\sphinxAtStartPar
N=O
&
\sphinxAtStartPar
2.0
\\
\hline
\end{tabulary}
\par
\sphinxattableend\end{savenotes}


\begin{savenotes}\sphinxattablestart
\centering
\begin{tabulary}{\linewidth}[t]{|T|T|}
\hline
\sphinxstyletheadfamily 
\sphinxAtStartPar
Molecule
&\sphinxstyletheadfamily 
\sphinxAtStartPar
Dipole moment {[}D{]}
\\
\hline
\sphinxAtStartPar
hexane
&
\sphinxAtStartPar
0
\\
\hline
\sphinxAtStartPar
water
&
\sphinxAtStartPar
1.85
\\
\hline
\sphinxAtStartPar
ethanol
&
\sphinxAtStartPar
1.7
\\
\hline
\sphinxAtStartPar
acetone
&
\sphinxAtStartPar
2.9
\\
\hline
\end{tabulary}
\par
\sphinxattableend\end{savenotes}

\sphinxAtStartPar
The dipole moment of a molecule is measured by the displacement of two charges \(±q\) from each other:

\sphinxAtStartPar
\begin{equation}
\vec{u}=q\vec{l}.
\end{equation}

\sphinxAtStartPar
Its direction is from the negative to the positive side. It creates an electric field that is given by

\sphinxAtStartPar
\begin{equation}
\vec{E}=\frac{3(\vec{u}\cdot \hat{r})\hat{r}-\vec{u}}{4\pi \epsilon_{0}\epsilon r^{3}},
\end{equation}

\sphinxAtStartPar
where \(\hat{r}=\vec{r}/|r|\). The dipole self\sphinxhyphen{}energy, i.e., the energy to create the dipole in a solvent is given by

\sphinxAtStartPar
\begin{equation}
\mu^{0}=\frac{1}{4\pi \epsilon_{0}\epsilon}\left [ \frac{q^2}{2a}+\frac{q^2}{2a}-\frac{q^2}{l}\right ].
\end{equation}

\sphinxAtStartPar
For \(l=2a\) this results in \(\mu^0=q^2/(8\pi\epsilon_0 \epsilon a)=u^2/(4\pi\epsilon_0 \epsilon l^3 )\) and thus yields a similar dependence of the chemical potential on the dielectric function \(\epsilon\) as in the case of a single charge. The result is a similar dependence of the solubility on the dielectric function.

\sphinxAtStartPar
The plot below shows the electric field of a dipole:

\begin{sphinxuseclass}{nbinput}
{
\sphinxsetup{VerbatimColor={named}{nbsphinx-code-bg}}
\sphinxsetup{VerbatimBorderColor={named}{nbsphinx-code-border}}
\begin{sphinxVerbatim}[commandchars=\\\{\}]
\llap{\color{nbsphinxin}[145]:\,\hspace{\fboxrule}\hspace{\fboxsep}}\PYG{n}{plt}\PYG{o}{.}\PYG{n}{figure}\PYG{p}{(}\PYG{n}{figsize}\PYG{o}{=}\PYG{p}{(}\PYG{l+m+mi}{10}\PYG{p}{,} \PYG{l+m+mi}{10}\PYG{p}{)}\PYG{p}{)}

\PYG{c+c1}{\PYGZsh{} generate grid}
\PYG{n}{x}\PYG{o}{=}\PYG{n}{np}\PYG{o}{.}\PYG{n}{linspace}\PYG{p}{(}\PYG{o}{\PYGZhy{}}\PYG{l+m+mi}{2}\PYG{p}{,} \PYG{l+m+mi}{2}\PYG{p}{,} \PYG{l+m+mi}{32}\PYG{p}{)}
\PYG{n}{y}\PYG{o}{=}\PYG{n}{np}\PYG{o}{.}\PYG{n}{linspace}\PYG{p}{(}\PYG{o}{\PYGZhy{}}\PYG{l+m+mi}{2}\PYG{p}{,} \PYG{l+m+mi}{2}\PYG{p}{,} \PYG{l+m+mi}{32}\PYG{p}{)}
\PYG{n}{x}\PYG{p}{,} \PYG{n}{y}\PYG{o}{=}\PYG{n}{np}\PYG{o}{.}\PYG{n}{meshgrid}\PYG{p}{(}\PYG{n}{x}\PYG{p}{,} \PYG{n}{y}\PYG{p}{)}

\PYG{k}{def} \PYG{n+nf}{E}\PYG{p}{(}\PYG{n}{q}\PYG{p}{,} \PYG{n}{a}\PYG{p}{,} \PYG{n}{x}\PYG{p}{,} \PYG{n}{y}\PYG{p}{)}\PYG{p}{:}
    \PYG{k}{return} \PYG{n}{q}\PYG{o}{*}\PYG{p}{(}\PYG{n}{x}\PYG{o}{\PYGZhy{}}\PYG{n}{a}\PYG{p}{[}\PYG{l+m+mi}{0}\PYG{p}{]}\PYG{p}{)}\PYG{o}{/}\PYG{p}{(}\PYG{p}{(}\PYG{n}{x}\PYG{o}{\PYGZhy{}}\PYG{n}{a}\PYG{p}{[}\PYG{l+m+mi}{0}\PYG{p}{]}\PYG{p}{)}\PYG{o}{*}\PYG{o}{*}\PYG{l+m+mi}{2}\PYG{o}{+}\PYG{p}{(}\PYG{n}{y}\PYG{o}{\PYGZhy{}}\PYG{n}{a}\PYG{p}{[}\PYG{l+m+mi}{1}\PYG{p}{]}\PYG{p}{)}\PYG{o}{*}\PYG{o}{*}\PYG{l+m+mi}{2}\PYG{p}{)}\PYG{o}{*}\PYG{o}{*}\PYG{p}{(}\PYG{l+m+mf}{1.5}\PYG{p}{)}\PYG{p}{,} \PYGZbs{}
        \PYG{n}{q}\PYG{o}{*}\PYG{p}{(}\PYG{n}{y}\PYG{o}{\PYGZhy{}}\PYG{n}{a}\PYG{p}{[}\PYG{l+m+mi}{1}\PYG{p}{]}\PYG{p}{)}\PYG{o}{/}\PYG{p}{(}\PYG{p}{(}\PYG{n}{x}\PYG{o}{\PYGZhy{}}\PYG{n}{a}\PYG{p}{[}\PYG{l+m+mi}{0}\PYG{p}{]}\PYG{p}{)}\PYG{o}{*}\PYG{o}{*}\PYG{l+m+mi}{2}\PYG{o}{+}\PYG{p}{(}\PYG{n}{y}\PYG{o}{\PYGZhy{}}\PYG{n}{a}\PYG{p}{[}\PYG{l+m+mi}{1}\PYG{p}{]}\PYG{p}{)}\PYG{o}{*}\PYG{o}{*}\PYG{l+m+mi}{2}\PYG{p}{)}\PYG{o}{*}\PYG{o}{*}\PYG{p}{(}\PYG{l+m+mf}{1.5}\PYG{p}{)}

\PYG{c+c1}{\PYGZsh{} calculate vector field}
\PYG{n}{Ex1}\PYG{p}{,} \PYG{n}{Ey1}\PYG{o}{=}\PYG{n}{E}\PYG{p}{(}\PYG{o}{\PYGZhy{}}\PYG{l+m+mi}{1}\PYG{p}{,} \PYG{p}{[}\PYG{o}{\PYGZhy{}}\PYG{l+m+mf}{0.1}\PYG{p}{,} \PYG{l+m+mi}{0}\PYG{p}{]}\PYG{p}{,} \PYG{n}{x}\PYG{p}{,} \PYG{n}{y}\PYG{p}{)}
\PYG{n}{Ex2}\PYG{p}{,} \PYG{n}{Ey2}\PYG{o}{=}\PYG{n}{E}\PYG{p}{(}\PYG{l+m+mi}{1}\PYG{p}{,} \PYG{p}{[}\PYG{l+m+mf}{0.1}\PYG{p}{,} \PYG{l+m+mi}{0}\PYG{p}{]}\PYG{p}{,} \PYG{n}{x}\PYG{p}{,} \PYG{n}{y}\PYG{p}{)}
\PYG{n}{Ex}\PYG{o}{=}\PYG{n}{Ex1}\PYG{o}{+}\PYG{n}{Ex2}
\PYG{n}{Ey}\PYG{o}{=}\PYG{n}{Ey1}\PYG{o}{+}\PYG{n}{Ey2}

\PYG{n}{E\PYGZus{}max}\PYG{o}{=}\PYG{l+m+mi}{5}
\PYG{n}{E}\PYG{o}{=}\PYG{n}{np}\PYG{o}{.}\PYG{n}{sqrt}\PYG{p}{(}\PYG{n}{Ex}\PYG{o}{*}\PYG{o}{*}\PYG{l+m+mi}{2}\PYG{o}{+}\PYG{n}{Ey}\PYG{o}{*}\PYG{o}{*}\PYG{l+m+mi}{2}\PYG{p}{)}

\PYG{n}{Ex}\PYG{o}{.}\PYG{n}{flat}\PYG{p}{[}\PYG{n}{E}\PYG{o}{.}\PYG{n}{flat}\PYG{p}{[}\PYG{p}{:}\PYG{p}{]}\PYG{o}{\PYGZgt{}}\PYG{n}{E\PYGZus{}max}\PYG{p}{]}\PYG{o}{=}\PYG{n}{np}\PYG{o}{.}\PYG{n}{nan}
\PYG{n}{Ey}\PYG{o}{.}\PYG{n}{flat}\PYG{p}{[}\PYG{n}{E}\PYG{o}{.}\PYG{n}{flat}\PYG{p}{[}\PYG{p}{:}\PYG{p}{]}\PYG{o}{\PYGZgt{}}\PYG{n}{E\PYGZus{}max}\PYG{p}{]}\PYG{o}{=}\PYG{n}{np}\PYG{o}{.}\PYG{n}{nan}
\PYG{c+c1}{\PYGZsh{} plot vecor field}
\PYG{n}{plt}\PYG{o}{.}\PYG{n}{quiver}\PYG{p}{(}\PYG{n}{x}\PYG{p}{,} \PYG{n}{y}\PYG{p}{,} \PYG{n}{Ex}\PYG{o}{/}\PYG{n}{E}\PYG{p}{,} \PYG{n}{Ey}\PYG{o}{/}\PYG{n}{E}\PYG{p}{,} \PYG{n}{pivot}\PYG{o}{=}\PYG{l+s+s1}{\PYGZsq{}}\PYG{l+s+s1}{middle}\PYG{l+s+s1}{\PYGZsq{}}\PYG{p}{,} \PYG{n}{headwidth}\PYG{o}{=}\PYG{l+m+mi}{3}\PYG{p}{,} \PYG{n}{headlength}\PYG{o}{=}\PYG{l+m+mi}{5}\PYG{p}{,}\PYG{n}{scale}\PYG{o}{=}\PYG{l+m+mi}{40}\PYG{p}{)}
\PYG{n}{plt}\PYG{o}{.}\PYG{n}{xlabel}\PYG{p}{(}\PYG{l+s+s1}{\PYGZsq{}}\PYG{l+s+s1}{\PYGZdl{}x\PYGZdl{}}\PYG{l+s+s1}{\PYGZsq{}}\PYG{p}{)}
\PYG{n}{plt}\PYG{o}{.}\PYG{n}{ylabel}\PYG{p}{(}\PYG{l+s+s1}{\PYGZsq{}}\PYG{l+s+s1}{\PYGZdl{}y\PYGZdl{}}\PYG{l+s+s1}{\PYGZsq{}}\PYG{p}{)}
\PYG{n}{plt}\PYG{o}{.}\PYG{n}{title}\PYG{p}{(}\PYG{l+s+s1}{\PYGZsq{}}\PYG{l+s+s1}{dipole field}\PYG{l+s+s1}{\PYGZsq{}}\PYG{p}{)}
\PYG{n}{plt}\PYG{o}{.}\PYG{n}{show}\PYG{p}{(}\PYG{p}{)}
\end{sphinxVerbatim}
}

\end{sphinxuseclass}
\begin{sphinxuseclass}{nboutput}
\begin{sphinxuseclass}{nblast}
\hrule height -\fboxrule\relax
\vspace{\nbsphinxcodecellspacing}

\makeatletter\setbox\nbsphinxpromptbox\box\voidb@x\makeatother

\begin{nbsphinxfancyoutput}

\begin{sphinxuseclass}{output_area}
\begin{sphinxuseclass}{}
\noindent\sphinxincludegraphics[width=650\sphinxpxdimen,height=627\sphinxpxdimen]{{notebooks_L7_1_Forces_and_Interactions_10_0}.png}

\end{sphinxuseclass}
\end{sphinxuseclass}
\end{nbsphinxfancyoutput}

\end{sphinxuseclass}
\end{sphinxuseclass}

\subsubsection{Ion\textendash{}dipole interaction}
\label{\detokenize{notebooks/L7/1_Forces and Interactions:Ion_dipole-interaction}}
\sphinxAtStartPar
The interaction energy of a dipole with a charge can be calculated by

\sphinxAtStartPar
\begin{equation}
w(r)=-\frac{qQ}{4\pi\epsilon_{0}\epsilon}\left [ \frac{1}{r-\frac{1}{2}l \cos(\theta)}-\frac{1}{r+\frac{1}{2}l \cos(\theta)}\right ]=-\frac{qQ}{4\pi\epsilon_{0}\epsilon r^2}\cos(\theta)=-uE\cos(\theta),
\label{eq:energy}
\end{equation}

\sphinxAtStartPar
where the last two equations are assuming that the distance between both objects \(r\) is much larger than the extent of the dipole \(l\) itself. From the last equation, we see that the interaction can be either attractive or repulsive. An angle \(θ=0^°\) results in an attractive interaction, while \(θ=180^°\) yields repulsive interaction. Using a single charge (e.g., an \(\mathrm{Na^+}\) ion) and a dipole of \(u=1.85\, {\rm D}\) (water molecule) results in an interaction
energy of about \(39\, k_{\rm B}T\). Ions align and bind polar molecules like water, for example. The alignment is caused by the torque \(\tau=\vec{u}\times \vec{E}\). For arbitrary polar molecules this is called \sphinxstylestrong{solvation}, while for water the term \sphinxstylestrong{hydration} is used. The strength of the hydration can effect the mobility of ions in solution as it makes them effective charges and is of interest, for example, in the study of ion transport through ion channels, as this requires the
stripping of the hydration shell.

\begin{sphinxadmonition}{note}{}\unskip
\sphinxAtStartPar
\sphinxstylestrong{Example: Sodium Ion Hydration}
\end{sphinxadmonition}

\sphinxAtStartPar
The plot below shows the ion dipole interaction for a sodium ion and a water molecule as a function of distance.

\begin{sphinxuseclass}{nbinput}
\begin{sphinxuseclass}{nblast}
{
\sphinxsetup{VerbatimColor={named}{nbsphinx-code-bg}}
\sphinxsetup{VerbatimBorderColor={named}{nbsphinx-code-border}}
\begin{sphinxVerbatim}[commandchars=\\\{\}]
\llap{\color{nbsphinxin}[2]:\,\hspace{\fboxrule}\hspace{\fboxsep}}\PYG{k}{def} \PYG{n+nf}{ion\PYGZus{}dipole}\PYG{p}{(}\PYG{n}{u}\PYG{p}{,}\PYG{n}{r}\PYG{p}{,}\PYG{n}{l}\PYG{p}{,}\PYG{n}{theta}\PYG{p}{,}\PYG{n}{epsilon}\PYG{p}{)}\PYG{p}{:}
    \PYG{n}{q}\PYG{o}{=}\PYG{n}{u}\PYG{o}{/}\PYG{n}{l}
    \PYG{n}{f}\PYG{o}{=}\PYG{n}{e}\PYG{o}{*}\PYG{n}{q}\PYG{o}{/}\PYG{p}{(}\PYG{l+m+mi}{4}\PYG{o}{*}\PYG{n}{np}\PYG{o}{.}\PYG{n}{pi}\PYG{o}{*}\PYG{n}{epsilon}\PYG{o}{*}\PYG{n}{epsilon\PYGZus{}0}\PYG{p}{)}
    \PYG{k}{return}\PYG{p}{(}\PYG{o}{\PYGZhy{}}\PYG{n}{f}\PYG{o}{*}\PYG{p}{(}\PYG{l+m+mi}{1}\PYG{o}{/}\PYG{p}{(}\PYG{n}{r}\PYG{o}{\PYGZhy{}}\PYG{l+m+mf}{0.5}\PYG{o}{*}\PYG{n}{l}\PYG{o}{*}\PYG{n}{np}\PYG{o}{.}\PYG{n}{cos}\PYG{p}{(}\PYG{n}{theta}\PYG{p}{)}\PYG{p}{)}\PYG{o}{\PYGZhy{}}\PYG{l+m+mi}{1}\PYG{o}{/}\PYG{p}{(}\PYG{n}{r}\PYG{o}{+}\PYG{l+m+mf}{0.5}\PYG{o}{*}\PYG{n}{l}\PYG{o}{*}\PYG{n}{np}\PYG{o}{.}\PYG{n}{cos}\PYG{p}{(}\PYG{n}{theta}\PYG{p}{)}\PYG{p}{)}\PYG{p}{)}\PYG{p}{)}
\end{sphinxVerbatim}
}

\end{sphinxuseclass}
\end{sphinxuseclass}
\begin{sphinxuseclass}{nbinput}
\begin{sphinxuseclass}{nblast}
{
\sphinxsetup{VerbatimColor={named}{nbsphinx-code-bg}}
\sphinxsetup{VerbatimBorderColor={named}{nbsphinx-code-border}}
\begin{sphinxVerbatim}[commandchars=\\\{\}]
\llap{\color{nbsphinxin}[3]:\,\hspace{\fboxrule}\hspace{\fboxsep}}\PYG{n}{D}\PYG{o}{=}\PYG{n}{physical\PYGZus{}constants}\PYG{p}{[}\PYG{l+s+s2}{\PYGZdq{}}\PYG{l+s+s2}{atomic unit of electric dipole mom.}\PYG{l+s+s2}{\PYGZdq{}}\PYG{p}{]}\PYG{p}{[}\PYG{l+m+mi}{0}\PYG{p}{]}
\PYG{n}{J2eV}\PYG{o}{=}\PYG{n}{physical\PYGZus{}constants}\PYG{p}{[}\PYG{l+s+s2}{\PYGZdq{}}\PYG{l+s+s2}{electron volt\PYGZhy{}joule relationship}\PYG{l+s+s2}{\PYGZdq{}}\PYG{p}{]}\PYG{p}{[}\PYG{l+m+mi}{0}\PYG{p}{]}
\end{sphinxVerbatim}
}

\end{sphinxuseclass}
\end{sphinxuseclass}
\begin{sphinxuseclass}{nbinput}
\begin{sphinxuseclass}{nblast}
{
\sphinxsetup{VerbatimColor={named}{nbsphinx-code-bg}}
\sphinxsetup{VerbatimBorderColor={named}{nbsphinx-code-border}}
\begin{sphinxVerbatim}[commandchars=\\\{\}]
\llap{\color{nbsphinxin}[4]:\,\hspace{\fboxrule}\hspace{\fboxsep}}\PYG{n}{r}\PYG{o}{=}\PYG{n}{np}\PYG{o}{.}\PYG{n}{linspace}\PYG{p}{(}\PYG{l+m+mf}{0.1e\PYGZhy{}9}\PYG{p}{,}\PYG{l+m+mf}{0.5e\PYGZhy{}9}\PYG{p}{,}\PYG{l+m+mi}{200}\PYG{p}{)}
\PYG{n}{l}\PYG{o}{=}\PYG{l+m+mf}{0.02e\PYGZhy{}9}
\end{sphinxVerbatim}
}

\end{sphinxuseclass}
\end{sphinxuseclass}
\begin{sphinxuseclass}{nbinput}
{
\sphinxsetup{VerbatimColor={named}{nbsphinx-code-bg}}
\sphinxsetup{VerbatimBorderColor={named}{nbsphinx-code-border}}
\begin{sphinxVerbatim}[commandchars=\\\{\}]
\llap{\color{nbsphinxin}[5]:\,\hspace{\fboxrule}\hspace{\fboxsep}}\PYG{n}{plt}\PYG{o}{.}\PYG{n}{figure}\PYG{p}{(}\PYG{n}{figsize}\PYG{o}{=}\PYG{p}{(}\PYG{l+m+mi}{6}\PYG{p}{,}\PYG{l+m+mi}{5}\PYG{p}{)}\PYG{p}{)}
\PYG{n}{plt}\PYG{o}{.}\PYG{n}{plot}\PYG{p}{(}\PYG{n}{r}\PYG{o}{*}\PYG{l+m+mf}{1e9}\PYG{p}{,}\PYG{n}{ion\PYGZus{}dipole}\PYG{p}{(}\PYG{l+m+mf}{1.8}\PYG{o}{*}\PYG{n}{D}\PYG{p}{,}\PYG{n}{r}\PYG{p}{,}\PYG{n}{l}\PYG{p}{,}\PYG{l+m+mi}{0}\PYG{p}{,}\PYG{l+m+mi}{1}\PYG{p}{)}\PYG{o}{/}\PYG{n}{J2eV}\PYG{p}{,}\PYG{l+s+s1}{\PYGZsq{}}\PYG{l+s+s1}{k\PYGZhy{}\PYGZhy{}}\PYG{l+s+s1}{\PYGZsq{}}\PYG{p}{)}
\PYG{n}{plt}\PYG{o}{.}\PYG{n}{plot}\PYG{p}{(}\PYG{n}{r}\PYG{o}{*}\PYG{l+m+mf}{1e9}\PYG{p}{,}\PYG{n}{ion\PYGZus{}dipole}\PYG{p}{(}\PYG{l+m+mf}{1.8}\PYG{o}{*}\PYG{n}{D}\PYG{p}{,}\PYG{n}{r}\PYG{p}{,}\PYG{n}{l}\PYG{p}{,}\PYG{n}{np}\PYG{o}{.}\PYG{n}{pi}\PYG{p}{,}\PYG{l+m+mi}{1}\PYG{p}{)}\PYG{o}{/}\PYG{n}{J2eV}\PYG{p}{,}\PYG{l+s+s1}{\PYGZsq{}}\PYG{l+s+s1}{k\PYGZhy{}\PYGZhy{}}\PYG{l+s+s1}{\PYGZsq{}}\PYG{p}{)}
\PYG{n}{plt}\PYG{o}{.}\PYG{n}{plot}\PYG{p}{(}\PYG{n}{r}\PYG{o}{*}\PYG{l+m+mf}{1e9}\PYG{p}{,}\PYG{n}{ion\PYGZus{}dipole}\PYG{p}{(}\PYG{l+m+mf}{1.8}\PYG{o}{*}\PYG{n}{D}\PYG{p}{,}\PYG{n}{r}\PYG{p}{,}\PYG{n}{l}\PYG{o}{*}\PYG{l+m+mi}{5}\PYG{p}{,}\PYG{l+m+mi}{0}\PYG{p}{,}\PYG{l+m+mi}{1}\PYG{p}{)}\PYG{o}{/}\PYG{n}{J2eV}\PYG{p}{,}\PYG{l+s+s1}{\PYGZsq{}}\PYG{l+s+s1}{k}\PYG{l+s+s1}{\PYGZsq{}}\PYG{p}{)}
\PYG{n}{plt}\PYG{o}{.}\PYG{n}{plot}\PYG{p}{(}\PYG{n}{r}\PYG{o}{*}\PYG{l+m+mf}{1e9}\PYG{p}{,}\PYG{n}{ion\PYGZus{}dipole}\PYG{p}{(}\PYG{l+m+mf}{1.8}\PYG{o}{*}\PYG{n}{D}\PYG{p}{,}\PYG{n}{r}\PYG{p}{,}\PYG{n}{l}\PYG{o}{*}\PYG{l+m+mi}{5}\PYG{p}{,}\PYG{n}{np}\PYG{o}{.}\PYG{n}{pi}\PYG{p}{,}\PYG{l+m+mi}{1}\PYG{p}{)}\PYG{o}{/}\PYG{n}{J2eV}\PYG{p}{,}\PYG{l+s+s1}{\PYGZsq{}}\PYG{l+s+s1}{k}\PYG{l+s+s1}{\PYGZsq{}}\PYG{p}{)}
\PYG{n}{plt}\PYG{o}{.}\PYG{n}{ylim}\PYG{p}{(}\PYG{o}{\PYGZhy{}}\PYG{l+m+mf}{9.2}\PYG{p}{,}\PYG{l+m+mf}{9.2}\PYG{p}{)}
\PYG{n}{plt}\PYG{o}{.}\PYG{n}{xlim}\PYG{p}{(}\PYG{l+m+mi}{0}\PYG{p}{,}\PYG{l+m+mf}{0.5}\PYG{p}{)}
\PYG{n}{plt}\PYG{o}{.}\PYG{n}{xlabel}\PYG{p}{(}\PYG{l+s+s2}{\PYGZdq{}}\PYG{l+s+s2}{distance [nm]}\PYG{l+s+s2}{\PYGZdq{}}\PYG{p}{)}
\PYG{n}{plt}\PYG{o}{.}\PYG{n}{ylabel}\PYG{p}{(}\PYG{l+s+s2}{\PYGZdq{}}\PYG{l+s+s2}{energy [eV]}\PYG{l+s+s2}{\PYGZdq{}}\PYG{p}{)}
\PYG{n}{plt}\PYG{o}{.}\PYG{n}{title}\PYG{p}{(}\PYG{l+s+s2}{\PYGZdq{}}\PYG{l+s+s2}{dipole charge interaction}\PYG{l+s+s2}{\PYGZdq{}}\PYG{p}{)}
\PYG{n}{plt}\PYG{o}{.}\PYG{n}{show}\PYG{p}{(}\PYG{p}{)}
\end{sphinxVerbatim}
}

\end{sphinxuseclass}
\begin{sphinxuseclass}{nboutput}
\begin{sphinxuseclass}{nblast}
\hrule height -\fboxrule\relax
\vspace{\nbsphinxcodecellspacing}

\makeatletter\setbox\nbsphinxpromptbox\box\voidb@x\makeatother

\begin{nbsphinxfancyoutput}

\begin{sphinxuseclass}{output_area}
\begin{sphinxuseclass}{}
\noindent\sphinxincludegraphics[width=424\sphinxpxdimen,height=355\sphinxpxdimen]{{notebooks_L7_1_Forces_and_Interactions_15_0}.png}

\end{sphinxuseclass}
\end{sphinxuseclass}
\end{nbsphinxfancyoutput}

\end{sphinxuseclass}
\end{sphinxuseclass}
\sphinxAtStartPar
The table below shows some selected hydration properties of ions. The hydrated radius determines the diffusion of the ion in water. The hydration number is the number of orientationally bound water molecules. Typically, smaller ions have a larger hydration number.


\begin{savenotes}\sphinxattablestart
\centering
\begin{tabulary}{\linewidth}[t]{|T|T|T|T|}
\hline
\sphinxstyletheadfamily 
\sphinxAtStartPar
Ion
&\sphinxstyletheadfamily 
\sphinxAtStartPar
Bare Ion radius (nm)
&\sphinxstyletheadfamily 
\sphinxAtStartPar
Hydrated radius (nm)
&\sphinxstyletheadfamily 
\sphinxAtStartPar
Hydration number
\\
\hline
\sphinxAtStartPar
Na+
&
\sphinxAtStartPar
0.095
&
\sphinxAtStartPar
0.36
&
\sphinxAtStartPar
4
\\
\hline
\sphinxAtStartPar
Mg2+
&
\sphinxAtStartPar
0.065
&
\sphinxAtStartPar
0.43
&
\sphinxAtStartPar
6
\\
\hline
\sphinxAtStartPar
Cl\sphinxhyphen{}
&
\sphinxAtStartPar
0.181
&
\sphinxAtStartPar
0.33
&
\sphinxAtStartPar
1
\\
\hline
\sphinxAtStartPar
OH\sphinxhyphen{}
&
\sphinxAtStartPar
0.176
&
\sphinxAtStartPar
0.3
&
\sphinxAtStartPar
3
\\
\hline
\end{tabulary}
\par
\sphinxattableend\end{savenotes}



\sphinxAtStartPar
The following section was created from \sphinxcode{\sphinxupquote{notebooks/L8/1\_Forces\_and\_Interactions.ipynb}}.


\chapter{Dipole\sphinxhyphen{}Dipole Interactions}
\label{\detokenize{notebooks/L8/1_Forces_and_Interactions:Dipole-Dipole-Interactions}}\label{\detokenize{notebooks/L8/1_Forces_and_Interactions::doc}}
\sphinxAtStartPar
The interaction of two dipole will depend on three angles, the two angles of the dipoles with the connecting axis \(\theta_1, \theta_2\) and the angle \(\phi\) between the two planes which contain the individual dipoles and the axis. The figure below shows the corresponding geometry.

\noindent\sphinxincludegraphics[width=300\sphinxpxdimen,height=132\sphinxpxdimen]{{dipole_dipole}.png}

\sphinxAtStartPar
The energy of the two dipoles at a distance \(r\) follows then wihtout a detailed calculation with

\sphinxAtStartPar
\begin{equation}
w(r,\theta_2,\theta_2,\phi)=-\frac{u_1 u_2}{4\pi \epsilon_0 \epsilon r^3}\left [ 2\cos(\theta_1) \cos(\theta_2)-\sin(\theta_1)\sin(\theta_2)\cos(\phi)\right ]
\end{equation}

\sphinxAtStartPar
The image below (taken from the book by Israelachvili) shows the dipole dipole interaction energy for the \sphinxstylestrong{parallel} and the \sphinxstylestrong{in\sphinxhyphen{}line} configuration as indicated. As shown, the in\sphinxhyphen{}line configutation with facing opposite charges is more favorable.

\noindent\sphinxincludegraphics[width=300\sphinxpxdimen,height=225\sphinxpxdimen]{{dipole_energy}.png}

\sphinxAtStartPar
The energies that are typically found are also larger than the thermal energy at room temperature and thefore lead to ordering effects.


\chapter{Rotating Dipoles, angle averaged Potential}
\label{\detokenize{notebooks/L8/1_Forces_and_Interactions:Rotating-Dipoles,-angle-averaged-Potential}}
\sphinxAtStartPar
The equations we derived so far deliver the energy for a specific fixed orientation of dipoles for example. However, molecules with dipoles may rotate and undergo rotational Brownian motion driven by thermal energy. This rotational diffusion can be very fast and the rotational sampling creates an average interaction that may scale differently with the distance. To get an effective distance dependence in the interaction, we have to integrate the Boltzmann factor over the orientational degrees of
freedoms, i.e.

\sphinxAtStartPar
\begin{equation}
\exp\left ( \frac{w(r)}{k_B T} \right)=\frac{\int \exp\left (-\frac{w(r,\Omega)}{k_B T} d\Omega\right )}{\int d\Omega}=\left \langle  \exp\left (-\frac{w(r,\Omega)}{k_B T} d\Omega\right ) \right \rangle
\end{equation}

\sphinxAtStartPar
with
\begin{equation*}
\begin{split}d\Omega =\sin(\theta)d\theta d\phi\end{split}
\end{equation*}
\sphinxAtStartPar
and
\begin{equation*}
\begin{split}\int d\Omega=4\pi\end{split}
\end{equation*}


\sphinxAtStartPar
The following section was created from \sphinxcode{\sphinxupquote{notebooks/L9/1\_Electric\_Double\_Layer.ipynb}}.


\chapter{Electric Double Layer}
\label{\detokenize{notebooks/L9/1_Electric_Double_Layer:Electric-Double-Layer}}\label{\detokenize{notebooks/L9/1_Electric_Double_Layer::doc}}
\begin{sphinxuseclass}{nbinput}
\begin{sphinxuseclass}{nblast}
{
\sphinxsetup{VerbatimColor={named}{nbsphinx-code-bg}}
\sphinxsetup{VerbatimBorderColor={named}{nbsphinx-code-border}}
\begin{sphinxVerbatim}[commandchars=\\\{\}]
\llap{\color{nbsphinxin}[66]:\,\hspace{\fboxrule}\hspace{\fboxsep}}\PYG{k+kn}{import} \PYG{n+nn}{matplotlib}\PYG{n+nn}{.}\PYG{n+nn}{pyplot} \PYG{k}{as} \PYG{n+nn}{plt}
\PYG{k+kn}{import} \PYG{n+nn}{numpy} \PYG{k}{as} \PYG{n+nn}{np}
\PYG{k+kn}{from} \PYG{n+nn}{scipy}\PYG{n+nn}{.}\PYG{n+nn}{constants} \PYG{k+kn}{import} \PYG{o}{*}
\PYG{k+kn}{from} \PYG{n+nn}{scipy}\PYG{n+nn}{.}\PYG{n+nn}{optimize} \PYG{k+kn}{import} \PYG{n}{fsolve}
\PYG{k+kn}{from} \PYG{n+nn}{ipywidgets} \PYG{k+kn}{import} \PYG{n}{interact}\PYG{p}{,} \PYG{n}{interactive}\PYG{p}{,} \PYG{n}{fixed}\PYG{p}{,} \PYG{n}{interact\PYGZus{}manual}
\PYG{k+kn}{import} \PYG{n+nn}{ipywidgets} \PYG{k}{as} \PYG{n+nn}{widgets}
\PYG{k+kn}{import} \PYG{n+nn}{json}

\PYG{o}{\PYGZpc{}}\PYG{k}{matplotlib} inline
\PYG{o}{\PYGZpc{}}\PYG{k}{config} InlineBackend.figure\PYGZus{}format = \PYGZsq{}retina\PYGZsq{}

\PYG{k}{with} \PYG{n+nb}{open}\PYG{p}{(}\PYG{l+s+s1}{\PYGZsq{}}\PYG{l+s+s1}{style.json}\PYG{l+s+s1}{\PYGZsq{}}\PYG{p}{,} \PYG{l+s+s1}{\PYGZsq{}}\PYG{l+s+s1}{r}\PYG{l+s+s1}{\PYGZsq{}}\PYG{p}{)} \PYG{k}{as} \PYG{n}{fp}\PYG{p}{:}
    \PYG{n}{style} \PYG{o}{=} \PYG{n}{json}\PYG{o}{.}\PYG{n}{load}\PYG{p}{(}\PYG{n}{fp}\PYG{p}{)}

\PYG{n}{plt}\PYG{o}{.}\PYG{n}{rcParams}\PYG{o}{.}\PYG{n}{update}\PYG{p}{(}\PYG{n}{style}\PYG{p}{)}
\end{sphinxVerbatim}
}

\end{sphinxuseclass}
\end{sphinxuseclass}

\section{Charges surfaces in solution, no salt}
\label{\detokenize{notebooks/L9/1_Electric_Double_Layer:Charges-surfaces-in-solution,-no-salt}}
\sphinxAtStartPar
We are going to consider first a situation, in which there is a solid surface (for example glass) in contact with a solution, which is presumably water. In this case, surface chemical groups on the solid may dissociate due to the high dielectric constant of water such that the residular surface is charged and some ions are dissolved in the solution. We will assume that there is no extra salt which is dissolved in water.

\sphinxAtStartPar
The questions that arise now are
\begin{itemize}
\item {} 
\sphinxAtStartPar
How are the dissolved ions distributed in space?

\item {} 
\sphinxAtStartPar
How does the ion distribution depend on the surface charge?

\item {} 
\sphinxAtStartPar
How is the electrostatic potential inside the liquid?

\end{itemize}

\sphinxAtStartPar
As the ions populate the regions according to their potential energy in the electric field of the surface charges, we have to combine the Poisson euqation with the Boltzmann equation to have a complete description. This combination will deliver us a mean field appraoch.


\section{Poisson\sphinxhyphen{}Boltzmann equation}
\label{\detokenize{notebooks/L9/1_Electric_Double_Layer:Poisson-Boltzmann-equation}}
\sphinxAtStartPar
If we name the electrostatic potential \(\psi\) and number density of ions of a valancy \(z\) \(\rho\), we can on one side write down the Boltzmann equation

\sphinxAtStartPar
\begin{equation}
\rho=\rho_0 \exp\left ( -\frac{z e\psi}{k_B T}\right)\tag{Boltzmann equation}
\end{equation}

\sphinxAtStartPar
end the Poisson equation as

\sphinxAtStartPar
\begin{equation}
ze\rho =-\epsilon_0 \epsilon \frac{d^{2}\psi}{dx^2}\tag{Poisson equation}
\end{equation}

\sphinxAtStartPar
As both contain the charge density \(\rho\) we can combine them to yield

\sphinxAtStartPar
\begin{equation}
\frac{d^2\psi}{dx^2} = -\frac{ze\rho}{\epsilon_0 \epsilon} =-\frac{ze \rho_0}{\epsilon_0 \epsilon}\exp\left ( -\frac{z e\psi}{k_B T}\right)\tag{Poisson-Boltzmann equation}
\end{equation}

\sphinxAtStartPar
the Poisson\sphinxhyphen{}Boltzmann equation, which is a non\sphinxhyphen{}linear differential equation in the potential \(\psi\) and therefore difficult to solve.

\sphinxAtStartPar
In the following we would like to determine the surface values for the potential \(\psi_s\), the electric field \(E_s\) and the density \(\rho_s\), which are called contact values. We will do that for a system of two solid surfaces each with a surface charge density \(\sigma\) according to the configuration shown below.

\noindent\sphinxincludegraphics[width=555\sphinxpxdimen,height=309\sphinxpxdimen]{{configuration}.png}

\sphinxAtStartPar
This system is symmetric with respect to the mid\sphinxhyphen{}plane at \(x=0\) and thus the potential is \(\psi_0=\psi(x=0)=0\) and the charge number density is \(\rho_o=\rho(x=0)\). Also, due to symmetry we have \(\frac{d\psi}{dx}|_{x=0}=0\). It follows from this symmetry that

\sphinxAtStartPar
\begin{equation}
\sigma=-\int_{0}^{D / 2} z e q d x=\epsilon_{0} \epsilon \int_0^{D / 2}\left(\frac{d^{2} \psi}{d x^{2}}\right)^{2} dx=-\epsilon_{0} \epsilon\left(\frac{d \psi}{d x}\right)|_{D / 2}=-\epsilon_{0} \epsilon\left(\frac{d \psi}{d x}\right)|_{S}=-\epsilon_0 \epsilon E_{s}
\end{equation}

\sphinxAtStartPar
To obtain the charge number density, we use the derivative of the Boltzmann equation with respect to the position

\sphinxAtStartPar
\begin{equation}
\frac{d\rho}{dx}=-\frac{ze\rho_0}{k_B T}\exp\left ( -\frac{z e\psi}{k_B T}\right)\frac{d\psi}{dx}
\end{equation}

\sphinxAtStartPar
The term on the right side can be replaced to yield

\sphinxAtStartPar
\begin{equation}
\frac{d\rho}{dx}=-\frac{ze\rho_0}{k_B T}\exp\left ( -\frac{z e\psi}{k_B T}\right)\frac{d\psi}{dx}=\frac{\epsilon_0 \epsilon}{k_B T}\left ( \frac{d\psi}{dx}\right ) \left (\frac{d^2\psi}{dx^2} \right )=\frac{\epsilon_0 \epsilon}{k_B T}\frac{d}{dx}\left ( \frac{d\psi}{dx}\right )^2
\end{equation}

\sphinxAtStartPar
thus we can integrate both sides

\sphinxAtStartPar
\begin{equation}
\rho_x- \rho_0=\int_0^{x}d\rho = \frac{\epsilon_0 \epsilon}{2k_B T} \left (\frac{d\psi}{dx} \right )^2_x
\end{equation}

\sphinxAtStartPar
or

\sphinxAtStartPar
\begin{equation}
\rho_x=\rho_0 + \frac{\epsilon_0 \epsilon}{2k_B T} \left (\frac{d\psi}{dx} \right )^2_x
\end{equation}

\sphinxAtStartPar
The term \(\frac{d\psi}{dx}\), which appears as a square in the last formula is just the electric field evaluated at the position \(x\). With the contact value of the electric field \(E_s\) obtained above, we can immediately write the contact value of the charge number density as

\sphinxAtStartPar
\begin{equation}
\rho_s=\rho_s+\frac{\sigma^2}{2\epsilon_0 \epsilon k_B T}
\end{equation}

\sphinxAtStartPar
Thus even in the case, when the two surfaces are seperated by an infinite distance \(\rho_0\rightarrow 0\), the surface charge number density never falls below \(\frac{\sigma^2}{2\epsilon_0 \epsilon k_B T}\).

\sphinxAtStartPar
To obtain the charge number density, the potential and the electric field we must solve the Poisson Boltzmann equation. The solution in the case of no additional electrolyte is given by

\sphinxAtStartPar
\begin{equation}
\psi=\frac{k_B T }{z e}\ln(\cos^{2}(Kx))
\end{equation}

\sphinxAtStartPar
or

\sphinxAtStartPar
\begin{equation}
\exp\left ( -\frac{z e \psi}{k_B T}\right )=\frac{1} { \cos ^{2} (K x)}
\end{equation}

\sphinxAtStartPar
where the constant \(K\) is given by

\sphinxAtStartPar
\begin{equation}
K^{2}=\frac{(z e)^{2} \rho_{0} }{ 2 \epsilon_{0} \epsilon k_B T}
\end{equation}

\sphinxAtStartPar
To find the value of the constant \(K\), we can take the derivative of the potential to obtain the electric field

\sphinxAtStartPar
\begin{equation}
E_{x}=-\frac{d\psi}{dx}= \frac{2k_B T K}{z e}\tan(K x)
\end{equation}

\sphinxAtStartPar
which is given at the surface to be

\sphinxAtStartPar
\begin{equation}
E_{s}=-\frac{d\psi}{dx}\bigg|_s = \frac{2k_B T K}{z e}\tan\left (K \frac{D}{2}\right)=-\frac{\sigma}{\epsilon_0 \epsilon}
\end{equation}

\sphinxAtStartPar
If the surface charge density \(\sigma\) is fixed and the thickness of the liquid layer \(D\) is known, the latter equation can be graphically solved to obtain the value of \(K\). The counter ion density in the solution is then given by

\sphinxAtStartPar
\begin{equation}
\rho(x)=\rho_0\exp\left (\frac{-ze\psi }{k_B T}\right)=\frac{\rho_0}{\cos^{2}(Kx)}
\end{equation}


\subsection{Example}
\label{\detokenize{notebooks/L9/1_Electric_Double_Layer:Example}}
\sphinxAtStartPar
Two surfaces with a charge density of \(\sigma=0.2\, C\cdot m^{-2}\) are at a distance of \(D=2 nm\) in water at \(T=293\,K\). We calculate the counter ion surface charge, the charge density, the electric field and the potential inside the water film between the two layers. We therefore first calculate a value of

\sphinxAtStartPar
\begin{equation}
KTK=\frac{ze\sigma}{2k_B T\epsilon_0\epsilon}
\end{equation}

\sphinxAtStartPar
which shall be equal to \(K\tan\left (KD/2\right )\).

\begin{sphinxuseclass}{nbinput}
\begin{sphinxuseclass}{nblast}
{
\sphinxsetup{VerbatimColor={named}{nbsphinx-code-bg}}
\sphinxsetup{VerbatimBorderColor={named}{nbsphinx-code-border}}
\begin{sphinxVerbatim}[commandchars=\\\{\}]
\llap{\color{nbsphinxin}[225]:\,\hspace{\fboxrule}\hspace{\fboxsep}}\PYG{n}{sigma}\PYG{o}{=}\PYG{l+m+mf}{0.2} \PYG{c+c1}{\PYGZsh{} C/m\PYGZca{}2}
\PYG{n}{epsilon}\PYG{o}{=}\PYG{l+m+mi}{80} \PYG{c+c1}{\PYGZsh{} water}
\PYG{n}{D}\PYG{o}{=}\PYG{l+m+mf}{2e\PYGZhy{}9} \PYG{c+c1}{\PYGZsh{} m}
\PYG{n}{T}\PYG{o}{=}\PYG{l+m+mi}{293} \PYG{c+c1}{\PYGZsh{} K}

\PYG{n}{KTK}\PYG{o}{=}\PYG{n}{e}\PYG{o}{*}\PYG{n}{sigma}\PYG{o}{/}\PYG{p}{(}\PYG{l+m+mi}{2}\PYG{o}{*}\PYG{n}{k}\PYG{o}{*}\PYG{n}{T}\PYG{o}{*}\PYG{n}{epsilon\PYGZus{}0}\PYG{o}{*}\PYG{n}{epsilon}\PYG{p}{)}
\end{sphinxVerbatim}
}

\end{sphinxuseclass}
\end{sphinxuseclass}
\sphinxAtStartPar
With the help of the previous definitions, we may search for the solution which gives the value of \(K\).

\begin{sphinxuseclass}{nbinput}
\begin{sphinxuseclass}{nblast}
{
\sphinxsetup{VerbatimColor={named}{nbsphinx-code-bg}}
\sphinxsetup{VerbatimBorderColor={named}{nbsphinx-code-border}}
\begin{sphinxVerbatim}[commandchars=\\\{\}]
\llap{\color{nbsphinxin}[226]:\,\hspace{\fboxrule}\hspace{\fboxsep}}\PYG{c+c1}{\PYGZsh{} numerically solve for K}
\PYG{n}{func} \PYG{o}{=} \PYG{k}{lambda} \PYG{n}{K} \PYG{p}{:} \PYG{n}{KTK} \PYG{o}{\PYGZhy{}} \PYG{n}{K}\PYG{o}{*}\PYG{n}{np}\PYG{o}{.}\PYG{n}{tan}\PYG{p}{(}\PYG{n}{K}\PYG{o}{*}\PYG{n}{D}\PYG{o}{/}\PYG{l+m+mi}{2}\PYG{p}{)}
\PYG{n}{K\PYGZus{}initial\PYGZus{}guess} \PYG{o}{=} \PYG{l+m+mf}{0.8}\PYG{o}{*}\PYG{n}{np}\PYG{o}{.}\PYG{n}{pi}\PYG{o}{/}\PYG{n}{D}
\PYG{n}{K} \PYG{o}{=} \PYG{n}{fsolve}\PYG{p}{(}\PYG{n}{func}\PYG{p}{,} \PYG{n}{K\PYGZus{}initial\PYGZus{}guess}\PYG{p}{)}
\end{sphinxVerbatim}
}

\end{sphinxuseclass}
\end{sphinxuseclass}

\subsubsection{Plot of the charge density}
\label{\detokenize{notebooks/L9/1_Electric_Double_Layer:Plot-of-the-charge-density}}
\begin{sphinxuseclass}{nbinput}
\begin{sphinxuseclass}{nblast}
{
\sphinxsetup{VerbatimColor={named}{nbsphinx-code-bg}}
\sphinxsetup{VerbatimBorderColor={named}{nbsphinx-code-border}}
\begin{sphinxVerbatim}[commandchars=\\\{\}]
\llap{\color{nbsphinxin}[227]:\,\hspace{\fboxrule}\hspace{\fboxsep}}\PYG{c+c1}{\PYGZsh{} ion density in the mid plane}
\PYG{n}{rho0}\PYG{o}{=}\PYG{n}{K}\PYG{o}{*}\PYG{o}{*}\PYG{l+m+mi}{2}\PYG{o}{*}\PYG{l+m+mi}{2}\PYG{o}{*}\PYG{n}{epsilon\PYGZus{}0}\PYG{o}{*}\PYG{n}{epsilon}\PYG{o}{*}\PYG{n}{k}\PYG{o}{*}\PYG{n}{T}\PYG{o}{/}\PYG{n}{e}\PYG{o}{*}\PYG{o}{*}\PYG{l+m+mi}{2}
\end{sphinxVerbatim}
}

\end{sphinxuseclass}
\end{sphinxuseclass}
\begin{sphinxuseclass}{nbinput}
{
\sphinxsetup{VerbatimColor={named}{nbsphinx-code-bg}}
\sphinxsetup{VerbatimBorderColor={named}{nbsphinx-code-border}}
\begin{sphinxVerbatim}[commandchars=\\\{\}]
\llap{\color{nbsphinxin}[228]:\,\hspace{\fboxrule}\hspace{\fboxsep}}\PYG{c+c1}{\PYGZsh{} ion density in the gap}
\PYG{n}{x}\PYG{o}{=}\PYG{n}{np}\PYG{o}{.}\PYG{n}{linspace}\PYG{p}{(}\PYG{o}{\PYGZhy{}}\PYG{n}{D}\PYG{o}{/}\PYG{l+m+mi}{2}\PYG{p}{,}\PYG{n}{D}\PYG{o}{/}\PYG{l+m+mi}{2}\PYG{p}{,}\PYG{l+m+mi}{1000}\PYG{p}{)}
\PYG{n}{plt}\PYG{o}{.}\PYG{n}{plot}\PYG{p}{(}\PYG{n}{x}\PYG{o}{/}\PYG{n}{D}\PYG{p}{,}\PYG{n}{rho0}\PYG{o}{/}\PYG{p}{(}\PYG{n}{np}\PYG{o}{.}\PYG{n}{cos}\PYG{p}{(}\PYG{n}{K}\PYG{o}{*}\PYG{n}{x}\PYG{p}{)}\PYG{o}{*}\PYG{o}{*}\PYG{l+m+mi}{2}\PYG{p}{)}\PYG{p}{)}
\PYG{n}{plt}\PYG{o}{.}\PYG{n}{xlabel}\PYG{p}{(}\PYG{l+s+sa}{r}\PYG{l+s+s1}{\PYGZsq{}}\PYG{l+s+s1}{x/D}\PYG{l+s+s1}{\PYGZsq{}}\PYG{p}{,}\PYG{n}{fontsize}\PYG{o}{=}\PYG{l+m+mi}{16}\PYG{p}{)}
\PYG{n}{plt}\PYG{o}{.}\PYG{n}{ylabel}\PYG{p}{(}\PYG{l+s+sa}{r}\PYG{l+s+s1}{\PYGZsq{}}\PYG{l+s+s1}{ion density [m\PYGZdl{}\PYGZca{}}\PYG{l+s+s1}{\PYGZob{}}\PYG{l+s+s1}{\PYGZhy{}3\PYGZcb{}\PYGZdl{}]}\PYG{l+s+s1}{\PYGZsq{}}\PYG{p}{,}\PYG{n}{fontsize}\PYG{o}{=}\PYG{l+m+mi}{16}\PYG{p}{)}
\PYG{n}{plt}\PYG{o}{.}\PYG{n}{show}\PYG{p}{(}\PYG{p}{)}
\end{sphinxVerbatim}
}

\end{sphinxuseclass}
\begin{sphinxuseclass}{nboutput}
\begin{sphinxuseclass}{nblast}
\hrule height -\fboxrule\relax
\vspace{\nbsphinxcodecellspacing}

\makeatletter\setbox\nbsphinxpromptbox\box\voidb@x\makeatother

\begin{nbsphinxfancyoutput}

\begin{sphinxuseclass}{output_area}
\begin{sphinxuseclass}{}
\noindent\sphinxincludegraphics[width=397\sphinxpxdimen,height=298\sphinxpxdimen]{{notebooks_L9_1_Electric_Double_Layer_15_0}.png}

\end{sphinxuseclass}
\end{sphinxuseclass}
\end{nbsphinxfancyoutput}

\end{sphinxuseclass}
\end{sphinxuseclass}

\subsubsection{Plot of the electric potential}
\label{\detokenize{notebooks/L9/1_Electric_Double_Layer:Plot-of-the-electric-potential}}
\begin{sphinxuseclass}{nbinput}
{
\sphinxsetup{VerbatimColor={named}{nbsphinx-code-bg}}
\sphinxsetup{VerbatimBorderColor={named}{nbsphinx-code-border}}
\begin{sphinxVerbatim}[commandchars=\\\{\}]
\llap{\color{nbsphinxin}[229]:\,\hspace{\fboxrule}\hspace{\fboxsep}}\PYG{c+c1}{\PYGZsh{} potential in the gap}
\PYG{n}{x}\PYG{o}{=}\PYG{n}{np}\PYG{o}{.}\PYG{n}{linspace}\PYG{p}{(}\PYG{o}{\PYGZhy{}}\PYG{n}{D}\PYG{o}{/}\PYG{l+m+mi}{2}\PYG{p}{,}\PYG{n}{D}\PYG{o}{/}\PYG{l+m+mi}{2}\PYG{p}{,}\PYG{l+m+mi}{1000}\PYG{p}{)}
\PYG{n}{plt}\PYG{o}{.}\PYG{n}{plot}\PYG{p}{(}\PYG{n}{x}\PYG{o}{/}\PYG{n}{D}\PYG{p}{,}\PYG{n}{k}\PYG{o}{*}\PYG{n}{T}\PYG{o}{*}\PYG{n}{np}\PYG{o}{.}\PYG{n}{log}\PYG{p}{(}\PYG{n}{np}\PYG{o}{.}\PYG{n}{cos}\PYG{p}{(}\PYG{n}{K}\PYG{o}{*}\PYG{n}{x}\PYG{p}{)}\PYG{o}{*}\PYG{o}{*}\PYG{l+m+mi}{2}\PYG{p}{)}\PYG{o}{/}\PYG{n}{e}\PYG{p}{)}
\PYG{n}{plt}\PYG{o}{.}\PYG{n}{xlabel}\PYG{p}{(}\PYG{l+s+sa}{r}\PYG{l+s+s1}{\PYGZsq{}}\PYG{l+s+s1}{x/D}\PYG{l+s+s1}{\PYGZsq{}}\PYG{p}{,}\PYG{n}{fontsize}\PYG{o}{=}\PYG{l+m+mi}{16}\PYG{p}{)}
\PYG{n}{plt}\PYG{o}{.}\PYG{n}{ylabel}\PYG{p}{(}\PYG{l+s+sa}{r}\PYG{l+s+s1}{\PYGZsq{}}\PYG{l+s+s1}{electric potential [V]}\PYG{l+s+s1}{\PYGZsq{}}\PYG{p}{,}\PYG{n}{fontsize}\PYG{o}{=}\PYG{l+m+mi}{16}\PYG{p}{)}
\end{sphinxVerbatim}
}

\end{sphinxuseclass}
\begin{sphinxuseclass}{nboutput}
{

\kern-\sphinxverbatimsmallskipamount\kern-\baselineskip
\kern+\FrameHeightAdjust\kern-\fboxrule
\vspace{\nbsphinxcodecellspacing}

\sphinxsetup{VerbatimColor={named}{white}}
\sphinxsetup{VerbatimBorderColor={named}{nbsphinx-code-border}}
\begin{sphinxuseclass}{output_area}
\begin{sphinxuseclass}{}


\begin{sphinxVerbatim}[commandchars=\\\{\}]
\llap{\color{nbsphinxout}[229]:\,\hspace{\fboxrule}\hspace{\fboxsep}}Text(0, 0.5, 'electric potential [V]')
\end{sphinxVerbatim}



\end{sphinxuseclass}
\end{sphinxuseclass}
}

\end{sphinxuseclass}
\begin{sphinxuseclass}{nboutput}
\begin{sphinxuseclass}{nblast}
\hrule height -\fboxrule\relax
\vspace{\nbsphinxcodecellspacing}

\makeatletter\setbox\nbsphinxpromptbox\box\voidb@x\makeatother

\begin{nbsphinxfancyoutput}

\begin{sphinxuseclass}{output_area}
\begin{sphinxuseclass}{}
\noindent\sphinxincludegraphics[width=437\sphinxpxdimen,height=281\sphinxpxdimen]{{notebooks_L9_1_Electric_Double_Layer_17_1}.png}

\end{sphinxuseclass}
\end{sphinxuseclass}
\end{nbsphinxfancyoutput}

\end{sphinxuseclass}
\end{sphinxuseclass}

\subsubsection{Plot of the electric field}
\label{\detokenize{notebooks/L9/1_Electric_Double_Layer:Plot-of-the-electric-field}}
\begin{sphinxuseclass}{nbinput}
{
\sphinxsetup{VerbatimColor={named}{nbsphinx-code-bg}}
\sphinxsetup{VerbatimBorderColor={named}{nbsphinx-code-border}}
\begin{sphinxVerbatim}[commandchars=\\\{\}]
\llap{\color{nbsphinxin}[232]:\,\hspace{\fboxrule}\hspace{\fboxsep}}\PYG{c+c1}{\PYGZsh{} field in the gap}
\PYG{n}{x}\PYG{o}{=}\PYG{n}{np}\PYG{o}{.}\PYG{n}{linspace}\PYG{p}{(}\PYG{o}{\PYGZhy{}}\PYG{n}{D}\PYG{o}{/}\PYG{l+m+mi}{2}\PYG{p}{,}\PYG{n}{D}\PYG{o}{/}\PYG{l+m+mi}{2}\PYG{p}{,}\PYG{l+m+mi}{1000}\PYG{p}{)}
\PYG{n}{plt}\PYG{o}{.}\PYG{n}{plot}\PYG{p}{(}\PYG{n}{x}\PYG{o}{/}\PYG{n}{D}\PYG{p}{,}\PYG{l+m+mi}{2}\PYG{o}{*}\PYG{n}{k}\PYG{o}{*}\PYG{n}{T}\PYG{o}{*}\PYG{n}{K}\PYG{o}{*}\PYG{n}{np}\PYG{o}{.}\PYG{n}{tan}\PYG{p}{(}\PYG{n}{K}\PYG{o}{*}\PYG{n}{x}\PYG{p}{)}\PYG{o}{/}\PYG{n}{e}\PYG{p}{)}
\PYG{n}{plt}\PYG{o}{.}\PYG{n}{xlabel}\PYG{p}{(}\PYG{l+s+sa}{r}\PYG{l+s+s1}{\PYGZsq{}}\PYG{l+s+s1}{x/D}\PYG{l+s+s1}{\PYGZsq{}}\PYG{p}{,}\PYG{n}{fontsize}\PYG{o}{=}\PYG{l+m+mi}{16}\PYG{p}{)}
\PYG{n}{plt}\PYG{o}{.}\PYG{n}{ylabel}\PYG{p}{(}\PYG{l+s+sa}{r}\PYG{l+s+s1}{\PYGZsq{}}\PYG{l+s+s1}{electric field [V/m]}\PYG{l+s+s1}{\PYGZsq{}}\PYG{p}{,}\PYG{n}{fontsize}\PYG{o}{=}\PYG{l+m+mi}{16}\PYG{p}{)}
\end{sphinxVerbatim}
}

\end{sphinxuseclass}
\begin{sphinxuseclass}{nboutput}
{

\kern-\sphinxverbatimsmallskipamount\kern-\baselineskip
\kern+\FrameHeightAdjust\kern-\fboxrule
\vspace{\nbsphinxcodecellspacing}

\sphinxsetup{VerbatimColor={named}{white}}
\sphinxsetup{VerbatimBorderColor={named}{nbsphinx-code-border}}
\begin{sphinxuseclass}{output_area}
\begin{sphinxuseclass}{}


\begin{sphinxVerbatim}[commandchars=\\\{\}]
\llap{\color{nbsphinxout}[232]:\,\hspace{\fboxrule}\hspace{\fboxsep}}Text(0, 0.5, 'electric field [V/m]')
\end{sphinxVerbatim}



\end{sphinxuseclass}
\end{sphinxuseclass}
}

\end{sphinxuseclass}
\begin{sphinxuseclass}{nboutput}
\begin{sphinxuseclass}{nblast}
\hrule height -\fboxrule\relax
\vspace{\nbsphinxcodecellspacing}

\makeatletter\setbox\nbsphinxpromptbox\box\voidb@x\makeatother

\begin{nbsphinxfancyoutput}

\begin{sphinxuseclass}{output_area}
\begin{sphinxuseclass}{}
\noindent\sphinxincludegraphics[width=408\sphinxpxdimen,height=298\sphinxpxdimen]{{notebooks_L9_1_Electric_Double_Layer_19_1}.png}

\end{sphinxuseclass}
\end{sphinxuseclass}
\end{nbsphinxfancyoutput}

\end{sphinxuseclass}
\end{sphinxuseclass}
\begin{sphinxuseclass}{nbinput}
\begin{sphinxuseclass}{nblast}
{
\sphinxsetup{VerbatimColor={named}{nbsphinx-code-bg}}
\sphinxsetup{VerbatimBorderColor={named}{nbsphinx-code-border}}
\begin{sphinxVerbatim}[commandchars=\\\{\}]
\llap{\color{nbsphinxin}[ ]:\,\hspace{\fboxrule}\hspace{\fboxsep}}
\end{sphinxVerbatim}
}

\end{sphinxuseclass}
\end{sphinxuseclass}


\sphinxAtStartPar
The following section was created from \sphinxcode{\sphinxupquote{notebooks/L10/1\_van\_der\_Waals.ipynb}}.


\chapter{van der Waals Interactions}
\label{\detokenize{notebooks/L10/1_van_der_Waals:van-der-Waals-Interactions}}\label{\detokenize{notebooks/L10/1_van_der_Waals::doc}}
\sphinxAtStartPar
So far we have considered electrostatic forces between charged and dipolar molecules. We have also introduced interactions which rely on the polarizability of molecules, either just the electronic polarizbility or also an orientational polarization of dipolar molecules.

\sphinxAtStartPar
A part of these interactions showed a specific distance dependence, which was \(r^{-6}\). In particular, we identified the \sphinxstylestrong{Keesom interaction}, i.e. the interaction of two freely rotating dipoles as well as the \sphinxstylestrong{Debye interaction}, i.e. the interaction of an induced dipole with a permanent dipole as two parts of the so\sphinxhyphen{}called van der Waals interaction.

\sphinxAtStartPar
Yet there also interaction between non\sphinxhyphen{}charged and non\sphinxhyphen{}polar molecules and this interaction is also belonging to this class and is specifically called \sphinxstylestrong{dispersion interaction} or \sphinxstylestrong{London interaction}.

\sphinxAtStartPar
As compared to other interactions, van der Waals interactions are typically
\begin{itemize}
\item {} 
\sphinxAtStartPar
long range

\item {} 
\sphinxAtStartPar
attractive and orienting

\item {} 
\sphinxAtStartPar
not additive

\end{itemize}

\begin{sphinxuseclass}{nbinput}
\begin{sphinxuseclass}{nblast}
{
\sphinxsetup{VerbatimColor={named}{nbsphinx-code-bg}}
\sphinxsetup{VerbatimBorderColor={named}{nbsphinx-code-border}}
\begin{sphinxVerbatim}[commandchars=\\\{\}]
\llap{\color{nbsphinxin}[150]:\,\hspace{\fboxrule}\hspace{\fboxsep}}\PYG{k+kn}{import} \PYG{n+nn}{numpy} \PYG{k}{as} \PYG{n+nn}{np}
\PYG{k+kn}{import} \PYG{n+nn}{matplotlib}\PYG{n+nn}{.}\PYG{n+nn}{pyplot} \PYG{k}{as} \PYG{n+nn}{plt}
\PYG{k+kn}{from} \PYG{n+nn}{numpy}\PYG{n+nn}{.}\PYG{n+nn}{linalg} \PYG{k+kn}{import} \PYG{n}{norm}
\PYG{k+kn}{from} \PYG{n+nn}{scipy}\PYG{n+nn}{.}\PYG{n+nn}{constants} \PYG{k+kn}{import} \PYG{n}{c}\PYG{p}{,}\PYG{n}{epsilon\PYGZus{}0}\PYG{p}{,}\PYG{n}{e}\PYG{p}{,}\PYG{n}{physical\PYGZus{}constants}
\PYG{k+kn}{import} \PYG{n+nn}{json}

\PYG{o}{\PYGZpc{}}\PYG{k}{config} InlineBackend.figure\PYGZus{}format = \PYGZsq{}retina\PYGZsq{}

\PYG{k}{with} \PYG{n+nb}{open}\PYG{p}{(}\PYG{l+s+s1}{\PYGZsq{}}\PYG{l+s+s1}{style.json}\PYG{l+s+s1}{\PYGZsq{}}\PYG{p}{,} \PYG{l+s+s1}{\PYGZsq{}}\PYG{l+s+s1}{r}\PYG{l+s+s1}{\PYGZsq{}}\PYG{p}{)} \PYG{k}{as} \PYG{n}{fp}\PYG{p}{:}
    \PYG{n}{style} \PYG{o}{=} \PYG{n}{json}\PYG{o}{.}\PYG{n}{load}\PYG{p}{(}\PYG{n}{fp}\PYG{p}{)}

\PYG{n}{plt}\PYG{o}{.}\PYG{n}{rcParams}\PYG{o}{.}\PYG{n}{update}\PYG{p}{(}\PYG{n}{style}\PYG{p}{)}
\end{sphinxVerbatim}
}

\end{sphinxuseclass}
\end{sphinxuseclass}

\section{Dispersion Interaction}
\label{\detokenize{notebooks/L10/1_van_der_Waals:Dispersion-Interaction}}
\sphinxAtStartPar
The dispersion part in particular, will require a quantum electrodynamic approach, which is beyond the scope of this lecture. We will will consider a much simpler approach at the beginning and later study the more general approach by McLachlan.

\sphinxAtStartPar
Consider first two atoms which have
\begin{itemize}
\item {} 
\sphinxAtStartPar
no time averaged dipole

\item {} 
\sphinxAtStartPar
no residual charge

\end{itemize}

\sphinxAtStartPar
Depite this fact, the atoms may have an instantaneous dipole, which is causing and induced dipole in the other atom. Similarly, the second atom may cause a corresponding dipole in the first atom as well. This will effectively lead to an attractive interaction, which carries the spirit of the dispersion interaction. We can put that into a simple and very crude model based on the most basic semi\sphinxhyphen{}classical description of an atom.

\sphinxAtStartPar
This atom may consist of an electron and proton, which are separated by a distance \(a_0\), which corresponds to the Bohr radius. In this atom, the Coulomb interaction is given by

\sphinxAtStartPar
\begin{equation}
E_{\rm pot}=\frac{e^2}{4\pi \epsilon_0 a_0}
\end{equation}

\sphinxAtStartPar
This potential energy corresponds for an hydrogen atom to the ionization potential \(I=13.6\) eV. To ionize the atom, we can use electromagnetic radiation of the frequency \(\nu=3.3\times 10 ^{15}\) s\(^{-1}\). Thus, in principle a photon of energy \(h\nu=2.2\times 10^{-18}\) J would be sufficient to ionize the atom.

\sphinxAtStartPar
Accordingly, we have
\begin{equation*}
\begin{split}h \nu= \frac{e^2}{4\pi \epsilon_0 a_0}\end{split}
\end{equation*}
\sphinxAtStartPar
or we can write, that the Bohr radius of the electron orbit is given by
\begin{equation*}
\begin{split}a_0=\frac{e^2}{4\pi\epsilon_0 h\nu}\end{split}
\end{equation*}
\sphinxAtStartPar
.

\sphinxAtStartPar
In this simple classical picture the atom has an instantaneous dipole which corresponds to \(u=a_0 e\). This dipole creates a dipole field, that induces a dipole in the second atom. Using our previous findings for the dipole \sphinxhyphen{} induced dipole interaction yields

\sphinxAtStartPar
\begin{equation}
w(r)=-\frac{u^2 \alpha_0}{(4\pi \epsilon_0)^2 r^6}=-\frac{(a_0 e)^2 \alpha_0}{(4\pi \epsilon_0)^2 r^6}
\end{equation}

\sphinxAtStartPar
where \(\alpha_0\) is the electronic polarizability \(\alpha_0=4\pi \epsilon_0 a_0^3\). The latter gives
\begin{equation*}
\begin{split}a_0^{2}=\frac{\alpha_0}{4\pi \epsilon_0 a_0}\end{split}
\end{equation*}
\sphinxAtStartPar
from which we finally find the interaction energy
\begin{equation*}
\begin{split}w(r)=-\frac{a_0^2 e^2 \alpha_0}{(4\pi \epsilon_0)^2 r^6}=-\frac{\alpha_0^2}{(4\pi \epsilon_0)^2}\frac{e^2}{4\pi \epsilon_0 a_0 }\frac{1}{r^6}\end{split}
\end{equation*}
\sphinxAtStartPar
or
\begin{equation*}
\begin{split}w(r)=-\frac{\alpha_0^2 h\nu}{(4\pi \epsilon_0)^2 r^6}\end{split}
\end{equation*}
\sphinxAtStartPar
This simple semi\sphinxhyphen{}classical description corresponds to the result London (up to a factor of 3/4) obtained with a quantum\sphinxhyphen{}mechanical pertubation theory, which is

\sphinxAtStartPar
\begin{equation}
w(r)=-\frac{C_{\rm disp}}{r^6}=-\frac{3}{4}\frac{\alpha_0^2 h\nu}{(4\pi \epsilon_0)^2 r^6}=-\frac{3}{4}\frac{\alpha_0^2I}{(4\pi \epsilon_0)^2 r^6}
\end{equation}

\sphinxAtStartPar
So far, we assumed that both molecules have the same polarizability and thus are of the same type. If this is not the case, the interaction energy is given by

\sphinxAtStartPar
\begin{equation}
w(r)=-\frac{3}{2}\frac{\alpha_{01},\alpha_{02}}{(4\pi \epsilon_0)^2 r^6}\frac{I_1 I_2}{I_1+I_2}
\end{equation}

\begin{sphinxadmonition}{note}{}\unskip
\sphinxAtStartPar
\sphinxstylestrong{Example: Estimating the Boiling Point of Noble Gases}

\sphinxAtStartPar
We can use this formula for the dispersion energy to estimate the boiling point of noble gases.

\sphinxAtStartPar
\begin{equation}
w(r)+E_{\mathrm{kin}}=-\frac{3 \alpha_{0}^{2}}{4\left(4 \pi \epsilon_{0}\right)^{2} \sigma^{6}} h v_{\mathrm{I}}+\frac{3}{2} k_{\mathrm{B}} T_{\mathrm{m}}=0
\end{equation}

\sphinxAtStartPar
For Neon and Argon for example, we have the following parameters:
\begin{itemize}
\item {} 
\sphinxAtStartPar
Ne: \(\sigma=3.08\) Angstroem, \(h\nu_I=21.6\) eV, \(\frac{\alpha_0}{4\pi \epsilon_0}=0.39\times 10^{-30}\) m\(^{-3}\), from which we obtain a boiling temperature of \(T_{\rm b}=22\) K, which nicely corresponds to the experimental value of \(T_{\rm b}=27\) K

\item {} 
\sphinxAtStartPar
Ar: \(\sigma=3.76\) Angstroem, \(h\nu_I=15.8\) eV, \(\frac{\alpha_0}{4\pi \epsilon_0}=1.63\times 10^{-30}\) m\(^{-3}\), from which we obtain a boiling temperature of \(T_{\rm b}=85\) K, which nicely corresponds to the experimental value of \(T_{\rm b}=87\) K

\end{itemize}
\end{sphinxadmonition}

\noindent\sphinxincludegraphics[width=2302\sphinxpxdimen,height=962\sphinxpxdimen]{{c_disp}.png}

\sphinxAtStartPar
While the theory above provides us only with some idea about the disperion interaction, we can now summarize all three controbution to the van der Waals interaction

\sphinxAtStartPar
\begin{equation}
w_{\rm vdW}(r)=-\underbrace{\frac{u_{1}^{2} u_{2}^{2}}{3 k_{\mathrm{B}} T\left(4 \pi \epsilon_{0} \right)^{2} r^{6}}}_{\text {Keesom }}-\underbrace{\frac{u_{1}^{2} \alpha_{02}+u_{2}^{2} \alpha_{01}}{\left(4 \pi \epsilon_{0} \right)^{2} r^{6}}}_{\text {Debye }}-\underbrace{\frac{3 h\nu_1\nu_2 \alpha_{01} \alpha_{02}}{\left(4 \pi \epsilon_{0} \right)^{2} 2(\nu_1+\nu_2)r^{6}}}_{\text {London} }
\end{equation}

\sphinxAtStartPar
or simply

\sphinxAtStartPar
\begin{equation}
w_{\rm vdW}(r)=-\frac{C_{\rm vdW}}{r^6}=-\frac{C_{\rm Debye}+C_{\rm Keesom}+C_{\rm disp}}{r^6}
\end{equation}

\sphinxAtStartPar
The individual contributions have different strength, but it is not difficult to see that the dispersion interaction is typically the biggest one as shown in the table below. The reason for that is the ionization potential and we will address this issue later in the section of the McLachlan theory.

\noindent\sphinxincludegraphics[width=2262\sphinxpxdimen,height=882\sphinxpxdimen]{{comp_vdw}.png}


\section{McLachlan Theory}
\label{\detokenize{notebooks/L10/1_van_der_Waals:McLachlan-Theory}}
\sphinxAtStartPar
The first complete theory for the vdW interaction involving 2 atoms in a medium, was proposed by McLachlan, which is expressed as

\sphinxAtStartPar
\begin{equation}
w_{\mathrm{vdW}}(r)=-\frac{6 k_{\mathrm{B}} T}{\left(4 \pi \epsilon_{0}\right)^{2} r^{6}} \sum_{n=0}^{\infty}{}^{\prime} \frac{\alpha_{1}\left(i v_{n}\right) \alpha_{2}\left(i v_{n}\right)}{\epsilon_{3}^{2}\left(i v_{n}\right)}
\end{equation}

\sphinxAtStartPar
The \(\sum^{\prime}\) notation denotes that the first term in the summation is multiplied by \(1/2\). The frequencies are sampled, only at discreet values that \(h v_{n}=2 \pi k_{\mathrm{B}} \operatorname{Tn}\) (known as the Matsubara frequencies). The typical \(\alpha(i v)-v\) plots of polar and non\sphinxhyphen{}polar molecules using the Lorentz model can be seen in the Figure below.

\noindent\sphinxincludegraphics[width=1394\sphinxpxdimen,height=758\sphinxpxdimen]{{polarizability}.png}


\subsection{Frequency Dependent Polarizability}
\label{\detokenize{notebooks/L10/1_van_der_Waals:Frequency-Dependent-Polarizability}}
\sphinxAtStartPar
The electronic polarizability of a single atom can classically be approximated by a damped harmonic oscillator, i.e.

\sphinxAtStartPar
\begin{equation}
m_{e}\ddot{x}+\Gamma m_e \dot{x}+m_e \omega_0^2  x=-e E(\omega)
\end{equation}

\sphinxAtStartPar
Here \(\omgea_0\) is the resonance frequency, \(m_e\) the electron mass, \(\Gamma\) the damping constant and \(E\) the external electric field. Solving this differential equation and using \(\omega=2\pi \nu\) yields the frequency dependent electronic polarizability

\sphinxAtStartPar
\begin{equation}
\alpha(\nu)=\frac{\alpha_0}{1-i\Gamma \frac{\nu}{\nu_0^2}-\left ( \frac{\nu}{\nu_0}\right)^2}
\end{equation}

\begin{sphinxuseclass}{nbinput}
\begin{sphinxuseclass}{nblast}
{
\sphinxsetup{VerbatimColor={named}{nbsphinx-code-bg}}
\sphinxsetup{VerbatimBorderColor={named}{nbsphinx-code-border}}
\begin{sphinxVerbatim}[commandchars=\\\{\}]
\llap{\color{nbsphinxin}[151]:\,\hspace{\fboxrule}\hspace{\fboxsep}}\PYG{k}{def} \PYG{n+nf}{alpha}\PYG{p}{(}\PYG{n}{nu}\PYG{p}{,}\PYG{n}{nu\PYGZus{}0}\PYG{p}{,}\PYG{n}{gamma}\PYG{p}{)}\PYG{p}{:}
    \PYG{k}{return}\PYG{p}{(}\PYG{l+m+mi}{10}\PYG{o}{/}\PYG{p}{(}\PYG{l+m+mi}{1}\PYG{o}{\PYGZhy{}}\PYG{l+m+mi}{1}\PYG{n}{j}\PYG{o}{*}\PYG{n}{gamma}\PYG{o}{*}\PYG{n}{nu}\PYG{o}{/}\PYG{n}{nu\PYGZus{}0}\PYG{o}{*}\PYG{o}{*}\PYG{l+m+mi}{2}\PYG{o}{\PYGZhy{}}\PYG{p}{(}\PYG{n}{nu}\PYG{o}{/}\PYG{n}{nu\PYGZus{}0}\PYG{p}{)}\PYG{o}{*}\PYG{o}{*}\PYG{l+m+mi}{2}\PYG{p}{)}\PYG{p}{)}
\end{sphinxVerbatim}
}

\end{sphinxuseclass}
\end{sphinxuseclass}
\begin{sphinxuseclass}{nbinput}
\begin{sphinxuseclass}{nblast}
{
\sphinxsetup{VerbatimColor={named}{nbsphinx-code-bg}}
\sphinxsetup{VerbatimBorderColor={named}{nbsphinx-code-border}}
\begin{sphinxVerbatim}[commandchars=\\\{\}]
\llap{\color{nbsphinxin}[152]:\,\hspace{\fboxrule}\hspace{\fboxsep}}\PYG{n}{nu}\PYG{o}{=}\PYG{n}{np}\PYG{o}{.}\PYG{n}{linspace}\PYG{p}{(}\PYG{l+m+mf}{1e5}\PYG{p}{,}\PYG{l+m+mf}{1e17}\PYG{p}{,}\PYG{l+m+mi}{1000}\PYG{p}{)}
\end{sphinxVerbatim}
}

\end{sphinxuseclass}
\end{sphinxuseclass}
\begin{sphinxuseclass}{nbinput}
{
\sphinxsetup{VerbatimColor={named}{nbsphinx-code-bg}}
\sphinxsetup{VerbatimBorderColor={named}{nbsphinx-code-border}}
\begin{sphinxVerbatim}[commandchars=\\\{\}]
\llap{\color{nbsphinxin}[153]:\,\hspace{\fboxrule}\hspace{\fboxsep}}\PYG{n}{plt}\PYG{o}{.}\PYG{n}{figure}\PYG{p}{(}\PYG{n}{figsize}\PYG{o}{=}\PYG{p}{(}\PYG{l+m+mi}{8}\PYG{p}{,}\PYG{l+m+mi}{5}\PYG{p}{)}\PYG{p}{)}

\PYG{n}{nu\PYGZus{}0}\PYG{o}{=}\PYG{l+m+mf}{3.3e15}

\PYG{n}{plt}\PYG{o}{.}\PYG{n}{semilogx}\PYG{p}{(}\PYG{n}{nu}\PYG{p}{,}\PYG{n}{alpha}\PYG{p}{(}\PYG{n}{nu}\PYG{p}{,}\PYG{n}{nu\PYGZus{}0}\PYG{p}{,}\PYG{l+m+mf}{1e15}\PYG{p}{)}\PYG{o}{.}\PYG{n}{real}\PYG{p}{,}\PYG{n}{label}\PYG{o}{=}\PYG{l+s+s2}{\PYGZdq{}}\PYG{l+s+s2}{real}\PYG{l+s+s2}{\PYGZdq{}}\PYG{p}{)}
\PYG{n}{plt}\PYG{o}{.}\PYG{n}{semilogx}\PYG{p}{(}\PYG{n}{nu}\PYG{p}{,}\PYG{n}{alpha}\PYG{p}{(}\PYG{n}{nu}\PYG{p}{,}\PYG{n}{nu\PYGZus{}0}\PYG{p}{,}\PYG{l+m+mf}{1e15}\PYG{p}{)}\PYG{o}{.}\PYG{n}{imag}\PYG{p}{,}\PYG{n}{label}\PYG{o}{=}\PYG{l+s+s2}{\PYGZdq{}}\PYG{l+s+s2}{imag}\PYG{l+s+s2}{\PYGZdq{}}\PYG{p}{)}
\PYG{n}{plt}\PYG{o}{.}\PYG{n}{semilogx}\PYG{p}{(}\PYG{n}{nu}\PYG{p}{,}\PYG{n}{alpha}\PYG{p}{(}\PYG{l+m+mi}{1}\PYG{n}{j}\PYG{o}{*}\PYG{n}{nu}\PYG{p}{,}\PYG{n}{nu\PYGZus{}0}\PYG{p}{,}\PYG{l+m+mf}{1e15}\PYG{p}{)}\PYG{o}{.}\PYG{n}{real}\PYG{p}{,}\PYG{n}{label}\PYG{o}{=}\PYG{l+s+sa}{r}\PYG{l+s+s2}{\PYGZdq{}}\PYG{l+s+s2}{\PYGZdl{}}\PYG{l+s+s2}{\PYGZbs{}}\PYG{l+s+s2}{alpha(i}\PYG{l+s+s2}{\PYGZbs{}}\PYG{l+s+s2}{nu)\PYGZdl{}}\PYG{l+s+s2}{\PYGZdq{}}\PYG{p}{)}
\PYG{n}{plt}\PYG{o}{.}\PYG{n}{xlabel}\PYG{p}{(}\PYG{l+s+sa}{r}\PYG{l+s+s2}{\PYGZdq{}}\PYG{l+s+s2}{frequency \PYGZdl{}}\PYG{l+s+s2}{\PYGZbs{}}\PYG{l+s+s2}{nu\PYGZdl{} [Hz]}\PYG{l+s+s2}{\PYGZdq{}}\PYG{p}{)}
\PYG{n}{plt}\PYG{o}{.}\PYG{n}{ylabel}\PYG{p}{(}\PYG{l+s+sa}{r}\PYG{l+s+s2}{\PYGZdq{}}\PYG{l+s+s2}{\PYGZdl{}}\PYG{l+s+s2}{\PYGZbs{}}\PYG{l+s+s2}{alpha(}\PYG{l+s+s2}{\PYGZbs{}}\PYG{l+s+s2}{nu)\PYGZdl{} [a.u.]}\PYG{l+s+s2}{\PYGZdq{}}\PYG{p}{)}
\PYG{n}{plt}\PYG{o}{.}\PYG{n}{xlim}\PYG{p}{(}\PYG{l+m+mf}{1e12}\PYG{p}{)}
\PYG{n}{plt}\PYG{o}{.}\PYG{n}{legend}\PYG{p}{(}\PYG{p}{)}
\PYG{n}{plt}\PYG{o}{.}\PYG{n}{show}\PYG{p}{(}\PYG{p}{)}
\end{sphinxVerbatim}
}

\end{sphinxuseclass}
\begin{sphinxuseclass}{nboutput}
\begin{sphinxuseclass}{nblast}
\hrule height -\fboxrule\relax
\vspace{\nbsphinxcodecellspacing}

\makeatletter\setbox\nbsphinxpromptbox\box\voidb@x\makeatother

\begin{nbsphinxfancyoutput}

\begin{sphinxuseclass}{output_area}
\begin{sphinxuseclass}{}
\noindent\sphinxincludegraphics[width=534\sphinxpxdimen,height=340\sphinxpxdimen]{{notebooks_L10_1_van_der_Waals_21_0}.png}

\end{sphinxuseclass}
\end{sphinxuseclass}
\end{nbsphinxfancyoutput}

\end{sphinxuseclass}
\end{sphinxuseclass}

\subsubsection{Zero Frequency Contribution}
\label{\detokenize{notebooks/L10/1_van_der_Waals:Zero-Frequency-Contribution}}
\sphinxAtStartPar
At zero frequency, we know that the polarizability reduces to the form of
\begin{equation*}
\begin{split}\alpha=\alpha_{0}+\frac{u^{2} }{3 k_{\mathrm{B}} T}\end{split}
\end{equation*}
\sphinxAtStartPar
which results in the first term of the sum

\begin{aligned}
w_{\mathrm{vdW}}(\nu=0, r) &=-\frac{1}{2} \frac{6 k_{\mathrm{B}} T\left(\alpha_{01}+u_{1}^{2} /\left(3 k_{\mathrm{B}} T\right)\right)\left(\alpha_{02}+u_{2}^{2} /\left(3 k_{\mathrm{B}} T\right)\right)}{\left(4 \pi \epsilon_{0} \epsilon_{3}^{2}\right)^{2} r^{6}} \\
&=-\underbrace{\frac{u_{1}^{2} u_{2}^{2}}{3 k_{\mathrm{B}} T\left(4 \pi \epsilon_{0} \epsilon_{3}\right)^{2} r^{6}}}_{\text {Keesom }}-\underbrace{\frac{u_{1}^{2} \alpha_{02}+u_{2}^{2} \alpha_{01}}{\left(4 \pi \epsilon_{0} \epsilon_{3}\right)^{2} r^{6}}}_{\text {Debye }}-\underbrace{\frac{3 k_{\mathrm{B}} T \alpha_{01} \alpha_{02}}{\left(4 \pi \epsilon_{0} \epsilon_{3}\right)^{2} r^{6}}}_{\text {London(n=0) }}
\end{aligned}

\sphinxAtStartPar
We immediately see the recovery of the Keesom and Debye energies, as well as the \(\alpha_{01} \alpha_{02}\) term from mathematical derivation. In fact, the last part is the zero\sphinxhyphen{}frequency part of the dispersion energy. Comparing the magnitudes of \(k_{\mathrm{B}} T\) and \(h v\), we can see that the zero\sphinxhyphen{}frequency contribution to the dispersion energy is negligible.


\subsection{Optical Frequency Contribution}
\label{\detokenize{notebooks/L10/1_van_der_Waals:Optical-Frequency-Contribution}}
\sphinxAtStartPar
The lowest legal frequency \(h v_{1}=2 \pi k_{\mathrm{B}} T \approx 0.16 \mathrm{eV}\). The permanent dipoles cannot respond to such high frequency, therefore the dipole polarizability has no effect on the dispersion energy at optical frequencies. The electronic polarizability govern the dispersion energy at such frequencies. We first consider \(\epsilon_{3}=1\), that the 2 molecules are in vacuum.

\sphinxAtStartPar
The summation in the original equation can be estimated using continuous integral at optical frequencies, if temperature is very low. Since \(h \mathrm{~d} \nu=2 \pi k_{\mathrm{B}} T \mathrm{~d} n\), we can rewrite the integral from \(n=1\) to

\sphinxAtStartPar
\begin{equation}
w_{\mathrm{vdw}}(\nu>0)=\frac{h}{2 \pi} \frac{6}{\left(4 \pi \epsilon_{0}\right)^{2} r^{6}} \int_{\nu_{1}}^{\infty} \alpha_{1}(i v) \alpha_{2}(i v) \mathrm{d} v
\end{equation}

\sphinxAtStartPar
Using the approximate form of the polarizability
\begin{equation*}
\begin{split}\alpha(i v)=\frac{\alpha_{0}}{\left[1+\left(v / v_{\mathrm{I}}\right)^{2}\right]}\end{split}
\end{equation*}
\sphinxAtStartPar
and \(\epsilon_3=1\) we end up at

\sphinxAtStartPar
\begin{equation}
w(\nu>0, r)=-\frac{3 \alpha_{01} \alpha_{02}}{2\left(4 \pi \epsilon_{0}\right)^{2} r^{6}} \frac{h v_{\mathrm{I} 1} v_{\mathrm{I} 2}}{v_{\mathrm{I} 1}+v_{\mathrm{I} 2}}
\end{equation}

\sphinxAtStartPar
which is finally the London form of dispersion energy. Now we know why this interaction is called “dispersion”. The largest contribution to the energy comes from the range where \(v\) is close to \(v_{\mathrm{I}}\). Since \(v_{\mathrm{I}}\) is usually in UV range, such interaction is dominated by the polarizabilities from Vis to UV frequencies. As the polarizability and permittivity are closely related to the dispersion of light, it is not hard to understand why it is originally
coined as “dispersion interaction”.



\sphinxAtStartPar
The following section was created from \sphinxcode{\sphinxupquote{notebooks/L11/1\_van\_der\_Waals.ipynb}}.


\chapter{van der Waals Interaction}
\label{\detokenize{notebooks/L11/1_van_der_Waals:van-der-Waals-Interaction}}\label{\detokenize{notebooks/L11/1_van_der_Waals::doc}}
\sphinxAtStartPar
The semiclassical theory we have developed in the last lecture can also be extended to objects and bodies which are embedded in media. We therefore consider a spherical particle of radius \(a_i\) of dielectric constant \(\epsilon_i\) in a mediums of dielectric constant \(\epsilon\). If an external electric field is applied, that the particle is polarized to obtain a dipole

\sphinxAtStartPar
\begin{equation}
u_{\rm ind}=4\pi \epsilon_0 \epsilon \frac{\epsilon_i-\epsilon}{\epsilon_i+2\epsilon}a_i^3E
\end{equation}

\sphinxAtStartPar
This result is obtained as the spheres polarization is also influenced by the electric field generated in the medium around the sphere. This local electric field that is relevant has been considered by \sphinxstylestrong{Clausius and Mossotti}.

\sphinxAtStartPar
According to \(u_{\rm ind}=\alpha E\), we readily obtain the polarizability

\sphinxAtStartPar
\begin{equation}
\alpha(\nu)=4\pi \epsilon_0 \epsilon_3(\nu)\frac{\epsilon_1(\nu)-\epsilon_3(\nu)}{\epsilon_1(\nu)+2\epsilon_3(\nu)}a_1^3
\end{equation}

\sphinxAtStartPar
This result shows that the dielectric function of the particle \(\epsilon_1\) needs to be different from the environment to obtain a contribution, which justifies the name excess polarizability.

\sphinxAtStartPar
Using this polarizability, we can write down the the van der Waals interaction energy by McLachlan as

\sphinxAtStartPar
\sphinxstylestrong{Zero Frequency}

\sphinxAtStartPar
\begin{equation}
w(r)=-\frac{3k_B T a_{1}^3a_2^3}{r^6}\left ( \frac{\epsilon_1(0)-\epsilon_3(0)}{\epsilon_1(0)+2\epsilon_3(0)}\right )\left ( \frac{\epsilon_2(0)-\epsilon_3(0)}{\epsilon_2(0)+2\epsilon_3(0)}\right )
\end{equation}

\sphinxAtStartPar
and at

\sphinxAtStartPar
\sphinxstylestrong{Optical Frequencies}

\sphinxAtStartPar
\begin{equation}
w(r)=-\frac{3h a_{1}^3a_2^3}{\pi r^6}\int_{\nu_1}^\infty\left ( \frac{\epsilon_1(i\nu)-\epsilon_3(i\nu)}{\epsilon_1(i\nu)+2\epsilon_3(i\nu)}\right )\left ( \frac{\epsilon_2(i\nu)-\epsilon_3(i\nu)}{\epsilon_2(i\nu)+2\epsilon_3(i\nu)}\right )d\nu
\end{equation}

\sphinxAtStartPar
To calculate the interaction energy we therefore need to know the dielectric function over the whole frequency range, which is often not easily accessible. Yet, if there is a strong absorption of the material at \(\nu_e\) then the dielectric function can be written as
\begin{equation*}
\begin{split}\epsilon(\nu)=1+\frac{n^2-1}{1-\left (\frac{\nu}{\nu_e}\right )^2}\end{split}
\end{equation*}
\sphinxAtStartPar
which can be inserted and integrated in the above equations as a replacement.

\begin{sphinxadmonition}{warning}{}\unskip
\sphinxAtStartPar
\sphinxstylestrong{van der Waals interactions}

\sphinxAtStartPar
Overall the van der Waals interactions have the following general properties
\begin{itemize}
\item {} 
\sphinxAtStartPar
since \(h\nu_e >>k_B T\) the dispersion interaction is much bigger that the rest of the interaction. For \(\epsilon=n^2\approx 2\), and \(\nu_e=3\times 10^{15}\) s\(^{-1}\) we find \(h\nu_e/2\sqrt{3}k_B T\approx 140\)

\item {} 
\sphinxAtStartPar
the van der Waals interaction is greatly reduced in a solvent as compared to vacuum. For \(n_1=n_2=1.5\) and \(n_3=1.4\) we obtain a factor of 32 as compared to \(n_3=1\)

\item {} 
\sphinxAtStartPar
the van der Waals force between like molecules is always attractive. The dispersion part can be repulsive, however, if \(n_3\) is between \(n_1\) and \(n_2\)

\item {} 
\sphinxAtStartPar
the van der Waals interaction is not additive

\item {} 
\sphinxAtStartPar
the dispersion interaction shows retardation effects, at distances larger than \(100\) nm, it decays with \(r^{-7}\)

\end{itemize}
\end{sphinxadmonition}


\section{Interaction Between Macroscopic Bodies}
\label{\detokenize{notebooks/L11/1_van_der_Waals:Interaction-Between-Macroscopic-Bodies}}
\sphinxAtStartPar
The above equations, that we derived before are essentially valid between small objects at distances larger than their size. If objects become macroscopic, we have to consider the interaction of all components of a body with all components of the second body. To calculate those interactions, we will assume that all interactions are additive, while we know that this is not completely true for van der Waals forces.

\sphinxAtStartPar
Our general law of interaction shall be
\begin{equation*}
\begin{split}w(r)=-\frac{C}{r^n}\end{split}
\end{equation*}
\sphinxAtStartPar
where \(C\) is the interaction constant, which comprises all additional constants in the interaction law.

\sphinxAtStartPar
\sphinxstylestrong{a) a single molecule in front of a wall}

\noindent\sphinxincludegraphics[width=411\sphinxpxdimen,height=436\sphinxpxdimen]{{molecule_body}.png}

\sphinxAtStartPar
According to the image above we see that the single molecule interacts with all molecules in a tiny volume element at a distance \(r=\sqrt{x^2+y^2}\) in the same way. This cylindral volume is given by
\begin{equation*}
\begin{split}2\pi x dx dz\end{split}
\end{equation*}
\sphinxAtStartPar
and contains, based on the number density \(\rho\) of molecules in the wall and amount of \(2\pi x dx dz\rho\) molecules. We then only have to sum up all contributions of shells with different \(x\) and different \(z\) which readily yields

\sphinxAtStartPar
\begin{eqnarray}
w(D)&=&-2\pi C\rho\int_z=D^{\infty} dz \int_{x=0}^{\infty}\frac{x dx}{(z^2+x^2)^(n/2)}=-\frac{2\pi C \rho}{n-2}\int_{D}^{\infty}\frac{dz}{z^{n-2}}\\
&=&-\frac{2\pi C \rho}{(n-2)(n-3)D^{n-3}}\; \forall\; n>3
\end{eqnarray}

\sphinxAtStartPar
Thus, if we would use the van der Waals interaction with \(n=6\) we obtain

\sphinxAtStartPar
\begin{equation}
w(D)=-\frac{\pi C \rho}{6D^3}
\end{equation}

\sphinxAtStartPar
which decays much weaker than the original power law of the van der Waals interaction law. This is a very general results, that the interactions law, which were obtained for point\sphinxhyphen{}like objects turn now into distance dependences that are more complex due to the extended shape of macroscopic bodies.

\sphinxAtStartPar
\sphinxstylestrong{b) Interaction of a sphere with a wall}

\sphinxAtStartPar
If we now turn now to a sphere in front of a wall, we have to add up all contributions of molecules in the sphere, which are at a certain distance \(D\). Each of these molecules contributes according to a), so we have to add interactions of slices of the sphere, which are at a distance \(D+z\).

\noindent\sphinxincludegraphics[width=357\sphinxpxdimen,height=301\sphinxpxdimen]{{sphere_wall}.png}

\sphinxAtStartPar
According to the figure, the radius of the slices is given by
\begin{equation*}
\begin{split}x^2=(2R-z)z\end{split}
\end{equation*}
\sphinxAtStartPar
which gives a volume of the slice, which is \(\pi x^2 dz=\pi (2R-z)z dz\) that finally gives the number of molecules \(\rho \pi (2R-z)dz\) with \(\rho\) being the number density of wall and sphere. Summing now up over all slices at a distance \(z\) results in

\sphinxAtStartPar
\begin{equation}
w(D)=-\frac{2\pi^2 C \rho^2}{(n-2)(n-3)}\int_0^{2R}\frac{(2R-z)zdz}{(D+z)^{n-3}}
\end{equation}

\sphinxAtStartPar
This results can be considered in different limits.

\sphinxAtStartPar
For \(D\ll R\) we have mainly contributions from \(z\approx D\), which yield then

\sphinxAtStartPar
\begin{equation}
w(D)=-\frac{4\pi^2 C \rho^2 R}{(n-2)(n-3)(n-4)(n-5)D^{n-5}}
\end{equation}

\sphinxAtStartPar
which would give for van der Waals interactions

\sphinxAtStartPar
\begin{equation}
w(D)=-\frac{\pi^2 C \rho^2 R}{6D}
\end{equation}

\sphinxAtStartPar
which reveals a much weaker distance dependence than the original formula for the van der Waals interaction. This is again a general result for the interaction for the interaction of bodies at small distances.

\sphinxAtStartPar
If the seperation of sphere and wall is much larger than the sphere itself \(D\gg R\), we can approximate the above integral with

\sphinxAtStartPar
\begin{equation}
w(D)=-\frac{2\pi C \rho (4\pi R^3 \rho/3 )}{(n-2)(n-3)D^{n-3}}
\end{equation}


\section{Lifshitz Theory}
\label{\detokenize{notebooks/L11/1_van_der_Waals:Lifshitz-Theory}}


\sphinxAtStartPar
The following section was created from \sphinxcode{\sphinxupquote{notebooks/L12/1\_depletion\_forces.ipynb}}.


\chapter{Depletion Forces}
\label{\detokenize{notebooks/L12/1_depletion_forces:Depletion-Forces}}\label{\detokenize{notebooks/L12/1_depletion_forces::doc}}
\sphinxAtStartPar
Depletion forces are entropic forces which arise from osmotic pressure differences. These osmotic pressure differences are the result of regions in space which are excludeded from the access of certain components of a solution. In the simplest case, the a component of s solution is excluded due to steric interactions, i.e. the component does not fit into the region due to its size.

\begin{sphinxuseclass}{nbinput}
\begin{sphinxuseclass}{nblast}
{
\sphinxsetup{VerbatimColor={named}{nbsphinx-code-bg}}
\sphinxsetup{VerbatimBorderColor={named}{nbsphinx-code-border}}
\begin{sphinxVerbatim}[commandchars=\\\{\}]
\llap{\color{nbsphinxin}[55]:\,\hspace{\fboxrule}\hspace{\fboxsep}}\PYG{k+kn}{import} \PYG{n+nn}{numpy} \PYG{k}{as} \PYG{n+nn}{np}
\PYG{k+kn}{import} \PYG{n+nn}{matplotlib}\PYG{n+nn}{.}\PYG{n+nn}{pyplot} \PYG{k}{as} \PYG{n+nn}{plt}
\PYG{k+kn}{from} \PYG{n+nn}{numpy}\PYG{n+nn}{.}\PYG{n+nn}{linalg} \PYG{k+kn}{import} \PYG{n}{norm}
\PYG{k+kn}{from} \PYG{n+nn}{scipy}\PYG{n+nn}{.}\PYG{n+nn}{constants} \PYG{k+kn}{import} \PYG{n}{c}\PYG{p}{,}\PYG{n}{epsilon\PYGZus{}0}\PYG{p}{,}\PYG{n}{e}\PYG{p}{,}\PYG{n}{physical\PYGZus{}constants}
\PYG{k+kn}{import} \PYG{n+nn}{json}

\PYG{o}{\PYGZpc{}}\PYG{k}{config} InlineBackend.figure\PYGZus{}format = \PYGZsq{}retina\PYGZsq{}

\PYG{k}{with} \PYG{n+nb}{open}\PYG{p}{(}\PYG{l+s+s1}{\PYGZsq{}}\PYG{l+s+s1}{style.json}\PYG{l+s+s1}{\PYGZsq{}}\PYG{p}{,} \PYG{l+s+s1}{\PYGZsq{}}\PYG{l+s+s1}{r}\PYG{l+s+s1}{\PYGZsq{}}\PYG{p}{)} \PYG{k}{as} \PYG{n}{fp}\PYG{p}{:}
    \PYG{n}{style} \PYG{o}{=} \PYG{n}{json}\PYG{o}{.}\PYG{n}{load}\PYG{p}{(}\PYG{n}{fp}\PYG{p}{)}

\PYG{n}{plt}\PYG{o}{.}\PYG{n}{rcParams}\PYG{o}{.}\PYG{n}{update}\PYG{p}{(}\PYG{n}{style}\PYG{p}{)}
\end{sphinxVerbatim}
}

\end{sphinxuseclass}
\end{sphinxuseclass}

\section{Depletion force between two plates}
\label{\detokenize{notebooks/L12/1_depletion_forces:Depletion-force-between-two-plates}}
\sphinxAtStartPar
Consider the picture below, where colloids are contained in a suspension together with two plates. Then there is a region, which is inaccessible by the centers of the colloids of radius \(\sigma/2\) that is indicated by the dahsed lines. If these regions between the plates overlap, not colloid can enter this region and the number density of the colloids is different inside as compared to the outside. This creates an osmotic pressure difference.

\noindent\sphinxincludegraphics[width=346\sphinxpxdimen,height=291\sphinxpxdimen]{{depletion_plates}.png}

\sphinxAtStartPar
The osmotic pressure difference can be calculated with the help of the van’t Hoff equation for the osmotic pressure \(\Pi=nk_B T\), where \(n\) is the number density. In the case the distance of the two plates is larger than the diamater of the two plates, \(h>\sigma\), we have \(\Pi_i=\Pi_o= nk_B T\) so the difference \(P=\Pi_i-\Pi_o=0\), so there is not pressure difference. In the case when there is no colloid between the plates we have \(\Pi_i =0\) and
\(\Pi_o =nk_B T\) such that

\sphinxAtStartPar
\begin{equation}
P=\Pi_i-\Pi_o=-nk_B T
\end{equation}

\sphinxAtStartPar
Thus there is an effective pressure that is compressing the two plates. As the force per area that is required is the pressure and thus

\sphinxAtStartPar
\begin{equation}
P=-\frac{dw}{dh}
\end{equation}

\sphinxAtStartPar
we can calculate the energy per area that is needed to seperate the two plates to a distance \(h\). This is then found to be linear in the distance \(h\)

\sphinxAtStartPar
\begin{equation}
w(h)=-n k_B T (\sigma-h)
\end{equation}

\sphinxAtStartPar
for all \(h<\sigma\), while it is zero for \(h>\sigma\).

\begin{sphinxuseclass}{nbinput}
\begin{sphinxuseclass}{nblast}
{
\sphinxsetup{VerbatimColor={named}{nbsphinx-code-bg}}
\sphinxsetup{VerbatimBorderColor={named}{nbsphinx-code-border}}
\begin{sphinxVerbatim}[commandchars=\\\{\}]
\llap{\color{nbsphinxin}[10]:\,\hspace{\fboxrule}\hspace{\fboxsep}}\PYG{k}{def} \PYG{n+nf}{depletion}\PYG{p}{(}\PYG{n}{h}\PYG{p}{,}\PYG{n}{sigma}\PYG{p}{,}\PYG{n}{n}\PYG{p}{)}\PYG{p}{:}
    \PYG{k}{return}\PYG{p}{(}\PYG{n}{np}\PYG{o}{.}\PYG{n}{where}\PYG{p}{(}\PYG{n}{h}\PYG{o}{\PYGZlt{}}\PYG{n}{sigma}\PYG{p}{,} \PYG{o}{\PYGZhy{}}\PYG{n}{n}\PYG{o}{*}\PYG{p}{(}\PYG{n}{sigma}\PYG{o}{\PYGZhy{}}\PYG{n}{h}\PYG{p}{)}\PYG{p}{,}\PYG{l+m+mi}{0}\PYG{p}{)}\PYG{p}{)}
\end{sphinxVerbatim}
}

\end{sphinxuseclass}
\end{sphinxuseclass}
\begin{sphinxuseclass}{nbinput}
\begin{sphinxuseclass}{nblast}
{
\sphinxsetup{VerbatimColor={named}{nbsphinx-code-bg}}
\sphinxsetup{VerbatimBorderColor={named}{nbsphinx-code-border}}
\begin{sphinxVerbatim}[commandchars=\\\{\}]
\llap{\color{nbsphinxin}[16]:\,\hspace{\fboxrule}\hspace{\fboxsep}}\PYG{n}{sigma}\PYG{o}{=}\PYG{l+m+mi}{1}
\PYG{n}{h}\PYG{o}{=}\PYG{n}{np}\PYG{o}{.}\PYG{n}{linspace}\PYG{p}{(}\PYG{l+m+mi}{0}\PYG{p}{,}\PYG{l+m+mi}{5}\PYG{p}{,}\PYG{l+m+mi}{100}\PYG{p}{)}
\end{sphinxVerbatim}
}

\end{sphinxuseclass}
\end{sphinxuseclass}
\begin{sphinxuseclass}{nbinput}
{
\sphinxsetup{VerbatimColor={named}{nbsphinx-code-bg}}
\sphinxsetup{VerbatimBorderColor={named}{nbsphinx-code-border}}
\begin{sphinxVerbatim}[commandchars=\\\{\}]
\llap{\color{nbsphinxin}[17]:\,\hspace{\fboxrule}\hspace{\fboxsep}}\PYG{n}{plt}\PYG{o}{.}\PYG{n}{plot}\PYG{p}{(}\PYG{n}{h}\PYG{p}{,}\PYG{n}{depletion}\PYG{p}{(}\PYG{n}{h}\PYG{p}{,}\PYG{n}{sigma}\PYG{p}{,}\PYG{l+m+mi}{5}\PYG{p}{)}\PYG{p}{)}
\PYG{n}{plt}\PYG{o}{.}\PYG{n}{xlabel}\PYG{p}{(}\PYG{l+s+sa}{r}\PYG{l+s+s2}{\PYGZdq{}}\PYG{l+s+s2}{distance h \PYGZdl{}[}\PYG{l+s+s2}{\PYGZbs{}}\PYG{l+s+s2}{mu m]\PYGZdl{}}\PYG{l+s+s2}{\PYGZdq{}}\PYG{p}{)}
\PYG{n}{plt}\PYG{o}{.}\PYG{n}{ylabel}\PYG{p}{(}\PYG{l+s+s2}{\PYGZdq{}}\PYG{l+s+s2}{energy per area \PYGZdl{}[k\PYGZus{}B T]\PYGZdl{}}\PYG{l+s+s2}{\PYGZdq{}}\PYG{p}{)}
\PYG{n}{plt}\PYG{o}{.}\PYG{n}{ylim}\PYG{p}{(}\PYG{o}{\PYGZhy{}}\PYG{l+m+mi}{5}\PYG{p}{,}\PYG{l+m+mi}{5}\PYG{p}{)}
\PYG{n}{plt}\PYG{o}{.}\PYG{n}{show}\PYG{p}{(}\PYG{p}{)}
\end{sphinxVerbatim}
}

\end{sphinxuseclass}
\begin{sphinxuseclass}{nboutput}
\begin{sphinxuseclass}{nblast}
\hrule height -\fboxrule\relax
\vspace{\nbsphinxcodecellspacing}

\makeatletter\setbox\nbsphinxpromptbox\box\voidb@x\makeatother

\begin{nbsphinxfancyoutput}

\begin{sphinxuseclass}{output_area}
\begin{sphinxuseclass}{}
\noindent\sphinxincludegraphics[width=412\sphinxpxdimen,height=285\sphinxpxdimen]{{notebooks_L12_1_depletion_forces_5_0}.png}

\end{sphinxuseclass}
\end{sphinxuseclass}
\end{nbsphinxfancyoutput}

\end{sphinxuseclass}
\end{sphinxuseclass}

\section{Depletion force between two spheres}
\label{\detokenize{notebooks/L12/1_depletion_forces:Depletion-force-between-two-spheres}}
\sphinxAtStartPar
To describe the depletion interaction between two spheres (radius \(R\)) in a solution of smaller spheres (radius \(σ/2\)), we may use the same approach of assuming an isotropic osmotic pressure of the smaller spheres. The smaller spheres can actually not approach a spherical shell of thickness \(σ/2\) around the larger spheres or a total volume of \(4π(R+σ/2)^3/3\) . The pressure on the large spheres has no consequences except in the case where the two spheres approach closer
than \(r≤2(R+σ/2)=2R_d\). In this case the two excluded volumes around each sphere overlap to a lens\sphinxhyphen{}like volume. Due to this overall, the forces which create the pressure on the spherical surface up to an angle \(θ_0\) are unbalanced from the other side of the surface (see image) and result in an attractive interaction. The surface element between \(\theta\) and \(\theta d\theta\) is then given by

\sphinxAtStartPar
\begin{equation}
2\pi R_{d}^2 \sin(\theta)d\theta
\end{equation}

\sphinxAtStartPar
but only the force components along the connecting line contribute.

\noindent\sphinxincludegraphics[width=544\sphinxpxdimen,height=300\sphinxpxdimen]{{depletion_spheres}.png}

\sphinxAtStartPar
The total force is then calculated from the osmotic pressure times the surface area of the spherical cap by just taking into account the force components along the connecting line (an additional factor \(\cos(\theta)\)). This the read

\sphinxAtStartPar
\begin{equation}
F(r)=-2nk_{\rm B}T\pi (R+\sigma/2)^2\int\limits_{0}^{\theta_{0}}\sin(\theta)\cos(\theta)\mathrm d\theta
\end{equation}

\sphinxAtStartPar
resulting in

\sphinxAtStartPar
\begin{equation}
F(r)=-nk_{\rm B}T\pi (R_{d})^2\left [ 1+\left ( \frac{r}{2R_{d}}\right )^2\right ]
\end{equation}

\sphinxAtStartPar
which is valid if the distance between the centers of the large spheres is \(r<2(R+σ/2)\). For \(r≥2(R+σ/2)\) there is no depletion of the smaller spheres from the region between the larger spheres and the force is zero, i.e., \(F(r)=0\).

\sphinxAtStartPar
Calculating again the interction free energy \(w(r)\), we have to integrate the force betwen \(r\) and \(2R_{d}\), which is given by
\begin{equation*}
\begin{split}w(r)=\int_r^{2R_d}F(r)dr=-nk_B T V_{ov}(r)\end{split}
\end{equation*}
\sphinxAtStartPar
where
\begin{equation*}
\begin{split}V_{ov}=\frac{4\pi}{3}R_{d}^3\left [ 1-\frac{3}{4}\frac{r}{R_d}+\frac{1}{16}\left ( \frac{r}{R_d}\right )^3 \right ]\end{split}
\end{equation*}
\begin{sphinxuseclass}{nbinput}
\begin{sphinxuseclass}{nblast}
{
\sphinxsetup{VerbatimColor={named}{nbsphinx-code-bg}}
\sphinxsetup{VerbatimBorderColor={named}{nbsphinx-code-border}}
\begin{sphinxVerbatim}[commandchars=\\\{\}]
\llap{\color{nbsphinxin}[38]:\,\hspace{\fboxrule}\hspace{\fboxsep}}\PYG{k}{def} \PYG{n+nf}{depletion\PYGZus{}sphere}\PYG{p}{(}\PYG{n}{r}\PYG{p}{,}\PYG{n}{R}\PYG{p}{,}\PYG{n}{sigma}\PYG{p}{,}\PYG{n}{n}\PYG{p}{)}\PYG{p}{:}
    \PYG{n}{Rd}\PYG{o}{=}\PYG{l+m+mi}{2}\PYG{o}{*}\PYG{n}{R}\PYG{o}{+}\PYG{n}{sigma}
    \PYG{n}{Vov}\PYG{o}{=}\PYG{l+m+mi}{4}\PYG{o}{*}\PYG{n}{np}\PYG{o}{.}\PYG{n}{pi}\PYG{o}{*}\PYG{n}{Rd}\PYG{o}{*}\PYG{o}{*}\PYG{l+m+mi}{3}\PYG{o}{/}\PYG{l+m+mi}{3}\PYG{o}{*}\PYG{p}{(}\PYG{l+m+mi}{1}\PYG{o}{\PYGZhy{}}\PYG{l+m+mi}{3}\PYG{o}{*}\PYG{n}{r}\PYG{o}{/}\PYG{l+m+mi}{4}\PYG{o}{/}\PYG{n}{Rd}\PYG{o}{+}\PYG{p}{(}\PYG{n}{r}\PYG{o}{/}\PYG{n}{Rd}\PYG{p}{)}\PYG{o}{*}\PYG{o}{*}\PYG{l+m+mi}{3}\PYG{o}{/}\PYG{l+m+mi}{16}\PYG{p}{)}
    \PYG{k}{return}\PYG{p}{(}\PYG{n}{np}\PYG{o}{.}\PYG{n}{where}\PYG{p}{(}\PYG{n}{r}\PYG{o}{\PYGZgt{}}\PYG{o}{=}\PYG{l+m+mi}{2}\PYG{o}{*}\PYG{n}{R}\PYG{p}{,}\PYG{n}{np}\PYG{o}{.}\PYG{n}{where}\PYG{p}{(}\PYG{n}{r}\PYG{o}{\PYGZlt{}}\PYG{l+m+mi}{2}\PYG{o}{*}\PYG{n}{R}\PYG{o}{+}\PYG{n}{sigma}\PYG{p}{,} \PYG{o}{\PYGZhy{}}\PYG{n}{n}\PYG{o}{*}\PYG{n}{Vov}\PYG{p}{,}\PYG{l+m+mi}{0}\PYG{p}{)}\PYG{p}{,}\PYG{l+m+mi}{0}\PYG{p}{)}\PYG{p}{)}
\end{sphinxVerbatim}
}

\end{sphinxuseclass}
\end{sphinxuseclass}
\begin{sphinxuseclass}{nbinput}
\begin{sphinxuseclass}{nblast}
{
\sphinxsetup{VerbatimColor={named}{nbsphinx-code-bg}}
\sphinxsetup{VerbatimBorderColor={named}{nbsphinx-code-border}}
\begin{sphinxVerbatim}[commandchars=\\\{\}]
\llap{\color{nbsphinxin}[53]:\,\hspace{\fboxrule}\hspace{\fboxsep}}\PYG{n}{sigma}\PYG{o}{=}\PYG{l+m+mi}{2}
\PYG{n}{R}\PYG{o}{=}\PYG{l+m+mi}{5}
\PYG{n}{n}\PYG{o}{=}\PYG{l+m+mf}{1e\PYGZhy{}3}
\PYG{n}{r}\PYG{o}{=}\PYG{n}{np}\PYG{o}{.}\PYG{n}{linspace}\PYG{p}{(}\PYG{l+m+mi}{2}\PYG{o}{*}\PYG{n}{R}\PYG{p}{,}\PYG{l+m+mi}{15}\PYG{p}{,}\PYG{l+m+mi}{500}\PYG{p}{)}
\end{sphinxVerbatim}
}

\end{sphinxuseclass}
\end{sphinxuseclass}
\begin{sphinxuseclass}{nbinput}
{
\sphinxsetup{VerbatimColor={named}{nbsphinx-code-bg}}
\sphinxsetup{VerbatimBorderColor={named}{nbsphinx-code-border}}
\begin{sphinxVerbatim}[commandchars=\\\{\}]
\llap{\color{nbsphinxin}[54]:\,\hspace{\fboxrule}\hspace{\fboxsep}}\PYG{n}{plt}\PYG{o}{.}\PYG{n}{plot}\PYG{p}{(}\PYG{n}{r}\PYG{p}{,}\PYG{n}{depletion\PYGZus{}sphere}\PYG{p}{(}\PYG{n}{r}\PYG{p}{,}\PYG{n}{R}\PYG{p}{,}\PYG{n}{sigma}\PYG{p}{,}\PYG{n}{n}\PYG{p}{)}\PYG{p}{)}
\PYG{n}{plt}\PYG{o}{.}\PYG{n}{xlabel}\PYG{p}{(}\PYG{l+s+sa}{r}\PYG{l+s+s2}{\PYGZdq{}}\PYG{l+s+s2}{distance r \PYGZdl{}[}\PYG{l+s+s2}{\PYGZbs{}}\PYG{l+s+s2}{mu m]\PYGZdl{}}\PYG{l+s+s2}{\PYGZdq{}}\PYG{p}{)}
\PYG{n}{plt}\PYG{o}{.}\PYG{n}{ylabel}\PYG{p}{(}\PYG{l+s+s2}{\PYGZdq{}}\PYG{l+s+s2}{energy \PYGZdl{}[k\PYGZus{}B T]\PYGZdl{}}\PYG{l+s+s2}{\PYGZdq{}}\PYG{p}{)}
\PYG{n}{plt}\PYG{o}{.}\PYG{n}{ylim}\PYG{p}{(}\PYG{o}{\PYGZhy{}}\PYG{l+m+mi}{5}\PYG{p}{,}\PYG{l+m+mi}{5}\PYG{p}{)}
\PYG{n}{plt}\PYG{o}{.}\PYG{n}{show}\PYG{p}{(}\PYG{p}{)}
\end{sphinxVerbatim}
}

\end{sphinxuseclass}
\begin{sphinxuseclass}{nboutput}
\begin{sphinxuseclass}{nblast}
\hrule height -\fboxrule\relax
\vspace{\nbsphinxcodecellspacing}

\makeatletter\setbox\nbsphinxpromptbox\box\voidb@x\makeatother

\begin{nbsphinxfancyoutput}

\begin{sphinxuseclass}{output_area}
\begin{sphinxuseclass}{}
\noindent\sphinxincludegraphics[width=412\sphinxpxdimen,height=285\sphinxpxdimen]{{notebooks_L12_1_depletion_forces_9_0}.png}

\end{sphinxuseclass}
\end{sphinxuseclass}
\end{nbsphinxfancyoutput}

\end{sphinxuseclass}
\end{sphinxuseclass}

\section{General description}
\label{\detokenize{notebooks/L12/1_depletion_forces:General-description}}
\sphinxAtStartPar
A more general description of the depletion interaction may be obtained based on our introduction into statistical physics at the beginning of the course. There we stated that the probability of finding a system in a stet of energy \(E\) is

\sphinxAtStartPar
\begin{equation}
p(E)=\frac{e^{-\beta E}}{Z},
\end{equation}

\sphinxAtStartPar
where \(\beta=1/k_{\rm B}T\) and the partition function \(Z\) is

\sphinxAtStartPar
\begin{equation}
Z=\sum_{i}e^{-\beta E_{i}}
\end{equation}

\sphinxAtStartPar
for a system with a discrete number of states numbered by the indexi.For a single particle with a continuous number of energies the energy may look like

\sphinxAtStartPar
\begin{equation}
E=\frac{p^2}{2m}+U(q),
\end{equation}

\sphinxAtStartPar
where \(q\) is some general coordinate. The classical partition function for a system of N particles is then

\sphinxAtStartPar
\begin{equation}
Z=\frac{1}{N!h^{3N}}\int \mathrm d^{3}p^{N}\int \mathrm d^{3}q^{N}\exp\left ( -\beta \left [ \sum_{i}\frac{p^2}{2m}+U(q^N)\right ]\right ),
\end{equation}

\sphinxAtStartPar
where we use the notation \(p^N\) and \(q^N\) to denote the whole set of variables. The prefactor is appropriate for indistinguishable particles and the phase space normalization factor \(h\) (Planck’s constant) for every pair of \(p\) and \(q\). Since there is no issue of the non\sphinxhyphen{}commutation of positions and momenta we can perform the momentum integrals exactly, yielding

\sphinxAtStartPar
\begin{equation}
Z=\frac{1}{N!}\frac{1}{\Lambda^{3N}}\int \mathrm d^{3}q^{N}\exp(-\beta U(q^{N})),
\end{equation}

\sphinxAtStartPar
where the thermal de Broglie wavelength \(Λ\) is

\sphinxAtStartPar
\begin{equation}
\Lambda=\frac{h}{\sqrt{2 \pi m k_{\rm B}T}}.
\end{equation}

\sphinxAtStartPar
In the case of two interacting large spheres in a solution of small spheres with all spheres being hard spheres, the integration over the exponential function yields

\sphinxAtStartPar
\begin{equation}
Z=\frac{V_{A}^{N}}{N! \Lambda^{3N}}
\end{equation}

\sphinxAtStartPar
with \(V_A\) being the volume available to the small spheres, i.e., \$V\_A=V\sphinxhyphen{}V\_\{E\}\textasciicircum{}\{’\} \$ for \(r<D+d\) and \(V_A=V-V_E\) for \(r>D+d\) with \(V_E=π(D+d)^3/3\) and \(V_{E}^{'}=V_E-(2πl^2)/3 [3(D+d)/2-l]\) and \(l=(D+d)/2-r/2\). The calculation then yields the free energy

\sphinxAtStartPar
\begin{equation}
G=-k_{\rm B}T\ln(Z) =-k_{\rm B}T\ln\left (\frac{V_{A}^N}{N!\Lambda^{3N}}\right).
\end{equation}

\sphinxAtStartPar
Using Stirling’s formula again this can be turned into

\sphinxAtStartPar
\begin{equation}
G=G_{\rm ideal}-Nk_{\rm B}\ln\left (\frac{V_{A}}{V}\right).
\end{equation}

\sphinxAtStartPar
The ideal contribution to the free energy is constant with the separation of the large sphere, so it does not contribute to the depletion force. It reads

\sphinxAtStartPar
\begin{equation}
G_{\rm ideal}=-Nk_{\rm B}T\left (1-\ln\left (\frac{N\Lambda^3}{V}\right)\right).
\end{equation}

\sphinxAtStartPar
The distance\sphinxhyphen{}dependent part still contains the logarithm which we can approximate by

\sphinxAtStartPar
\begin{equation}
\ln\left (\frac{V_{A}}{V}\right)\approx -\frac{V_{E}}{V}+\frac{\pi}{6V}(D+d-r)^2(D+d+r/2)
\end{equation}

\sphinxAtStartPar
for the case of the overlapping excluded volumes. This gives then finally a force

\sphinxAtStartPar
\begin{equation}
F=-\frac{N}{4V}k_{\rm B}T\pi(D+d-r)(D+d+r)
\end{equation}

\sphinxAtStartPar
for \(r<d+D\). For all other distances of the two centers of the spheres, the depletion force is zero.



\sphinxAtStartPar
The following section was created from \sphinxcode{\sphinxupquote{notebooks/L16/1\_langevin.ipynb}}.


\chapter{Langevin Theory}
\label{\detokenize{notebooks/L16/1_langevin:Langevin-Theory}}\label{\detokenize{notebooks/L16/1_langevin::doc}}

\section{Langevin Equation}
\label{\detokenize{notebooks/L16/1_langevin:Langevin-Equation}}
\sphinxAtStartPar
Langevin theory provides now an equation of motion for objects whoch are subject to fluctuating forces. In the most general way, Newtons equation of motion is written as
\begin{equation*}
\begin{split}m\frac{d\vec{v}}{dt}=\vec{F}(t)\end{split}
\end{equation*}
\sphinxAtStartPar
and contains on the right side the sum of all forces \(\vec{F}\) acting on a particle. This total force can be separated into
\begin{itemize}
\item {} 
\sphinxAtStartPar
a viscous drag force, e.g. \(6\pi \eta R \vec{v}\) for a spherical particle

\item {} 
\sphinxAtStartPar
a force coming from and external potential \(U\)

\item {} 
\sphinxAtStartPar
random fluctuating force \(\vec{\zeta}\) due to the collisions with the solvent

\end{itemize}

\sphinxAtStartPar
With the force arising from an external potential we find the Langevin equation
\begin{equation*}
\begin{split}m\frac{d\vec{v}}{dt}=-6\pi \eta R \vec{v}-\nabla U +\vec{\zeta}(t) \tag{Langevin Equation}\end{split}
\end{equation*}
\sphinxAtStartPar
The time dependence of the fluctuating force is not known in detail. Thus we can only refer to the statistical properties (e.g. its moments) when solving the Langevin equation. Therefore also not analytical solution for the position can be written down, only average proerties over an ensemble or over time (given ergodicity).

\sphinxAtStartPar
The first moment (the mean) of the fluctuating force gives
\begin{equation*}
\begin{split}\langle\vec{\zeta}(t) \rangle =0\end{split}
\end{equation*}
\sphinxAtStartPar
i.e. there is no net force acting on average on the particle considered. Yet the mean velocity of the particle is not zero, \(\langle \vec{v}\rangle\neq 0\). Considering the mean velocity at short times, for example, yields
\begin{equation*}
\begin{split}m\frac{d\langle \vec{v}\rangle}{dt}=-6\pi \eta R \langle \vec{v}\rangle\end{split}
\end{equation*}
\sphinxAtStartPar
giving
\begin{equation*}
\begin{split}\langle \vec{v}\rangle=\vec{v}(0)\exp\left ( -\frac{t}{\tau_{mr}}\right)\end{split}
\end{equation*}
\sphinxAtStartPar
with \(\tau_{mr}\) being the momentum relaxation time we discussed already earlier. We can further continue to derive the mean squared position by taking the scalar product of the Langevin equation with \(\vec{r}\) and neglecting the potential. Taking the ensemble average yields
\begin{equation*}
\begin{split}m\frac{d}{dt}\langle \vec{r}\cdot \vec{v}\rangle =-6\pi \eta R \langle \vec{r}\cdot \vec{v}\rangle + m\langle v^2\rangle\end{split}
\end{equation*}
\sphinxAtStartPar
since
\begin{equation*}
\begin{split}\vec{r}\cdot \frac{d\vec{v}}{dt}=\frac{d}{dt}(\vec{r}\cdot \vec{v})-v^2\end{split}
\end{equation*}
\sphinxAtStartPar
and \(\langle \vec{r}\cdot \vec{\zeta}\rangle =0\). Using our previous result for the mean squared velocity, i.e. \(\langle v^2\rangle =3k_B T\) we obtain after integrating over time
\begin{equation*}
\begin{split}\langle \vec{r}\cdot \vec{v}\rangle =C\exp\left ( -\frac{t}{\tau_{mr}}\right)+ \frac{k_B T}{2\pi\eta R}\end{split}
\end{equation*}
\sphinxAtStartPar
Using the initial condition \(\vec{r}(t=0)=0\) this gives
\begin{equation*}
\begin{split}\langle \vec{r}\cdot \vec{v}\rangle =\frac{k_B T}{2\pi\eta R}\left (1 -\exp\left ( -\frac{t}{\tau_{mr}}\right)\right)\end{split}
\end{equation*}
\sphinxAtStartPar
A second integration over time of \(\langle \vec{r}\cdot \vec{v}\rangle\) can be carried out using
\begin{equation*}
\begin{split}\langle \vec{r}\cdot \vec{v}\rangle=\frac{1}{2}\frac{d}{dt}\langle r^2\rangle\end{split}
\end{equation*}
\sphinxAtStartPar
and we finally obtain the mean squared displacement
\begin{equation*}
\begin{split}\langle r^2\rangle =\frac{k_B T}{\pi \eta R}\left [ t-\tau_{mr}\left ( 1-\exp\left (-\frac{t}{\tau_{mr}}\right )\right )\right ]\end{split}
\end{equation*}
\sphinxAtStartPar
This is the mean squared displacement for an Ornstein Uhlenbeck process, a process, which has a persistence for a certain time and then is randomized. It described the transition from the ballistic to diffusive regime for a Brownian particle except that it misses the hydrodynamic memory in the intermediate regime.

\noindent\sphinxincludegraphics{{test1}.pdf}

\sphinxAtStartPar
We can look at different limits of the process with respect to the momentum relaxation time, ie.

\sphinxAtStartPar
\sphinxstylestrong{1) long time limits:} \(t\gg \tau_{mr}\)

\sphinxAtStartPar
In this regime, the exponential function has already decayed to zero and we obtain
\begin{equation*}
\begin{split}\langle r^{2}\rangle =6Dt\end{split}
\end{equation*}
\sphinxAtStartPar
with
\begin{equation*}
\begin{split}D=\frac{k_B T}{6\pi\eta R}\end{split}
\end{equation*}
\sphinxAtStartPar
\sphinxstylestrong{1) short times:} \(t\approx \tau_{mr}\)

\sphinxAtStartPar
For short times, we can expand the exponential function up to second order
\begin{equation*}
\begin{split}\exp\left ( -\frac{t}{\tau_{mr}}\right )\approx 1-\frac{t}{\tau_{mr}}+\frac{t^2}{\tau_{mr}^2}\end{split}
\end{equation*}
\sphinxAtStartPar
to obtain for
\begin{equation*}
\begin{split}t-\tau_{mr}\left ( 1-\exp\left (-\frac{t}{\tau_{mr}}\right )\right )\approx \frac{t^2}{2\tau_{mr}}\end{split}
\end{equation*}
\sphinxAtStartPar
and for the mean squared displacement
\begin{equation*}
\begin{split}\langle r^2 \rangle \propto t^2\end{split}
\end{equation*}
\sphinxAtStartPar
which is the result we expect for ballistic motion. Between both regimes, the MSD could be complicated as the short time motion starts a hydrodynamic flux in the environment which is influencing the particle motion again. These hydrodynamic memory effects cause the so called long time tails in the mean squared displacement. The Langevin equation in one dimension then reads like

\sphinxAtStartPar
\begin{equation}
M^{\prime} \ddot{x}(t)=
-6 \pi \eta R \dot{x}(t)-6 R^{2} \sqrt{\pi \rho_{f} \eta} \int_{0}^{t}\left(t-t^{\prime}\right)^{-1 / 2} \ddot{x}\left(t^{\prime}\right) \mathrm{d} t^{\prime} -\nabla U+\zeta(t)
\end{equation}

\sphinxAtStartPar
and contains the effective mass of the colloidal \(M^{\prime}\), the fluid density \(\rho_f\) and the external potential \$U. The full MSD for a colloidal particle is displayed in the figure below as measured for example in an optical tweezer in the Molecular Nanophotonics group. Note that the MSD axis goes down to \(10^{-21}\, {\rm m}\), which corresponds to 30 picometer displacement, which is far below the size of a hydrodgen atom.

\noindent\sphinxincludegraphics[width=405\sphinxpxdimen,height=410\sphinxpxdimen]{{msd_bead}.png}

\sphinxAtStartPar
While we so far only know that \(\langle\vec{\zeta}(t) \rangle =0\), we also know that there must be a certain magnitude of the random force that is connected to the temperature of the sample. The short time autocorrelation of the noise should be of the type
\begin{equation*}
\begin{split}\langle\vec{\zeta}(t_1)\cdot \vec{\zeta}(t_2) \rangle =A\delta (t_1-t_2)\end{split}
\end{equation*}
\sphinxAtStartPar
stating the the forces are truly random and only correlated, when the times conincide. We would like to determine the prefactor \(A\) now. We can write down the Langevin equation with
\begin{equation*}
\begin{split}\frac{d}{dt}\vec{v}(t)=-\frac{1}{\tau_{mr}}\vec{v}(t)+\frac{1}{m}\vec{\zeta}(t)\end{split}
\end{equation*}
\sphinxAtStartPar
and multiply by \(\exp(t/\tau_{mr})\), which results in
\begin{equation*}
\begin{split}\exp\left (\frac{t}{\tau_{mr}}\right )\frac{d}{dt}\vec{v}(t)+\frac{1}{\tau_{mr}}\exp\left (\frac{t}{\tau_{mr}}\right )\vec{v}(t)=\frac{d}{dt}\left (\exp\left (\frac{t}{\tau_{mr}}\right )\vec{v}(t) \right )=\exp\left(\frac{t}{\tau_{mr}}\right )\frac{1}{m}\vec{\zeta}(t)\end{split}
\end{equation*}
\sphinxAtStartPar
Integrating both sides of the two right parts over time results in
\begin{equation*}
\begin{split}\vec{v}(t)=\vec{v}(0)\exp \left (-\frac{t}{\tau_{mr}} \right )+\exp \left (-\frac{t}{\tau_{mr}} \right )\frac{1}{m}\int_0^t \exp \left (\frac{t'}{\tau_{mr}} \right )\vec{\zeta}(t')dt'\end{split}
\end{equation*}
\sphinxAtStartPar
which is the time evolution of the velocity. To obtain \(\langle\vec{\zeta}(t_1)\cdot \vec{\zeta}(t_2) \rangle\), we need to calculate the product of the velocity at two times and take the ensemble average. The calculation is essentially a writing exercise which at the end yields three different terms.

\sphinxAtStartPar
The first term contains
\begin{equation*}
\begin{split}\langle v_0^2 \rangle \exp\left (-\frac{2t}{\tau_{mr}} \right )\rightarrow 0\end{split}
\end{equation*}
\sphinxAtStartPar
which decays to zero for long times. The second term contains two mixed terms of the type
\begin{equation*}
\begin{split}\frac{v_0}{m}\exp\left (-\frac{t}{\tau_{mr}} \right ) \int \ldots \rightarrow 0\end{split}
\end{equation*}
\sphinxAtStartPar
which also decay to zero for long times. The last term contains
\begin{equation*}
\begin{split}\frac{1}{m^2}\int_{-\infty}^{t}dt' \int_{-\infty}^{t}dt''\exp\left ( -\frac{t-t'}{\tau_{mr}} \right )\exp\left ( -\frac{t-t''}{\tau_{mr}} \right )\langle \vec{\zeta}(t')\vec{\zeta}(t'')\rangle\end{split}
\end{equation*}
\sphinxAtStartPar
whcih is the only nonzero term. Inserting our initial assumption \(\langle\vec{\zeta}(t_1)\cdot \vec{\zeta}(t_2) \rangle =A\delta (t_1-t_2)\) results in the mean squared velocity
\begin{equation*}
\begin{split}\langle v^2\rangle =\frac{A}{2m}\int_{-\infty}^{t}dt' \exp\left ( -\frac{2(t-t')}{\tau_{mr}}\right)=\frac{A\tau_{mr}}{2m^2}\end{split}
\end{equation*}
\sphinxAtStartPar
Since the velocity also has to comply with equipartition, i.e.
\begin{equation*}
\begin{split}\langle v^2\rangle=\frac{k_B T}{m}=\frac{A\tau_{mr}}{2m^2}\end{split}
\end{equation*}
\sphinxAtStartPar
we find
\begin{equation*}
\begin{split}A=2\gamma k_B T\end{split}
\end{equation*}
\sphinxAtStartPar
with \(\gamma=6\pi\eta R\) or
\begin{equation*}
\begin{split}\langle\vec{\zeta}(t_1)\cdot \vec{\zeta}(t_2) \rangle =2\gamma k_B T\delta (t_1-t_2)\end{split}
\end{equation*}
\sphinxAtStartPar
which is a very fundamental result. It states that the fluctuating force \(\vec{\zeta}(t)\) are related to the viscous forces expressed by the friction coefficient \(\gamma\). This makes sensse, since the friction forces must dissipate energy into heat. The motion would thus come to a rest if heat and dynamics would be decoupled. The law we obtained is thus stateing the dissipated energy goes back again into kinetic energy. This is the statement of the fluctuation dissipation theorem,


\section{Fluctuation Dissipation Theorem}
\label{\detokenize{notebooks/L16/1_langevin:Fluctuation-Dissipation-Theorem}}
\sphinxAtStartPar
The fluctuation dissipation theorem (FDT) generalizes the previous finding into
\begin{equation*}
\begin{split}\chi''(\omega)=\frac{\omega}{2k_B T}\langle |x_{\omega}|^2\rangle \tag{Fluctuation dissipation relation}\end{split}
\end{equation*}
\sphinxAtStartPar
which is now converted into frequency space. On the left side is the imaginary part of a response function or susceptibility \(\chi\), which is connected to the dissipation. The right side contains the power spectral density of the fluctuation of a variable \(x_{\omega}\) at the frequency \(\omega\). So the amplitude of a fluctuation at a certain frequency is connected to the dissipation in equilibrium.


\subsection{Damped Driven Harmonic Oscillator}
\label{\detokenize{notebooks/L16/1_langevin:Damped-Driven-Harmonic-Oscillator}}
\sphinxAtStartPar
We would like to have a look at the FDT using a 1d damped driven oscillator given by the following equation
\begin{equation*}
\begin{split}m\ddot{x}+\gamma \dot{x}+kx=F(t)\end{split}
\end{equation*}
\sphinxAtStartPar
with \(k\) being the spring constant of the harmonic potential and \(\gamma\) the friction coefficient. Solving this differential equation yields for the amplitude of the oscillation at a certain driving frequency of \(F(t)\)
\begin{equation*}
\begin{split}x_{\omega}=\frac{1}{k-m\omega^2-i\gamma \omega}F_{\omega}\end{split}
\end{equation*}
\sphinxAtStartPar
which is clearly a linear response relation as amplitude and force dependent lineraly on each other. The term converting the force into an amplitude is the response function
\begin{equation*}
\begin{split}\chi(\omega)=\frac{1}{k-m\omega^2-i\gamma \omega} \tag{response function}\end{split}
\end{equation*}
\sphinxAtStartPar
which can be split into real and imaginary part, i.e. \(\chi=\chi'(\omega)+i\chi'' (\omega)\). A quick calculation shows that
\begin{equation*}
\begin{split}\chi''(\omega)=\frac{\omega \gamma}{(k-m\omega^2)^2+(\omega \gamma)^2}\end{split}
\end{equation*}
\sphinxAtStartPar
In the overdamped regime (i.e. \(m=0\)) we therefore find for the imaginary part
\begin{equation*}
\begin{split}\chi''(\omega)=\frac{\omega \gamma}{k^2+(\omega \gamma)^2}\end{split}
\end{equation*}
\sphinxAtStartPar
If we furthe rswitch off the harmonic potential we just have
\begin{equation*}
\begin{split}\chi''(\omega)=\frac{\omega \gamma}{\omega^2 \gamma^2}\end{split}
\end{equation*}
\sphinxAtStartPar
which is the response function for Brownian motion. From this we find
\begin{equation*}
\begin{split}\frac{\chi''(\omega)}{\omega}=\frac{1}{\omega^2 \gamma}\langle |x_{\omega}|^2\rangle=\frac{1}{\gamma \omega^2}\end{split}
\end{equation*}
\sphinxAtStartPar
or
\begin{equation*}
\begin{split}\langle |x_{\omega}|^2\rangle=\frac{2k_B T}{\gamma \omega^2}\end{split}
\end{equation*}
\sphinxAtStartPar
for the amplitude and
\begin{equation*}
\begin{split}\langle |v_{\omega}|^2\rangle=\frac{2k_B T}{\gamma}\end{split}
\end{equation*}
\sphinxAtStartPar
for the velocity amplitude. This is the frequency spectrum of the positional and speed fluctuations of a Brownian particle. This is exactly what is found in an optical tweezer for a Brownian particle at high frequencies. At lower frequencies the particle bounces from the harmonic potential and the motion of the colloid is restricted.
\begin{equation*}
\begin{split}\frac{\chi''(\omega)}{\omega}=\frac{\gamma}{k^2+(\omega \gamma)^2}=\frac{1}{2k_B T}\langle |x_{\omega}|^2\rangle\end{split}
\end{equation*}
\sphinxAtStartPar
From thsi follows the power spectral density of tge positional fluctuations in an optical tweezers
\begin{equation*}
\begin{split}\langle |x_{\omega}|^2\rangle =\frac{2k_B T \gamma}{k^2}\frac{1}{1+(\omega/\omega_0)^2}\end{split}
\end{equation*}
\sphinxAtStartPar
Here \(\omega_0=k/\gamma\) is a particular frequency at which the \(1/\omega^2\) dependence of the power spectrum turns into a plateau at small frequencies. The last equation is heavily used to calibrate optical tweezers, as it can be used to measure the force constant \(k\).

\noindent\sphinxincludegraphics{{test}.pdf}



\sphinxAtStartPar
The following section was created from \sphinxcode{\sphinxupquote{notebooks/L17/1\_hydrodynamics.ipynb}}.


\chapter{Hydrodynamics}
\label{\detokenize{notebooks/L17/1_hydrodynamics:Hydrodynamics}}\label{\detokenize{notebooks/L17/1_hydrodynamics::doc}}
\sphinxAtStartPar
Hydrodynamics provides the fundamental equations to describe the motion of a fluid. Note that a fluid can therby be a gas or a liquid. Gases and liquids may have completely different flow properties as in liquids the mean free path of a molecule is much smaller than the size of the liquif container. This is something, that is not alway true for gases.

\sphinxAtStartPar
In fluid dynamics, we are interested in the motion of a value element of a fluid, which flows with a velocity \(\vec{u}\). As the whole fluid moves, each volume element may have a different flow velocity, which results in a flow field \(\vec{u}(\vec{r},t)\). Each component of the flow field depends now on each coordinate, i.e.
\begin{equation*}
\begin{split}\vec{u}=
\begin{Bmatrix}
u(x,y,z,t)\\
v(x,y,z,t)\\
w(x,y,z,t)
\end{Bmatrix}\end{split}
\end{equation*}
\sphinxAtStartPar
In case of a stationary flow, the components do not explicitly depend on time, which means
\begin{equation*}
\begin{split}\frac{\partial \vec{u}}{\partial t}=0\end{split}
\end{equation*}
\begin{sphinxadmonition}{warning}{}\unskip
\sphinxAtStartPar
\sphinxstylestrong{Streamline}

\sphinxAtStartPar
A streamline is a curve where each point on the curve has the same direction as \(\vec{u}\) without requiring the \(\vec{u}\) is constant along the streamline. This means that there are no perpendicular velocity components along a streamline, i.e. \(d\vec{r}\times \vec{u}=0\).

\sphinxAtStartPar
\sphinxincludegraphics[width=1536\sphinxpxdimen,height=468\sphinxpxdimen]{{stream}.png}
\end{sphinxadmonition}

\sphinxAtStartPar
When we now follow the motion of a fluid element in space and time we notice that a change in the velocity \(d\vec{u}\) can be achieved by
\begin{equation*}
\begin{split}d\vec{u}=\frac{\partial \vec{u}}{\partial t}\bigg|_{\vec{r}}dt+\frac{\partial \vec{u}}{\partial x}\bigg|_{\vec{r}}dx+\frac{\partial \vec{u}}{\partial y}\bigg|_{\vec{r}}dy+\frac{\partial \vec{u}}{\partial z}\bigg|_{\vec{r}}dz\end{split}
\end{equation*}
\sphinxAtStartPar
which when dividing by \(dt\) results in

\sphinxAtStartPar
\begin{eqnarray}
\frac{d\vec{u}}{dt} &= &\frac{\partial \vec{u}}{\partial t}+\frac{\partial \vec{u}}{\partial x}\frac{dx}{dt}+\frac{\partial \vec{u}}{\partial y}\frac{dy}{dt}+\frac{\partial \vec{u}}{\partial z}\frac{dz}{dt}\\
&=& \frac{\partial \vec{u}}{\partial t}+(\vec{u}\cdot \nabla)\vec{u}
\end{eqnarray}

\sphinxAtStartPar
This is the total derivative, which is called the \sphinxstyleemphasis{substantial derivative}:
\begin{equation*}
\begin{split}\frac{D}{Dt}=\frac{\partial}{\partial t} + \vec{u}\nabla\end{split}
\end{equation*}
\sphinxAtStartPar
and states, that a change in the velocity can be caused by a temporal change but also by a convective motion (different places have different velocities). The term \$:nbsphinx\sphinxhyphen{}math:\sphinxtitleref{vec\{u\}}:nbsphinx\sphinxhyphen{}math:{\color{red}\bfseries{}\textasciigrave{}}nabla {\color{red}\bfseries{}\textasciigrave{}}\$ is therefore called the \sphinxstyleemphasis{convective derivative}.


\section{Navier Stokes Equation}
\label{\detokenize{notebooks/L17/1_hydrodynamics:Navier-Stokes-Equation}}
\sphinxAtStartPar
We obtain a force density from the above equations if we multiply the above derived velocity change with the mass density \(\rho\) of the fluid. This force density is the result of all possible forces that are present in the system. We thus have to consider possible forces on a volume element.

\sphinxAtStartPar
\sphinxstylestrong{a) We would like to simplify our considerations by assuming an incompressibility of the liquid, i.e.}
\begin{equation*}
\begin{split}\nabla \cdot \vec{u}=0 \tag{incompressibility}\end{split}
\end{equation*}
\sphinxAtStartPar
\sphinxstylestrong{b) One of the causes of a flow could be a pressure acting on the surface of a volume element}

\noindent\sphinxincludegraphics[width=1536\sphinxpxdimen,height=422\sphinxpxdimen]{{surface}.png}

\sphinxAtStartPar
If the pressure \(p\) acts on the surface we have for a surface element \(\delta S\) with a normal vector \(\hat{n}\) a force \(-\hat{n}p\delta S\) on the surface element. If we integrate the force over the whole surface we can write
\begin{equation*}
\begin{split}-\oint p\hat{n}\delta S = -\int \nabla p dV\end{split}
\end{equation*}
\sphinxAtStartPar
stating the the integral of the pressure over the whole surface should amount for the volume integral (over the volume enclosed by the surface) of the gradient pressure. Thus the net force cause by a pressure gradient is \(-\nabla p dV\) and the pressure gradient is just the force density.
\begin{equation*}
\begin{split}-\nabla p \tag{pressure force density}\end{split}
\end{equation*}
\sphinxAtStartPar
\sphinxstylestrong{c) Additionally, flow may be caused by external forces, like gravity (or others)}

\sphinxAtStartPar
The force density by an external force, like gravity is just given by
\begin{equation*}
\begin{split}\rho \vec{g} \tag{external force density }\end{split}
\end{equation*}
\sphinxAtStartPar
The sum of these components of the force density make up the \sphinxstylestrong{Euler equation}

\sphinxAtStartPar
\begin{equation}
\rho \frac{D\vec{u}}{Dt}=\rho \left [\frac{\partial \vec{u}}{\partial t} +(\vec{u}\nabla)\vec{u}\right ]=-\nabla p + \rho \vec{g} \tag{Euler equation}
\end{equation}

\sphinxAtStartPar
The Euler equation describes the flow of ideal fluids due to pressure gradients or external forces. It is, however, only valid for ideal fluids with no internal friction. Thus, it essentially describes the flow of dilute gases or superfluids like superfluid Helium.

\sphinxAtStartPar
To go beyond that limitation, we have to introduce frction into the equations. Since this is a good occasion, we will introduce the stress tensor at the same time, which is expressing all possible force on a volume element of a fluid.

\sphinxAtStartPar
\sphinxstylestrong{d) Introduce internal friction}

\sphinxAtStartPar
Let’s consider the following system with two fluid sheets, which only have a constant velocity component along the x\sphinxhyphen{}direction, i.e. \(\vec{u}=\{u(y),0,0\}\) and \(\partial u(y)/\partial x=0\).

\noindent\sphinxincludegraphics[width=1536\sphinxpxdimen,height=630\sphinxpxdimen]{{viscosity}.png}

\sphinxAtStartPar
The sheet at \(y\) moves along the x\sphinxhyphen{}direction with a speed \(u(y)\). The sheet at \(y+\Delta y\), which is \(\Delta y\) away from the otehr sheet moves with a speed \(u(y+\Delta y)\). Consider now the momentum \(\vec{p}=\{m u(y),0,0 \}\) of the fluid sheet (don’t confuse with the pressure \(p\)). The total momentum change \(d\vec{p}\) can be written as a substantial derivative

\sphinxAtStartPar
\begin{equation}
d\vec{p}=\frac{\partial \vec{p}}{\partial t}dt +\underbrace{\frac{\partial \vec{p}}{\partial x}dx}_{=0} + \frac{\partial \vec{p}}{\partial y}dy+\underbrace{\frac{\partial \vec{p}}{\partial z}dz}_{=0}
\end{equation}

\sphinxAtStartPar
from which we obtain

\sphinxAtStartPar
\begin{equation}
\frac{d\vec{p}}{dt}=\frac{\partial \vec{p}}{\partial t}+\frac{\partial \vec{p}}{\partial y}\frac{dy}{dt}=\frac{\partial m u(y)}{\partial y}\frac{dy}{dt}
\end{equation}

\sphinxAtStartPar
which is a force. This is the force that is responsible for the change in the fluid velocity along the y direction, which is parallel to the x\sphinxhyphen{}direction.

\noindent\sphinxincludegraphics[width=673\sphinxpxdimen,height=422\sphinxpxdimen]{{stress}.png}

\sphinxAtStartPar
Such a force tangential to an area in the direction of motion is called a shear stress \(\tau\), i.e.

\sphinxAtStartPar
\begin{equation}
\tau=\frac{F}{A}=\frac{1}{A}\frac{d\vec{p}}{dt}=\underbrace{\frac{dy}{dt}\frac{\partial m u(y)}{\partial y}}_{\frac{d\vec{p}}{dt}}\frac{1}{A}
\end{equation}

\sphinxAtStartPar
With \(m=\rho V\) and \(v=A L\) this turns into

\sphinxAtStartPar
\begin{equation}
\tau=\underbrace{L\frac{dy}{dt}}_{\frac{m^2}{s}}\frac{\partial \rho u(y)}{\partial y}
\end{equation}

\sphinxAtStartPar
where \(\rho u(y)\) indicates a momentum density. This equation is actually a diffusion equation. On the left side we have a momentum current density (\(\tau\)) and on the right side a momentum density gradient, which is multiplied by a prefactor with a unit of the diffusion coefficient. This prefactor, the momentum diffusion coefficient, is the kinematic viscosity
\begin{equation*}
\begin{split}\nu=\frac{\eta}{\rho} \tag{kinematic viscosity}\end{split}
\end{equation*}
\sphinxAtStartPar
and thus

\sphinxAtStartPar
\begin{equation}
\tau_{yx}=\frac{\eta}{\rho}\frac{\partial \rho u(y)}{\partial y}=\eta \frac{\partial u(y)}{dy}
\end{equation}

\sphinxAtStartPar
wher I have sneaked in the \sphinxstylestrong{dynamic viscosity} \(\eta\) and the index \(yx\), as this is one element of the stress tensor for a momentum transport along the y\sphinxhyphen{}axis when stresses are applied along the x\sphinxhyphen{}axis.


\subsection{Viscous Stress Tensor}
\label{\detokenize{notebooks/L17/1_hydrodynamics:Viscous-Stress-Tensor}}
\sphinxAtStartPar
A similar treatment can be done for other velocity and force components can be done as well to construct the so\sphinxhyphen{}called visous stress tensor, which contains all the viscous stress components. They are

\sphinxAtStartPar
\begin{eqnarray}
\tau_{xy}=\tau_{yx}&=&\eta \left ( \frac{\partial u}{\partial y} +\frac{\partial v}{\partial x}\right )\\
\tau_{xz}=\tau_{zx}&=&\eta \left ( \frac{\partial u}{\partial z} +\frac{\partial w}{\partial x}\right )\\
\tau_{yz}=\tau_{zy}&=&\eta \left ( \frac{\partial v}{\partial z} +\frac{\partial w}{\partial y}\right )
\end{eqnarray}

\sphinxAtStartPar
for the off\sphinxhyphen{}diagonal elements and

\sphinxAtStartPar
\begin{eqnarray}
\tau_{xx}&=&2\eta \frac{\partial u }{\partial x}\\
\tau_{yy}&=&2\eta \frac{\partial v }{\partial y}\\
\tau_{zz}&=&2\eta \frac{\partial w }{\partial z}
\end{eqnarray}

\sphinxAtStartPar
for the diagonal elements. Together they create the viscous stress tensor

\sphinxAtStartPar
\begin{equation}
\tau=\begin{bmatrix}
\tau_{xx} & \tau_{xy} & \tau_{xz}\\
\tau_{yx} & \tau_{yy} & \tau_{yz}\\
\tau_{zx} & \tau_{zy} & \tau_{zz}
\end{bmatrix}
\end{equation}

\sphinxAtStartPar
which is, as you see from the equations a symmetric tensor.

\sphinxAtStartPar
So far we have only considered the situation where \(\eta/\rho={\rm const.}\), without saying it. This situation actually corresponds to what we call \sphinxstylestrong{Newtonian Liquids}. The transport coefficient for the momentum is constant in this case.

\sphinxAtStartPar
More generall, the transport coefficient may depend on the \sphinxstyleemphasis{shear rate} \(\frac{\partial u}{\partial y},\ldots\). Thus we write more generally

\sphinxAtStartPar
\begin{equation}
\tau_{yx}=k\left( \frac{\partial u}{\partial y} \right)^n
\end{equation}

\sphinxAtStartPar
where \(k\) is the so\sphinxhyphen{}called \sphinxstyleemphasis{consistency index} and \(n\) is the \sphinxstyleemphasis{flow behavior index}. The above equation can be transformed into

\sphinxAtStartPar
\begin{equation}
\tau_{yx}=\underbrace{k\left( \frac{\partial u}{\partial y} \right)^{n-1} }_{\rm apparent\, viscosity} \frac{\partial u}{\partial y}
\end{equation}

\sphinxAtStartPar
which takes the known form of the momentum diffusion equation with a shear rate dependent prefactor, which is the momentum diffusion coefficient or the \sphinxstyleemphasis{apparent viscosity}.

\sphinxAtStartPar
Depending on the shear rate dependence of the apparent viscosity, we can now classify liquids into
\begin{itemize}
\item {} 
\sphinxAtStartPar
Newtonian liquids

\item {} 
\sphinxAtStartPar
Non\sphinxhyphen{}Newtonian liquids
\begin{itemize}
\item {} 
\sphinxAtStartPar
shear thinning (ketchup)

\item {} 
\sphinxAtStartPar
shear thickening (corn starch)

\end{itemize}

\end{itemize}

\noindent\sphinxincludegraphics[width=533\sphinxpxdimen,height=512\sphinxpxdimen]{{nonnewton}.png}

\sphinxAtStartPar
Stress vector
\begin{equation*}
\begin{split}\vec{S}=\hat{n} \tau =[T_{xx},T_{xy},T_{xz}]\end{split}
\end{equation*}
\sphinxAtStartPar
total force
\begin{equation*}
\begin{split}\vec{F}=\oint \vec{S} dS=\oint \hat{n}\tau dS=\int \nabla \tau dV\end{split}
\end{equation*}\begin{equation*}
\begin{split}\nabla \tau=\left [\frac{\partial \tau_{xx}}{\partial x} + \frac{\partial \tau_{yx}}{\partial y}+\frac{\partial \tau_{zx}}{\partial z},\ldots \right ]\end{split}
\end{equation*}
\sphinxAtStartPar
wher \(\nabla \tau\) is the visous force density, which we can add to the Euler euqation to introduce friction

\sphinxAtStartPar
\begin{equation}
\rho \frac{D\vec{u}}{Dt}=\rho \left [\frac{\partial \vec{u}}{\partial t} +(\vec{u}\nabla)\vec{u}\right ]=-\nabla p + \rho \vec{g} +\nabla \tau
\end{equation}


\subsection{Mechanical Stress Tensor}
\label{\detokenize{notebooks/L17/1_hydrodynamics:Mechanical-Stress-Tensor}}
\sphinxAtStartPar
Together with the pressure, we can define the mechanical stress tensor
\begin{equation*}
\begin{split}T=-pI +\tau\end{split}
\end{equation*}
\sphinxAtStartPar
where

\sphinxAtStartPar
\begin{equation}
I=\begin{bmatrix}
1 & 0 & 0\\
0 & 1 & 0\\
0 & 0 & 1
\end{bmatrix}
\end{equation}

\sphinxAtStartPar
is the identity matrix. As we need a force density for the Navier Stokes equation, we can take as before the divergence of the mechnical stress tensor which yields

\sphinxAtStartPar
\begin{eqnarray}
\nabla T &=&\left ( -\frac{\partial p}{\partial x} +\frac{\partial \tau_{xx}}{\partial x}+\frac{\partial \tau_{yx}}{\partial y}+\frac{\partial \tau_{zx}}{\partial z}\right)\hat{e}_x+\\
&=& \left (\ldots \right )\hat{e}_y+ \\
&=& \left ( \ldots \right ) \hat{e}_z
\end{eqnarray}

\sphinxAtStartPar
Just looking for the moment at the x\sphinxhyphen{}component only and inserting the viscous stress tensor elements results in
\begin{equation*}
\begin{split}-\frac{\partial p}{\partial x}+2\eta \underline{\frac{\partial^2 u }{\partial x^2}}+ \eta \left ( \frac{\partial^2 u }{\partial y^2}+\underline{\frac{\partial^2 v }{\partial x \partial y}}\right)+ \eta \left ( \frac{\partial^2 u }{\partial z^2}+\underline{\frac{\partial^2 w }{\partial x \partial z}}\right)\end{split}
\end{equation*}
\sphinxAtStartPar
The underlined components can actually be understood as
\begin{equation*}
\begin{split}\frac{\partial }{\partial x}(\nabla \cdot \vec{u})\end{split}
\end{equation*}
\sphinxAtStartPar
which is actually \sphinxstyleemphasis{zero} due to the fact that we assue an incompressible fluid (\(\nabla\cdot \vec{u}=0\)).

\sphinxAtStartPar
\begin{eqnarray}
\rho \frac{Du}{Dt}&=&-p_x+\eta (u_{xx}+u_{yy}+u_{zz})+f_{B,x}\\
\rho \frac{Dv}{Dt}&=&-p_y+\eta (v_{xx}+v_{yy}+v_{zz})+f_{B,y}\\
\rho \frac{Dw}{Dt}&=&-p_z+\eta (w_{xx}+w_{yy}+w_{zz})+f_{B,z}
\end{eqnarray}

\sphinxAtStartPar
which corresponds to the components of the Navier\sphinxhyphen{}Stokes equation

\sphinxAtStartPar
\begin{equation}
\rho\frac{D\vec{u}}{Dt}=-\nabla p + \underbrace{ \vec{f}_{B}}_{{\rm e.g. \,}\rho \vec{g}}+\eta \Delta \vec{u} \tag {Navier-Stokes-Equation}
\end{equation}

\sphinxAtStartPar
under the incompressibility condition
\begin{equation*}
\begin{split}\nabla \cdot \vec{u}=0 \tag{incompressibility condition}\end{split}
\end{equation*}


\sphinxAtStartPar
The following section was created from \sphinxcode{\sphinxupquote{notebooks/L18/1\_Reynolds\_Number.ipynb}}.


\chapter{Reynolds Number}
\label{\detokenize{notebooks/L18/1_Reynolds_Number:Reynolds-Number}}\label{\detokenize{notebooks/L18/1_Reynolds_Number::doc}}
\begin{sphinxuseclass}{nbinput}
\begin{sphinxuseclass}{nblast}
{
\sphinxsetup{VerbatimColor={named}{nbsphinx-code-bg}}
\sphinxsetup{VerbatimBorderColor={named}{nbsphinx-code-border}}
\begin{sphinxVerbatim}[commandchars=\\\{\}]
\llap{\color{nbsphinxin}[9]:\,\hspace{\fboxrule}\hspace{\fboxsep}}\PYG{k+kn}{import} \PYG{n+nn}{numpy} \PYG{k}{as} \PYG{n+nn}{np}
\PYG{k+kn}{import} \PYG{n+nn}{matplotlib}\PYG{n+nn}{.}\PYG{n+nn}{pyplot} \PYG{k}{as} \PYG{n+nn}{plt}
\PYG{k+kn}{from} \PYG{n+nn}{numpy}\PYG{n+nn}{.}\PYG{n+nn}{linalg} \PYG{k+kn}{import} \PYG{n}{norm}
\PYG{k+kn}{from} \PYG{n+nn}{scipy}\PYG{n+nn}{.}\PYG{n+nn}{constants} \PYG{k+kn}{import} \PYG{n}{c}\PYG{p}{,}\PYG{n}{epsilon\PYGZus{}0}\PYG{p}{,}\PYG{n}{e}\PYG{p}{,}\PYG{n}{physical\PYGZus{}constants}
\PYG{k+kn}{from} \PYG{n+nn}{IPython}\PYG{n+nn}{.}\PYG{n+nn}{display} \PYG{k+kn}{import} \PYG{n}{YouTubeVideo}
\PYG{k+kn}{import} \PYG{n+nn}{json}

\PYG{o}{\PYGZpc{}}\PYG{k}{config} InlineBackend.figure\PYGZus{}format = \PYGZsq{}retina\PYGZsq{}

\PYG{k}{with} \PYG{n+nb}{open}\PYG{p}{(}\PYG{l+s+s1}{\PYGZsq{}}\PYG{l+s+s1}{style.json}\PYG{l+s+s1}{\PYGZsq{}}\PYG{p}{,} \PYG{l+s+s1}{\PYGZsq{}}\PYG{l+s+s1}{r}\PYG{l+s+s1}{\PYGZsq{}}\PYG{p}{)} \PYG{k}{as} \PYG{n}{fp}\PYG{p}{:}
    \PYG{n}{style} \PYG{o}{=} \PYG{n}{json}\PYG{o}{.}\PYG{n}{load}\PYG{p}{(}\PYG{n}{fp}\PYG{p}{)}

\PYG{n}{plt}\PYG{o}{.}\PYG{n}{rcParams}\PYG{o}{.}\PYG{n}{update}\PYG{p}{(}\PYG{n}{style}\PYG{p}{)}
\end{sphinxVerbatim}
}

\end{sphinxuseclass}
\end{sphinxuseclass}
\sphinxAtStartPar
Hydrodynamics is full of dimensionless number, mainly also due to its relevance for engineering. For example, you may want to compare the flow around the same object at large and small scales. Ideally, you could just build a model of a small airplane and study the flow field around the small airplane in a lab, as it is naturally less expensive than studying it on a large one.

\sphinxAtStartPar
A number which is useful with this respect is the \sphinxstylestrong{Reynold number}, which tells of how to scale the flow velocity when scaling the object.

\sphinxAtStartPar
To obtain the Reynold number and its meaning, we introduce a Navier Stokes equation with dimensionless quatities. These use characteristic dimensions of the system to rescale. We use

\sphinxAtStartPar
\begin{eqnarray}
x&=&x^{\prime} L\\
\vec{u}&=&\vec{u}^{\prime} U\\
t&=&t^{\prime} \frac{L}{U}
\end{eqnarray}

\sphinxAtStartPar
for the rescaling, where \(L\) is the characteristic size of the object, \(U\) a characteristic velocity and \(L/U\) the characteristic time for a fluid volume element to pass the object of size \(L\). Using these quantities, the differential operators turn into

\sphinxAtStartPar
\begin{eqnarray}
\nabla &=& \frac{\nabla^{\prime}}{L}\\
\frac{\partial}{\partial t} &=& \frac{\partial}{\partial t^{\prime}}\frac{U}{L}\\
p&=&p^{\prime} U^2/\rho
\end{eqnarray}

\sphinxAtStartPar
Using these relations in the Navier Stokes equation

\sphinxAtStartPar
\begin{equation}
\rho \left [ \frac{\partial }{\partial t} +(\vec{u}\nabla)\right]\vec{u}=-\nabla p +\eta \Delta \vec{u}
\end{equation}

\sphinxAtStartPar
yields

\sphinxAtStartPar
\begin{equation}
\rho \left [ \frac{\partial }{\partial t^{\prime}}\frac{U}{L} +U\left (\vec{u}^{\prime}\frac{\nabla^{\prime}}{L}\right)\right]\vec{u}^{\prime}U=-\frac{\nabla^{\prime}}{L}p^{\prime} U^2 \rho +\eta \frac{1}{L^2}\Delta^{\prime} \vec{u}^{\prime}
\end{equation}

\sphinxAtStartPar
Dividing the previous equation by \(\rho U^2/L\) gives us the dimensionless Navier\sphinxhyphen{}Stokes equation

\sphinxAtStartPar
\begin{equation}
\left [\frac{\partial }{\partial t^{\prime}} +\vec{u}^{\prime} \nabla \right ]\vec{u}^{\prime}=-\nabla p^{\prime} +\frac{1}{\rm Re}\Delta \vec{u}^{\prime}
\end{equation}

\sphinxAtStartPar
where

\sphinxAtStartPar
\begin{equation}
{\rm Re}=\frac{\rho}{\eta}U L \tag{Reynolds number}
\end{equation}

\sphinxAtStartPar
is the Reynolds number. This number now tells us, that if we scale the size of the object by a factor of 2, we have to increase the velocity by a factor of two to get the same flow field as for the larger object. This is probably not what we would have simply anticipated by our intuition.

\sphinxAtStartPar
The Reynolds number has also a very important meaning for the classification of flows. It may help you to make a distinction between the realm of turbulent and laminar flow. When we start just start from the stationary Navier Stokes equation

\sphinxAtStartPar
\begin{equation}
\rho \left [(\vec{u}\nabla)\right]\vec{u}=-\nabla p +\eta \Delta \vec{u}
\end{equation}

\sphinxAtStartPar
we can also make a dimension analysis with the help of the individual quantities. In this way we find

\sphinxAtStartPar
\begin{equation}
\underbrace{\rho \frac{U^2}{L}}_{\rm inertia}=-\nabla p +\underbrace{\eta \frac{U}{L^2}}_{\rm viscous}
\end{equation}

\sphinxAtStartPar
where the left side correspopnds to the inertial and the right side to the viscous force densities. If we compare the intertial to the viscous force densities, we obtain

\sphinxAtStartPar
\begin{equation}
\frac{\rho \frac{U^2}{L}}{\eta \frac{U}{L^2}}=\rho\frac{U L}{\eta})={\rm Re}
\end{equation}

\sphinxAtStartPar
which is just the Reynolds number. Thus large Reynolds numbers (\({\rm Re >1}\)) state that inertia are important as compared to viscous forces and the flow may be turbulent. Otherwise, for \({\rm Re}<1\), viscous forces dominate the flow and the flowfield will always be laminar.

\begin{sphinxuseclass}{nbinput}
{
\sphinxsetup{VerbatimColor={named}{nbsphinx-code-bg}}
\sphinxsetup{VerbatimBorderColor={named}{nbsphinx-code-border}}
\begin{sphinxVerbatim}[commandchars=\\\{\}]
\llap{\color{nbsphinxin}[68]:\,\hspace{\fboxrule}\hspace{\fboxsep}}\PYG{n}{YouTubeVideo}\PYG{p}{(}\PYG{l+s+s1}{\PYGZsq{}}\PYG{l+s+s1}{\PYGZus{}dbnH\PYGZhy{}BBSNo}\PYG{l+s+s1}{\PYGZsq{}}\PYG{p}{,} \PYG{n}{width}\PYG{o}{=}\PYG{l+m+mi}{800}\PYG{p}{,} \PYG{n}{height}\PYG{o}{=}\PYG{l+m+mi}{600}\PYG{p}{)}
\end{sphinxVerbatim}
}

\end{sphinxuseclass}
\begin{sphinxuseclass}{nboutput}
\begin{sphinxuseclass}{nblast}
\hrule height -\fboxrule\relax
\vspace{\nbsphinxcodecellspacing}

\savebox\nbsphinxpromptbox[0pt][r]{\color{nbsphinxout}\Verb|\strut{[68]:}\,|}

\begin{nbsphinxfancyoutput}

\begin{sphinxuseclass}{output_area}
\begin{sphinxuseclass}{}
\noindent\sphinxincludegraphics[width=480\sphinxpxdimen,height=360\sphinxpxdimen]{{notebooks_L18_1_Reynolds_Number_4_0}.jpg}

\end{sphinxuseclass}
\end{sphinxuseclass}
\end{nbsphinxfancyoutput}

\end{sphinxuseclass}
\end{sphinxuseclass}

\section{Stokes Equation}
\label{\detokenize{notebooks/L18/1_Reynolds_Number:Stokes-Equation}}
\sphinxAtStartPar
In the realm of very small Reynolds number we may even neglect the inertial terms in the Navier Stokes equation and obtain just the Stokes equation

\sphinxAtStartPar
\begin{equation}
0= -\nabla p+\eta \Delta \vec{u}+\vec{f}\tag{Stokes Equation}
\end{equation}

\sphinxAtStartPar
with the incompressibility condition

\sphinxAtStartPar
\begin{equation}
\nabla \cdot \vec{u}=0
\end{equation}


\section{Solutions of the Stokes Equation}
\label{\detokenize{notebooks/L18/1_Reynolds_Number:Solutions-of-the-Stokes-Equation}}
\sphinxAtStartPar
The Stokes equation is easier to solve than the Navier\sphinxhyphen{}Stokes equation as it has no time dependence. This also means that it is time symmetric and the a reversal of the motion of an object typically also creates a reversed flow field. This is very important for example for micro\sphinxhyphen{}organisms that would like to swim in water. Due to their small dimension they live at low Reynolds numbers and every symmetric motion they take is just yielding a wiggeling back and forth but no net motion. They
therefore have to come up with some time\sphinxhyphen{}asymmetric motion to swim. This is summarized in a theorem that has been put forward by Edward Purcell. We will talk about this later.

\sphinxAtStartPar
We would first like to obtain two general solutions for the flow field from the Stokes equation and thereby consider two solid surfaces which confine a water film of height \(h\) according to the drawing below.

\noindent\sphinxincludegraphics[width=632\sphinxpxdimen,height=254\sphinxpxdimen]{{flow_field}.png}

\sphinxAtStartPar
We chose the ccordinate system in the way that the horizontal axis is the \(x\)\sphinxhyphen{}axis and the vertical one the \(y\)\sphinxhyphen{}axis. In two dimensions the Stokes equation is now the following

\sphinxAtStartPar
\begin{equation}
\eta \frac{d^2 u}{d y^2 }=\frac{dp}{dx}
\end{equation}

\sphinxAtStartPar
without any external forces. We can readily integrate the both sides two times

\sphinxAtStartPar
\begin{equation}
\int\frac{d^2 u}{d y^2 }=\frac{1}{\eta}\int \frac{dp}{dx}
\end{equation}

\sphinxAtStartPar
which results in

\sphinxAtStartPar
\begin{equation}
\int\frac{d u}{d y}=\int \left [ \frac{1}{\eta} \frac{dp}{dx}y+C_1 \right ] dy
\end{equation}

\sphinxAtStartPar
and finally gives

\sphinxAtStartPar
\begin{equation}
u(y)=\frac{1}{2\eta}\frac{dp}{dx}y^2+ C_1 y+ C_2
\end{equation}

\sphinxAtStartPar
where \(C_1\) and \(C_2\) are integration constants, which we have to determine from the boundary conditions.

\sphinxAtStartPar
We assume the following no\sphinxhyphen{}slip boundary conditions

\sphinxAtStartPar
\begin{equation}
u(y=0)=0
\end{equation}

\sphinxAtStartPar
and

\sphinxAtStartPar
\begin{equation}
u(y=h)=U
\end{equation}

\sphinxAtStartPar
With the help of these we obtain

\sphinxAtStartPar
\begin{equation}
u(y)=\frac{1}{2\eta}\frac{dp}{dx}y(y-h)+\frac{Uy}{h}
\end{equation}

\sphinxAtStartPar
for the flow profile of the liquid film.

\sphinxAtStartPar
We can now recognize two different solutions in the flow field.


\subsection{Couette Flow \sphinxhyphen{} Shear Driven Flow}
\label{\detokenize{notebooks/L18/1_Reynolds_Number:Couette-Flow---Shear-Driven-Flow}}
\sphinxAtStartPar
To obtain a flow field purely driven by shear, we assume that \(\frac{dp}{dx}=0\), which is leaving a flow velocity which linearly increases with the position \(y\)

\sphinxAtStartPar
\begin{equation}
u(y)=U\frac{y}{h}
\end{equation}

\begin{sphinxuseclass}{nbinput}
{
\sphinxsetup{VerbatimColor={named}{nbsphinx-code-bg}}
\sphinxsetup{VerbatimBorderColor={named}{nbsphinx-code-border}}
\begin{sphinxVerbatim}[commandchars=\\\{\}]
\llap{\color{nbsphinxin}[54]:\,\hspace{\fboxrule}\hspace{\fboxsep}}\PYG{n}{plt}\PYG{o}{.}\PYG{n}{figure}\PYG{p}{(}\PYG{n}{figsize}\PYG{o}{=}\PYG{p}{(}\PYG{l+m+mi}{6}\PYG{p}{,}\PYG{l+m+mi}{4}\PYG{p}{)}\PYG{p}{)}
\PYG{n}{plt}\PYG{o}{.}\PYG{n}{axhline}\PYG{p}{(}\PYG{n}{y}\PYG{o}{=}\PYG{l+m+mi}{0}\PYG{p}{,}\PYG{n}{linestyle}\PYG{o}{=}\PYG{l+s+s2}{\PYGZdq{}}\PYG{l+s+s2}{\PYGZhy{}\PYGZhy{}}\PYG{l+s+s2}{\PYGZdq{}}\PYG{p}{)}
\PYG{n}{plt}\PYG{o}{.}\PYG{n}{axhline}\PYG{p}{(}\PYG{n}{y}\PYG{o}{=}\PYG{l+m+mi}{1}\PYG{p}{,}\PYG{n}{linestyle}\PYG{o}{=}\PYG{l+s+s2}{\PYGZdq{}}\PYG{l+s+s2}{\PYGZhy{}\PYGZhy{}}\PYG{l+s+s2}{\PYGZdq{}}\PYG{p}{)}
\PYG{p}{[}\PYG{n}{plt}\PYG{o}{.}\PYG{n}{arrow}\PYG{p}{(}\PYG{l+m+mf}{0.5}\PYG{p}{,}\PYG{n}{y}\PYG{p}{,}\PYG{n}{y}\PYG{o}{*}\PYG{l+m+mf}{0.31}\PYG{o}{\PYGZhy{}}\PYG{l+m+mf}{0.03}\PYG{p}{,}\PYG{l+m+mi}{0}\PYG{p}{,}\PYG{n}{head\PYGZus{}width}\PYG{o}{=}\PYG{l+m+mf}{0.02}\PYG{p}{)} \PYG{k}{for} \PYG{n}{y} \PYG{o+ow}{in} \PYG{n}{np}\PYG{o}{.}\PYG{n}{arange}\PYG{p}{(}\PYG{l+m+mf}{0.1}\PYG{p}{,}\PYG{l+m+mf}{1.1}\PYG{p}{,}\PYG{l+m+mf}{0.1}\PYG{p}{)}\PYG{p}{]}
\PYG{n}{y}\PYG{o}{=}\PYG{n}{np}\PYG{o}{.}\PYG{n}{arange}\PYG{p}{(}\PYG{l+m+mi}{0}\PYG{p}{,}\PYG{l+m+mf}{1.1}\PYG{p}{,}\PYG{l+m+mf}{0.1}\PYG{p}{)}
\PYG{n}{plt}\PYG{o}{.}\PYG{n}{plot}\PYG{p}{(}\PYG{n}{y}\PYG{o}{*}\PYG{l+m+mf}{0.31}\PYG{o}{/}\PYG{l+m+mi}{1}\PYG{o}{+}\PYG{l+m+mf}{0.5}\PYG{p}{,}\PYG{n}{y}\PYG{p}{)}
\PYG{n}{plt}\PYG{o}{.}\PYG{n}{ylim}\PYG{p}{(}\PYG{o}{\PYGZhy{}}\PYG{l+m+mf}{0.1}\PYG{p}{,}\PYG{l+m+mf}{1.1}\PYG{p}{)}
\PYG{n}{plt}\PYG{o}{.}\PYG{n}{xlim}\PYG{p}{(}\PYG{l+m+mi}{0}\PYG{p}{,}\PYG{l+m+mi}{1}\PYG{p}{)}
\PYG{n}{plt}\PYG{o}{.}\PYG{n}{xlabel}\PYG{p}{(}\PYG{l+s+s2}{\PYGZdq{}}\PYG{l+s+s2}{position x }\PYG{l+s+s2}{\PYGZdq{}}\PYG{p}{)}
\PYG{n}{plt}\PYG{o}{.}\PYG{n}{ylabel}\PYG{p}{(}\PYG{l+s+s2}{\PYGZdq{}}\PYG{l+s+s2}{position y }\PYG{l+s+s2}{\PYGZdq{}}\PYG{p}{)}
\PYG{n}{plt}\PYG{o}{.}\PYG{n}{show}\PYG{p}{(}\PYG{p}{)}
\end{sphinxVerbatim}
}

\end{sphinxuseclass}
\begin{sphinxuseclass}{nboutput}
\begin{sphinxuseclass}{nblast}
\hrule height -\fboxrule\relax
\vspace{\nbsphinxcodecellspacing}

\makeatletter\setbox\nbsphinxpromptbox\box\voidb@x\makeatother

\begin{nbsphinxfancyoutput}

\begin{sphinxuseclass}{output_area}
\begin{sphinxuseclass}{}
\noindent\sphinxincludegraphics[width=438\sphinxpxdimen,height=283\sphinxpxdimen]{{notebooks_L18_1_Reynolds_Number_8_0}.png}

\end{sphinxuseclass}
\end{sphinxuseclass}
\end{nbsphinxfancyoutput}

\end{sphinxuseclass}
\end{sphinxuseclass}

\subsection{Poiseuille Flow \sphinxhyphen{} Pressure Driven Flow}
\label{\detokenize{notebooks/L18/1_Reynolds_Number:Poiseuille-Flow---Pressure-Driven-Flow}}
\sphinxAtStartPar
If we assume that both boundaries are at rest in the laboratory frame and the pressure gradient along the x\sphinxhyphen{}axis is constant, we find a parabolic flow field, which is typical for pressure driven flows.

\sphinxAtStartPar
\begin{equation}
u(y)=\frac{1}{2\eta}\frac{dp}{dx}y(y-h)
\end{equation}

\begin{sphinxuseclass}{nbinput}
{
\sphinxsetup{VerbatimColor={named}{nbsphinx-code-bg}}
\sphinxsetup{VerbatimBorderColor={named}{nbsphinx-code-border}}
\begin{sphinxVerbatim}[commandchars=\\\{\}]
\llap{\color{nbsphinxin}[56]:\,\hspace{\fboxrule}\hspace{\fboxsep}}\PYG{n}{plt}\PYG{o}{.}\PYG{n}{figure}\PYG{p}{(}\PYG{n}{figsize}\PYG{o}{=}\PYG{p}{(}\PYG{l+m+mi}{6}\PYG{p}{,}\PYG{l+m+mi}{4}\PYG{p}{)}\PYG{p}{)}
\PYG{n}{plt}\PYG{o}{.}\PYG{n}{axhline}\PYG{p}{(}\PYG{n}{y}\PYG{o}{=}\PYG{l+m+mi}{0}\PYG{p}{,}\PYG{n}{linestyle}\PYG{o}{=}\PYG{l+s+s2}{\PYGZdq{}}\PYG{l+s+s2}{\PYGZhy{}\PYGZhy{}}\PYG{l+s+s2}{\PYGZdq{}}\PYG{p}{)}
\PYG{n}{plt}\PYG{o}{.}\PYG{n}{axhline}\PYG{p}{(}\PYG{n}{y}\PYG{o}{=}\PYG{l+m+mi}{1}\PYG{p}{,}\PYG{n}{linestyle}\PYG{o}{=}\PYG{l+s+s2}{\PYGZdq{}}\PYG{l+s+s2}{\PYGZhy{}\PYGZhy{}}\PYG{l+s+s2}{\PYGZdq{}}\PYG{p}{)}
\PYG{p}{[}\PYG{n}{plt}\PYG{o}{.}\PYG{n}{arrow}\PYG{p}{(}\PYG{l+m+mf}{0.5}\PYG{p}{,}\PYG{n}{y}\PYG{p}{,}\PYG{o}{\PYGZhy{}}\PYG{n}{y}\PYG{o}{*}\PYG{p}{(}\PYG{n}{y}\PYG{o}{\PYGZhy{}}\PYG{l+m+mi}{1}\PYG{p}{)}\PYG{o}{\PYGZhy{}}\PYG{l+m+mf}{0.03}\PYG{p}{,}\PYG{l+m+mi}{0}\PYG{p}{,}\PYG{n}{head\PYGZus{}width}\PYG{o}{=}\PYG{l+m+mf}{0.02}\PYG{p}{)} \PYG{k}{for} \PYG{n}{y} \PYG{o+ow}{in} \PYG{n}{np}\PYG{o}{.}\PYG{n}{arange}\PYG{p}{(}\PYG{l+m+mf}{0.1}\PYG{p}{,}\PYG{l+m+mf}{1.}\PYG{p}{,}\PYG{l+m+mf}{0.1}\PYG{p}{)}\PYG{p}{]}
\PYG{n}{y}\PYG{o}{=}\PYG{n}{np}\PYG{o}{.}\PYG{n}{arange}\PYG{p}{(}\PYG{l+m+mi}{0}\PYG{p}{,}\PYG{l+m+mf}{1.1}\PYG{p}{,}\PYG{l+m+mf}{0.1}\PYG{p}{)}
\PYG{n}{plt}\PYG{o}{.}\PYG{n}{plot}\PYG{p}{(}\PYG{o}{\PYGZhy{}}\PYG{n}{y}\PYG{o}{*}\PYG{p}{(}\PYG{n}{y}\PYG{o}{\PYGZhy{}}\PYG{l+m+mi}{1}\PYG{p}{)}\PYG{o}{+}\PYG{l+m+mf}{0.5}\PYG{p}{,}\PYG{n}{y}\PYG{p}{)}
\PYG{n}{plt}\PYG{o}{.}\PYG{n}{ylim}\PYG{p}{(}\PYG{o}{\PYGZhy{}}\PYG{l+m+mf}{0.1}\PYG{p}{,}\PYG{l+m+mf}{1.1}\PYG{p}{)}
\PYG{n}{plt}\PYG{o}{.}\PYG{n}{xlim}\PYG{p}{(}\PYG{l+m+mi}{0}\PYG{p}{,}\PYG{l+m+mi}{1}\PYG{p}{)}
\PYG{n}{plt}\PYG{o}{.}\PYG{n}{xlabel}\PYG{p}{(}\PYG{l+s+s2}{\PYGZdq{}}\PYG{l+s+s2}{position x }\PYG{l+s+s2}{\PYGZdq{}}\PYG{p}{)}
\PYG{n}{plt}\PYG{o}{.}\PYG{n}{ylabel}\PYG{p}{(}\PYG{l+s+s2}{\PYGZdq{}}\PYG{l+s+s2}{position y }\PYG{l+s+s2}{\PYGZdq{}}\PYG{p}{)}
\PYG{n}{plt}\PYG{o}{.}\PYG{n}{show}\PYG{p}{(}\PYG{p}{)}
\end{sphinxVerbatim}
}

\end{sphinxuseclass}
\begin{sphinxuseclass}{nboutput}
\begin{sphinxuseclass}{nblast}
\hrule height -\fboxrule\relax
\vspace{\nbsphinxcodecellspacing}

\makeatletter\setbox\nbsphinxpromptbox\box\voidb@x\makeatother

\begin{nbsphinxfancyoutput}

\begin{sphinxuseclass}{output_area}
\begin{sphinxuseclass}{}
\noindent\sphinxincludegraphics[width=438\sphinxpxdimen,height=283\sphinxpxdimen]{{notebooks_L18_1_Reynolds_Number_11_0}.png}

\end{sphinxuseclass}
\end{sphinxuseclass}
\end{nbsphinxfancyoutput}

\end{sphinxuseclass}
\end{sphinxuseclass}


\sphinxAtStartPar
The following section was created from \sphinxcode{\sphinxupquote{notebooks/L18/2\_boundary\_hydrodynamics.ipynb}}.


\chapter{Boundary Hydrodynamics}
\label{\detokenize{notebooks/L18/2_boundary_hydrodynamics:Boundary-Hydrodynamics}}\label{\detokenize{notebooks/L18/2_boundary_hydrodynamics::doc}}
\begin{sphinxuseclass}{nbinput}
\begin{sphinxuseclass}{nblast}
{
\sphinxsetup{VerbatimColor={named}{nbsphinx-code-bg}}
\sphinxsetup{VerbatimBorderColor={named}{nbsphinx-code-border}}
\begin{sphinxVerbatim}[commandchars=\\\{\}]
\llap{\color{nbsphinxin}[9]:\,\hspace{\fboxrule}\hspace{\fboxsep}}\PYG{k+kn}{import} \PYG{n+nn}{numpy} \PYG{k}{as} \PYG{n+nn}{np}
\PYG{k+kn}{import} \PYG{n+nn}{matplotlib}\PYG{n+nn}{.}\PYG{n+nn}{pyplot} \PYG{k}{as} \PYG{n+nn}{plt}
\PYG{k+kn}{from} \PYG{n+nn}{numpy}\PYG{n+nn}{.}\PYG{n+nn}{linalg} \PYG{k+kn}{import} \PYG{n}{norm}
\PYG{k+kn}{from} \PYG{n+nn}{scipy}\PYG{n+nn}{.}\PYG{n+nn}{constants} \PYG{k+kn}{import} \PYG{n}{c}\PYG{p}{,}\PYG{n}{epsilon\PYGZus{}0}\PYG{p}{,}\PYG{n}{e}\PYG{p}{,}\PYG{n}{physical\PYGZus{}constants}
\PYG{k+kn}{import} \PYG{n+nn}{json}

\PYG{o}{\PYGZpc{}}\PYG{k}{config} InlineBackend.figure\PYGZus{}format = \PYGZsq{}retina\PYGZsq{}

\PYG{k}{with} \PYG{n+nb}{open}\PYG{p}{(}\PYG{l+s+s1}{\PYGZsq{}}\PYG{l+s+s1}{style.json}\PYG{l+s+s1}{\PYGZsq{}}\PYG{p}{,} \PYG{l+s+s1}{\PYGZsq{}}\PYG{l+s+s1}{r}\PYG{l+s+s1}{\PYGZsq{}}\PYG{p}{)} \PYG{k}{as} \PYG{n}{fp}\PYG{p}{:}
    \PYG{n}{style} \PYG{o}{=} \PYG{n}{json}\PYG{o}{.}\PYG{n}{load}\PYG{p}{(}\PYG{n}{fp}\PYG{p}{)}

\PYG{n}{plt}\PYG{o}{.}\PYG{n}{rcParams}\PYG{o}{.}\PYG{n}{update}\PYG{p}{(}\PYG{n}{style}\PYG{p}{)}
\end{sphinxVerbatim}
}

\end{sphinxuseclass}
\end{sphinxuseclass}
\sphinxAtStartPar
As another important example of hydrodynamical flow fields we would like to have a look at hydrodynamic flows that arise from interfacial forces. In most of the cases, these interfacial forces are often caused by osmotic pressure differences an are termed osmotic flows. The osmotic pressure difference may thereby arise from various properties, e.g. temperature gradients, electrostatic potential gradients, concentration gradients.

\sphinxAtStartPar
To analyze this type of boundary flow, we first look at the geometry below.

\noindent\sphinxincludegraphics[width=782\sphinxpxdimen,height=224\sphinxpxdimen]{{boundary_flow}.png}

\sphinxAtStartPar
The boundary flows are the results of the fact that the interaction between liquid molecules is altered by the presence of the substrate depicted below (some symmetry breaking). This altered interaction is present at a length scale of nanometers (!) only. Think for example about the influence of van der Waals forces or the electric double layer. Beyond this interfacial region of a few nanometers the influence of the solid surface has decayed and the liquid exhibits bulk\sphinxhyphen{}like properties. It is
this tiny interfacial region which we will consider in the description below.

\sphinxAtStartPar
According to the above picture, we can use the Stokes equation

\sphinxAtStartPar
\begin{equation}
\eta \Delta \vec{u}= -\nabla p-\vec{f}
\end{equation}

\sphinxAtStartPar
and split it according to the \(x\) (tangential) and \(z\) (nromal) components.







\sphinxAtStartPar
normal to surface



\sphinxAtStartPar
\(\frac{\partial^2 \vec{u}}{\partial z^2}=0\)



\sphinxAtStartPar
\(0=\frac{\partial p}{\partial z}-f_z\)







\sphinxAtStartPar
tangential to surface



\sphinxAtStartPar
\(\eta \frac{\partial^2 \vec{u}}{\partial z^2}=\frac{\partial p}{\partial x}-f_x\)





\sphinxAtStartPar
We assume that close to the boundary there will be no flow component normal to the surface, which simplifies the normal components of the Stokes equation into a balance of pressure gradient (presumably osmotic pressure) and normal force density.

\sphinxAtStartPar
For the tangential component we have a tangential pressure gradient (osmotic) and a tangential force density which determine the flow.

\sphinxAtStartPar
We can now integrate the equation for the tangential component by parts with two boundary conditions, i.e. \(u(z=0)=0\) and \(u(z\gg\lambda)=u_B={\rm const.}\) and find

\sphinxAtStartPar
\begin{equation}
u_B=\frac{1}{\eta}\int_0^{\infty} z\left (f_x-\frac{dp}{dx}\right )dz
\end{equation}

\sphinxAtStartPar
for the value of the boundary velocity \(u_B\), which is established rounghly at a distance \(\lambda\) from the interface. Note that this upper boundary condition says that there is no solid boundary (with no\sphinxhyphen{}slip boundary condition) at infinity. We can introduce the boundary condition at large distance later. The question is now what in detail the term in the brackets of the previous equation is.


\section{Thermo\sphinxhyphen{}osmosis}
\label{\detokenize{notebooks/L18/2_boundary_hydrodynamics:Thermo-osmosis}}
\sphinxAtStartPar
We want to evaluate the boundary flow for the case when the surface is charged and in contact with an electrolyte solution. In this case, we can refer to the Debye\sphinxhyphen{}Hückel theory, which we earlier developed. In addition, this double layer is subject to a temperature gradient tangential to the surface, which will be the source of the osmotic pressure gradient along the surface.

\noindent\sphinxincludegraphics[width=1026\sphinxpxdimen,height=244\sphinxpxdimen]{{charged_surface}.png}

\sphinxAtStartPar
For this situation we have all equations at hand. We have first of all
\begin{itemize}
\item {} 
\sphinxAtStartPar
excess charge density: \(\rho = e (n_+-n_-)\)

\item {} 
\sphinxAtStartPar
total ion density: \(n=n_++n_-\)

\item {} 
\sphinxAtStartPar
the ion density distribution: \(n_{\pm}=n_0(\exp\left (\mp \frac{e\psi}{k_B T}\right )-1)\)

\item {} 
\sphinxAtStartPar
\(n_0\) the bulk ion density

\item {} 
\sphinxAtStartPar
\(\psi\) the electrostatic potential

\end{itemize}

\sphinxAtStartPar
The force density in the system is given by

\sphinxAtStartPar
\begin{equation}
f=-\rho \nabla \psi + \ldots
\end{equation}

\sphinxAtStartPar
where the dots denote other terms (electrostriction, change of dielectric constant …), which exist, but which we neglect.

\sphinxAtStartPar
Further, we can write the pressure as the osmotic pressure of the ion density

\sphinxAtStartPar
\begin{equation}
p=n k_B T=(n_++n_-)k_B T=\left ( n_0(\exp\left (- \frac{e\psi}{k_B T}\right )-1)+ n_0(\exp\left ( \frac{e\psi}{k_B T}\right )-1) \right ) k_B T
\end{equation}

\sphinxAtStartPar
The task is now to calculate the term f\_x\sphinxhyphen{}\frac{\partial p}{\partial x} using derivatives of the above written pressure. Note that we have to take the derivative of the temperature with respect to the position but assume the temperature in the Boltzmann factors constant.

\sphinxAtStartPar
The result of this a bit more lengthy calculation is that

\sphinxAtStartPar
\begin{equation}
f_x-\frac{\partial p}{\partial x} =-nk_B T \frac{1}{T}\frac{\partial T}{\partial x} - \rho \psi \frac{1}{T}\frac{\partial T}{\partial x}
\end{equation}

\sphinxAtStartPar
or in 3\sphinxhyphen{}dimensions

\sphinxAtStartPar
\begin{equation}
{\bf f}-\nabla p= -\left(\varrho \psi+n k_{\mathrm{B}} T\right) \frac{\nabla T}{T}
\end{equation}

\sphinxAtStartPar
The term in fron of the fraction is an electrostatic interaction energy related term and a term corresponding to the ideal gas pressure. Both terms together clearly make up an enthalpy (energy + pressure x volume). In this case, they correspond to an excess enthalpy density as there is no contribution if there is no electrostatic surface charge involved.

\sphinxAtStartPar
We can therefore also write the boundary velocity \(u_B\) as

\sphinxAtStartPar
\begin{equation}
u_B= -\frac{1}{\eta}\int_0^{\infty} z h(z)dz\frac{\nabla T_{||}}{T}
\end{equation}

\sphinxAtStartPar
where \(h(z)\) is the excess enthalpy density of the interaction of the liquid with the solid as compared to the liquid alone. As the interaction between liquid and solid is often attractive (van der Waals alone already is attractive), the excess enthalpy density is negative and the difference of force density and pressure density results in forces along the temperarture gradient, e.g. from the cold region to the hot region. If the excess enthalpy is positive, the flow is driven from hot to
cold regions. Note that this flow is established in a thin liquid layer of thickness \(\lambda\), which is only a few nanometers thin. Within these few nanometers the flow velocity rises from 0 to a few 10 \(\mu m/s\).

\sphinxAtStartPar
The plot below calculates the integral over the product of \(z\) and the excess enthalpy for the electric double layer, assuming that some of the constants are 1 and the temperature gradient just gives a minus sign (assuming hot is on the left and cold is on the right).

\begin{sphinxuseclass}{nbinput}
\begin{sphinxuseclass}{nblast}
{
\sphinxsetup{VerbatimColor={named}{nbsphinx-code-bg}}
\sphinxsetup{VerbatimBorderColor={named}{nbsphinx-code-border}}
\begin{sphinxVerbatim}[commandchars=\\\{\}]
\llap{\color{nbsphinxin}[62]:\,\hspace{\fboxrule}\hspace{\fboxsep}}\PYG{c+c1}{\PYGZsh{}\PYGZsh{} electrostatic potential}
\PYG{k}{def} \PYG{n+nf}{psi}\PYG{p}{(}\PYG{n}{z}\PYG{p}{,}\PYG{n}{lam\PYGZus{}D}\PYG{p}{)}\PYG{p}{:}
    \PYG{k}{return}\PYG{p}{(}\PYG{n}{np}\PYG{o}{.}\PYG{n}{exp}\PYG{p}{(}\PYG{o}{\PYGZhy{}}\PYG{n}{z}\PYG{o}{/}\PYG{n}{lam\PYGZus{}D}\PYG{p}{)}\PYG{p}{)}
\end{sphinxVerbatim}
}

\end{sphinxuseclass}
\end{sphinxuseclass}
\begin{sphinxuseclass}{nbinput}
\begin{sphinxuseclass}{nblast}
{
\sphinxsetup{VerbatimColor={named}{nbsphinx-code-bg}}
\sphinxsetup{VerbatimBorderColor={named}{nbsphinx-code-border}}
\begin{sphinxVerbatim}[commandchars=\\\{\}]
\llap{\color{nbsphinxin}[63]:\,\hspace{\fboxrule}\hspace{\fboxsep}}\PYG{c+c1}{\PYGZsh{}\PYGZsh{} n+}
\PYG{k}{def} \PYG{n+nf}{npl}\PYG{p}{(}\PYG{n}{z}\PYG{p}{,}\PYG{n}{lam\PYGZus{}D}\PYG{p}{,}\PYG{n}{n}\PYG{p}{)}\PYG{p}{:}
    \PYG{k}{return}\PYG{p}{(}\PYG{n}{n}\PYG{o}{*}\PYG{p}{(}\PYG{n}{np}\PYG{o}{.}\PYG{n}{exp}\PYG{p}{(}\PYG{o}{\PYGZhy{}}\PYG{n}{psi}\PYG{p}{(}\PYG{n}{z}\PYG{p}{,}\PYG{n}{lam\PYGZus{}D}\PYG{p}{)}\PYG{p}{)}\PYG{o}{\PYGZhy{}}\PYG{l+m+mi}{1}\PYG{p}{)}\PYG{p}{)}

\PYG{c+c1}{\PYGZsh{}\PYGZsh{} n\PYGZhy{}}
\PYG{k}{def} \PYG{n+nf}{nmi}\PYG{p}{(}\PYG{n}{z}\PYG{p}{,}\PYG{n}{lam\PYGZus{}D}\PYG{p}{,}\PYG{n}{n}\PYG{p}{)}\PYG{p}{:}
    \PYG{k}{return}\PYG{p}{(}\PYG{n}{n}\PYG{o}{*}\PYG{p}{(}\PYG{n}{np}\PYG{o}{.}\PYG{n}{exp}\PYG{p}{(}\PYG{n}{psi}\PYG{p}{(}\PYG{n}{z}\PYG{p}{,}\PYG{n}{lam\PYGZus{}D}\PYG{p}{)}\PYG{p}{)}\PYG{o}{\PYGZhy{}}\PYG{l+m+mi}{1}\PYG{p}{)}\PYG{p}{)}
\end{sphinxVerbatim}
}

\end{sphinxuseclass}
\end{sphinxuseclass}
\begin{sphinxuseclass}{nbinput}
\begin{sphinxuseclass}{nblast}
{
\sphinxsetup{VerbatimColor={named}{nbsphinx-code-bg}}
\sphinxsetup{VerbatimBorderColor={named}{nbsphinx-code-border}}
\begin{sphinxVerbatim}[commandchars=\\\{\}]
\llap{\color{nbsphinxin}[64]:\,\hspace{\fboxrule}\hspace{\fboxsep}}\PYG{c+c1}{\PYGZsh{}\PYGZsh{} total n}
\PYG{k}{def} \PYG{n+nf}{nt}\PYG{p}{(}\PYG{n}{z}\PYG{p}{,}\PYG{n}{lam\PYGZus{}D}\PYG{p}{,}\PYG{n}{n}\PYG{p}{)}\PYG{p}{:}
    \PYG{k}{return}\PYG{p}{(}\PYG{l+m+mi}{2}\PYG{o}{*}\PYG{n}{n}\PYG{o}{*}\PYG{p}{(}\PYG{n}{np}\PYG{o}{.}\PYG{n}{cosh}\PYG{p}{(}\PYG{n}{psi}\PYG{p}{(}\PYG{n}{z}\PYG{p}{,}\PYG{n}{lam\PYGZus{}D}\PYG{p}{)}\PYG{p}{)}\PYG{o}{\PYGZhy{}}\PYG{l+m+mi}{1}\PYG{p}{)}\PYG{p}{)}
\end{sphinxVerbatim}
}

\end{sphinxuseclass}
\end{sphinxuseclass}
\begin{sphinxuseclass}{nbinput}
\begin{sphinxuseclass}{nblast}
{
\sphinxsetup{VerbatimColor={named}{nbsphinx-code-bg}}
\sphinxsetup{VerbatimBorderColor={named}{nbsphinx-code-border}}
\begin{sphinxVerbatim}[commandchars=\\\{\}]
\llap{\color{nbsphinxin}[65]:\,\hspace{\fboxrule}\hspace{\fboxsep}}\PYG{c+c1}{\PYGZsh{}\PYGZsh{} excess enhalpy}

\PYG{k}{def} \PYG{n+nf}{h}\PYG{p}{(}\PYG{n}{z}\PYG{p}{,}\PYG{n}{lam\PYGZus{}D}\PYG{p}{,}\PYG{n}{n}\PYG{p}{)}\PYG{p}{:}
    \PYG{k}{return}\PYG{p}{(}\PYG{p}{(}\PYG{n}{npl}\PYG{p}{(}\PYG{n}{z}\PYG{p}{,}\PYG{n}{lam\PYGZus{}D}\PYG{p}{,}\PYG{l+m+mi}{1}\PYG{p}{)}\PYG{o}{\PYGZhy{}}\PYG{n}{nmi}\PYG{p}{(}\PYG{n}{z}\PYG{p}{,}\PYG{n}{lam\PYGZus{}D}\PYG{p}{,}\PYG{l+m+mi}{1}\PYG{p}{)}\PYG{p}{)}\PYG{o}{*}\PYG{n}{psi}\PYG{p}{(}\PYG{n}{z}\PYG{p}{,}\PYG{n}{lam\PYGZus{}D}\PYG{p}{)}\PYG{o}{+}\PYG{n}{nt}\PYG{p}{(}\PYG{n}{z}\PYG{p}{,}\PYG{n}{lam\PYGZus{}D}\PYG{p}{,}\PYG{l+m+mi}{1}\PYG{p}{)}\PYG{p}{)}
\end{sphinxVerbatim}
}

\end{sphinxuseclass}
\end{sphinxuseclass}
\begin{sphinxuseclass}{nbinput}
{
\sphinxsetup{VerbatimColor={named}{nbsphinx-code-bg}}
\sphinxsetup{VerbatimBorderColor={named}{nbsphinx-code-border}}
\begin{sphinxVerbatim}[commandchars=\\\{\}]
\llap{\color{nbsphinxin}[61]:\,\hspace{\fboxrule}\hspace{\fboxsep}}\PYG{n}{plt}\PYG{o}{.}\PYG{n}{figure}\PYG{p}{(}\PYG{n}{figsize}\PYG{o}{=}\PYG{p}{(}\PYG{l+m+mi}{6}\PYG{p}{,}\PYG{l+m+mi}{4}\PYG{p}{)}\PYG{p}{)}
\PYG{n}{plt}\PYG{o}{.}\PYG{n}{axhline}\PYG{p}{(}\PYG{n}{y}\PYG{o}{=}\PYG{l+m+mi}{0}\PYG{p}{,}\PYG{n}{linestyle}\PYG{o}{=}\PYG{l+s+s2}{\PYGZdq{}}\PYG{l+s+s2}{\PYGZhy{}\PYGZhy{}}\PYG{l+s+s2}{\PYGZdq{}}\PYG{p}{)}
\PYG{n}{plt}\PYG{o}{.}\PYG{n}{axhline}\PYG{p}{(}\PYG{n}{y}\PYG{o}{=}\PYG{l+m+mi}{1}\PYG{p}{,}\PYG{n}{linestyle}\PYG{o}{=}\PYG{l+s+s2}{\PYGZdq{}}\PYG{l+s+s2}{\PYGZhy{}\PYGZhy{}}\PYG{l+s+s2}{\PYGZdq{}}\PYG{p}{)}
\PYG{n}{z}\PYG{o}{=}\PYG{n}{np}\PYG{o}{.}\PYG{n}{linspace}\PYG{p}{(}\PYG{l+m+mf}{0.1}\PYG{p}{,}\PYG{l+m+mi}{5}\PYG{p}{,}\PYG{l+m+mi}{1000}\PYG{p}{)}\PYG{n}{b}
\PYG{n}{inth}\PYG{o}{=}\PYG{o}{\PYGZhy{}}\PYG{n}{np}\PYG{o}{.}\PYG{n}{cumsum}\PYG{p}{(}\PYG{n}{z}\PYG{o}{*}\PYG{p}{(}\PYG{n}{h}\PYG{p}{(}\PYG{n}{z}\PYG{p}{,}\PYG{l+m+mi}{1}\PYG{p}{,}\PYG{l+m+mi}{1}\PYG{p}{)}\PYG{p}{)}\PYG{o}{*}\PYG{p}{(}\PYG{n}{z}\PYG{p}{[}\PYG{l+m+mi}{1}\PYG{p}{]}\PYG{o}{\PYGZhy{}}\PYG{n}{z}\PYG{p}{[}\PYG{l+m+mi}{0}\PYG{p}{]}\PYG{p}{)}\PYG{p}{)} \PYG{c+c1}{\PYGZsh{}\PYGZsh{} numerically integrate the excess enthalpy *z}
\PYG{p}{[}\PYG{n}{plt}\PYG{o}{.}\PYG{n}{arrow}\PYG{p}{(}\PYG{l+m+mi}{0}\PYG{p}{,}\PYG{n}{z}\PYG{p}{[}\PYG{n}{l}\PYG{p}{]}\PYG{p}{,}\PYG{o}{\PYGZhy{}}\PYG{n}{inth}\PYG{p}{[}\PYG{n}{l}\PYG{p}{]}\PYG{o}{*}\PYG{l+m+mi}{5}\PYG{p}{,}\PYG{l+m+mi}{0}\PYG{p}{,}\PYG{n}{head\PYGZus{}width}\PYG{o}{=}\PYG{l+m+mf}{0.2}\PYG{p}{,}\PYG{n}{head\PYGZus{}length}\PYG{o}{=}\PYG{l+m+mf}{0.1}\PYG{p}{)} \PYG{k}{for} \PYG{n}{l} \PYG{o+ow}{in} \PYG{n+nb}{range}\PYG{p}{(}\PYG{l+m+mi}{0}\PYG{p}{,}\PYG{n+nb}{len}\PYG{p}{(}\PYG{n}{z}\PYG{p}{)}\PYG{p}{,}\PYG{l+m+mi}{50}\PYG{p}{)}\PYG{p}{]}
\PYG{n}{plt}\PYG{o}{.}\PYG{n}{xlim}\PYG{p}{(}\PYG{o}{\PYGZhy{}}\PYG{l+m+mi}{4}\PYG{p}{,}\PYG{l+m+mi}{2}\PYG{p}{)}
\PYG{n}{plt}\PYG{o}{.}\PYG{n}{xlabel}\PYG{p}{(}\PYG{l+s+s2}{\PYGZdq{}}\PYG{l+s+s2}{position x }\PYG{l+s+s2}{\PYGZdq{}}\PYG{p}{)}
\PYG{n}{plt}\PYG{o}{.}\PYG{n}{ylabel}\PYG{p}{(}\PYG{l+s+s2}{\PYGZdq{}}\PYG{l+s+s2}{position y }\PYG{l+s+s2}{\PYGZdq{}}\PYG{p}{)}
\PYG{n}{plt}\PYG{o}{.}\PYG{n}{show}\PYG{p}{(}\PYG{p}{)}
\end{sphinxVerbatim}
}

\end{sphinxuseclass}
\begin{sphinxuseclass}{nboutput}
\begin{sphinxuseclass}{nblast}
\hrule height -\fboxrule\relax
\vspace{\nbsphinxcodecellspacing}

\makeatletter\setbox\nbsphinxpromptbox\box\voidb@x\makeatother

\begin{nbsphinxfancyoutput}

\begin{sphinxuseclass}{output_area}
\begin{sphinxuseclass}{}
\noindent\sphinxincludegraphics[width=400\sphinxpxdimen,height=285\sphinxpxdimen]{{notebooks_L18_2_boundary_hydrodynamics_13_0}.png}

\end{sphinxuseclass}
\end{sphinxuseclass}
\end{nbsphinxfancyoutput}

\end{sphinxuseclass}
\end{sphinxuseclass}
\sphinxAtStartPar
The flow profile we have calculated saturates at the boundary velocity \(u_B\) as we have assumed this as the boundary condition at infinite distance. If there is a second surface at some distance, even though at finite distance but still large as compared to the interaction range, the velocity magnitude has to decay to zero towards this surface. Instead of assuming the complicated flow profile as depicted below, we can in cases where all other distances are large as compared to the
interaction range \(\lambda_D\) assume that the velocity directly at the interface is \(u_B\). The hydrodynamic boundary condition is thus altered from a \sphinxstylestrong{no\sphinxhyphen{}slip boundary condition} to a \sphinxstylestrong{slip boundary condition} with the slip velocity \(u_B\).



\sphinxAtStartPar
The following section was created from \sphinxcode{\sphinxupquote{notebooks/L19/1\_electroosmosis.ipynb}}.


\chapter{Electro\sphinxhyphen{}osmosis}
\label{\detokenize{notebooks/L19/1_electroosmosis:Electro-osmosis}}\label{\detokenize{notebooks/L19/1_electroosmosis::doc}}
\begin{sphinxuseclass}{nbinput}
\begin{sphinxuseclass}{nblast}
{
\sphinxsetup{VerbatimColor={named}{nbsphinx-code-bg}}
\sphinxsetup{VerbatimBorderColor={named}{nbsphinx-code-border}}
\begin{sphinxVerbatim}[commandchars=\\\{\}]
\llap{\color{nbsphinxin}[ ]:\,\hspace{\fboxrule}\hspace{\fboxsep}}
\end{sphinxVerbatim}
}

\end{sphinxuseclass}
\end{sphinxuseclass}


\sphinxAtStartPar
The following section was created from \sphinxcode{\sphinxupquote{notebooks/L19/2\_polymers.ipynb}}.


\chapter{Polymers}
\label{\detokenize{notebooks/L19/2_polymers:Polymers}}\label{\detokenize{notebooks/L19/2_polymers::doc}}
\begin{sphinxuseclass}{nbinput}
\begin{sphinxuseclass}{nblast}
{
\sphinxsetup{VerbatimColor={named}{nbsphinx-code-bg}}
\sphinxsetup{VerbatimBorderColor={named}{nbsphinx-code-border}}
\begin{sphinxVerbatim}[commandchars=\\\{\}]
\llap{\color{nbsphinxin}[2]:\,\hspace{\fboxrule}\hspace{\fboxsep}}\PYG{k+kn}{import} \PYG{n+nn}{numpy} \PYG{k}{as} \PYG{n+nn}{np}
\PYG{k+kn}{import} \PYG{n+nn}{matplotlib}\PYG{n+nn}{.}\PYG{n+nn}{pyplot} \PYG{k}{as} \PYG{n+nn}{plt}
\PYG{k+kn}{from} \PYG{n+nn}{numpy}\PYG{n+nn}{.}\PYG{n+nn}{linalg} \PYG{k+kn}{import} \PYG{n}{norm}
\PYG{k+kn}{from} \PYG{n+nn}{scipy}\PYG{n+nn}{.}\PYG{n+nn}{constants} \PYG{k+kn}{import} \PYG{n}{c}\PYG{p}{,}\PYG{n}{epsilon\PYGZus{}0}\PYG{p}{,}\PYG{n}{e}\PYG{p}{,}\PYG{n}{physical\PYGZus{}constants}
\PYG{k+kn}{import} \PYG{n+nn}{json}

\PYG{o}{\PYGZpc{}}\PYG{k}{config} InlineBackend.figure\PYGZus{}format = \PYGZsq{}retina\PYGZsq{}

\PYG{k}{with} \PYG{n+nb}{open}\PYG{p}{(}\PYG{l+s+s1}{\PYGZsq{}}\PYG{l+s+s1}{style.json}\PYG{l+s+s1}{\PYGZsq{}}\PYG{p}{,} \PYG{l+s+s1}{\PYGZsq{}}\PYG{l+s+s1}{r}\PYG{l+s+s1}{\PYGZsq{}}\PYG{p}{)} \PYG{k}{as} \PYG{n}{fp}\PYG{p}{:}
    \PYG{n}{style} \PYG{o}{=} \PYG{n}{json}\PYG{o}{.}\PYG{n}{load}\PYG{p}{(}\PYG{n}{fp}\PYG{p}{)}

\PYG{n}{plt}\PYG{o}{.}\PYG{n}{rcParams}\PYG{o}{.}\PYG{n}{update}\PYG{p}{(}\PYG{n}{style}\PYG{p}{)}
\end{sphinxVerbatim}
}

\end{sphinxuseclass}
\end{sphinxuseclass}


\sphinxAtStartPar
The following section was created from \sphinxcode{\sphinxupquote{notebooks/L22/1\_real\_polymers.ipynb}}.


\chapter{Real Polymers}
\label{\detokenize{notebooks/L22/1_real_polymers:Real-Polymers}}\label{\detokenize{notebooks/L22/1_real_polymers::doc}}
\sphinxAtStartPar
\sphinxhref{1\_real\_polymers.pdf}{Download Lecture as PDF}

\sphinxAtStartPar
We want to look at properties of real polymers in the following, which means that we have to incorporate interactions between monomers, which are of finite size and also have interactions with the solvent. We will do that in a mean field model introducing the so\sphinxhyphen{}called Mayer f\sphinxhyphen{}function.


\section{Mayer f\sphinxhyphen{}function and excluded volume}
\label{\detokenize{notebooks/L22/1_real_polymers:Mayer-f-function-and-excluded-volume}}
\sphinxAtStartPar
Consider for that purpose the interaction by two monomers in a solvent with an effective potential \(U(r)\). This potential typically has some large repulsive component at very short distances, some attractive components (negative) at intermediate length scales and a zero value at very large length scales.

\noindent\sphinxincludegraphics[width=441\sphinxpxdimen,height=293\sphinxpxdimen]{{interaction}.png}


\subsection{Interaction potential}
\label{\detokenize{notebooks/L22/1_real_polymers:Interaction-potential}}
\sphinxAtStartPar
The plots below show some example interaction potentials. The left graph displays the Lennard\sphinxhyphen{}Jones potential, the right one the ghard sphere potential, which is also called an athermal potential, as there is no temperature in it due to the missing minimum.

\sphinxAtStartPar
Note again, that this potential includes all effects of the solvent asl well.

\noindent\sphinxincludegraphics{{tmp1}.pdf}


\subsection{Probability distribution}
\label{\detokenize{notebooks/L22/1_real_polymers:Probability-distribution}}
\sphinxAtStartPar
The probability distribution for finding the the monomers at a certain distance \(r\) is then given by the Boltzman factor

\sphinxAtStartPar
\begin{equation}
p(r)=\exp\left (-\frac{U(r)}{k_B T}\right)
\end{equation}

\sphinxAtStartPar
For the two potentials shown above, the probability therefore is zero wherever the potential is infinitely large. Whenever the potential is zero we find a probability density value of 1 while it is larger than 1 in the regions, where the potential is attractive.

\noindent\sphinxincludegraphics{{tmp2}.pdf}


\subsection{Mayer f\sphinxhyphen{}function}
\label{\detokenize{notebooks/L22/1_real_polymers:Mayer-f-function}}
\sphinxAtStartPar
The Mayer f\sphinxhyphen{}function measures now the deviation of the probability from the regions where the potential is zero or the probability density gives 1. It is defined by

\sphinxAtStartPar
\begin{equation}
f(r)=\exp\left (-\frac{U(r)}{k_B T}\right)-1 \tag{Mayer f-function}
\end{equation}

\noindent\sphinxincludegraphics{{tmp3}.pdf}

\sphinxAtStartPar
This definition makes sense if we consider the right example of a hard sphere interaction. The free volume can then be calculated by

\sphinxAtStartPar
\begin{equation}
v=-\int f(r)d^3r=\int 1-\exp\left (-\frac{U(r)}{k_B T}\right) d^3r
\end{equation}

\sphinxAtStartPar
which is giving in the case of the hard sphere interaction just \(4\pi R^3/3\), where R is the contact separation distance of the two spheres. Thus the volume integral over the Mayer f\sphinxhyphen{}function is directly giving the excluded volume for the pairwise interaction. Note that the excluded volume can also be negative. This happens, when the attractive interaction is very strong. It can also be zero, when both positive and negative areas under the Mayer f\sphinxhyphen{}function are of the same size.


\subsection{Polymer chain as a real gas}
\label{\detokenize{notebooks/L22/1_real_polymers:Polymer-chain-as-a-real-gas}}
\sphinxAtStartPar
Our further calculations will now base the excluded volume influence on the conformation of a polymer chain on the assumtion that all monomers (or actually Kuhn segments) are independet and not correlation in their position. They will thus behave like a real gas and show effect which we know from the \sphinxstylestrong{van der Waals gas}, such as co\sphinxhyphen{}volume and cohesive pressure. The two effects are actually corrections to the equation of state of the \sphinxstylestrong{ideal gas}, which can be written as

\sphinxAtStartPar
\begin{equation}
Z\equiv = \frac{pV_m}{R T}=1 \tag{compressibility factor}
\end{equation}

\sphinxAtStartPar
where \(p\) is the pressure, \(V_m\) the molar volume and \(R\) is the gas constant, i.e. \(R=k_B N_A\). For a real gas, the corrections have to scale with the density of the objects \(\rho=1/V_m\) and we can write the compressibility factor as a Taylor series expansion

\sphinxAtStartPar
\begin{equation}
Z=\frac{pV_m}{R T}= 1+ B(T)\rho +C(T)\rho^2 + D(T)\rho^3+\ldots \tag{virial expansion}
\end{equation}

\sphinxAtStartPar
which is the viral expansion of the compressibility factor for a real gas. The coefficients \(B(T),C(T),D(T)\) are called the viral expansion coefficient.

\sphinxAtStartPar
Let’s have a look at the first one in the case of a van der Waals gas, which is \(B(T)\). The presure of a vand der Waals gas reads

\sphinxAtStartPar
\begin{equation}
p=\frac{RT}{(V_m-b)}-\frac{a}{V_m^2}
\end{equation}

\sphinxAtStartPar
where \(b\) denotes the co\sphinxhyphen{}volume (the volume taken by the molecules themselves) and \(a\) amounts for the cohesive pressure of the gas molecules. If we set \(a=0\) we can write the compressibility factor

\sphinxAtStartPar
\begin{equation}
Z=\frac{PV_m}{RT}=\frac{1}{1-\frac{b}{V_m}}
\end{equation}

\sphinxAtStartPar
For \(b/V_m<1\), we can do a Taylor expansion which yields

\sphinxAtStartPar
\begin{equation}
Z=1+b\frac{1}{V_m}+b^2\frac{1}{V_m^2}+\ldots=1+b\rho +b^2 \rho^2 + \ldots
\end{equation}

\sphinxAtStartPar
Comparing with the virial expansion above yields the fact that the virial expansion coefficient
\begin{equation*}
\begin{split}B(T)=b\end{split}
\end{equation*}
\sphinxAtStartPar
where b was the molar co\sphinxhyphen{}volume in the van der Waals equation, e.g. the excluded volume. Correspondingly, the virial expansion coefficient is given by the integral over the Mayer f\sphinxhyphen{}function, i.e. in spherical coordinates
\begin{equation*}
\begin{split}B(T)=-\int f(r)d^3r=-4 \pi \int_{0}^{\infty}\left(e^{-\frac{U(r)}{k_B T}}-1\right) r^{2} d^3r\end{split}
\end{equation*}
\sphinxAtStartPar
Using this relation, we have a way to introduce the monomer\sphinxhyphen{}monomer and monomer\sphinxhyphen{}solvent interaction that is hidden in the May f\sphinxhyphen{}function into the equation of state of a real gas of polymer segements. We will come back to this solution.


\section{Non\sphinxhyphen{}spherical segments and Free energy of interaction of a real chain}
\label{\detokenize{notebooks/L22/1_real_polymers:Non-spherical-segments-and-Free-energy-of-interaction-of-a-real-chain}}
\sphinxAtStartPar
While we have assumed so far (without emphasizing it to much) that the interaction of the monomers is radially symmatric (monomers are spheres), typical monomers are rather rod\sphinxhyphen{}like. Thus we should get an idea about how the free volume changes when we go from a sphere to the rod. The procedure below considers the free energy of the interaction and replaces a rod by a set of spheres. We can get an idea about the free energy of a chain by coming back to our osmotic pressure formula

\sphinxAtStartPar
\begin{equation}
\Pi=n k_B T
\end{equation}

\sphinxAtStartPar
where \(n\) is the number density of molecules. The unit of pressure is \(N/m^2\) or \(J/m^3\), which is an energy density. The formula above tells us that adding a single molecule comes at some cost, which is \(k_B T\). Yet, this is an ideal gas consideration. If we want to add a contribution of the excluded volume, we can refer to the overlap fraction
\begin{equation*}
\begin{split}\Phi^*=b^3 \frac{N}{R^3}\end{split}
\end{equation*}
\sphinxAtStartPar
where \(b\) was the Kuhn length and \(R\) the root mean squared end\sphinxhyphen{}to\sphinxhyphen{}end distance. The number of monomer\sphinxhyphen{}monomer contacts was \(N\Phi^*\), which finally gives \(N^2/R^6\) contacts per volume. The free energy of interaction per volume is therefore

\sphinxAtStartPar
\begin{equation}
\frac{F_{\rm int}}{V}=k_B T(vc_m^2 + w c_m^3+\ldots)=k_B T (v\frac{N^2}{R^6}+w\frac{N^3}{R^9}+\ldots)
\end{equation}

\sphinxAtStartPar
where \(c_m=N/R^3\) is the monomer concentration. The first term in the above equation results from the monomer\sphinxhyphen{}monomer contacts as we introduced. The second term is accordingly the result of the contact between three monomers and so on. Therefore, this is also a virial expansion and the first virial expansion coefficient \(v\) is the excluded volume of the bimolecular monomer contact.

\sphinxAtStartPar
This consideration helps us to understand the role of non\sphinxhyphen{}spherical rodlike monomers or segments. If we have a rodlike segment of length \(b\) and radius \(d\), then the ro can be replaced by spherical monomers of radius \(d\). The number of spherical monomers per segment is then \(b/d\). Thus if a chain has \(N\) rodlike segments, the it can be replaced by \(n=Nb/d\) spherical monomers.

\sphinxAtStartPar
In case of hard\sphinxhyphen{}sphere interactions the overall contribution of the excluded volume to the free energy shall be independent of the fact if the rods are replaced by spheres so the term
\begin{equation*}
\begin{split}vN^2/R^6\end{split}
\end{equation*}
\sphinxAtStartPar
shall give the same results, and therefore
\begin{equation*}
\begin{split}v_s n^2 = v_c N^2\end{split}
\end{equation*}
\sphinxAtStartPar
or
\begin{equation*}
\begin{split}w_sn^3=w_c N^3\end{split}
\end{equation*}
\sphinxAtStartPar
With the spherical volume \(v_s=d^3\) and \(w_s=d^6\) we find
\begin{equation*}
\begin{split}v_c=v_s\left (\frac{n}{N}\right )^2=v_s\left(\frac{b}{d} \right )^2=b^2 d\end{split}
\end{equation*}
\sphinxAtStartPar
Thus the excluded volume for the interaction of two rods is larger than the actual rod volume itself, which is
\begin{equation*}
\begin{split}v_0=bd^2\end{split}
\end{equation*}
\sphinxAtStartPar
which is the result of random orientations. If the aspect ratio, i.e. the ratio of the two volumes \(v_c/v_0\) is large, this excess excluded volume is the driving force for a nematic (aligment) ordering of the rodlike segments as originally described by Onsager for liquid crystalline systems.


\section{Solvent Classification}
\label{\detokenize{notebooks/L22/1_real_polymers:Solvent-Classification}}\begin{itemize}
\item {} 
\sphinxAtStartPar
\sphinxstylestrong{Athermal Solvents} Here, \(v\) is independent of temperature. The system only features hard core repulsion and \(v ≈ b^2d\). Monomer\sphinxhyphen{}monomer contact is energetically indistinguishable from monomer\sphinxhyphen{}solvent contact, for example.

\item {} 
\sphinxAtStartPar
\sphinxstylestrong{Good solvents} Excluded volume is reduced due to monomer\sphinxhyphen{}monomer attraction. The effect of this attraction is greater at lower temperatures, causing a reduction in the excluded volume. \(0 < v < b^2d\)

\item {} 
\sphinxAtStartPar
\sphinxstylestrong{Theta solvent} The (positive) contribution to excluded volume from hard core repulsion is exactly balanced by that (negative) due to attractions and so \(v = 0\). The chains thus have nearly ideal conformations. This occurs at a temperature called the theta temperature,\(Θ\), which is analogous to the Boyle temperature in thermodynamics.

\item {} 
\sphinxAtStartPar
\sphinxstylestrong{Poor solvents} Excluded volume is negative due to large attractive interactions between the monomers, which prefer monomer\sphinxhyphen{}monomer contact strongly over monomer\sphinxhyphen{}solvent contact. Chain dimensions are reduced relative to ideal. \(−b^2d < v < 0\).

\item {} 
\sphinxAtStartPar
\sphinxstylestrong{Non\sphinxhyphen{}solvents} Here, \(v ≈ −b^2d\) and the polymer collapses into a very compact structure that excludes all solvent.

\end{itemize}


\section{Flory theory (in a good solvent)}
\label{\detokenize{notebooks/L22/1_real_polymers:Flory-theory-(in-a-good-solvent)}}
\sphinxAtStartPar
Flory treated the question of equilibrium conformation of real chains using a mean field approach. The equilibrium size is set by a balance between excluded volume which tends to expand the chain size, and a restoring force due to loss of conformational entropy due to swelling. The energetic contribution due to excluded volume is given by the number of excluded volume interactions within a coil and the cost of each exclusion, \(k_B T\). The number of excluded volume interactions is just the
probability of finding a monomer within the excluded volume of another. If we assume a mean density of monomers in the coil, \(N/R^3\), then the number of excluded volume interactions per monomer is \(vN/R^3\) and for \(N\) monomers in the coil, the energetic contribution is
\begin{equation*}
\begin{split}F_{\text {int }} \approx k_B T v \frac{N^{2}}{R^{3}}\end{split}
\end{equation*}
\sphinxAtStartPar
The entropic energy due to expansion of the coil is, as we have calculated before, given as
\begin{equation*}
\begin{split}F_{\text {entropic }} \approx k_B T \frac{R^{2}}{N b^{2}}\end{split}
\end{equation*}
\sphinxAtStartPar
which gives a total free energy
\begin{equation*}
\begin{split}F \approx k_B T\left(v \frac{N^{2}}{R^{3}}+\frac{R^{2}}{N b^{2}}\right)\end{split}
\end{equation*}
\sphinxAtStartPar
\sphinxstylestrong{insert plot here}

\sphinxAtStartPar
The total free enery as a function of the end to end distance \(R\) has two components which either decay with \(R^{-3}\) or increase with \(R^2\). Therefore this function has a minimum, i.e.
\begin{equation*}
\begin{split}\frac{\partial F}{ \partial R}\bigg |_{R_{F}}=0\end{split}
\end{equation*}
\sphinxAtStartPar
at a particular end to end distance, which is the so\sphinxhyphen{}called Flory radius \(R_F\). From the derivative we find
\begin{equation*}
\begin{split}R_{F} \approx v^{1 / 5} b^{2 / 5} N^{3 / 5}\end{split}
\end{equation*}
\sphinxAtStartPar
which shows, that the size of a real polymer chain scales with \(N^{3 / 5}\), where \(\nu=3/5=0.6\) is the Flory exponent, which has to be compared to \(\nu=1/2\) for an ideal (Gaussian chain). The polymer size thus scales stronger with the number of segments in the polymer, which seems at first glance a small difference, but due to the large numbers involved for \(N\), this could make quite some difference.

\sphinxAtStartPar
\sphinxstylestrong{insert experimental plot here}

\sphinxAtStartPar
The simple approach taken by Flory provides surprisingly good results \sphinxhyphen{} more modern theories/calculations provide \(R ∼ N^{0.588}\). However, the success of the Flory theory is due to a cancelation of errors. The excluded volume contributions are overestimated as correlations between monomers (which decrease the probability of overlap) are not considered. At the same time, the entropic restoring force is also overestimated. Nevertheless, such approaches based on a mean field approximation of
density combined with ideal chain conformation\sphinxhyphen{}derived entropy can provide quite useful results, for example in the case of an adsorbed chain. The treatment of the case for \(v < 0\) in the preceding simple manner leads to an aphysical result for the coil size that minimizes the total free energy. Stabilizing terms need to be considered.


\section{Flory Theory (in all solvents)}
\label{\detokenize{notebooks/L22/1_real_polymers:Flory-Theory-(in-all-solvents)}}
\sphinxAtStartPar
To extend the Flory theory for all kinds of solvents you have to go back to the virial expansion of the free energy density.
\begin{equation*}
\begin{split}\frac{F_{int}}{V k_B T}=F_{0}+ vc^2+ wc3+\ldots\end{split}
\end{equation*}
\sphinxAtStartPar
which essentially contains a contribution of the ideal chain, plus the corrections in orders of the segment density \(c\). The ideal chain therby contributes the entropic part and the real chain correction gives the free volume correction. The main difference now to the good solvent model is now that the entropic part above has to include some additional term to include a repulsive
\begin{equation*}
\begin{split}F_{\text {entropic }} \approx k_B T \frac{R^{2}}{N b^{2}}+ \underbrace{\frac{Nb^2}{R^2}}_{\rm new\, term}\end{split}
\end{equation*}
\sphinxAtStartPar
In this way, the free energy is
\begin{equation*}
\begin{split}\frac{F}{k_BT}=\frac{R^2}{Nb^2}+\frac{Nb^2}{R}+v \frac{N^2}{R^3}+w \frac{N^3}{R^6}+\ldots\end{split}
\end{equation*}
\sphinxAtStartPar
Defining now the ratio of the end to end distance of the real chain and the ideal chain gives
\begin{equation*}
\begin{split}\alpha^2=\frac{\langle R^2\rangle}{\langle R^2\rangle_0=\frac{R^2}{Nb^2}}\end{split}
\end{equation*}
\sphinxAtStartPar
results in
\begin{equation*}
\begin{split}\frac{F}{k_B T}=\alpha^2+\alpha^{-2}+ \frac{vN^{172}}{b^3}\alpha^{-3}+\frac{w}{b^6}\alpha^{-6}\end{split}
\end{equation*}
\sphinxAtStartPar
This allows us to see that for
\begin{itemize}
\item {} 
\sphinxAtStartPar
\(\alpha \gg 1\) we have a good solvent with \(R_F\propto N^{3/5}\) and an extended polymer conformation

\item {} 
\sphinxAtStartPar
\(\alpha \ll 1\) we find \(R_F\propto N^{1/3}\) with a collapsed polymer

\end{itemize}


\section{Temperature dependence of the chain size}
\label{\detokenize{notebooks/L22/1_real_polymers:Temperature-dependence-of-the-chain-size}}
\sphinxAtStartPar
Using the Mayer f\sphinxhyphen{}function we can now also write down some basic ideas about the temperature dependence of the chain size. This is essentially hidden in the Boltzman factor including the potential energy \(U(r)\).

\sphinxAtStartPar
If this potential energy is much bigger than the thermal eneryg, i.e. \(U(r)\gg k_B T\), then we are commonly at small distances \(r<b\) in the repulsive region. In this region, we can reduce the Mayer f\sphinxhyphen{}function to
\begin{equation*}
\begin{split}f(r)=\exp\left ( \frac{U(r)}{k_B T}\right)-1 \approx -1\end{split}
\end{equation*}
\sphinxAtStartPar
essentially to \(-1\).

\sphinxAtStartPar
If, on the other side the distance is larger than the Kuhn length (\(r>b\)), the interaction potential is small as compared to \(k_B T\) and we may write
\begin{equation*}
\begin{split}f(r)=\exp\left ( \frac{U(r)}{k_B T}\right)-1 \approx -\frac{U(r)}{k_B T}\end{split}
\end{equation*}
\sphinxAtStartPar
Folowing these approximation we may split the intergral, which yields the excluded volume into two parts
\begin{equation*}
\begin{split}v=-4\pi \int_0^{\infty}f(r)r^2 dr\approx 4\pi \int_0^{b}r^2 dr+\frac{4\pi}{k_B T}\int_b^{\infty}U(r)r^2dr\end{split}
\end{equation*}
\sphinxAtStartPar
This gives two terms, one the voume of the hard core repulsion and the second a term which comprises all the temperature dependent interaction. The dependencies can then be written as
\begin{equation*}
\begin{split}v\approx \left ( 1-\frac{\theta}{T}\right ) b^3\end{split}
\end{equation*}
\sphinxAtStartPar
where \(\theta\) is the theta temperature, which is defined as
\begin{equation*}
\begin{split}\theta\approx -\frac{1}{b^3 k_B}\int_B^{\infty}U(r) r^{2}dr\end{split}
\end{equation*}
\sphinxAtStartPar
According to this simplified formula for the excluded volume we see now the individual effects of solvent
\begin{itemize}
\item {} 
\sphinxAtStartPar
\(T<\theta\) means the free volume is negative and we are in a \sphinxstylestrong{poor solvent}

\item {} 
\sphinxAtStartPar
\(T=\theta\) means that we have \(v=0\), and we are in a \(\theta\)\sphinxhyphen{}solvent

\item {} 
\sphinxAtStartPar
\(T>\theta\) means that we have \(v>0\) and a \sphinxstylestrong{swelling} (growth) of the polymer

\item {} 
\sphinxAtStartPar
\(T\gg\theta\) means we are in the \sphinxstylestrong{athermal} situation such that \(v=b^{3}\)

\end{itemize}

\sphinxAtStartPar
\sphinxstylestrong{display experimental graph from Colby}



\sphinxAtStartPar
The following section was created from \sphinxcode{\sphinxupquote{notebooks/L23/1\_Scattering.ipynb}}.


\chapter{Scattering Techniques for Polymer Conformation}
\label{\detokenize{notebooks/L23/1_Scattering:Scattering-Techniques-for-Polymer-Conformation}}\label{\detokenize{notebooks/L23/1_Scattering::doc}}
\sphinxAtStartPar
The scattering of electromagnetic or matter waves is a commonly used tool to obtain information about the conformation of polymer chains. Light wave, X\sphinxhyphen{}rays (SAXS) but also neutrons (SANS) are used to study polymer solutions. While all waves have their own pecularities, they are all based on the intereference of partial waves scattered at different segments of the polymer.

\sphinxAtStartPar
A solution of different polymer chains (assume its dilute) has different length scales involved. There is an average distance between the polymer chains which is large as compared to the polymer size. Waves that interfere when being scattered from different polymers reflect the structure of the solution and is commonly addressed in a scattering quantity, which is the \sphinxstylestrong{structure factor}. Waves that are scattered by the same chain and interfer are a measure of the polymer conformation and
described by the \sphinxstylestrong{form factor}. Yet, this scale seperation might not always be possible in dense polymer solutions.


\section{Form Factor}
\label{\detokenize{notebooks/L23/1_Scattering:Form-Factor}}
\sphinxAtStartPar
We would like to consider the form factor only, which gives us information on the polymer conformation. We have an incident plane wave with the wave vector \(\vec{q}_i\) which is falling on our polymer chain. The wavevector has a direction and magnitude according to
\begin{equation*}
\begin{split}\vec{q}_i=\frac{2\pi}{\lambda}\vec{u}_i\end{split}
\end{equation*}
\sphinxAtStartPar
where \(\lambda\) is the wavelength. Note that you should take care of the corresponding refractive index \(n\) in the case of light scattering (\(\lambda=\lambda_0/n\)). We will neglect the refractive index in the following, as it does not change the qualitative results.

\noindent\sphinxincludegraphics[width=672\sphinxpxdimen,height=424\sphinxpxdimen]{{scattering}.png}

\sphinxAtStartPar
The scattered light is then observed in a different direction, e.g. along the direction \(\vec{u}_s\), which results in a scattered wavevector \(\vec{q}_s\). Considering now the scattered wave from two segments of the polymer at \(\vec{R}_i\) and \(\vec{R}_j\) results in a path length difference \(\Delta\) between the two waves, that is given by the indcident wave path different and the scattered wave path difference.
\begin{equation*}
\begin{split}\Delta = \vec{u}_i\cdot \vec{R}_j - \vec{u}_s\cdot \vec{R}_j=(\vec{u}_i-\vec{u}_s)\cdot \vec{R}_j\end{split}
\end{equation*}
\sphinxAtStartPar
This path difference translates into a phase difference if the waves
\begin{equation*}
\begin{split}\varphi_{j}=\frac{2 \pi}{\lambda}\left(\vec{u}_{\mathrm{i}}-\vec{u}_{\mathrm{s}}\right) \cdot \vec{R}_{j}=\left(\vec{q}_{\mathrm{i}}-\vec{q}_{\mathrm{s}}\right) \cdot \vec{R}_{j}=\vec{q} \cdot \vec{R}_{j}\end{split}
\end{equation*}
\sphinxAtStartPar
Since the magnitude of the incident and the scattered wavevectors are the same, i.e.
\begin{equation*}
\begin{split}\left|\vec{q}_{i}\right|=\left|\vec{q}_{s}\right|=\frac{2 \pi}{\lambda}\end{split}
\end{equation*}
\sphinxAtStartPar
we can express the magnitude of the vector \(\vec{q}\) denoting the momentum exchange during scattering, in terms of the scattering angle \(\theta\)
\begin{equation*}
\begin{split}q \equiv|\vec{q}|=2\left|\vec{q}_{i}\right| \sin \left(\frac{\theta}{2}\right)=\frac{4 \pi}{\lambda} \sin \left(\frac{\theta}{2}\right)\end{split}
\end{equation*}
\sphinxAtStartPar
With the help of the phase angle, we can now find the electric field amplitude of the scattered waves of all segments \(j\) with
\begin{equation*}
\begin{split}E_{\mathrm{s}}=E_{\mathrm{i}} \sum_{j=1}^{N} A \cos \left(2 \pi \nu t-\varphi_{j}\right)\end{split}
\end{equation*}
\sphinxAtStartPar
which provides with its magnitude square also the intensity of the scattered wave.
\begin{equation*}
\begin{split}\begin{aligned}
I_{\mathrm{s}} &=2 I_{\mathrm{i}} A^{2} \nu \int_{0}^{1 / \nu}\left[\sum_{j=1}^{N} \cos \left(2 \pi \nu t-\varphi_{j}\right)\right]^{2} \mathrm{~d} t \\
&=2 I_{\mathrm{i}} A^{2} \nu \int_{0}^{1 / \nu}\left[\sum_{j=1}^{N} \sum_{k=1}^{N} \cos \left(2 \pi \nu t-\varphi_{j}\right) \cos \left(2 \pi \nu t-\varphi_{k}\right)\right] \mathrm{d} t \\
&=I_{\mathrm{i}} A^{2} \nu \int_{0}^{1 / \nu}\left[\sum_{j=1}^{N} \sum_{k=1}^{N}\left(\cos \left(4 \pi \nu t-\varphi_{j}-\varphi_{k}\right)+\cos \left(\varphi_{k}-\varphi_{j}\right)\right)\right] \mathrm{d} t
\end{aligned}\end{split}
\end{equation*}
\sphinxAtStartPar
The first term in the sum turns out to be zero when integrated over time, such that only the difference of the the two individual phase angles is important.
\begin{equation*}
\begin{split}I_{s}(\vec{q})=\underbrace{I_{i} A^{2} N^{2}}_{I_{s}(0)} \underbrace{\frac{1}{N^{2}} \sum_{k=1}^{N} \sum_{j=1}^{N} \cos \left(\varphi_{k}-\varphi_{j}\right)}_{P(\vec{q})}\end{split}
\end{equation*}
\sphinxAtStartPar
The latter double sum inlcuding a prefactor defines the \sphinxstylestrong{from factor}
\begin{equation*}
\begin{split}P(\vec{q}) \equiv \frac{I_{\mathrm{s}}(\vec{q})}{I_{\mathrm{s}}(0)}\tag{form factor}\end{split}
\end{equation*}
\sphinxAtStartPar
Inserting the expression for the individual phase angles yields
\begin{equation*}
\begin{split}P(\vec{q})=\frac{1}{N^{2}} \sum_{i=1}^{N} \sum_{j=1}^{N} \cos \left[\vec{q} \cdot\left(\vec{R}_{i}-\vec{R}_{j}\right)\right]\end{split}
\end{equation*}
\sphinxAtStartPar
We therefore have to calculate the average over the phase angles or exchanged momentum times the position difference between two segments
\begin{equation*}
\begin{split}\left\langle\cos \left[\vec{q} \cdot\left(\vec{R}_{i}-\vec{R}_{j}\right)\right]\right\rangle=\frac{1}{4 \pi} \int_{0}^{2 \pi}\left[\int_{0}^{\pi} \cos \left(q R_{i j} \cos \alpha\right) \sin \alpha \mathrm{d} \alpha\right] \mathrm{d} \beta\end{split}
\end{equation*}
\sphinxAtStartPar
Using
\begin{equation*}
\begin{split}\vec{q} \cdot\left(\vec{R}_{i}-\vec{R}_{j}\right)=q R_{i j} \cos \alpha\end{split}
\end{equation*}
\sphinxAtStartPar
directly gives us the idea that the averaged cosine results in a sinus cardinalis
\begin{equation*}
\begin{split}\left\langle\cos \left[\vec{q} \cdot\left(\vec{R}_{i}-\vec{R}_{j}\right)\right]\right\rangle=\frac{1}{2} \int_{-1}^{1} \cos \left(q R_{i j} x\right) \mathrm{d} x=\frac{\sin \left(q R_{i j}\right)}{q R_{i j}}\end{split}
\end{equation*}
\sphinxAtStartPar
which is reminiscent of the single slit result or the Fourier transform of the segment distribution in the scattering objects components:
\begin{equation*}
\begin{split}P(q)=\frac{1}{N^{2}} \sum_{i=1}^{N} \sum_{j=1}^{N} \frac{\sin \left(q R_{i j}\right)}{q R_{i j}}\end{split}
\end{equation*}

\section{Radius of Gyration}
\label{\detokenize{notebooks/L23/1_Scattering:Radius-of-Gyration}}
\sphinxAtStartPar
We can relate the latter result directly with properties of the polymer chain, like the radius of gyration. If we consider only values of the sinc argument which are \(q R_ij<1\), i.e. we consider \sphinxstylestrong{small angle scattering}, we can expand the sinc function into
\begin{equation*}
\begin{split}\frac{\sin x}{x}=1-\frac{x^{2}}{3 !}+\frac{x^{4}}{5 !}-\cdots\end{split}
\end{equation*}
\sphinxAtStartPar
This yields a form factor
\begin{equation*}
\begin{split}P(q)=1-\frac{q^{2}}{6 N^{2}} \sum_{i=1}^{N} \sum_{j=1}^{N} R_{i j}^{2}+\cdots \quad \text { for } q R<1\end{split}
\end{equation*}
\sphinxAtStartPar
which gives with the definition of the radius of gyration directly
\begin{equation*}
\begin{split}P(q)=1-\frac{1}{3} q^{2}\left\langle R_{\mathrm{g}}^{2}\right\rangle+\cdots \quad \text { for } q R_{\mathrm{g}}<1\end{split}
\end{equation*}
\sphinxAtStartPar
or
\begin{equation*}
\begin{split}P(q)=1-\frac{16 \pi^{2} n^{2}}{3 \lambda^{2}}\left\langle R_{\mathrm{g}}^{2}\right\rangle \sin ^{2}\left(\frac{\theta}{2}\right)+\cdots\end{split}
\end{equation*}
\sphinxAtStartPar
when inserting \(q\). We will see in the derivation of the Debye function, that this inverse parabola can also be approximated with a Gaussian function
\begin{equation*}
\begin{split}P(q) \cong \exp \left(-\frac{q^{2} R_{\mathrm{g}}^{2}}{3}\right) \quad \text { for } q R_{\mathrm{g}}<1\end{split}
\end{equation*}
\sphinxAtStartPar
which is termed the \sphinxstylestrong{Guinier function}.


\section{Debye Function}
\label{\detokenize{notebooks/L23/1_Scattering:Debye-Function}}
\sphinxAtStartPar
The above consideration is not taking into account that not all distances \(R_{ij}\) have the same probability for a polymer chain. This was first taken into account by Pieter Debye in form of the ideal chain. The end\sphinxhyphen{}to\sphinxhyphen{}end distance distribution we have obtained earlier delivers the probability to find the ending segments of a polymer with \(N\) Kuhn segments of length \(b\) at a distance \(R\). We can use this expression as well to give a probability density to find two
segments \(i,j\) at a distance \(R_{ij}\), which is then given by
\begin{equation*}
\begin{split}p(R_{i j} ; \underbrace{|i-j|}_{N}) d R_{i j}=\left(\frac{3}{2 \pi|i-j| b^{2}}\right)^{\frac{3}{2}} e^{-3 R^{2}\left(2|i-j| b^{2}\right)} 4 \pi R_{i j}^{2} d R_{i j}\end{split}
\end{equation*}
\sphinxAtStartPar
Note that the number of segments takes here the value N=|i\sphinxhyphen{}j|. Using this density distribution, we have to weight each contribution of the form factor with the corresponding probability density, i.e.

\sphinxAtStartPar
\begin{eqnarray}
P(q)&=&\frac{1}{N^{2}} \sum_{i=1}^{N} \sum_{j=1}^{N} \int_{0}^{\infty} \frac{\sin \left(q R_{i j}\right)}{q R_{i j}} P\left(R_{i j},|i-j|\right) 4 \pi R_{i j}^{2} \mathrm{~d} R_{i j}\\
&=&\frac{1}{N^{2}} \sum_{i=1}^{N} \sum_{j=1}^{N} \exp \left(-\frac{q^{2} b^{2}|i-j|}{6}\right)
\end{eqnarray}

\sphinxAtStartPar
One can transform the double sum above in the case of large segment number into a double integral. The result is the Debye function
\begin{equation*}
\begin{split}P(q)=\frac{2}{\left(q^{2}\left\langle R_{\mathrm{g}}^{2}\right\rangle\right)^{2}}\left[\exp \left(-q^{2}\left\langle R_{\mathrm{g}}^{2}\right\rangle\right)-1+q^{2}\left\langle R_{\mathrm{g}}^{2}\right\rangle\right]\tag{Debye function}\end{split}
\end{equation*}
\sphinxAtStartPar
describing the scattering of an ideal chain, which is displayed in the graph below for two different radii of gyration.

\noindent\sphinxincludegraphics{{tmp11}.pdf}

\sphinxAtStartPar
We can have a look at different limits again, e.g. the small angle limit for which we find the same result as in the derivation in the previous section
\begin{equation*}
\begin{split}P(q) \cong\left(1-\frac{q^{2}\left\langle R_{\mathrm{g}}^{2}\right\rangle}{3}+\cdots\right) \quad \text { for } q \sqrt{\left\langle R_{\mathrm{g}}^{2}\right\rangle}<1\end{split}
\end{equation*}
\sphinxAtStartPar
Similarly, we find
\begin{equation*}
\begin{split}P(q) \cong \frac{2}{q^{2}\left\langle R_{\mathrm{g}}^{2}\right\rangle} \quad \text { for } q \sqrt{\left\langle R_{\mathrm{g}}^{2}\right\rangle}>1\end{split}
\end{equation*}
\sphinxAtStartPar
in the limit of large q, where the form factor decays as \(1/q^2\). Both limits are included in the plot above to see their range of validity.

\sphinxAtStartPar
While the Debye function is the result for a simple ideal Gaussian chain, the form factor can also be derived for many other polymer shapes. The table below reports some examples.

\noindent\sphinxincludegraphics[width=700\sphinxpxdimen,height=256\sphinxpxdimen]{{form}.png}



\sphinxAtStartPar
The following section was created from \sphinxcode{\sphinxupquote{notebooks/L23/2\_Viscoelasticity.ipynb}}.


\chapter{Viscoelasticity}
\label{\detokenize{notebooks/L23/2_Viscoelasticity:Viscoelasticity}}\label{\detokenize{notebooks/L23/2_Viscoelasticity::doc}}
\noindent\sphinxincludegraphics[width=680\sphinxpxdimen,height=257\sphinxpxdimen]{{ranges}.png}

\sphinxAtStartPar
Viscoelsatic behavior is a mixture of viscous behavior, which we know already from simple liquids and elastic behavior, which is typical for solids. Yet, this mixture is not just a simple superposition but often quite complex and depdning on the way mechanical deformation is introduced.

\sphinxAtStartPar
\sphinxstylestrong{insert sketch}

\sphinxAtStartPar
As already previously introduced, we can define a \sphinxstylestrong{shear stress}
\begin{equation*}
\begin{split}\sigma=\frac{F}{A}\end{split}
\end{equation*}
\sphinxAtStartPar
as the tangential force \(F\) to an area \(A\). As a response to such a stress, the material deforms and the deformation is called \sphinxstylestrong{shear strain}
\begin{equation*}
\begin{split}\gamma =\frac{\Delta x}{L}\end{split}
\end{equation*}
\sphinxAtStartPar
which corresponds directly to the angle in the above sketch as long as the displacement is small. Within this description all parts of the sample experience the same shear stress and strain if the material is uniform.

\sphinxAtStartPar
A perfectly elastic solid would give a very simple relation between stress and strain, i.e.
\begin{equation*}
\begin{split}\sigma=G \gamma\end{split}
\end{equation*}
\sphinxAtStartPar
where \(G\) is the shear modulus. This corresponds to Hooke’s law of elasticity, which is valid for small deformations or small strain. In case of a simple liquid with a dynamic viscosity \(\eta\), a constant strain leads to a zero shear stress. The liquid flow in response to the deformation and all flows have dissipated the initial deformation due to the internal friction of the liquid. Thus for liquids and viscous dissipation the \sphinxstylestrong{shear rate} may be of larger importance
\begin{equation*}
\begin{split}\dot{\gamma}=\frac{d\gamma}{dt}\end{split}
\end{equation*}
\sphinxAtStartPar
A constant stress in the liquid thus needs a constant strain rate such that
\begin{equation*}
\begin{split}\sigma=\eta \dot{\gamma} \tag{Newton's law of viscosity}\end{split}
\end{equation*}
\sphinxAtStartPar
The response of an elastic or viscous material to stepwise introduced stress or strain is therefore different and helps o classify the mechanical material response.

\noindent\sphinxincludegraphics[width=828\sphinxpxdimen,height=505\sphinxpxdimen]{{response}.png}

\sphinxAtStartPar
Accoding to this picture, and
\begin{itemize}
\item {} 
\sphinxAtStartPar
\sphinxstylestrong{elastic solid} responds with a step stress to a step strain, but also with a step strain to a step stress

\item {} 
\sphinxAtStartPar
\sphinxstylestrong{viscous liquid} responds with a spike in the stress to a step strain and a linear strain to the step stress

\item {} 
\sphinxAtStartPar
\sphinxstylestrong{viscoelastic material} responds with an (exponential) decay to a stepp strain and an (exponential) increase to a limiting value to a step stress

\end{itemize}

\sphinxAtStartPar
Viscoelasstic materials are thus materials that exhibit both viscous and elastic responses under applied mechanical stress and strain. As these responses are time dependent, viscoeleastic materials may show elastic behavior on short timescales but liquid on long timescales. Viscoelastic solids will always reach non\sphinxhyphen{}zero values for stress or strain after a certain amount of time, while for a viscoelastic material, the stress always decays to zero but the strain is able to grow without bounds.


\section{Models for Viscoelastic behavior}
\label{\detokenize{notebooks/L23/2_Viscoelasticity:Models-for-Viscoelastic-behavior}}

\section{Maxwell Model}
\label{\detokenize{notebooks/L23/2_Viscoelasticity:Maxwell-Model}}
\sphinxAtStartPar
A model for a viscoelastic liquid is the Maxwell model being a serial combination of a spring and a dashpot (viscous element).

\noindent\sphinxincludegraphics[width=310\sphinxpxdimen,height=158\sphinxpxdimen]{{maxwell}.png}


\section{Kelvin\sphinxhyphen{}Voigt}
\label{\detokenize{notebooks/L23/2_Viscoelasticity:Kelvin-Voigt}}
\sphinxAtStartPar
The Kelvin\sphinxhyphen{}Voigt model is a parallel combination of both elements and can grasp some features of a viscoelastic solid. It can however not describe the behavior for step strain.

\noindent\sphinxincludegraphics[width=219\sphinxpxdimen,height=176\sphinxpxdimen]{{kelvinvoigt}.png}


\section{Standard Model}
\label{\detokenize{notebooks/L23/2_Viscoelasticity:Standard-Model}}
\sphinxAtStartPar
A better model for visoelastic solids is the standard linear solid model comprising three mechanical elements.

\noindent\sphinxincludegraphics[width=271\sphinxpxdimen,height=195\sphinxpxdimen]{{standard}.png}

\begin{sphinxuseclass}{nbinput}
\begin{sphinxuseclass}{nblast}
{
\sphinxsetup{VerbatimColor={named}{nbsphinx-code-bg}}
\sphinxsetup{VerbatimBorderColor={named}{nbsphinx-code-border}}
\begin{sphinxVerbatim}[commandchars=\\\{\}]
\llap{\color{nbsphinxin}[ ]:\,\hspace{\fboxrule}\hspace{\fboxsep}}
\end{sphinxVerbatim}
}

\end{sphinxuseclass}
\end{sphinxuseclass}


\sphinxAtStartPar
The following section was created from \sphinxcode{\sphinxupquote{notebooks/L24/1\_Viscoelasticity.ipynb}}.


\chapter{Viscoelasticity}
\label{\detokenize{notebooks/L24/1_Viscoelasticity:Viscoelasticity}}\label{\detokenize{notebooks/L24/1_Viscoelasticity::doc}}
\sphinxAtStartPar
\sphinxhref{1\_Viscoelasticity.pdf}{Download Lecture as PDF}


\section{Maxwell Model}
\label{\detokenize{notebooks/L24/1_Viscoelasticity:Maxwell-Model}}
\noindent\sphinxincludegraphics[width=310\sphinxpxdimen,height=158\sphinxpxdimen]{{maxwell1}.png}

\sphinxAtStartPar
Combing back to the Maxwell model, we have a serial combination of viscous and elastic element. Thus both elements feel the same stress \(\sigma\) and the deformation is the sum of the elastic and viscous deformation, i.e. \(\gamma=\gamma_e+\gamma_v\).

\sphinxAtStartPar
We therefore find
\begin{equation*}
\begin{split}\sigma=G_M\gamma_e= \eta_M\frac{d\gamma}{dt}\end{split}
\end{equation*}
\sphinxAtStartPar
or
\begin{equation*}
\begin{split}\gamma_e=\frac{\eta_M}{G_M}\frac{d\gamma_v}{dt}\end{split}
\end{equation*}
\sphinxAtStartPar
where \(\eta_M/G_M\) defines a timescale for the viscous relaxation. Such a timescale is always present in viscoelastic materials. In most materials even multiple timescales are relevant.

\sphinxAtStartPar
In the following sections we explore the response of a viscoelastic material to different pertubations. They can be either formed by a \sphinxstylestrong{step strain}, a \sphinxstylestrong{steady shear} or a \sphinxstylestrong{step stress or creep}.


\section{Stress Relaxation after Step Strain}
\label{\detokenize{notebooks/L24/1_Viscoelasticity:Stress-Relaxation-after-Step-Strain}}
\sphinxAtStartPar
Here we impose a step strain \(\gamma\) at a time \(t=0\), such that the strain is constant for \(t>0\). Depending on the material we will get a different response. If we have for example an
\begin{itemize}
\item {} 
\sphinxAtStartPar
\sphinxstylestrong{elastic solid} will repond with a jump in the stress according to \(G\gamma\) and the stress will stay constant as long as the strain is constant

\item {} 
\sphinxAtStartPar
\sphinxstylestrong{Newtonian liquid} will respond with a stress spike that instantaneously decays to zero

\item {} 
\sphinxAtStartPar
\sphinxstylestrong{viscoelastic system} will respond with some time dependent stress \(\sigma(t)\)

\end{itemize}

\sphinxAtStartPar
Due to the time dependence of the stress \(\sigma(t)\) for a viscoelastic material we can generalize Hooke’s law to
\begin{equation*}
\begin{split}G(t)=\frac{\sigma(t)}{\gamma}\end{split}
\end{equation*}
\sphinxAtStartPar
The plot shows the reponse of either a viscoelastic liquid or a viscoelastic solid. The latter comprise a general elastic part, which lets the stress converge for infinity time to some constant value
\begin{equation*}
\begin{split}G_{\rm eq}=\lim_{t\rightarrow \infty} G(t)\end{split}
\end{equation*}
\sphinxAtStartPar
For viscoelestic liquids this residual stress is absent and the stress decays to zero.

\noindent\sphinxincludegraphics{{tmp}.pdf}

\sphinxAtStartPar
The Maxwell model, which is represented by a viscous dashpot element and an elastic spring represents such a viscoelastic liquid. The elastic strain thereby is
\begin{equation*}
\begin{split}\gamma_e=\frac{\eta_M}{G_M}\frac{d\gamma_v}{dt}=\tau_M\frac{d\gamma_v}{dt}\end{split}
\end{equation*}
\sphinxAtStartPar
Since the two elements are in series the elastic strain is also expressed by
\begin{equation*}
\begin{split}\gamma_e=\gamma-\gamma_v(t)\end{split}
\end{equation*}
\sphinxAtStartPar
Separation off variables leads to
\begin{equation*}
\begin{split}\frac{d\gamma_v(t)}{\gamma-\gamma_v(t)}=\frac{dt}{\tau_M}\end{split}
\end{equation*}
\sphinxAtStartPar
from which we obtain
\begin{equation*}
\begin{split}\ln(\gamma-\gamma_v(t))=\frac{-t}{\tau_M}+C\end{split}
\end{equation*}
\sphinxAtStartPar
with \(C=\ln(\gamma)\) based on the initial condition that \(\gamma(t=0)=\gamma\). As a result we obtain
\begin{equation*}
\begin{split}\gamma_e(t)=\gamma-\gamma_v(t)=\gamma \exp\left (- \frac{t}{\tau_M} \right )\end{split}
\end{equation*}
\sphinxAtStartPar
and for the stress
\begin{equation*}
\begin{split}\sigma(t)=G_M\gamma_e(t)=G_M \gamma \exp \left ( -\frac{t}{\tau_M}\right )\end{split}
\end{equation*}
\sphinxAtStartPar
Thus in the Maxwell model, the stress relaxation has a simple exponential decay with the characteristic time constant \(\tau_M\). Such a characteristic time constant is characteristic for all viscoelastic materials. Many materials, e.g., polymers have multiple relaxation modes, each with its own time constant \(\tau\) as we will see later.It turns out, that any stress relaxation modulus can be described by a series of Maxwell elements.


\subsection{Boltzmann Superposition}
\label{\detokenize{notebooks/L24/1_Viscoelasticity:Boltzmann-Superposition}}
\sphinxAtStartPar
The mechanical response of all materials has a region of linear response, where the relaxation modulus is independent of the strain. A manifestation of this linear response is the Boltzmann superposition principle.

\sphinxAtStartPar
Stress from any combination of small step strains is simply the linear combination of stresses resulting from each individual step.

\sphinxAtStartPar
If this individual strain step \(\delta \gamma\) is applied at \(t_i\), then the stress at a time \(t\) is given by
\begin{equation*}
\begin{split}\sigma(t)=\sum_{i}G(t-t_i)\delta \gamma_i\end{split}
\end{equation*}
\sphinxAtStartPar
This means that the stress of each individual step is independent of the other steps and the system remembers the deformations that were imposed earlier and continue to relaxa from earlier deformations as new ones are applied.

\sphinxAtStartPar
The stress relaxation modulus tells then how much stress remains at \(t\) past each deformation \(\delta \gamma_i\) through \(t-t_i\). With
\begin{equation*}
\begin{split}\delta \gamma_i=\dot{\gamma}_i\delta t_i\end{split}
\end{equation*}
\sphinxAtStartPar
we obtain
\begin{equation*}
\begin{split}\sigma(t)=\sum_i G(t-t_i)\dot{\gamma}_i\delta t_i\end{split}
\end{equation*}
\sphinxAtStartPar
or from a smooth history of strains
\begin{equation*}
\begin{split}\sigma(t)=\int_{-\infty}^{t}G(t-t^{\prime})\dot{\gamma}(t^{\prime})dt^{\prime}\end{split}
\end{equation*}
\sphinxAtStartPar
which tells you that the stress in any material is the result of all past deformations, which is expressed by this convolution. The momory of each past deformation only decays as the relaxation modulus decay ovet the elapsed time \(t-t^{\prime}\)


\section{Steady Shear}
\label{\detokenize{notebooks/L24/1_Viscoelasticity:Steady-Shear}}
\sphinxAtStartPar
We now switch to a different mode of mechanical perturbation of a material, which is the steady shear deformation. There we already considered the flow profile in the low Reynolds number section.

\noindent\sphinxincludegraphics[width=400\sphinxpxdimen,height=259\sphinxpxdimen]{{shear}.png}

\sphinxAtStartPar
Here, the shear rate is given by the speed of the top surface \(\dot{\gamma}=|\vec{v}|/h\), where \(h\) is the height of the film. According to our previous considerations the stress follows from

\sphinxAtStartPar
\begin{eqnarray}
\sigma(t)&=&\dot{\gamma}\int_{-\infty}^{t}G(t-t^{\prime})dt^{\prime}\\
&=& \dot{\gamma}\int_0^{\infty}G(s)dt
\end{eqnarray}

\sphinxAtStartPar
with \(s=t-t^{\prime}\). With Newtons law of viscosity follows that
\begin{equation*}
\begin{split}\eta=\int_{0}^{\infty}G(t)dt\end{split}
\end{equation*}
\sphinxAtStartPar
Using the Maxwell model this means that the viscosity is
\begin{equation*}
\begin{split}\eta=G\int_{0}^{\infty}\exp\left (-\frac{t}{\tau}\right )dt=G\tau=\eta_M\end{split}
\end{equation*}
\sphinxAtStartPar
which is also the result for a viscoelastic liquid. If the modulus \(G\) is not constant, the viscosity can be also approximated by \(\eta=G(\tau)\tau\).

\sphinxAtStartPar
For a viscoelastic solid, the moduls \(G(t)\) does not decay to zero in time, but stays constant for long times. As a result the integral will diverge, as solids have an infinite viscosity.

\sphinxAtStartPar
If the shear rate becomes very large, the linear response approximation we have asummed for the Boltzmann superposition will not hold and we will observe non\sphinxhyphen{}linear effects. This leads for example to a \sphinxstylestrong{shear thinning} in polymeric liquids but also to a \sphinxstylestrong{shear thickening} for other materials such as the corn starch.

\sphinxAtStartPar
In these nonlinear regions, we can still define an apparent viscosity as the ration of shear stress divided by shear strain. The viscosity we commonly refer to is, however, the viscosity measured in the limit of \(\gamma\ll \tau^{-1}\).

\sphinxAtStartPar
Note that all liquids display viscoelasticity even though the range where the viscosity is independent of the shear rate is very wide. For water, for example, the molecular relaxations are fast and the viscoelastic regimes starts at shear rates of about \(10^{10}\, s^{-1}\)


\section{Creep and Creep Recovery}
\label{\detokenize{notebooks/L24/1_Viscoelasticity:Creep-and-Creep-Recovery}}

\subsection{Creep}
\label{\detokenize{notebooks/L24/1_Viscoelasticity:Creep}}
\sphinxAtStartPar
Another mechanical perturbation is the step stress that can be applied to watch the strain relax. This is called \sphinxstylestrong{creep} and switching of a constant stress is called \sphinxstylestrong{creep recovery}. For the study of creep relaxation processes we define a new quantity, which is the inverse of the shear modulus. It is called the \sphinxstylestrong{shear creep compliance}

\sphinxAtStartPar
\begin{equation}
J(t)\equiv \frac{\gamma(t)}{\sigma} \tag{Creep Compliance}
\end{equation}

\sphinxAtStartPar
The relation of modulus and compliance is much like the relation of resistivity and conductivity. We can use again the Maxwell model to gain some insight. There the elastic element responds with an instantaneous strain
\begin{equation*}
\begin{split}\gamma_e=\frac{\sigma}{G_M}\end{split}
\end{equation*}
\sphinxAtStartPar
For the viscous element on the other hand we have
\begin{equation*}
\begin{split}\frac{d\gamma_v(t)}{dt}=\frac{\sigma}{\eta_M}\end{split}
\end{equation*}
\sphinxAtStartPar
which results in a linear growth of the viscous strain with time
\begin{equation*}
\begin{split}\gamma_v(t)=\frac{\sigma}{\eta_M}t\end{split}
\end{equation*}
\sphinxAtStartPar
The creep compliance in the Maxwell model is therefore
\begin{equation*}
\begin{split}J(t)=\frac{\gamma_e+\gamma_v(t)}{\sigma}=\frac{1}{G_M}+\frac{t}{\eta_M}\end{split}
\end{equation*}
\sphinxAtStartPar
which is overall linear in time. At time \(t=0\), the compliance is given by the elastic part \(1/G_M\), while the slope at long times is given by \(1/\eta\). For viscoelastic liquids we have in general
\begin{equation*}
\begin{split}J(t)=J_{eq}+\frac{t}{\eta}\end{split}
\end{equation*}
\sphinxAtStartPar
where \(J_{eq}\) corresponds to the energy stored in the elastic part of the liquid, but the compliance grows linearly with time. For a viscoelastic solid we know, however, that
\begin{equation*}
\begin{split}J_{eq}=\lim_{t\rightarrow \infty}=\frac{1}{\sigma}\lim_{t\rightarrow \infty}\gamma(t)=\frac{1}{G_{eq}}\end{split}
\end{equation*}
\sphinxAtStartPar
We can therefore use a different model to better describe a viscoelastic solid. This is done by the Kelvin Voigt model, where viscous and elastic part are connected in parallel.

\noindent\sphinxincludegraphics[width=219\sphinxpxdimen,height=176\sphinxpxdimen]{{kelvinvoigt1}.png}

\sphinxAtStartPar
In this case the stresses add up
\begin{equation*}
\begin{split}\sigma=\sigma_e(t)+\sigma_{v}(t)=G\gamma(t)+\eta \frac{d\gamma}{dt}\end{split}
\end{equation*}
\sphinxAtStartPar
Separation of variables yields
\begin{equation*}
\begin{split}\int_0^{\gamma}\frac{d\gamma(t)}{\frac{\sigma}{G}-\gamma(t)}=\frac{G}{\eta}\int_0^{t}dt\end{split}
\end{equation*}
\sphinxAtStartPar
which results in the strain
\begin{equation*}
\begin{split}\gamma(t)=\frac{\sigma}{G}(1-\exp(-t/(\eta/G))\end{split}
\end{equation*}
\sphinxAtStartPar
which gives for the creep compliance
\begin{equation*}
\begin{split}J(t)=\frac{\gamma(t)}{\sigma}=\frac{1}{G}\left (1-\exp(-t/\tau)\right)\end{split}
\end{equation*}
\sphinxAtStartPar
with \(J_{eq}=\lim_{t\rightarrow \infty}J(t)\).

\noindent\sphinxincludegraphics{{tmp12}.pdf}


\subsection{Creep Recovery}
\label{\detokenize{notebooks/L24/1_Viscoelasticity:Creep-Recovery}}
\sphinxAtStartPar
When switching of the stress, the viscoelastic material recovers and we get an elastic recoil. The reocovery strain is defined as
\begin{equation*}
\begin{split}\gamma_R =\gamma_0-\gamma(t)\end{split}
\end{equation*}
\sphinxAtStartPar
where the switch off happens at \(t=0\). The corresponding recovery compliance is then
\begin{equation*}
\begin{split}J_R(t)\equiv\frac{\gamma_R(t)}{\sigma}\end{split}
\end{equation*}
\sphinxAtStartPar
which directly gives
\begin{equation*}
\begin{split}J_R(t)=J(t)-\frac{t}{\eta}\end{split}
\end{equation*}
\sphinxAtStartPar
for a viscoelastic liquid. For a solid, the viscosity is infinite an we obtain
\begin{equation*}
\begin{split}J_R(t)=J(t)\end{split}
\end{equation*}\begin{equation*}
\begin{split}\lim_{t\rightarrow\infty} J_R(t)=\lim_{t\rightarrow \infty}\left [ J(t)-\frac{t}{\eta}\right ]=J_{eq}\end{split}
\end{equation*}

\section{Oscillatory Shear}
\label{\detokenize{notebooks/L24/1_Viscoelasticity:Oscillatory-Shear}}
\sphinxAtStartPar
All the previous consideration were done in the time domain with step\sphinxhyphen{}like perturbations. The step\sphinxhyphen{}like perturbation responses have an analog in the frequency domain, which is commonly measured by a type of “spectroscopy”, which is termed \sphinxstylestrong{rheology}. In rheology you apply an oscillating strain at a frequency \(\omega\), which can be written in the complex form as
\begin{equation*}
\begin{split}\gamma(t)=\gamma_0 \exp(i\omega t)\end{split}
\end{equation*}
\sphinxAtStartPar
For a pure eslastic solid, the response in instantaneous an the stress reads:
\begin{equation*}
\begin{split}\sigma(t)=G \gamma(t)=G \gamma_{0} e^{i \omega t}\end{split}
\end{equation*}
\sphinxAtStartPar
The stress is osciallting in phase with the strain. For a viscous liquid, however, the stress is related to the strain rate and thus
\begin{equation*}
\begin{split}\sigma(t)=\eta \frac{d \gamma(t)}{d t}=i \omega \eta \gamma_{0} e^{i \omega t}\end{split}
\end{equation*}
\sphinxAtStartPar
we obtain an out of phase signal, a purely imaginary response, of the stress, which trails the train as shown in the Figure below.

\noindent\sphinxincludegraphics{{tmp21}.pdf}

\sphinxAtStartPar
For a viscoelastic material, we have now both responses, which we can summarize in a complex modulus \(G\). The real part is called the \sphinxstylestrong{storage modulus} and the imaginary part is called the \sphinxstylestrong{loss modulus}
\begin{equation*}
\begin{split}G(\omega)=G^{\prime}(\omega)+i G^{\prime\prime}(\omega)\end{split}
\end{equation*}
\sphinxAtStartPar
Therefore, the time dependent stress is also a complex quantity and given by
\begin{equation*}
\begin{split}\sigma(t)=\underbrace{G \gamma_{0}}_{\sigma_{0}} e^{i \omega t}=\underbrace{G^{\prime} \gamma_{0}}_{\operatorname{Re} \sigma_{0}} e^{i \omega t}+i \underbrace{G^{\prime \prime} \gamma_{0}}_{\operatorname{Im} \sigma_{0}} e^{i \omega t}\end{split}
\end{equation*}
\sphinxAtStartPar
The phase angle \(\delta\) is given by the ratio of imaginary and real part
\begin{equation*}
\begin{split}\tan \delta=\frac{\operatorname{Im} G}{\operatorname{Re} G}=\frac{G^{\prime \prime}}{G^{\prime}}\end{split}
\end{equation*}
\sphinxAtStartPar
and the tangens of the phase angle is called the \sphinxstylestrong{loss tangent}.
\begin{equation*}
\begin{split}P=\frac{d W}{d t}=\frac{F d x}{d t}=\underbrace{\sigma A}_{F} \frac{L d \gamma}{d t}\end{split}
\end{equation*}\begin{equation*}
\begin{split}\frac{P}{V}=\sigma\left(\frac{d \gamma}{d t}\right)\end{split}
\end{equation*}\begin{equation*}
\begin{split}\frac{P}{V}=\frac{1}{2} \sigma_{0}\left(\frac{d \gamma}{d t}\right)_{0}^{*} = \frac{1}{2}\left(G^{\prime} \gamma_{0}+i G^{\prime \prime} \gamma_{0}\right)\left(-i \omega \gamma_{0}\right) =\frac{1}{2}(\underbrace{G^{\prime \prime}}_{\begin{array}{c}
\text { active } \\
\text { power }
\end{array}}-\underbrace{i G^{\prime}}_{\text {reactive }}) \gamma_{0}^{2} \omega\end{split}
\end{equation*}
\sphinxAtStartPar
The last part of the equation shows, that the imaginary part is responsible for the dissipation, while the real part adds some pseudo power much like the ac resistance in electronic circuitry.


\subsection{Complex shear modulus of the Maxwell model}
\label{\detokenize{notebooks/L24/1_Viscoelasticity:Complex-shear-modulus-of-the-Maxwell-model}}
\sphinxAtStartPar
The elastic part in the Maxwell model is again given by
\begin{equation*}
\begin{split}\gamma_e=\frac{\sigma}{G}\end{split}
\end{equation*}
\sphinxAtStartPar
while the viscous part is determined by the following differential equation
\begin{equation*}
\begin{split}\frac{d \gamma_{v}}{d t}=\frac{\sigma}{\eta}=\frac{\sigma_{0}}{\eta} e^{i \omega t}\end{split}
\end{equation*}
\sphinxAtStartPar
which yields
\begin{equation*}
\begin{split}\gamma_{v}=\frac{1}{i} \frac{\sigma_{0}}{\omega \eta} e^{i \omega t}=-i \frac{\sigma}{\omega \eta}\end{split}
\end{equation*}
\sphinxAtStartPar
or
\begin{equation*}
\begin{split}\sigma=i \omega \eta \gamma_{v}\end{split}
\end{equation*}
\sphinxAtStartPar
Thus inserting both in the total strain
\begin{equation*}
\begin{split}\underbrace{\gamma_{0} e^{i \omega t}}_{\gamma}=\underbrace{\frac{\sigma_{0}}{G} e^{i \omega t}}_{\gamma_e}-i\underbrace{ \frac{\sigma_{0}}{\omega \eta} e^{i \omega t}}_{\gamma_v}\end{split}
\end{equation*}
\sphinxAtStartPar
from which we can read the complex modulus

\sphinxAtStartPar
\begin{eqnarray}
G^{*}&=&\frac{\sigma_{0}}{\gamma_{0}}=\frac{1}{1 / G-i /(\omega \eta)}=\frac{\omega \eta}{\omega \eta / G-i}=\frac{\omega \eta(\omega \eta / G+i)}{(\omega \underbrace{\eta / G}_{\tau})^{2}+1}\\
&=&\underbrace{G \frac{1}{1+1 /(\omega \tau)^{2}}}_{G^{\prime}}+i \underbrace{G \frac{\omega \tau}{(\omega \tau)^{2}+1}}_{G^{\prime \prime}}
\end{eqnarray}

\sphinxAtStartPar
The plot below shows the frequency dependence of the real and imaginary part of the modulus for the Maxwell model. The imaginary part has the shape of a Lorentzian, which is corresponding to the Fourier transform of an exponential in the time domain. The wings of \(G^{\prime \prime}\) increase or decay with \(\omega\), while the real part \(G^{\prime}\) shows a steeper increase with \(\omega^2\) until it saturates. The saturation and the peak occur at the characteristic timescale
\(\tau=\eta/G\), indicated by the vertical dahsed line.

\noindent\sphinxincludegraphics{{tmp31}.pdf}


\subsection{Complex shear modulus of the Kelvin\sphinxhyphen{}Voigt model}
\label{\detokenize{notebooks/L24/1_Viscoelasticity:Complex-shear-modulus-of-the-Kelvin-Voigt-model}}\begin{equation*}
\begin{split}\underbrace{\sigma_{0} e^{i \omega t}}_{\sigma}=\sigma_{e}+\sigma_{v}=G \gamma+\eta \frac{d \gamma}{d t}=G \gamma_{0} e^{i \omega t}+i \omega \eta \gamma_{0} e^{i \omega t}\end{split}
\end{equation*}\begin{equation*}
\begin{split}G^{*}=\frac{\sigma_{0}}{\gamma_{0}}=G+i \omega \eta=\underbrace{G}_{G^{\prime}}+i \underbrace{G \omega \tau}_{G^{\prime \prime}} \tag{Complex modulus Kelvin-Voigt model}\end{split}
\end{equation*}


\sphinxAtStartPar
The following section was created from \sphinxcode{\sphinxupquote{notebooks/L25/1\_polymer\_dynamics.ipynb}}.


\chapter{Dynamics of Polymers}
\label{\detokenize{notebooks/L25/1_polymer_dynamics:Dynamics-of-Polymers}}\label{\detokenize{notebooks/L25/1_polymer_dynamics::doc}}
\sphinxAtStartPar
\sphinxhref{1\_polymer\_dynamics.pdf}{Download Lecture as PDF}

\sphinxAtStartPar
After we have introduced a phenomenological model for the viscoelasticity of materials, we would like to connect this to the dynamics of polymers. This means, that we have to connect the reponse of a polmyer to deformations to the polymer chain in a way.


\section{Diffusion of a Single Polymer Chain}
\label{\detokenize{notebooks/L25/1_polymer_dynamics:Diffusion-of-a-Single-Polymer-Chain}}
\sphinxAtStartPar
A first clue on the dynamics is already given by a process of the diffusion of the polymer chain. During diffusion, all segments fluctuate with the position and experience hdrodynamic friction. If the molecule would be a rigid sphere of radius \(R\), the diffusion coefficient would be given by the Stokes\sphinxhyphen{}Einstein relation
\begin{equation*}
\begin{split}D=\frac{k_\mathrm{B}T}{\gamma}=\frac{k_\mathrm{B} T}{6\pi\eta R}.\end{split}
\end{equation*}
\sphinxAtStartPar
Using the mean\sphinxhyphen{}squared displacement \(\langle r^{2}\rangle=6Dt\), this also means that the particle will diffuse a distance corresponding to its own size in a time
\begin{equation*}
\begin{split}\tau=\frac{R^2}{D}=\frac{R^2}{k_\mathrm{B} T}\gamma.\end{split}
\end{equation*}
\sphinxAtStartPar
For a polymer, which is now a flexible entity, this time means that all conformational fluctuations have relaxed at this time and the polymer chain is displaced by its own radius. This, therefore, sets the longest timescale of relaxation. This relaxation timescale can now be obtained with different dynamic approaches, which are the \sphinxstylestrong{Rouse model} and the \sphinxstylestrong{Zimm model}. Both models differ mainly in how the hydrodynamic coupling of individual segments are considered.


\subsection{Rouse Model}
\label{\detokenize{notebooks/L25/1_polymer_dynamics:Rouse-Model}}
\sphinxAtStartPar
In the Rouse model, the polymer is approximated by a bead\sphinxhyphen{}spring model (Gaussian chain) of \(N\) segments:
\begin{equation*}
\begin{split}\gamma\frac{\mathrm{d}\vec{R}_n}{\mathrm{d}t}=k\left [ \vec{R}_{n-1} - \vec{R}_{n} + \vec{R}_{n+1} - \vec{R}_{n}\right ] +\vec{f}_n(t),\end{split}
\end{equation*}
\sphinxAtStartPar
where the force on the segment \(n\) at a position \(\vec{R}_n\) is given by the forces exerted from the neighboring elements with a spring constant \(k=3k_\mathrm{B} T/b^2\) and the thermal noise force \(\vec{f}_n(t)\).

\sphinxAtStartPar
Each of the segments is experiencing a friction given by the factor \(\gamma\). The total friction on the Rouse chain is now assumed to be
\begin{equation*}
\begin{split}\gamma_\mathrm{R}=N\gamma\end{split}
\end{equation*}
\sphinxAtStartPar
which needs some short explanation, as this is one key assumption of the Rouse model. Imagine you have 5 different rigid spheres each experiencing the same friction coefficient and they are far apart from each other. The total friction coefficient will be \(5\gamma\). Yet if the spheres come closer to each other, the hydrodynamic flow fields around each sphere influence each other until the point when they are in close contact and act as a new body. In this case, the friction coefficient
will not be just the sum of all friction coefficients. This means, that the assumption of the Rouse model is now that the individual beads cause only localized flow fields when fluctuating, which are, in turn, not influencing the motion of the other segments.

\sphinxAtStartPar
According to that, we may write down the time to diffuse the size \(R\) of the polymer as
\begin{equation*}
\begin{split}\tau_\mathrm{R}=\frac{R^2}{k_\mathrm{B} T}N \gamma\end{split}
\end{equation*}
\sphinxAtStartPar
which is the Rouse time. At times shorter than the Rouse time, the polymer exhibits viscoelastic relaxation modes. The longest relaxation mode is that of the whole chain, which is the Rouse time \(\tau_\mathrm{R}\). As the size of the polymer chain is approximately given by \(R\approx bN^{\nu}\) with \(\nu\) beeing the fractal dimension (e.g., the Flory exponent) of the chain we obtain for the Rouse time
\begin{equation*}
\begin{split}\tau_\mathrm{R}=N \frac{\gamma}{\underbrace{k_\mathrm{B} T}_{\tau_{0}}} b^{2} N^{2 \nu}=\tau_{0} N^{2 \nu+1},\end{split}
\end{equation*}
\sphinxAtStartPar
where \(\tau_0\) is the relaxation time for a single segment in the chain. Depending if we now consider an ideal or a real chain, we find different scaling of the Rouse time with the number of segments, i.e.:
\begin{equation*}
\begin{split}\tau_\mathrm{R} \propto\left\{\begin{array}{l}
\tau_{0} N^{2}  \qquad \text { for ideal chain }(\nu=1 / 2) \\
\tau_{0} N^{11 / 5} \quad \,\,\text {for real chain in good solvent }(\nu \approx 3 / 5)
\end{array}\right.\end{split}
\end{equation*}
\sphinxAtStartPar
which is just a simple estimate.

\sphinxAtStartPar
The full calculation by Rouse for an ideal chain shows a similar result, which is
\begin{equation*}
\begin{split}\tau_\mathrm{R}=\frac{1}{6 \pi^{2}} \frac{\gamma b^{2}}{k_\mathrm{B} T} N^{2}.\end{split}
\end{equation*}
\sphinxAtStartPar
To summarize, on time scales smaller than \(\tau_{\rm R}\), we expect to find \sphinxstylestrong{viscoelastic modes} of the polymer contributing to the modulus, while for modes larger than \(\tau_{\rm R}\) everything should be \sphinxstylestrong{diffusive}.


\subsection{Zimm Model}
\label{\detokenize{notebooks/L25/1_polymer_dynamics:Zimm-Model}}
\sphinxAtStartPar
The Zimm model takes care of the fact that each segment is generating a flow field that decays as \(1/r\) and is thus typically long range. Thus segments in the chain volume (the so\sphinxhyphen{}called pervaded volume) are coupled by hydrodynamics in solution. This has to be taken into account for the dynamics of the polymer chain and the fact that the Rouse model neglects that restricts its validity essentially to the melt region (a system of polymer chains without solvent) only.

\sphinxAtStartPar
This hydrodynamic coupling in the Zimm model changes the friction coefficient of the chain to
\begin{equation*}
\begin{split}\gamma_\mathrm{Z}\approx \eta R\end{split}
\end{equation*}
\sphinxAtStartPar
with \(R\) being the root\sphinxhyphen{}mean\sphinxhyphen{}squared end\sphinxhyphen{}to\sphinxhyphen{}end distance of the polymwer chain as used also in the Rouse model. Thus the Zimm chain behaves hydrodynamically more like a solid sphere rather than a collection of individual beads.

\sphinxAtStartPar
With \(R=bN^{\nu}\) we find for the Zimm relaxation time
\begin{equation*}
\begin{split}\tau_{\rm Z} \approx \frac{R^{2}}{D_{\rm Z}} \approx \frac{\gamma_{\rm Z}}{k_{\rm B} T} R^{2} \approx \frac{\eta}{k_{\rm B} T} R^{3} \approx \frac{\eta b^{3}}{k_{\rm B} T} N^{3 \nu}=\tau_{0} N^{3 \nu}\end{split}
\end{equation*}
\sphinxAtStartPar
while the full calculation by Zimm shows

\sphinxAtStartPar
\begin{equation}
\tau_{\rm Z}=\frac{1}{2 \sqrt{3 \pi}} \frac{\eta}{k_{\rm B} T} R^{3}\tag{Zimm time}.
\end{equation}

\sphinxAtStartPar
Using the exponents \(\nu\) for the ideal and the real chains, we can now make predictions for the scaling of the Zimm relaxation time for both chains, which gives
\begin{equation*}
\begin{split}\tau_{\rm Z} \propto \begin{cases}\tau_{0} N^{3 / 2}  & \text { for ideal chain }(\nu=1 / 2) \\ \tau_{0} N^{9 / 5}  & \text { for real chain in good solvent }(\nu=3 / 5)\end{cases}.\end{split}
\end{equation*}
\sphinxAtStartPar
Thus, the scaling of relaxation times as predicted by the Zimm and the Rouse model are different. Which regime is valid has to be decided based on experimental results. As mentioned before, the Zimm model is rather valid for dilute polymer solutions, while the Rouse model rather applies to the dynamics of polymer chains in the melt.


\section{Intrinsic viscosity of polymer solutions}
\label{\detokenize{notebooks/L25/1_polymer_dynamics:Intrinsic-viscosity-of-polymer-solutions}}
\sphinxAtStartPar
In dilute solutions, polymer chains are isolated and deformed in an affine way. In such solutions, polymers linearly increase the viscosity of the solution with increasing concentration. To study the contribution of the polymer chains to the viscosity of the solution we define a specific viscosity
\begin{equation*}
\begin{split}\eta_{\rm sp}=\frac{\eta-\eta_{\rm s}}{\eta_{\rm s}},\end{split}
\end{equation*}
\sphinxAtStartPar
where \(\eta\) is the viscosity of the solvent with polymers and \(\eta_{\rm s}\) is the one of the solvent only. One may also write the same expression as \(\eta_{\rm sp}=\eta_{\rm r}-1\) where \(\eta_{\rm r}=\eta/\eta_{\rm s}\) is the reduced viscosity. The contribution of a single polymer chain is then measured by the intrinsic viscosity
\begin{equation*}
\begin{split}[\eta]=\lim_{c\rightarrow 0}\frac{\eta_{\rm sp}}{c},\end{split}
\end{equation*}
\sphinxAtStartPar
where \(c\) is the polymer concentration. Note that the intrinsic vsicosity now has the unit of an inverse concentration. Using this intrinsic value we may connect now the viscosity to the shear modulus, e.g., by using the result of the Maxwell model:
\begin{equation*}
\begin{split}\eta \approx G(\tau)\int \exp(-t/\tau)\mathrm{d}t=G(\tau)\tau.\end{split}
\end{equation*}

\subsection{Affine Deformation and Entropy}
\label{\detokenize{notebooks/L25/1_polymer_dynamics:Affine-Deformation-and-Entropy}}
\sphinxAtStartPar
To include the polymer in the shear viscosity, we have to consider an affine deformation of the chain.

\sphinxAtStartPar
\sphinxstylestrong{insert image here}

\sphinxAtStartPar
As we have discussed earlier, the entropy of a single chain with an end\sphinxhyphen{}to\sphinxhyphen{}end distance \(R\) can be calculated from

\sphinxAtStartPar
\begin{eqnarray}
S(R)&=&k_\mathrm{B} \ln(p(\vec{R},N))+{\rm const}\\
&=&-\frac{3}{2}\frac{k_\mathrm{B} R^2}{Nb^2}+{\rm const}.
\end{eqnarray}

\sphinxAtStartPar
In our lab system the end\sphinxhyphen{}to\sphinxhyphen{}end vector length \(R^2\) can be expressed by its components such that
\begin{equation*}
\begin{split}S(R)=-\frac{3}{2}\frac{k_B (R_x^2+R_y^2+R_z^2)}{Nb^2}+{\rm const}.\end{split}
\end{equation*}
\sphinxAtStartPar
If we now introduce a deformation
\begin{equation*}
\begin{split}L_x=\lambda_x L_{X0},\, L_y=\lambda_y L_{Y0},\, L_z=\lambda_z L_{Z0}\end{split}
\end{equation*}
\sphinxAtStartPar
where the \(\lambda\) are just scaling factors for the volume, we can also assume that the components of the end\sphinxhyphen{}to\sphinxhyphen{}end vectro scale in the same way, i.e.
\begin{equation*}
\begin{split}R_x=\lambda_x R_{X0},\, R_y=\lambda_y R_{Y0},\, R_z=\lambda_z R_{Z0}\end{split}
\end{equation*}
\sphinxAtStartPar
such that we obtain for the change in entropy upon deformation
\begin{equation*}
\begin{split}\Delta S=S(R)-S(R_0)=\frac{3}{2}\frac{k_B}{Nb^2}\left [(\lambda_x^2-1)R_{X0}^2 +(\lambda_y^2-1)R_{Y0}^2+(\lambda_z^2-1)R_{Z0}^2\right ]\end{split}
\end{equation*}
\sphinxAtStartPar
If we have now a system of \(n\)\sphinxhyphen{}chains in the solution, we have to sum up all squared components of the end\sphinxhyphen{}to\sphinxhyphen{}end distance
\begin{equation*}
\begin{split}\sum_{i=1}^{n}(R_{X0})_i^2=n\frac{1}{n}\sum_{i=1}^{n}(R_{X0})_i^2=n\langle R_{XO}^2\rangle=n\frac{Nb^2}{3}\end{split}
\end{equation*}
\sphinxAtStartPar
using our previous result for the mean squared end\sphinxhyphen{}to\sphinxhyphen{}end distance. This finally results in
\begin{equation*}
\begin{split}\Delta S=-\frac{n k_B}{2}(\lambda_x^2+\lambda_y^2+\lambda_z^2-3)\end{split}
\end{equation*}
\sphinxAtStartPar
for the entropy change of all chains. The results lets us calculate the change in free energy \(\Delta F=-T \Delta S\), which is
\begin{equation*}
\begin{split}\Delta F=-\frac{n k_B T}{2}(\lambda_x^2+\lambda_y^2+\lambda_z^2-3)\end{split}
\end{equation*}
\sphinxAtStartPar
If we assume now that the volume is unchanged upon deformation, i.e. \(V=\lambda_x\lambda_y\lambda_z V\) or \(\lambda_x\lambda_y\lambda_z=1\), we can express a uniaxial deformation along the x\sphinxhyphen{}axis as
\begin{equation*}
\begin{split}\lambda_x=\lambda.\, \lambda_y=\lambda_z=\frac{1}{\sqrt{\lambda}}\end{split}
\end{equation*}
\sphinxAtStartPar
from whgich we obtain for the deformation
\begin{equation*}
\begin{split}\Delta F= \frac{n k_B T}{2}(\lambda^2+\frac{2}{\lambda}-3)\end{split}
\end{equation*}
\noindent\sphinxincludegraphics{{test2}.pdf}

\sphinxAtStartPar
The corresponding force that has to be applied to stretch the molecules can be calculated from the derivative

\sphinxAtStartPar
\begin{eqnarray}
f_x&=&\frac{\partial \Delta F}{\partial L_x}=\frac{\partial \Delta F}{\partial\lambda L_{x0}}=\frac{1}{L_{x0}}\frac{\partial \Delta F}{\partial \lambda}\\
&=&\frac{nK_B T}{L_{x0}}(\lambda-\frac{1}{\lambda^2})
\end{eqnarray}

\sphinxAtStartPar
which results in a stress

\sphinxAtStartPar
\begin{eqnarray}
\sigma_{xx}&=&\frac{f_x}{L_yL_z}=\frac{nK_B T}{L_{x0}L_{y}L_{z}}\\
&=&\frac{nk_B T}{L_{x0}L_{y0}L_{z0}}\lambda\left (\lambda-\frac{1}{\lambda^2}\right )\\
&=&\frac{n}{V}k_B T \left (\lambda^2-\frac{1}{\lambda}\right )
\end{eqnarray}

\sphinxAtStartPar
As the stress is in general the result of a modulus \(G\) multiplied with a deformation and the term with the \(\lambda\) denotes the deformation, we have found the modulus to be
\begin{equation*}
\begin{split}G=\frac{n}{V}k_B T=\frac{\rho R_M T}{M_s}\end{split}
\end{equation*}
\sphinxAtStartPar
where \(R_M\) is gas constant, \(\rho\) the density and \(M_s\) the molar mass of a single strand. The stress is this
\begin{equation*}
\begin{split}\sigma=G \left (\lambda^2-\frac{1}{\lambda}\right )\end{split}
\end{equation*}
\sphinxAtStartPar
Returning to our calculation of the contribution of the polymer chain to the intrinsic viscosity we may write
\begin{equation*}
\begin{split}[\eta]=\frac{1}{c}\frac{\eta-\eta_{s}}{\eta_{s}}=\frac{1}{c}\frac{G(\tau)\tau}{\eta_s}\end{split}
\end{equation*}
\sphinxAtStartPar
with
\begin{equation*}
\begin{split}G(\tau)=\frac{\rho R_M T}{M_s}=\frac{\rho N_A k_B T}{N M_b}\end{split}
\end{equation*}
\sphinxAtStartPar
with \(M_b\) being the molar mass of the monomer (Kuhn segment). The intrinsic viscosity is therefore
\begin{equation*}
\begin{split}[\eta]=\frac{N_A k_B T}{ \eta_s N M_b}\tau\end{split}
\end{equation*}
\sphinxAtStartPar
With the result of Rouse for the Rouse time
\begin{equation*}
\begin{split}\tau_R=\frac{1}{6 \pi^{2}} \frac{\gamma b^{2}}{k_{B} T} N^{2\nu+1}=N\frac{\eta_s b}{k_B T}b^2 N^{2\nu}\frac{1}{6\pi^2}\end{split}
\end{equation*}
\sphinxAtStartPar
we find for the \sphinxstylestrong{intrinsic viscosity of a Rouse chain}
\begin{equation*}
\begin{split}[\eta]_R \approx \frac{N_A b^3}{M_b}N^{2\nu}\propto
\begin{cases}
N, & \text{ideal chain}\\
N^{6/5},  & \text{real chain in good solvent}
\end{cases}\end{split}
\end{equation*}
\sphinxAtStartPar
A similar calculation can be done using the Zimm relaxation time
\begin{equation*}
\begin{split}\tau_{Z} \approx \frac{\eta b^{3}}{k_{B} T} N^{3 \nu}\end{split}
\end{equation*}
\sphinxAtStartPar
which gives for the \sphinxstylestrong{intrinsic viscosity of a Zimm chain}
\begin{equation*}
\begin{split}[\eta]_Z \approx \frac{N_A b^3}{M_b}N^{3\nu-1}\propto
\begin{cases}
N^{1/2}, & \text{ideal chain}\\
N^{4/5},  & \text{real chain in good solvent}
\end{cases}\end{split}
\end{equation*}
\sphinxAtStartPar
A more general expression for the intrisinsic viscosity of a dilute polymer solution is the \sphinxstylestrong{Fox Flory} law
\begin{equation*}
\begin{split}[\eta]=\Phi \frac{R^3}{M}\end{split}
\end{equation*}
\sphinxAtStartPar
with \(\Phi=0.425 N_A\approx 2.5\times 10^{23}\, {\rm mol^{-1}}\).

\sphinxAtStartPar
\sphinxstylestrong{plot sime experimental data}


\section{Relaxation Modes of a Polymer}
\label{\detokenize{notebooks/L25/1_polymer_dynamics:Relaxation-Modes-of-a-Polymer}}
\sphinxAtStartPar
Polymers are self\sphinxhyphen{}similar objects, which means that shorter parts of a polymer behave like shorter chains. Thus there must be a spectrum of relaxation modes present for a polymer chain and we considered in the section before just the longest one.

\sphinxAtStartPar
If there are \(N\) segments, the
\begin{itemize}
\item {} 
\sphinxAtStartPar
there are N modes index by \(p=1\ldots N\)

\item {} 
\sphinxAtStartPar
shortest mode \(p=N\) with \(\tau_0\)

\item {} 
\sphinxAtStartPar
largest mode \(p=1\) with \(\tau_R\)

\end{itemize}


\subsection{Rouse modes}
\label{\detokenize{notebooks/L25/1_polymer_dynamics:Rouse-modes}}
\sphinxAtStartPar
In the Rouse model, we therefore have the Rouse time
\begin{equation*}
\begin{split}\tau=\tau_R\approx\tau_0 N^2\end{split}
\end{equation*}
\sphinxAtStartPar
and for the individua model
\begin{equation*}
\begin{split}\tau_p\approx \tau_0 \left ( \frac{N}{p}\right )^2\, \text{for}\, p=1,2,\ldots, N\end{split}
\end{equation*}
\sphinxAtStartPar
or
\begin{equation*}
\begin{split}p\approx \left( \frac{\tau_p}{\tau_0} \right )^{1/2} N\end{split}
\end{equation*}
\sphinxAtStartPar
Each of these modes relaxes independently.

\sphinxAtStartPar
On a timescale \(\tau=\tau_0\) all of the \(N\) modes have not relaxed. At \(\tau=\tau_p\) \(N-p\) modes have not decayed. Eahc of these modes contributes \(k_B T\) to the energy density as we have seen before. With \(N-p\) modes devayed, there is \(pk_B T\) still stored in the deformation of the chain. Thus the modulus amplitude for each mode is found to be
\begin{equation*}
\begin{split}G(\tau_P)\approx \frac{n}{V}k_B T p\end{split}
\end{equation*}
\sphinxAtStartPar
and the modulus itself
\begin{equation*}
\begin{split}G(t)\approx \frac{n}{V}k_B T \sum_{p=1}^{N}\exp\left (-\frac{t}{\tau_P}\right )\end{split}
\end{equation*}
\sphinxAtStartPar
with
\begin{equation*}
\begin{split}\tau_p=\frac{\tau_0}{6\pi^2}\frac{N^2}{p^2}\end{split}
\end{equation*}
\sphinxAtStartPar
For large \(N\) we may convert the sum into an integral and obtain for the modulus
\begin{equation*}
\begin{split}G(t)=n_{p} k_{B} T \int_{0}^{\infty} d p \exp \left(-p^{2} t / \tau_{R}\right)=\frac{\sqrt{\pi}}{2} n_{p} k_{B} T\left(\frac{\tau_{R}}{t}\right)^{1 / 2}\end{split}
\end{equation*}
\sphinxAtStartPar
which gives an inversse square root dependence for the modulus with time. This is valid only for short times, as the chain has a longest relaxation, which is teh Rouse time. Including this cut\sphinxhyphen{}off at long times we may write
\begin{equation*}
\begin{split}G(t)\approx \underbrace{\frac{nNk_B T}{v}\frac{t}{\tau_0}^{-1/2}}_{\tau_0<t<\tau_R}\underbrace{\exp\left ( -\frac{t}{\tau_R}\right )}_{t>\tau_R}\end{split}
\end{equation*}
\noindent\sphinxincludegraphics{{tmp4}.pdf}


\subsubsection{Shear modulus with oscillating deformation}
\label{\detokenize{notebooks/L25/1_polymer_dynamics:Shear-modulus-with-oscillating-deformation}}
\sphinxAtStartPar
With the calculations done above we may now also determine the frequency dependence of the shear modulus. Both can be obtained by Fourier transforming the time dependent modulus. The storage modulus \(G^{\prime}(\omega)\)
\begin{equation*}
\begin{split}G^{\prime}(\omega)=G_{eq}+\omega \int_0^{\infty}[G(t)-G_eq]\sin(\omega t)dt\end{split}
\end{equation*}
\sphinxAtStartPar
which gives
\begin{equation*}
\begin{split}G^{\prime}(\omega) \approx \frac{n k_B T}{V} \frac{\left(\omega \tau_{\mathrm{R}}\right)^{2}}{\sqrt{\left[1+\left(\omega \tau_{\mathrm{R}}\right)^{2}\right]\left[\sqrt{1+\left(\omega \tau_{\mathrm{R}}\right)^{2}}+1\right]}} \text { for } \omega<1 / \tau_{0}\end{split}
\end{equation*}
\sphinxAtStartPar
The loss modulus is found by
\begin{equation*}
\begin{split}G^{\prime\prime}(\omega)=\omega \int_0^{\infty}[G(t)-G_eq]\cos(\omega t)dt\end{split}
\end{equation*}
\sphinxAtStartPar
resulting in
\begin{equation*}
\begin{split}G^{\prime \prime}(\omega) \approx \frac{n}{V} k_B T \omega \tau_{\mathrm{R}} \sqrt{\frac{\sqrt{1+\left(\omega \tau_{\mathrm{R}}\right)^{2}}+1}{1+\left(\omega \tau_{\mathrm{R}}\right)^{2}}} \text { for } \omega<1 / \tau_{0}\end{split}
\end{equation*}
\sphinxAtStartPar
According to that both the storage and the loss modulus scale in the range \(\frac{1}{\tau_R}<\omega<\frac{1}{\tau_0}\) as \(G^{\prime}\propto G^{\prime\prime}\propto \omega^{1/2}\), while for \(\omega<\frac{1}{\tau_R}\) the storage modulus scales as \(G^{\prime}\sim \omega^2\).

\sphinxAtStartPar
\sphinxstylestrong{frequency dependence here}

\noindent\sphinxincludegraphics{{tmp6}.pdf}

\sphinxAtStartPar
The viscosity of a dilute Rouse polymer solution is the obtained by integrating the time dependent modulus
\begin{equation*}
\begin{split}\eta=\int_0^{\infty}G(t)dt \approx \frac{k_B T}{V}nN \int_{0}^{\infty} \left ( \frac{t}{\tau_0}\right )^{-1/2}\exp\left (-\frac{t}{\tau_R} \right )dt\end{split}
\end{equation*}
\sphinxAtStartPar
\sphinxstylestrong{add part on the diffusion coefficient}


\subsection{Zimm modes}
\label{\detokenize{notebooks/L25/1_polymer_dynamics:Zimm-modes}}
\sphinxAtStartPar
The same calculation as for the Rouse model can be done for the Zimm modes as well. The relaxation times for the Zimm model scale as
\begin{equation*}
\begin{split}\tau_p=\tau_0\left (\frac{N}{p} \right )^{3\nu}\end{split}
\end{equation*}
\sphinxAtStartPar
and we obtain for the mode index
\begin{equation*}
\begin{split}p\approx N \left ( \frac{\tau_p}{\tau_0}\right)^{-1/3\nu}=N\left ( \frac{t}{\tau_0}\right )^{-1/3\nu}\end{split}
\end{equation*}
\sphinxAtStartPar
where we make a transition from the mode relaxation times \(\tau_p\rightarrow t\). This then leads to the time dependent stress modulus
\begin{equation*}
\begin{split}G(t)\approx \underbrace{\frac{n}{V}Nk_B T \left ( \frac{\tau_0}{t} \right)^{1/3\nu}}_{\tau_0<t<\tau_Z}\underbrace{\exp\left (-\frac{t}{\tau_Z} \right )}_{t>\tau_Z}\end{split}
\end{equation*}
\noindent\sphinxincludegraphics{{tmp5}.pdf}


\subsubsection{Shear modulus with oscillating deformation}
\label{\detokenize{notebooks/L25/1_polymer_dynamics:id1}}
\sphinxAtStartPar
Using this result we may now also obtained the frequency dependent storage and loss modulus of the Zimm chains.

\sphinxAtStartPar
\sphinxstylestrong{storage modulus}
\begin{equation*}
\begin{split}G^{\prime}\approx \frac{n}{V}k_B T \frac{\omega \tau_Z \sin\left [(1-\frac{1}{3\nu})\arctan(\omega \tau_Z) \right ]}{\left [ 1+(\omega \tau_Z)^2\right ]^{(1-1/3\nu)/2}}\end{split}
\end{equation*}
\sphinxAtStartPar
\sphinxstylestrong{loss modulus}
\begin{equation*}
\begin{split}G^{\prime\prime}\approx \frac{n}{V}k_B T \frac{\omega \tau_Z \cos\left [(1-\frac{1}{3\nu})\arctan(\omega \tau_Z) \right ]}{\left [ 1+(\omega \tau_Z)^2\right ]^{(1-1/3\nu)/2}}\end{split}
\end{equation*}
\noindent\sphinxincludegraphics{{tmp7}.pdf}

\sphinxAtStartPar
The viscosity contribution of solution of Zimm chains is then
\begin{equation*}
\begin{split}\eta-\eta_s=\int_0^{\infty}G(t)dt \approx \eta_s \frac{n}{V}Nb^{3}N^{3\nu-1}\end{split}
\end{equation*}
\sphinxAtStartPar
We can summarize
\begin{itemize}
\item {} 
\sphinxAtStartPar
The \sphinxstylestrong{Zimm limit} applies to dilute solutions, where the solvent within the pervaded volume of the polymer is hydrodynamically coupled to the polymer. Polymer dynamics are described by the Zimm model in dilute solutions.

\item {} 
\sphinxAtStartPar
The \sphinxstylestrong{Rouse limit} applies to unentangled polymer melts because hydrodynamic interactions are screened in melts (just as excluded volume interactions are screened in melts). Polymer dynamics in the melt state (with no solvent) are described by the Rouse model, for short chains that are not entangled.

\end{itemize}



\sphinxAtStartPar
The following section was created from \sphinxcode{\sphinxupquote{notebooks/L26/1\_semidilute\_polymers.ipynb}}.


\chapter{Semidilute Polymer Solutions}
\label{\detokenize{notebooks/L26/1_semidilute_polymers:Semidilute-Polymer-Solutions}}\label{\detokenize{notebooks/L26/1_semidilute_polymers::doc}}
\sphinxAtStartPar
The properties of polymer solutions change when you increase the density of the solution. In the case of dilite solutions, we have already seen that the viscosity changes linearly with the concentration of the polymer. Whe the chains start to overlap, the neigboring chains impose constraints which influence the relaxation of the chains upon deformation.

\sphinxAtStartPar
To identify, when such a semi\sphinxhyphen{}dilute regime is reached, we can have a look at the volume fraction
\begin{equation*}
\begin{split}\phi=\frac{n V b^{3}}{V}\end{split}
\end{equation*}
\sphinxAtStartPar
where \(n\) is the number of polymer chains, \(V\) is the sample vlume and \(N\) is the number of Kuhn segements of length \(b\). If we have now polymer chains of a size \(R\) (end\sphinxhyphen{}to\sphinxhyphen{}end distance), with a volume of \(R^3\) we can then fill our sample volume space filling with \(n\) chains such that the sample volume \(V=nR^3\). This results in a concentration that is given by
\begin{equation*}
\begin{split}\phi^{*}\approx \frac{Nb^3}{R^3}\end{split}
\end{equation*}
\sphinxAtStartPar
which is the overlap concentration indicating that the chains start to overlap at this concentration. Since the size \(R\) of the chain can be now expressed by
\begin{equation*}
\begin{split}R\approx b \left (\frac{v}{b^{3}}\right )^{2\nu-1}N^{\nu}\end{split}
\end{equation*}
\sphinxAtStartPar
according to our consideration of the Flory chain (\(v\) free volume, \(\nu\) fractal dimension/Flory exponent), we find that the overlap concentration scales as
\begin{equation*}
\begin{split}\phi^{*}\approx N^{1-3\nu}\end{split}
\end{equation*}
\sphinxAtStartPar
with the number of segments in the chain. We therefore have
\begin{itemize}
\item {} 
\sphinxAtStartPar
a dilute solution for \(\phi<\phi^{*}\)

\item {} 
\sphinxAtStartPar
a semidilute solution for \(\phi^{*}<\phi\ll 1\)

\item {} 
\sphinxAtStartPar
a melt for \(\phi=1\)

\end{itemize}

\sphinxAtStartPar
\sphinxstylestrong{insert sketch showing this}

\sphinxAtStartPar
Due to the overlap of the chains in the semidilute regime, there must be some length scale \(\xi\) (called correlation length),such that
\begin{itemize}
\item {} 
\sphinxAtStartPar
for \(r<\xi\) the monomers are just surrounded by solvent mostly or same chain monomers

\item {} 
\sphinxAtStartPar
for \(r>\xi\) the monomer starts to see other chains

\end{itemize}

\sphinxAtStartPar
This correlation length helps us to subdivide the chain into blobs of a size \(\xi\). This size is determined by the free volume and the chain size again via
\begin{equation*}
\begin{split}\xi \approx b \left ( \frac{\nu}{b^3}\right )^{2\nu-1} g^{\nu}\end{split}
\end{equation*}
\sphinxAtStartPar
with \(g\) segments in the blob. The correlation volume is then given by \(\phi=gb^3/\xi^3\) such that we can express the blob size as
\begin{equation*}
\begin{split}\xi \approx b\phi^{-\frac{\nu}{3\nu-1}}
\begin{cases}\phi^{-1}\,&\text{for }\nu=1/2\\
\phi^{-3/4}\,&\text{for }\nu=3/5\end{cases}\end{split}
\end{equation*}
\sphinxAtStartPar
From this, the number of segements per blob follow to scale as
\begin{equation*}
\begin{split}g\approx \phi^{-\frac{1}{3\nu-1}}\end{split}
\end{equation*}
\sphinxAtStartPar
Since for \(r>\xi\) the monomers sees other chains, excluded volume interactions are screened. Thus, a semidilute solution of polymer is like a “melt” of blobs. The blobs are the new segments of a random walk of the size
\begin{equation*}
\begin{split}R\approx \xi \left ( \frac{N}{g}\right )^{1/2}\approx bN^{1/2}\phi^{-\frac{2\nu-1}{6\nu-2}}\begin{cases}
\phi^{0}\, & \text{for }\nu=1/2\\
\phi^{-1/8}\, & \text{for }\nu=3/5
\end{cases}\end{split}
\end{equation*}
\sphinxAtStartPar
The polymer size therefore changes only very weakly with the concentration. To understand the dynamics of these polymer solutions, we now need the two models we have refered to already earlier
\begin{itemize}
\item {} 
\sphinxAtStartPar
\sphinxstylestrong{Zimm model} which was valid for a dilute solution where monomers are hydrodynamically coupled

\item {} 
\sphinxAtStartPar
\sphinxstylestrong{Rouse model} which typically is valid in a melt in the absence of entanglements and neglects hydrodynamic coupling

\end{itemize}

\sphinxAtStartPar
Since there is a characteristic length scale in the structure of the semidilute solutions, the hydrodynamic coupling must also possess a characteristic length scale \(\xi_h\) such that
\begin{itemize}
\item {} 
\sphinxAtStartPar
when \(r<\xi_h\) the segments are coupled by hydrodynamics and the Zimm model can be applied

\item {} 
\sphinxAtStartPar
when \(r>\xi_h\) the hydrodynamics is screened and the system is comparable to a melt of blobs

\end{itemize}

\sphinxAtStartPar
In fact, this hydrodynamic length scale is assumed to be the same as the structural length scale \(\xi_h=\xi\) such that we have for the short length scales

\sphinxAtStartPar
\sphinxstylestrong{Zimm dynamics}

\sphinxAtStartPar
Here the chain sections feel the hydrodynamic coupling which leads to a maximum relaxation time
\begin{equation*}
\begin{split}\tau_{\xi}\approx \frac{\eta_s \xi^{3}}{k_B T}\approx \frac{\eta_s b^{}3}{k_B T}\phi^{-\frac{3\nu}{3\nu-1}}\end{split}
\end{equation*}
\sphinxAtStartPar
as already used earlier. Beyond this length scale the dynamics becomes Rouse like and we have the

\sphinxAtStartPar
\sphinxstylestrong{Rouse dynamics}

\sphinxAtStartPar
with the longest relaxation time being for the whole chain of blobs
\begin{equation*}
\begin{split}\tau_{chain}\approx \tau_{\xi}\left ( \frac{N}{g}\right )^{2}\approx \frac{\eta_s b^{3}}{k_B T}N^2 \phi^{\frac{2-3\nu}{3\nu-1}}\end{split}
\end{equation*}
\sphinxAtStartPar
scaling as
\begin{equation*}
\begin{split}\tau_{chain}\approx \begin{cases}
\phi\, & \text{for } \nu=1/2\\
\phi^{0.31}\, & \text{for } \nu=3/5
\end{cases}\end{split}
\end{equation*}
\sphinxAtStartPar
with the concentration of the chain \(\phi\). With this longest relaxation time for the chain we can now obtain the diffusion coefficient of a chain in the semi\sphinxhyphen{}dilute regime, which is
\begin{equation*}
\begin{split}D\approx \frac{R^2}{\tau_{chain}}\approx \frac{k_B T}{\eta_s b N} \phi^{-\frac{1-\nu}{3\nu -1}}\approx D_{z}\left (\frac{\phi}{\phi^{*}}\right )^{-\frac{1-\nu}{3\nu-1}}\end{split}
\end{equation*}
\sphinxAtStartPar
with \(D_z=k_B T/\eta_s b N^{\nu}\). The diffusion coefficient therefore scales as
\begin{equation*}
\begin{split}D\approx \begin{cases}
\phi^{0}\, & \text{for } \nu=1/2\\
\phi^{-0.54}\, & \text{for } \nu=3/5
\end{cases}\end{split}
\end{equation*}
\sphinxAtStartPar
This scaling for the real chain with an exponent of \(-0.54\) is also found in the experiments.

\sphinxAtStartPar
\sphinxstylestrong{insert experimetal image}


\section{Stress Relaxation}
\label{\detokenize{notebooks/L26/1_semidilute_polymers:Stress-Relaxation}}
\sphinxAtStartPar
The stress relaxation now follows the dynamics in the two regimes and can be obtained from the correponding relaxation time spectrum as we have introduced that also already earlier.

\sphinxAtStartPar
For \(\tau_0<t<\tau_{\xi}\):
\begin{equation*}
\begin{split}G(t)\approx \frac{k_B T}{b^3}\phi \left ( \frac{t}{\tau_0}\right )^{-\frac{1}{3\nu}}\end{split}
\end{equation*}
\sphinxAtStartPar
For \(\tau_{\xi}<t<\tau_{chain}\):
\begin{equation*}
\begin{split}G(t)\approx \frac{k_B T}{b^3}\phi^{\frac{3\nu}{3\nu-1}} \left ( \frac{t}{\tau_0}\right )^{-\frac{1}{2}}\end{split}
\end{equation*}
\begin{sphinxuseclass}{nbinput}
\begin{sphinxuseclass}{nblast}
{
\sphinxsetup{VerbatimColor={named}{nbsphinx-code-bg}}
\sphinxsetup{VerbatimBorderColor={named}{nbsphinx-code-border}}
\begin{sphinxVerbatim}[commandchars=\\\{\}]
\llap{\color{nbsphinxin}[ ]:\,\hspace{\fboxrule}\hspace{\fboxsep}}
\end{sphinxVerbatim}
}

\end{sphinxuseclass}
\end{sphinxuseclass}



\chapter{Indices and tables}
\label{\detokenize{index:indices-and-tables}}\begin{itemize}
\item {} 
\sphinxAtStartPar
\DUrole{xref,std,std-ref}{genindex}

\item {} 
\sphinxAtStartPar
\DUrole{xref,std,std-ref}{modindex}

\item {} 
\sphinxAtStartPar
\DUrole{xref,std,std-ref}{search}

\end{itemize}



\renewcommand{\indexname}{Index}
\printindex
\end{document}